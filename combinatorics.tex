\section{Вероятностный метод в комбинаторике}

\begin{problem}
На турнир приехало $n$ игроков. Каждая пара игроков, согласно регламенту турнира, должна провести одну встречу (ничьих быть не может). Пусть 
$$
C_n^k\cdot (1-2^{-k})^{n-k}<1 . 
$$
Докажите, что тогда игроки могли сыграть так, что для каждого множества из $k$ игроков найдется игрок, который побеждает их всех. 

\end{problem}

\begin{ordre}
Введем $A_K$ --- событие, состоящее в том, что не существует игрока, побеждающего всех игроков из множества $K$. 
Докажем, что 
$$
{\mathbb P}\bigl(\bigcup\limits_{K\subset\{1,..,n\},|K|=k} A_K \bigr)\leqslant C_n^k\cdot (1-2^{-k})^{n-k} . 
$$

\end{ordre}


\begin{problem}
Рассмотрим матрицу $n\times n$, составленную из лампочек, каждая из которых любо включена $(a_{ij}=1)$, либо выключена $(a_{ij}=-1)$. 
Предположим, что для каждой строки и каждого столбца имеется переключатель, поворот которого ($x_i=-1$ для строки $i$ и 
$y_j=-1$ для столбца $j$) переключает все лампочки в соответствующей линии: с <<вкл.>> на <<выкл.>> и с <<выкл.>> на <<вкл.>>. 
Тогда для любой начальной конфигурации лампочек можно установить такое положение переключателей, что разность между числом включенных и 
выключенных лампочек будет не меньше $(\sqrt{2/\pi}+o(1))n^{3/2}$. 
\end{problem}

\begin{ordre}
Рассмотрите  переключатель по столбцам как случайную величину, принимающую с равной вероятностью значения 1, -1. Каждому переключателю по столбцам необходимо подобрать переключатель по строкам, максимизирующий разность включенных и 
выключенных лампочек. Распределение данной разности можно оценить при помощи ц.п.т.       
\end{ordre}


\begin{problem}
Поверхность некоторой шарообразной планеты состоит из океана и суши (множество мелких островков). Суша занимает больше половины 
площади планеты. Также известно, что суша есть множество, принадлежащее борелевской  $\sigma$-алгебре на сфере. На планету хочет 
совершить посадку космический корабль, сконструированный так, что концы всех шести его ножек лежат на поверхности планеты. 
Посадка окажется успешной, если не меньше четырех ножек из шести окажутся на суши. Возможна ли успешная посадка корабля на планету?
\end{problem}

\begin{ordre}
Задача на линейность математического ожидания.
\end{ordre}





\begin{problem}
Покажите, что можно так раскрасить в два цвета ребра полного графа с $n$ вершинами (т.е. графа (без петель), в котором любые две 
различные вершины соединены одним ребром), что любой его полный подграф с $m$ вершинами, где 
$2C_n^m (\left.1\right/2)^{C_m^2}<1$, имеет ребра разного цвета. 
\end{problem}


\begin{problem}[устойчивые системы большой размерности; В.И. Опойцев, 1985]
Из курсов функционального анализа и вычислительной математики хорошо известно, что если спектральный радиус матрицы 
$A=\| a_{ij}\|_{i,j=1}^{n}$ меньше единицы: $\rho(A)<1$, то итерационный процесс ${\vec x}^{n+1}=A{\vec x}^n +{\vec b}$ 
(СОДУ $\dot{\vec x}=-{\vec x}+A{\vec x}+{\vec b}$), вне зависимости от точки старта ${\vec x}^0$, 
сходится к единственному решению уравнения ${\vec x}^*=A{\vec x}^*+{\vec b}$. 
Скажем, если $\| A\|=\max\limits_{i} \sum\limits_j |a_{ij}|<1$, то и $\rho(A)<1$ (обратное, конечно, не верно). Предположим, что 
существует такое маленькое $\varepsilon>0$, что 
$$
\frac{1}{n}\sum\limits_{i,j} |a_{ij}|<1-\varepsilon . 
\quad (S)
$$
Очевидно, что отсюда, тем более, не следует: $\rho(A)<1$. 
Тем не менее, введя на множестве матриц, удовлетворяющих условию $(S)$, равномерную меру, покажите, что относительная мера тех матриц 
(удовлетворяющих условию $(S)$), для которых спектральный радиус не меньше единицы, стремится к нулю 
с ростом $n$ ( $\varepsilon$ --- фиксировано и от $n$ не зависит). 
\end{problem}


\begin{problem}[модель случайного графа Эрдёша -- Реньи, 1959-1961]
Задан случайный граф $G(n,p)$ (на n вершинах, любые две из которых 
соединяются ребром с вероятностью p, $p\in [0,1])$. Случайная величина $T_n 
(G)$ - равна числу треугольников, образованных ребрами, в случайном графе 
$G(n,p)$. Воспользовавшись неравенством Маркова, доказать, что если 
$p=o\left( {1 \mathord{\left/ {\vphantom {1 n}} \right. 
\kern-\nulldelimiterspace} n} \right)$, то почти наверное треугольников в 
случайном графе нет, т.е.
\[
P\left\{ {T_n (G)=0} 
\right\}\mathrel{\mathop{\kern0pt\longrightarrow}\limits_{n\to \infty }} 1.
\]
\end{problem}


\begin{problem}
$V=\left\{ {1,...,m} \right\}$, ${\rm M}=\left\{ {M_1 
,...,M_n } \right\}$, $M_k \subseteq V$.

$\chi :\quad V\to \left\{ {-1,1} \right\}$ (можно интерпретировать, как 
раскраску множества V в два цвета).

$\chi (M_i )=\sum\limits_{a\in M_i } {\chi (a)} $ ($\left| {\chi (M_i )} 
\right|$ отвечает за ``равномерность'' покраски множества $M_i $ в два 
цвета).

$disc({\rm M},\chi )=\mathop {\max }\limits_{i=1..n} \left| {\chi (M_i )} 
\right|$ (от слова discrepancy - уклонение) - мера того, что хотя бы один 
объект в ${\rm M}$ раскрашен ``неравномерно''.

$disc({\rm M})=\mathop {\min }\limits_\chi disc({\rm M},\chi )$(``поиск'' 
наилучшей раскраски).

Показать, что для $\forall n\;\forall m\;\forall {\rm M} \quad disc({\rm M})\le 
\sqrt {2m\ln (2n)} $. Т. е. $\exists \chi :\;disc({\rm M},\chi )\le \sqrt 
{2m\ln (2n)} $.

\end{problem}


\begin{problem}
Оцените, сколько можно найти подмножеств множества 
$\left\{ {1,2,...,n} \right\}$ таких, что симметрическая разность любых двух 
из них имеет мощность не менее $n \mathord{\left/ {\vphantom {n 3}} \right. 
\kern-\nulldelimiterspace} 3$?
\end{problem}



\begin{problem}[вероятностный метод в теории чисел; Харди -- Рамануджан -- 
Туран -- Эрдёш - Кац, 1920, 1934, 1940]
Пусть $\nu \left( n \right)$ 
обозначает количество простых чисел $p$, делящих $n$. Тогда для любого 
$\lambda $
\[
\mathop {\lim }\limits_{n\to \infty } \frac{1}{n}\left| {\left\{ {k:\;1\le 
k\le n,\;\nu \left( k \right)\ge \ln \ln n+\lambda \sqrt {\ln \ln n} } 
\right\}} \right|=\frac{1}{\sqrt {2\pi } }\int\limits_\lambda ^\infty 
{e^{-{t^2} \mathord{\left/ {\vphantom {{t^2} 2}} \right. 
\kern-\nulldelimiterspace} 2}dt} .
\]
\end{problem}



\begin{problem}[обобщенная схема размещений]
$ $

\begin{enumerate}
\item  Пусть для неотрицательных целочисленных с.в. $\eta _1 
$,{\ldots},$\eta _N $ существуют независимые одинаково распределенные с.в. 
$\xi _1 $,{\ldots},$\xi _N $ такие, что

$P\left( {\eta _1 =k_1 ,...,\eta _N =k_N } \right)=P\left( {\left. {\xi _1 
=k_1 ,...,\xi _N =k_N } \right|\xi _1 +...+\xi _N =n} \right).$ (**)

Введем независимые одинаково распределенные с.в. $\xi _1^{\left( r \right)} 
$,{\ldots},$\xi _N^{\left( r \right)} $, где $r$ целое неотрицательное число 
и
\[
P\left( {\xi _1^{\left( r \right)} =k} \right)=P\left( {\left. {\xi _1 =k} 
\right|\xi _1 \ne r} \right),
\quad
k=0,1,...
\]
Пусть $p_r =P\left( {\xi _1 =r} \right)$ и $S_N =\xi _1 +...+\xi _N $, 
$S_N^{\left( r \right)} =\xi _1^{\left( r \right)} +...+\xi _N^{\left( r 
\right)} $. Пусть $\mu _r \left( {n,N} \right)$ - число тех с.в. $\eta _1 
$,{\ldots},$\eta _N $, принявших значение $r$. Покажите, что с.в. типа $\mu 
_r \left( {n,N} \right)$ можно изучать с помощью \textit{обобщенной схемы размещений}: для любого $k=0,1,...,N$
\[
P\left( {\mu _r \left( {n,N} \right)=k} \right)=C_n^k p_r^k \left( {1-p_r } 
\right)^{N-k}\frac{P\left( {S_{N-k}^{\left( r \right)} =n-kr} 
\right)}{P\left( {S_N =n} \right)}.
\]
Напомним, что в классической схеме размещений $n$ различных частиц по $N$ 
различным ячейкам было доказано, что распределение заполнений ячеек $\eta _1 
$,{\ldots},$\eta _N $ имеет вид:
\[
P\left( {\eta _1 =k_1 ,...,\eta _n =k_n } \right)=\frac{n!}{k_1 !....k_N 
!N^n},
\]
где $k_1 $,{\ldots},$k_n $ - неотрицательные целые числа такие, что $k_1 
+....+k_n =n$. Если положить $\xi _1 $,{\ldots},$\xi _N \in Po\left( \lambda 
\right)$ - i.i.d. ($\lambda >0$ - произвольно), то получим (**).

\item Дан случайный граф (модель Эрдеша - Реньи) $G\left( {n,\;p} \right)$. Пусть 
$p=c\frac{\ln n}{n}$. Покажите, что при $c>1$ граф $G\left( {n,\;p} \right)$ 
почти наверное связен.


\end{enumerate}
\end{problem}







