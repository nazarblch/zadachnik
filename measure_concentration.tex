\section{Концентрация меры}
\subsection{Введение}

\textbf{
TODO Общие результаты про классы функций Леви, теорема Дворецкого, примеры без доказательств}

Число $\mu _f $ называют медианой функции $f$, если
\begin{equation*}
\mu \left( {\vec {x}\in S_1^n :\;\;f\left( {\vec {x}} \right)\ge \mu _f } 
\right)\ge 1 \mathord{\left/ {\vphantom {1 2}} \right. 
\kern-\nulldelimiterspace} 2
\quad \text{и} \quad 
\mu \left( {\vec {x}\in S_1^n :\;\;f\left( 
{\vec {x}} \right)\le \mu _f } \right)\ge 1 \mathord{\left/ {\vphantom {1 
2}} \right. \kern-\nulldelimiterspace} 2,
\end{equation*}

где $\mu \left( {d\vec {x}} \right)$ - равномерная мера на единичной сфере 
$S_1^n $ в ${\rm R}^n$. Пусть $A$ - измеримое (борелевское) множество на 
сфере $S_1^n $. Через $A_\delta $ - будем обозначать $\delta $-окрестность 
множества $A$ на сфере $S_1^n $.  Оказывается, что если $\mu \left( A \right)\ge 1 \mathord{\left/ 
{\vphantom {1 2}} \right. \kern-\nulldelimiterspace} 2$, то
\[
\mu \left( {A_\delta } \right)\ge 1-\sqrt {\pi \mathord{\left/ {\vphantom 
{\pi 2}} \right. \kern-\nulldelimiterspace} 2} \exp \left( {-{\delta ^2n} 
\mathord{\left/ {\vphantom {{\delta ^2n} 2}} \right. 
\kern-\nulldelimiterspace} 2} \right).
\]

Иллюстрацией этому результату является задача о царице Дидоне: царь предложил царице построить 
огород с заданной длиной забора,а царица хочет, чтобы её участок земли при заданном 
периметре имел наибольшую площадь. Таким образом, царице надо решить 
изопериметрическую задачу (такие задачи обычно рассматриваются в курсах 
вариационного исчисления). Решение этой задачи хорошо известно -- <<круглый 
огород>>. Рассмотрение двойственной задачи (при заданной площади огорода, минимизировать его периметр) приводит к результату для $\mu(A_{\delta})$. 


Пусть теперь на $S_1^n $ задана функция с модулем непрерывности
\[
\omega _f \left( \delta \right)=\sup \left\{ {\left| {f\left( {\vec {x}} 
\right)-f\left( {\vec {y}} \right)} \right|:\;\;\rho \left( {\vec {x},\vec 
{y}} \right)\le \delta ,\;\vec {x},\vec {y}\in S_1^n } \right\}.
\]
Тогда
\[
\mu \left( {\vec {x}\in S_1^n :\;\;\left| {f\left( {\vec {x}} \right)-\mu _f 
} \right|\ge \omega _f \left( \delta \right)} \right)\le \sqrt {\pi 
\mathord{\left/ {\vphantom {\pi 2}} \right. \kern-\nulldelimiterspace} 2} 
\exp \left( {-{\delta ^2n} \mathord{\left/ {\vphantom {{\delta ^2n} 2}} 
\right. \kern-\nulldelimiterspace} 2} \right).
\]




Можно показать, что при весьма естественных условиях медиана асимптотически 
близка к среднему значению (математическому ожиданию). Аналогичное 
неравенство можно получить (М. Талагран, 1994), например, для модели 
случайных графов (Эрдёша - Реньи). И исследовать плотную концентрацию около 
среднего значения различные функций на случайных графов: число 
независимости, хроматическое число и т.п..

\medskip

Уход от стандартных вероятностных неравенств заключается в обобщении неравенств об  отклонении линейной функции  случайных переменных от своего среднего значения  на более сложные функции случайных переменных, например, Липшицевы функции, функции из класса Гельдера.
%\medskip
%\textbf{
%TODO Неравенство Талаграна 
%}
\medskip

Интересным примером концентрации меры являются следующие результаты Вершика. 

\textbf{Предельные меры для задачи о распределении максимального цикла в подстановке и др.}
В качестве множества элементарных исходов рассматривается группа всевозможных 
подстановок (перестановок) $S_n $ (симметрическая группа), $n\gg 1$. В этой 
группе $n!$ элементов. Припишем каждой подстановке одинаковую вероятность $1 
\mathord{\left/ {\vphantom {1 {n!}}} \right. \kern-\nulldelimiterspace} 
{n!}$. Оказывается, что верны следующие результаты.
\begin{enumerate}
\item  Математическое ожидание числа циклов есть 
$\approx \ln n$.

\item  Нормированные длины циклов случайной 
подстановки убывают со скоростью геометрической прогрессии со знаменателем 
$e^{-1}$.

\item Положим $\rho _n \left( a \right)={\left| {\left\{ {g\in S_n 
:\;n_{\max } \left( g \right)\le an} \right\}} \right|} \mathord{\left/ 
{\vphantom {{\left| {\left\{ {g\in S_n :\;n_{\max } \left( g \right)\le an} 
\right\}} \right|} {n!}}} \right. \kern-\nulldelimiterspace} {n!}$, где 
$n_{\max } \left( g \right)$ - длина максимально цикла в подстановке $g$. 
Веливина $\rho _n \left( a \right)$ удовлетворяет \textit{уравнению Дикмана -- Гончарова} (40-ые годы XX 
века):
$
\rho _n \left( a \right)=\int\limits_0^a {\rho _n \left( {\frac{a}{1-t}} 
\right)dt} .
$
\item  Начиная с некоторого большого числа $N$ 99{\%} 
натуральных чисел $n$, больших, чем $N$ обладают свойством
$
n^{0.99}<p_1 \cdot ...\cdot p_{11} .
$
Иначе говоря, у основной части (99{\%}) натуральных чисел основная часть 
(99{\%}) числа есть произведение наибольших простых делителей. Число 11 
возникло из-за того, что мы выбрали 99{\%} и 99{\%}.
\end{enumerate}

\medskip 

Для доказательства неравенства Азумы необходимы следующие понятия и результаты.

 Пусть $\{Z_i\}^{n}_{i=1}$ и  $\{X_i\}^n_{i=1}$ последовательность случайных величин на одном вероятностном пространстве, таких что $\mathbf{E}[X_i|Z_1,\dots,Z_{i-1}] = X_{i-1}$ для всех $i$. Тогда $\{X_i\}^n_{i=1}$ называется мартингалом по отношению к $\{Z_i\}^{n}_{i=1}$. Более того последовательность $Y_i = X_i-X_{i-1}$ называется разностной мартингальной последовательностью. По определению  $\mathbf{E}[Y_i|Z_1,\dots,Z_{i-1}] = 0$ для всех $i$.
Более общее определение: пусть $\{\mathcal{F}_i\}_{i=1}^n$ фильтрация, то есть вложенная последовательность $\sigma$-алгебр $\varnothing \subseteq \mathcal{F}_1\subseteq \dots \subseteq\mathcal{F}_n$, случайные величины $X_i$ измеримы относительно $\mathcal{F}_i$. Тогда  последовательность $X_i$ называется мартингалом относительно фильтрации $\mathcal{F}_i$, если $\mathbf{E}(X_i|\mathcal{F}_{i-1})=X_{i-1}$ для всех $i$.
Для последовательности $\{\mathcal{F}_i\}_{i=1}^n$  случайную величину $\nu$ будем называть моментом остановки, если $\{\nu\leq n\}\in \mathcal{F}_n$ при любом $n\geq 0$.

\textbf{Теорема Дуба.}  Пусть $\{X_n,\, \mathcal{F}_n;\,n=1,\dots\}$ ~--- мартингал и $\nu_1$, $\nu_2$~--- моменты остановки такие, что 
\begin{equation*}
\mathbf{E}|X_{\nu_1}|\leq \infty,\quad i=1,2,
\end{equation*}
\begin{equation*}
\lim_{n\to\infty}\mathbf{E}(|X_n|:\nu_2\geq n) = 0.
\end{equation*}
Тогда на множестве $\{\nu_1\geq \nu_2\}$
\begin{equation*}
\mathbf{E}(X_{\nu_2}|\mathcal{F}_{\nu_1}) = X_{\nu_1}.
\end{equation*}

\medskip

\subsection {Примеры явления концентрации меры}
%По мотивам книги Зорича В. А. <<Математический анализ задач естествознания>>
%в евклидовом пространстве $\mathbf{R}^{n}$

\begin{problem}[Концентрация объема шара] 
Рассмотрим шар $B^{n}(r)$ радиуса $r$ в евклидовом пространстве $\mathbf{R}^n$ большой размерности. 
%Пусть $V\big[B^{n}(r)\bigl]$ ~--- объем шара.
 Необходимо убедиться в том, что объем шара сконцентрирован в малой окрестности его границы.
Рассмотреть также единичный куб.
\end{problem}

\begin{problem}[Концентрация площади сферы]

Рассмотрим сферу $S^{n-1}(r)$ в евклидовом пространстве $\mathbf{R}^r$ с радиусом в начале координат. %Необходимо убедиться в том, что выбранные наугад два единичных вектора в пространстве $\mathbf{R}^n$ большой размерности с большой вероятностью окажутся почти ортогональными.  
Зафиксируем координатную ось $x$.
Необходимо убедиться в том, что подавляющая часть площади многомерной сферы $S^{n-1}$ сосредоточена в малой окрестности экватора, перпендикулярного выбранной оси $x$. Каково взаимное расположение двух выбранных наугад единичных векторов в простанстве $\mathbf{R}^n$, если все направления считаются равновероятными?
\end{problem}


\begin{remark}
Достаточно доказать, что для всякого сколь угодно малого $\delta>0$ проекция второго вектора на ось $x_1$ с вероятностью, близкой к 
единице, лежит в промежутке $[-\delta, \delta]$ при $n\to\infty$. Это равносильно тому, что доля от площади всей сферы $S^{n-1}(r)$, 
которую занимает сферический слой $S^{n-1}_{\delta}(r)$, проектирующийся в отрезок $[-\delta, \delta]$ оси $x_1$, 
может быть сделана сколь угодно близкой к $1$ при $n\to\infty$. 

Перейдя к $n$-мерным сферическим координатам и обратно, показать, что мера сферического слоя $S^{n-1}_{\delta}(r)$ равна: 
$$
\mu_{n-1} S_{\delta}^{n-1}(r) = Cr^{n-1} 
\int\limits_{-\delta}^{\delta} \Bigl( 1-(x/r)^2\Bigr)^{(n-3)/2} \, dx . 
$$

Вероятность попадания в данный слой $S_{\delta}^{n-1}(r)$ равна 
$$
{\mathbf P}[-\delta, \delta]=\frac{\int\limits_{-\delta}^{\delta} \Bigl( 1-(x/r)^2\Bigr)^{(n-3)/2} \, dx}
{\int\limits_{-r}^{r} \Bigl( 1-(x/r)^2\Bigr)^{(n-3)/2} \, dx} . 
$$
Данное отношение на зависит от $r$, поэтому можно считать $r=1$. 

Для нахождения асимптотики имеющихся интегралов при $n\to\infty$ использовать классические результаты относительно асимптотики интеграла 
Лапласа $F(\lambda)=\int_a^b f(x)e^{\lambda S(x)}\, dx$ при $\lambda\to +\infty$. Если обе функции $f$ и $S$ определены и регулярны 
на промежутке $I=[a,b]$ и функция $S$ имеет единственный глобальный максимум на $I$, который достигается в точке $x_0\in I$, 
$f(x_0)\ne 0$, то асимптотика интеграла такая же, как в окрестности точки $x_0$ (принцип локализации). В зависимости от расположения 
точки $x_0$ и свойств функции $S(x)$ возможны следующие тейлоровские разложения при $\lambda\to +\infty$: 
$$
F(\lambda)=\frac{f(x_0)}{-S'(x_0)}e^{\lambda S(x_0)} \lambda^{-1}\bigl( 1+O(\lambda^{-1})\bigr) , 
$$
если $x_0=a$ и $S'(x_0)\ne 0$ (т.е. $S'(x_0)<0$); 
$$
F(\lambda)=\sqrt{\frac{\pi}{-2S''(x_0)}} f(x_0) e^{\lambda S(x_0)} \lambda^{-1/2}\bigl( 1+O(\lambda^{-1/2})\bigr) , 
$$
если $x_0=a$, $S'(x_0)=0$, $S''(x_0)\ne 0$ (т.е. $S''(x_0)<0$); 
$$
F(\lambda)=\sqrt{\frac{2\pi}{-S''(x_0)}} f(x_0) e^{\lambda S(x_0)} \lambda^{-1/2}\bigl( 1+O(\lambda^{-1/2})\bigr) , 
$$
если $a<x_0<b$, $S'(x_0)=0$, $S''(x_0)\ne 0$ (т.е. $S''(x_0)<0$). 

\end{remark}

\begin{problem}[Физическая интерпретация концентрации на сфере]
Провести аналогию между предыдущей задачей и задачей отыскания статистических характеристик ансамбля из $n$ частиц массы $m$  со скоростями $v_i$, $i=1,\dots,n$. Суммарная кинетическая энергия $E_n$ растет пропорционально $n$, то есть 
\begin{equation*}
\frac{1}{2}mv_1^2+\cdots+\frac{1}{2}m v_n^2 = E_n;\quad \sum_{i=1}^n v_i=\frac{2E_n}{m}\asymp n.
\end{equation*}
В решении предыдущей задачи перейти к термодинамическому пределу, когда $n\to\infty$, $r = \sigma n^{1/2}$, чтобы получить распределение Максвелла. 
\end{problem}

\begin{problem}
Пусть $X_n$ --- случайный вектор с равномерным распределением на единичной сфере в ${\mathbf R}^n$. Равномерное распределение 
характеризуется тем, что оно инвариантно относительно группы ортогональных преобразований. Пусть $Y_n$ обозначает первую координату $X_n$. 
Докажите, что $\sqrt{n}\, Y_n \xrightarrow{d}N(0,1)$ при $n\to\infty$. Заметим, что в статистической физике с помощью утверждения 
этой задачи получался закон распределения Максвелла скоростей частиц одномерного идеального газа. 
\end{problem}

\begin{remark}
Пусть $\xi_1,\ldots, \xi_n$ --- независимые в совокупности с.в., имеющие одинаковое распределение $N(0,1)$. Рассмотрим случайный вектор 
$Z_n=(\xi_1,\xi_2,\ldots,\xi_n)$. Тогда $Z_n\in N(0,E_n)$, $E_n$ --- единичная матрица размера $n$. Показать, что $Z_n$ инвариантно относительно группы ортогональных преобразований. Заметим, что распределения 
$$
X_n \quad\text{ и } \quad \frac{Z_n}{\|Z_n \|_{{\mathbf R}^n}} \quad \text{ совпадают. }
$$
Поэтому имеет место равенство по распределению с.в. 
$$
Y_n=\frac{\xi_1}{\sqrt{\xi_1^2+\ldots+ \xi_n^2}} 
$$
$$
\Rightarrow \quad \sqrt{n}Y_n = \frac{\xi_1}{\sqrt{(\xi_1^2+\ldots+ \xi_n^2)/n}} . 
$$
Применить теорему Колмогорова для у.з.б.ч. для $\frac{\xi_1^2+\ldots+ \xi_n^2}{n}$. 

\end{remark}

\begin{problem}[Геометрическая интерпретация закона больших чисел]
Рассмотрим куб $C^n = [-1,1]$ в евклидовом пространстве $\mathbf{R}^n$. Пусть $\xi_i$, $i=1,\dots,n$ независимые центрированные одинаково распределенные случайные величины с равномерным распределением на $[-1,1]$. Найти геометрическую интерпретацию закона больших чисел.
\end{problem}

\begin{remark} 
Рассмотреть объем  следующего множества ---   пусть $\mathcal{H}$ часть гиперплоскости, содержащаяся в кубе и перпендикулярная главной диагонали куба, т.е.  $f(x) =\sum_{i=1}^n x_i = 0$. Необходимо подсчитать объем $\epsilon\sqrt{n}$-окрестности $\mathcal{H}$. 
\end{remark}

\begin{problem}
Рассмотрим систему линейных алгебраических уравнений 
\begin{equation*}
x=Ax+b,
\end{equation*}
где $A$ положительно определенная неособенная матрица, собственные числа $\lambda_i$ которой меньше единицы, $b$~--- заданный и  $x$ ~--- искомый векторы $n$-мерного подпространства. Пусть эта система решается при помощи итерационного процесса 
\begin{equation*}
x_{m+1}= Ax_{m}+b,\quad m=1,2,\dots,
\end{equation*}
 который заканчивается на $p$-м шаге, если вектор--невязка $\delta_p = x_{p+1}-x_{p}$ попадает в некоторую заданную окрестность $G$ нуля (будем рассматривать шар радиуса $\alpha$). 
Ошибка итерационного процесса $\epsilon_m = x^{*} - x_m$, где  $x^{*}$ истинное решение системы уравнений, связана с невязкой  (проверить)
 \begin{equation*}
\epsilon_m = (I-A)^{-1}\delta_m,
\end{equation*}
поэтому  исходя из значений вектора невязки можно определить вероятностное распределение  ошибок. 

Оказывается, что наиболее вероятными ошибками являются максимальные. Необходимо проверить это для случая $n=2$.

\end{problem}
\begin{remark}
\end{remark}

\textbf{
TODO
}



\subsection{Неравенства концентрации для сумм случайных величин ---  неравенства Чернова, Хевдинга, Бернштейна, Азумы}


\textbf{
TODO Сравнение неравенства Бернштейна и Хефдинга.
Достижимость границ.
}

%По мотивам лекции Голубева, обзора Лугоши.

%\begin{enumerate}
%\item
\begin{problem}[Неравенство  Чернова]

Доказать, что неравенство Чернова для неотрицательной случайной величины $X$
\begin{equation*}
\mathbf{P}\{ X >t\}\leq \inf_{s>0}\mathbf{E}\exp(sX-st)
\end{equation*}
 дает более завышенную границу по сравнению с моментной границей
\begin{equation*}
\mathbf{P}\{ X >t\}\leq \min_{q>0}\mathbf{E}[X^q]t^{-q},
\end{equation*}
 то есть 
\begin{equation*}
\min_q\mathbf{E}[X^q]t^{-q}\leq \inf_{s>0}\mathbf{E}\big[\text{e}^{s(X-t)}\bigl]
\end{equation*}
\end{problem}

\begin{remark} Использовать следствие из неравенства Маркова: для монотонной возрастающей неотрицаиельной функции $\phi(\cdot)$ и произвольной неотрицательной случайной величины $X$ верно
\begin{equation*}
\mathbf{P}\{\phi(X)\geq \phi(t)\}\leq \frac{\mathbf{E}\phi(X)}{\phi(X)}.
\end{equation*}
\end{remark}

%\item 
%\begin{problem}[Неравенство Чернова для суммы случайных величин]

%\end{problem}

%\item 
\begin{problem}[Лемма Хефдинга] Пусть $X$--- случайная величина, такая что $\mathbf{E}X =0$, $a\leq X\leq b$. Тогда для $\lambda>0$ верно
\begin{equation*}
\mathbf{E}\exp(\lambda X)\leq \exp\bigg[\frac{\lambda^2(b-a)^2}{8}\biggr]
\end{equation*}
\end{problem}

\begin{remark}
Рассмотреть функцию
$
\phi(\lambda) = \log\bigg[\int_a^be^{\lambda x}p(x) \, d x\biggr],%= \log \big[\mathbf{E}\exp(\lambda\xi)\bigr]
$
найти $\phi^{\prime\prime}(\lambda)$,   $\phi(0)$, $\phi^{\prime}(0)$.

Определить область значений случайной величины  $Z$, для которой $\text{Var}(Z) = \phi''(\lambda)$. 
Воспользовавшись 
\begin{equation*}
\bigg[Z-\frac{a+b}{2}\biggr]^2\leq \bigg(\frac{b-a}{2}\biggl)^2.
\end{equation*}
И тем, что $\text{Var}(Z) = \min_{x}\mathbf{E}[Z-x]^2$
получить  
\begin{equation*}
\phi^{\prime\prime}(\lambda)\leq \bigg(\frac{b-a}{2}\biggl)^2.
\end{equation*}
Остается проинтегрировать полученное неравенство.
\end{remark}

\begin{problem}[Теорема Хефдинга] Пусть $\xi_t$, $t\in T$ ~--- независимые случайные величины, такие что $\xi_t\in[a,b]$. Тогда
\begin{equation*}
\mathbf{P}\bigg\{\bigg|\frac{1}{n}\sum_{t\in T}\big(\xi_t-\mathbf{E}\xi_t\bigr)\biggr|\geq x\biggr\}\leq 2\exp\bigg\{-\frac{2nx^2}{(b-a)^2}\biggr\}.
\end{equation*}
\end{problem}
\begin{remark}
Ввести случайную величину $\xi = \frac{1}{n}\sum_{i=1}^n(\xi_i-\mathbf{E}\xi_i)$.
Воспользоваться неравеством Чернова и леммой Хевдинга для $\xi$, чтобы получить
\begin{equation*}
\mathbf{P}\{\xi>x\}\leq \exp\bigg\{\min_{\lambda}\bigg[-\lambda x + \frac{\lambda^2}{8}\frac{(b-a)^2}{n}\biggr]\biggr\}.
\end{equation*}
Затем найти оптимальное $\lambda$.  Аналогичное неравенство справедливо для $-\xi$.
\end{remark}

\begin{problem}[Неравенство Беннетта]
Пусть $X_1,\dots, X_n$ независимые центрированные ограниченные случайные величины, такие, что с вероятностью $1$ выполнено $|X_i|\leq c$.
Пусть $\sigma^2 = \sum_{i=1}^n\text{Var}\{X_i\}$.
 Тогда для любого $t>0$ 
\begin{equation*}
\mathbf{P}\bigg\{\sum_{i=1}^n X_i>t\biggr\}\leq \exp\bigg(-\frac{n\sigma^2}{c^2}h\bigg(\frac{ct}{n\sigma^2}\biggr)\biggr),
\end{equation*}
где $h(u) = (1+u)\log(1+u)-u$ для $u\geq 0$.
\end{problem}
\begin{remark}
Введем $\sigma_i^2 = \mathbf{E}[X_i^r]$ и $F_i = \sum_{r=2}^{\infty}\frac{s^{r-2}\mathbf{E}[X_i^r]}{r!\sigma_i^2}$.
Используя разложение для ряда Тейлора $\exp(sX)$, показать, что 
\begin{equation*}
\mathbf{E}[e^{sX_i}]\leq \exp(s^2\sigma^2_iF_i).
\end{equation*}

Из ограниченности  $X_i$ получить оценку
\begin{equation*}
F_i\leq \frac{\exp(sc)-1-sc}{(sc)^2}.
\end{equation*} 
Далее воспользоваться неравенством Чернова для $X_i$ и минимизировать правую часть в неравенстве Чернова по $s$.
\end{remark}

\begin{problem}[Неравенство Бернштейна]
При выполнении условий предыдущей теоремы для любого  $\epsilon>0$ верно 
\begin{equation*}
\mathbf{P}\bigg\{\sum_{i=1}^n X_i>\epsilon\biggr\}\leq \exp\bigg(-\frac{n\epsilon^2}{2\sigma^2+2c\epsilon/3}\biggr).
\end{equation*}
\end{problem}

\begin{remark} 
Показать, что верно элементарное неравенство 
\begin{equation*}
h(u)\geq \frac{u^2}{2+2u/3}
\end{equation*}
и использовать неравенство Беннетта.
\end{remark}
%\end{enumerate}

\begin{problem} Пусть $Y$ случайная величина,  $Y\in [-1,+1]$ и $\mathbf{E}[Y]=0$. Тогда для любого $t\geq 0$ верно 
\begin{equation*}
\mathbf{E}[\exp(tY)]\leq \exp(t^2/2).
\end{equation*}
\textit{Указание}
Использовать выпуклость $\exp(tx)$, а именно для $x\in[-1,1]$ верно.
\begin{equation*}
\text{e}^{tx}\leq \frac{1}{2}(1+x)\text{e}^{t} +\frac{1}{2}(1-x)\text{e}^{-t}
\end{equation*}
Подсчитать оценку математического ожидания $\mathbf{E}[\text{e}^{tY}]$ используя разложение экспоненты в ряд Тейлора и элементарный факт $(2n)!>2^nn!$.  
\end{problem}

\begin{problem}[Мартингальное неравенство Азумы-Хевдинга]

Пусть $\{X_i\}_{i=0}^{\infty}$ мартингал по отношению к фильтрации $\{\mathcal{F}_i\}$, пусть $Y_i = X_i-X_{i-1}$ соответствующая последовательность приращений. Тогда, если существуют такие $c_i>0$, что $|Y_i|\leq c_i$ для всех $i$, то 
\begin{equation*}
\mathbf{P}\{|X_m-X_0|\geq t\}\leq 2\exp\bigg\{\frac{-t^2}{2\sum_{i=1}^{m}c^2_i}\biggl\}
\end{equation*}
\end{problem}

\begin{remark} Способы доказательства неравенства Азумы-Хевдинга. 
\medskip
\begin{enumerate}

\item \textit{Первый способ доказательства использует теорему Дуба (см. введение).} 

\textbf{TODO}


\item \textit{Второй способ доказательства использует результат задачи 12.}

Показать, что 
\begin{equation*}
\mathbf{E}\exp(sY_1+\dots+ sY_m) = \mathbf{E}\big[\exp(sY_1+\dots+sY_{m-1})\mathbf{E}[\exp(sY_m)|\mathcal{F}_{m-1}]\bigl] 
\end{equation*}
записать неравенство Чернова

\begin{equation*}
\mathbf{P}[Y_1+\dots+Y_m>t]\leq \exp\big[-st+\sum_{i=1}^mc^2_i s^2/2\bigr].
\end{equation*}
Остается оценить $s$ из минимизации правой части.
\end{enumerate}

\end{remark}

\subsection{Неравенства концентрации меры для функционалов от случайных величин.}

\medskip

\begin{problem}[Неравенство Эфрона-Стейна]
Пусть $X'_1,\dots,X'_n$ ~--- независимые копии $X_1,\dots,X_n$ и 
\begin{equation*}
Z'_i = g(X_1,\dots, X'_i,\dots,X_n).
\end{equation*}
Тогда верно неравенство 
\begin{equation*}
\text{Var}(Z)\leq \frac{1}{2}\sum_{i=1}^{n}\mathbf{E}[(Z-Z'_i)^2].
\end{equation*}
\end{problem}
\begin{remark}
 
Пусть $X_1,\dots, X_n$, произвольные независимые (не обязательно одинаково распределенные случайные величины) принимающие значения из $\mathcal{X}$ и пусть  $g: \mathcal{X}^n\to \mathbf{R}$ измеримая функция $n$ переменных. Показать, что для случайной величины $Z = g(X_1,\dots,X_n)$ верно 
\begin{equation*}
\text{Var}(Z) \leq \sum_{i=1}^n \mathbf{E}\big[ (Z-\mathbf{E}_iZ)^2\bigl],
\end{equation*}
где $\mathbf{E}_iZ = \mathbf{E}[Z|X_1,\dots,X_{i-1},X_{i+1},\dots,X_n]$.
\end{remark}
\begin{problem}[Случай функций с ограниченными  разностями]
Функция $g: \mathcal{X}^n\to \mathbf{R}$ является функцией с ограниченными разностями, если для некоторых $c_i,$  $1\leq i\leq n$.
\begin{equation*}
\sup_{x_1,\dots,x_n;\, x'_i\in\mathcal{X}} |g(x_1,\dots,x_n)-g(x_1,\dots,x_{i-1},x'_i,x_{i+1},\dots,x_{i+1},\dots,x_n)|\leq c_i. 
\end{equation*}
Выпишите неравенство Эфрона-Стейна для случая функций с ограниченными разностями.
\end{problem}
\medskip

\subsection{Вероятности больших уклонений}

\textbf{TODO}
%\begin{problem}[Задача о среднем функции в смысле Леви --- про концентрацию меры на сфере вокруг медианного значения "хорошей" функции]
%\end{problem}
%\subsection{Изопериметрические неравенства Талаграна(?)}
