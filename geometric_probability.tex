\section{Геометрические вероятности}

\begin{problem}

Три бабочки капустницы садятся на круглый кочан капусты радиуса 1 случайным образом (имеется в виду, что место положение каждой бабочки -- с.в., равномерно распределенная на сфере) и независимо друг от друга. Если между двумя бабочками (геодезическое) расстояние оказывается меньше ${\pi \mathord{\left/ {\vphantom {\pi  2}} \right. \kern-\nulldelimiterspace} 2} $, то обе улетают. Найдите вероятность того, что на капусте сидят все три бабочки.

\end{problem}

\begin{problem}
Какова вероятность того, что $n$-угольник с вершинами, случайно расположенных на окружности, содержит ее центр?
\end{problem}

\begin{problem}
Найти среднюю длину секущих трехмерного куба с единичной длиной.
\end{problem}

\begin{problem}
Пусть в пространстве $\mathbb R^n$ с евклидовой нормой задан $n$-мерный шар единичного радиуса. Внутри него имеются две случайные точки с радиус-векторами ${\bf{r}}_1$ и ${\bf{r}}_2$ соответственно, имеющие равномерное пространственное распределение внутри шара. Найти распределение случайной величины, являющейся средним расстоянием между этими двумя точками $r = \left|{\bf r}_1 - {\bf r}_2\right|$.
\end{problem}

\begin{problem} (в 2х местах)
Пусть случайный вектор $X^{n} $ имеет равномерное распределение на единичной сфере в ${\mathbb R}^{n} $. Пусть $Y^{n} $ -- проекция $X^{n} $ на первую координатную ось. Докажите, что последовательность $\sqrt{n} Y^{n} $ сходится по распределению к стандартной нормальной случайной величине.

\begin{ordre} 
Показать сходимость по распределению нормы $X^{n} $ к единице. Воспользоваться соотношением
\[
P\{ \frac{\xi_1}{\xi_2} < x \} = \underset{t}{\int} P\{\xi_1 < xt \} f_{\xi_2}(t) dt
\] 
\end{ordre}
\end{problem}

