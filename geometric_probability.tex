\section{Геометрические вероятности}

\begin{problem}

Три бабочки капустницы садятся на круглый кочан капусты радиуса 1 случайным образом (имеется в виду, что место положение каждой бабочки -- с.в., равномерно распределенная на сфере) и независимо друг от друга. Если между двумя бабочками (геодезическое) расстояние оказывается меньше ${\pi \mathord{\left/ {\vphantom {\pi  2}} \right. \kern-\nulldelimiterspace} 2} $, то обе улетают. Найдите вероятность того, что на капусте сидят все три бабочки.

\end{problem}

\begin{problem}
Какова вероятность того, что $n$-угольник с вершинами, случайно расположенных на окружности, содержит ее центр?
\end{problem}

\begin{problem}
Найти среднюю длину секущих трехмерного куба с единичной длиной.
\end{problem}

\begin{problem}
Пусть в пространстве $\mathbb R^n$ с евклидовой нормой задан $n$-мерный шар единичного радиуса. Внутри него имеются две случайные точки с радиус-векторами ${\bf{r}}_1$ и ${\bf{r}}_2$ соответственно, имеющие равномерное пространственное распределение внутри шара. Найти распределение случайной величины, являющейся средним расстоянием между этими двумя точками $r = \left|{\bf r}_1 - {\bf r}_2\right|$.
\end{problem}

\begin{problem}
Пусть случайный вектор $X^{n} $ имеет равномерное распределение на единичной сфере в ${\mathbb R}^{n} $. Пусть $Y^{n} $ -- проекция $X^{n} $ на первую координатную ось. Докажите, что последовательность $\sqrt{n} Y^{n} $ сходится по распределению к стандартной нормальной случайной величине.

\begin{ordre} 
Показать сходимость по распределению нормы $X^{n} $ к единице. Воспользоваться соотношением
\[
P\{ \frac{\xi_1}{\xi_2} < x \} = \underset{t}{\int} P\{\xi_1 < xt \} f_{\xi_2}(t) dt
\] 
\end{ordre}
\end{problem}

\begin{problem}
Найдите вероятность того, что пара случайно выбранных из $E^n=\{ 0,1\}^n$ векторов является ортогональной 
\begin{enumerate}
\item[а)] над полем $F_2=\{ 0,1\}$; 

\item[б)] над полем действительных чисел. 
\end{enumerate}
\end{problem}

\begin{solution}
\begin{enumerate}
\item[а)]
Пусть 
$$
\xi_{x,y}=\begin{cases}
1, &\text{ если } (x,y)=0,\\
0, &\text{ если } (x,y)=1.
\end{cases}
$$
Для искомой вероятности тогда имеем 
$$
P=\frac{1}{2^n\cdot 2^n}\sum\limits_{x,y}\xi_{x,y}=\frac{1}{2^{2n}} \sum\limits_{x}\sum\limits_{y}\xi_{x,y} . 
$$
Поскольку при $x\ne 0$ с.в. 
$$
\xi_{x,y}=1+x_1 y_1+\ldots +x_n y_n 
$$
является невырожденной линейной булевой функцией переменной $y\in E^n$, внутренняя сумма $\sum\limits_{y}\xi_{x,y}$ 
равна числу нулей функции $\xi_{x,y}$, т.е. равна $2^{n-1}$. При $x=0$, очевидно, $\sum\limits_{y}\xi_{x,y}=2^n$. Отсюда 
$$
P=\frac{1}{2^{2n}}\Bigl(\sum\limits_{x\ne 0}2^{n-1}+2^n \Bigr)=\frac{(2^n -1)\cdot 2^{n-1}+2^n}{2^{2n}}=\frac{1}{2}+
\frac{1}{2^{n+1}} . 
$$

\item[б)] Аналогично получим 
$$
P=\frac{1}{2^{2n}}\sum\limits_{x,y}\xi_{x,y}=\frac{1}{4^n}\sum\limits_{k=0}^{n} \sum\limits_{x:\, \| x\|=k} 
\sum\limits_{y} \xi_{x,y}=
$$
$$
=\frac{1}{4^n}\sum\limits_{k=0}^{n} {n\choose k}2^{n-k}=
\frac{1}{4^n}\cdot\bigl( 1+2\bigr)^n=\Bigl(\frac{3}{4}\Bigr)^n . 
$$

\end{enumerate}
\end{solution}