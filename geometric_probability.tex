\section{Геометрические вероятности}

\begin{comment}

\begin{problem}

Три бабочки капустницы садятся на круглый кочан капусты радиуса 1 случайным образом (имеется в виду, что место положение каждой бабочки -- с.в., равномерно распределенная на сфере) и независимо друг от друга. Если между двумя бабочками (геодезическое) расстояние оказывается меньше ${\pi \mathord{\left/ {\vphantom {\pi  2}} \right. \kern-\nulldelimiterspace} 2} $, то обе улетают. Найдите вероятность того, что на капусте сидят все три бабочки.

\end{problem}

\begin{problem}
Какова вероятность того, что $n$-угольник с вершинами, случайно расположенных на окружности, содержит ее центр?
\end{problem}

\begin{problem}
Найти среднюю длину секущих трехмерного куба с единичной длиной.
\end{problem}

\begin{problem}
Пусть в пространстве $\mathbb R^n$ с евклидовой нормой задан $n$-мерный шар единичного радиуса. Внутри него имеются две случайные точки с радиус-векторами ${\bf{r}}_1$ и ${\bf{r}}_2$ соответственно, имеющие равномерное пространственное распределение внутри шара. Найти распределение случайной величины, являющейся средним расстоянием между этими двумя точками $r = \left|{\bf r}_1 - {\bf r}_2\right|$.
\end{problem}

\begin{problem} (в 2х местах)
Пусть случайный вектор $X^{n} $ имеет равномерное распределение на единичной сфере в ${\mathbb R}^{n} $. Пусть $Y^{n} $ -- проекция $X^{n} $ на первую координатную ось. Докажите, что последовательность $\sqrt{n} Y^{n} $ сходится по распределению к стандартной нормальной случайной величине.

\begin{ordre} 
Показать сходимость по распределению нормы $X^{n} $ к единице. Воспользоваться соотношением
\[
P\{ \frac{\xi_1}{\xi_2} < x \} = \underset{t}{\int} P\{\xi_1 < xt \} f_{\xi_2}(t) dt
\] 
\end{ordre}
\end{problem}

\end{comment}

\begin{problem}
Двое условились о встрече между $10$ и $11$ часами утра, причем договорились ждать друг друга не более $10$ минут. Считая, что 
момент прихода на встречу каждым выбирается <<наудачу>> в пределах указанного часа, найти вероятность того, что встреча состоится. 
\end{problem}


\begin{problem}
На плоскости проведены параллельные прямые на единичном расстоянии друг от друга, и на плоскость наугад бросается иголка длиной $L<1$. 
Угол между прямыми и иголкой и расстояние от середины иглы до ближайшей прямой являются независимыми с.в., равномерно распределенными 
на соответственно $(0,2\pi)$ и $(-1/2,1/2)$. С помощью серии таких опытов вычислить число $\pi$ с заданной точностью 
$\delta=1\%$ и с вероятностью ошибки не больше $\varepsilon=5\%$. 
\end{problem}

\begin{ordre}

Рассмотрим окружность диаметра $1$, т.е. длины $\pi$. Такая окружность с вероятностью $1$ пересекает дважды одну из прямых. 
Тогда, исходя из линейности математического ожидания числа попаданий иглы на прямую относительно длины иглы, для иглы длиной $L<1$ 
имеем ${\mathbb E}\xi_L = 2L/\pi$. 

\end{ordre}



\begin{problem}
Покажите, что средняя площадь ортогональной проекции куба с ребром единица на случайную плоскость равна $3/2$. 
\end{problem}

\begin{ordre}
Покажите, что  средняя площадь  ортогональной проекции всякого измеримого тела 
линейно зависит от площади его границы. 
Рассмотрим вспомогательное (см. предыдущую задачу) тело, у которого легко вычисляется средняя площадь ортогональной проекции. 
\end{ordre}

\begin{comment}

\begin{problem}
Рассмотрим звезды, находящиеся на расстоянии, не превышающем $R$ от наблюдателя. Для простоты будем считать, что все звезды имеют одинаковый диаметр $\delta$ и равномерное пространственное распределение с количеством звезд $\lambda$ на единицу объема. Показать, что при $R \rightarrow \infty$ любой участок неба будет полностью светящимся.   

\begin{remark}
В действительности такое явление не наблюдается. В связи с конечным возрастом вселенной ($14 \cdot 10^9$ лет) ее радиус ограничен величиной $ct$.
\end{remark}
\end{problem}

\begin{problem}
Пусть  $N$ точек независимо распределены в области $D$ n-мерного пространства, $P$ - вероятность того, что фигура $F$, образованная $N$ точками, обладает определенным свойством,
зависящим только от взаимного расположения точек. Область $D$ является измеримой по Лебегу и ее мара равна $V$. Обозначим как $P_1$ вероятность того, что $F$ обладает требуемым свойством для случайных точек в области $D_1 \supset D$. Докажите следующее соотношение для малых приращений $\delta V$:
 \[
 \delta P = N (P_1 - P) V^{-1} \delta V
 \]  

\end{problem}



\begin{problem}[Теорема Дворецкого]
Доказать, что для любого $\epsilon > 0$ и $k \in \mathbb{N}$
существует $N = N(k, \epsilon) < exp(C\frac{\log \epsilon}{\epsilon^2}k)$ такое, что любое конечномерное банахово пространство ($X, \Vert\cdot\Vert)$, где $\dim X > N$, содержит $k$-мерное подпространство $E$, являющееся $\epsilon$-евклидовым, т.е. в нем можно задать такую норму $\vert \cdot \vert$, что $\Vert x \Vert \leqslant \vert x \vert \leqslant (1 + \epsilon) \Vert x \Vert$ $\forall x \in E$.     

\end{problem}

\begin{problem}
Выберем наугад (равновероятно) $k$ вершин $m$-мерного куба $[0,1]^m$. Обозначим как $X$ выпуклую оболочку выбранных вершин. Пусть $p_{km}$ - вероятность того, что все вершины многогранника попарно смежны. Докажите справедливость следующей оценки при $m > 3$:

\[
p_{km} > 1 - \frac{k^4 \cdot 5^m}{4 \cdot 8^m}
\]
    
\end{problem}

\end{comment}

\begin{problem}
На плоскости нарисована выпуклая фигура, ограниченная кривой длины $L$. Докажите, что ее диаметр, т.е. максимальное расстояние между двумя ее точками, не меньше $\frac{L}{\pi }$.
\end{problem}
\begin{ordre}
Проведите в случайном направлении прямую. Покажите, что 
математическое ожидание длины проекции фигуры на случайное направление равно 
$\frac{L}{\pi }$.
\end{ordre}

\begin{problem}
В московском метро можно провозить коробки, у которых сумма измерений (длины, ширины и высоты) не превосходит некоторой границы. Можно ли перехитрить правила, поместив одну коробку в другую (сумма измерений внутренней коробки больше суммы измерений внешней)?
\end{problem}
\begin{ordre}
Спроектируйте коробку на случайно выбранное (в пространстве) 
направление. Длина проекции коробки складывается из проекций отрезков, 
идущих по ее высоте, длине и ширине. Проекция внутренней коробки не 
превосходит проекции внешней.
\end{ordre}

\begin{problem}
Несамопересекающаяся кривая длины 22 находится внутри круга радиуса 1. Докажите, что найдется прямая, имеющая с этой кривой по крайней мере 8 общих точек.
Известно, что более половины поверхности Земли занимают океаны. Используя из географии только этот факт, докажите, что можно найти две диаметрально противоположные точки, обе попавшие в океан.
\end{problem}

\begin{comment}

\begin{problem}
На плоскости расположено $2n$ векторов, выходящих из начала координат и длиной не более 1. Доказать, что существует угол $\alpha$ такой, что при повороте каждого из векторов на угол $\pm \alpha$, их векторная сумма окажется не большей 1.  
\end{problem}

\end{comment}