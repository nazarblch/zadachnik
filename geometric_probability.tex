\section{Геометрические вероятности}

\begin{problem}

Три бабочки капустницы садятся на круглый кочан капусты радиуса 1 случайным образом (имеется в виду, что место положение каждой бабочки -- с.в., равномерно распределенная на сфере) и независимо друг от друга. Если между двумя бабочками (геодезическое) расстояние оказывается меньше ${\pi \mathord{\left/ {\vphantom {\pi  2}} \right. \kern-\nulldelimiterspace} 2} $, то обе улетают. Найдите вероятность того, что на капусте сидят все три бабочки.

\end{problem}

\begin{problem}
Какова вероятность того, что $n$-угольник с вершинами, случайно расположенных на окружности, содержит ее центр?
\end{problem}

\begin{problem}
Найти среднюю длину секущих трехмерного куба с единичной длиной.
\end{problem}

\begin{problem}
Пусть в пространстве $\mathbb R^n$ с евклидовой нормой задан $n$-мерный шар единичного радиуса. Внутри него имеются две случайные точки с радиус-векторами ${\bf{r}}_1$ и ${\bf{r}}_2$ соответственно, имеющие равномерное пространственное распределение внутри шара. Найти распределение случайной величины, являющейся средним расстоянием между этими двумя точками $r = \left|{\bf r}_1 - {\bf r}_2\right|$.
\end{problem}

\begin{problem} (в 2х местах)
Пусть случайный вектор $X^{n} $ имеет равномерное распределение на единичной сфере в ${\mathbb R}^{n} $. Пусть $Y^{n} $ -- проекция $X^{n} $ на первую координатную ось. Докажите, что последовательность $\sqrt{n} Y^{n} $ сходится по распределению к стандартной нормальной случайной величине.

\begin{ordre} 
Показать сходимость по распределению нормы $X^{n} $ к единице. Воспользоваться соотношением
\[
P\{ \frac{\xi_1}{\xi_2} < x \} = \underset{t}{\int} P\{\xi_1 < xt \} f_{\xi_2}(t) dt
\] 
\end{ordre}
\end{problem}



\begin{problem}
Двое условились о встрече между $10$ и $11$ часами утра, причем договорились ждать друг друга не более $10$ минут. Считая, что 
момент прихода на встречу каждым выбирается <<наудачу>> в пределах указанного часа, найти вероятность того, что встреча состоится. 
\end{problem}


\begin{problem}[изогнутая игла Бюффона]
Любопытный студент швейного техникума решил повторить опыты Бюффона по бросанию иглы (студент хочет оценить число $\pi$). 
Для этого он подготовил горизонтально расположенный лист бумаги, разлинованный параллельными прямыми так, что расстояние между 
соседними прямыми равно $1$. Однако в распоряжении студента оказалось только погнутая иголка. Иголка имеет форму кочерги, 
но студент не имеет точного представления о том, как именно погнута иголка. Ему известно лишь то, что длина иголки, до того как 
она погнулась, была равна $2$. Студент бросил погнутую иголку $1 000 000$ раз и посчитал суммарное число пересечений, учитывая кратность. 
Помогите студенту оценить число $\pi$: 
\begin{enumerate}
\item[а)] с помощью неравенства Чебышёва; 
\item[б)] с помощью ц.п.т. (центральной предельной теоремы) и оценок скорости сходимости в ц.п.т., например, 
с помощью неравенства Берри–Эссена или более точных аппроксимаций. 
\end{enumerate}
\end{problem}

\begin{ordre}

Рассмотрим окружность диаметра $1$, т.е. длины $\pi$. Такая окружность с вероятностью $1$ пересекает дважды одну из прямых. 
Тогда, исходя из линейности математического ожидания числа попаданий иглы на прямую относительно длины иглы, для иглы длиной $L<1$ 
имеем ${\mathbb E}\xi_L = 2L/\pi$. 

Докажите следующую оценку 
$$
\Var \xi =\sum\limits_{i,j=1}^3 \cov(\xi_{L_i}, \xi_{L_j})\le \Bigl( \sqrt{\Var \xi_{L_1}}+\sqrt{\Var \xi_{L_2}}+
\sqrt{\Var \xi_{L_3}} \Bigr)^2 . 
$$


Близость величины $\frac{S_N-{\mathbb E}S_N}{\sqrt{\Var S_N}}$ к стандартной нормально распределенной (согласно ц.п.т.) в смысле 
близости их функций распределения определяется из неравенства Берри-Эссена 
$$
\sup\limits_x \left| {\mathbb P}\Bigl( \frac{S_N-{\mathbb E}S_N}{\sqrt{\Var S_N}} <x \Bigr) - \Phi(x) 
\right| \le \frac{C_0 \mu^3}{\sigma^3 \sqrt{N}} , 
$$
где $C_0<0.7056$, $\mu^3={\mathbb E}|\xi_i - {\mathbb E}\xi_i|^3$, $\sigma^2=\Var \xi_i$, 
$\Phi(x)=\int_{-\infty}^x \frac{e^{-t^2/2}}{\sqrt{2\pi}}\, dt$. 

\end{ordre}



\begin{problem}
Покажите, что средняя площадь ортогональной проекции куба с ребром единица на случайную плоскость равна $3/2$. 
\end{problem}

\begin{ordre}
Покажите, что  средняя площадь  ортогональной проекции всякого измеримого тела 
линейно зависит от площади его границы. 
Рассмотрим вспомогательное (см. предыдущую задачу) тело, у которого легко вычисляется средняя площадь ортогональной проекции. 
\end{ordre}




