
\section{Примеры явления концентрации меры}
%По мотивам книги Зорича В. А. <<Математический анализ задач естествознания>>
%в евклидовом пространстве $\mathbf{R}^{n}$

\begin{problem}[Концентрация объема шара] 
Рассмотрим шар $B^{n}(r)$ радиуса $r$ в евклидовом пространстве $\mathbf{R}^n$ большой размерности, пусть в шаре задана равномерная мера. 
%Пусть $V\big[B^{n}(r)\bigl]$ ~--- объем шара.
 Необходимо убедиться в том, что мера сконцентрирована в малой окрестности  границы шара.
Рассмотреть также единичный куб.
\end{problem}

\begin{problem}[Концентрация площади сферы]

Рассмотрим сферу $S^{n-1}(r)$ в евклидовом пространстве $\mathbf{R}^r$ с радиусом в начале координат. %Необходимо убедиться в том, что выбранные наугад два единичных вектора в пространстве $\mathbf{R}^n$ большой размерности с большой вероятностью окажутся почти ортогональными.  
Зафиксируем координатную ось $x$.
Необходимо убедиться в том, что подавляющая часть площади многомерной сферы $S^{n-1}$ сосредоточена в малой окрестности экватора, перпендикулярного выбранной оси $x$. Каково взаимное расположение двух выбранных наугад единичных векторов в простанстве $\mathbf{R}^n$, если концы векторов распределены на сфере равномерно?
\end{problem}


\begin{remark}
Достаточно доказать, что для всякого сколь угодно малого $\delta>0$ проекция второго вектора на ось $x_1$ с вероятностью, близкой к 
единице, лежит в промежутке $[-\delta, \delta]$ при $n\to\infty$. Это равносильно тому, что доля от площади всей сферы $S^{n-1}(r)$, 
которую занимает сферический слой $S^{n-1}_{\delta}(r)$, проектирующийся в отрезок $[-\delta, \delta]$ оси $x_1$, 
может быть сделана сколь угодно близкой к $1$ при $n\to\infty$. 

Перейдя к $n$-мерным сферическим координатам и обратно, показать, что мера сферического слоя $S^{n-1}_{\delta}(r)$ равна: 
$$
\mu_{n-1} S_{\delta}^{n-1}(r) = Cr^{n-1} 
\int\limits_{-\delta}^{\delta} \Bigl( 1-(x/r)^2\Bigr)^{(n-3)/2} \, dx . 
$$

Вероятность попадания в данный слой $S_{\delta}^{n-1}(r)$ равна 
$$
{\mathbf P}[-\delta, \delta]=\frac{\int\limits_{-\delta}^{\delta} \Bigl( 1-(x/r)^2\Bigr)^{(n-3)/2} \, dx}
{\int\limits_{-r}^{r} \Bigl( 1-(x/r)^2\Bigr)^{(n-3)/2} \, dx} . 
$$
Данное отношение на зависит от $r$, поэтому можно считать $r=1$. 

Для нахождения асимптотики имеющихся интегралов при $n\to\infty$ использовать классические результаты относительно асимптотики интеграла 
Лапласа $F(\lambda)=\int_a^b f(x)e^{\lambda S(x)}\, dx$ при $\lambda\to +\infty$. %Если обе функции $f$ и $S$ определены и регулярны 
%на промежутке $I=[a,b]$ и функция $S$ имеет единственный глобальный максимум на $I$, который достигается в точке $x_0\in I$, 
%$f(x_0)\ne 0$, то асимптотика интеграла такая же, как в окрестности точки $x_0$ (принцип локализации). В зависимости от расположения 
%точки $x_0$ и свойств функции $S(x)$ возможны следующие тейлоровские разложения при $\lambda\to +\infty$: 
%$$
%F(\lambda)=\frac{f(x_0)}{-S'(x_0)}e^{\lambda S(x_0)} \lambda^{-1}\bigl( 1+O(\lambda^{-1})\bigr) , 
%$$
%если $x_0=a$ и $S'(x_0)\ne 0$ (т.е. $S'(x_0)<0$); 
%$$
%F(\lambda)=\sqrt{\frac{\pi}{-2S''(x_0)}} f(x_0) e^{\lambda S(x_0)} \lambda^{-1/2}\bigl( 1+O(\lambda^{-1/2})\bigr) , 
%$$
%если $x_0=a$, $S'(x_0)=0$, $S''(x_0)\ne 0$ (т.е. $S''(x_0)<0$); 
%$$
%F(\lambda)=\sqrt{\frac{2\pi}{-S''(x_0)}} f(x_0) e^{\lambda S(x_0)} \lambda^{-1/2}\bigl( 1+O(\lambda^{-1/2})\bigr) , 
%$$
%если $a<x_0<b$, $S'(x_0)=0$, $S''(x_0)\ne 0$ (т.е. $S''(x_0)<0$). 

\end{remark}

\begin{problem}[Физическая интерпретация концентрации на сфере]
Провести аналогию между предыдущей задачей и задачей отыскания статистических характеристик ансамбля из $n$ частиц массы $m$  со скоростями $v_i$, $i=1,\dots,n$. Суммарная кинетическая энергия $E_n$ растет пропорционально $n$, то есть 
\begin{equation*}
\frac{1}{2}mv_1^2+\cdots+\frac{1}{2}m v_n^2 = E_n;\quad \sum_{i=1}^n v^2_i=\frac{2E_n}{m}\asymp n.
\end{equation*}
В решении предыдущей задачи перейти к термодинамическому пределу, когда $n\to\infty$, $r = \sigma n^{1/2}$, чтобы получить распределение Максвелла. 
\end{problem}

\begin{problem}
Пусть $X_n$ --- случайный вектор с равномерным распределением на единичной сфере в ${\mathbf R}^n$. Равномерное распределение 
характеризуется тем, что оно инвариантно относительно группы ортогональных преобразований. Пусть $Y_n$ обозначает первую координату $X_n$. 
Докажите, что $\sqrt{n}\, Y_n \xrightarrow{d}N(0,1)$ при $n\to\infty$. Заметим, что в статистической физике с помощью утверждения 
этой задачи получался закон распределения Максвелла скоростей частиц одномерного идеального газа. 
\end{problem}

\begin{remark}
Решение задачи содержит в себе способ генерирования равномерного распределения. 
Пусть $\xi_1,\ldots, \xi_n$ --- независимые в совокупности с.в., имеющие одинаковое распределение $N(0,1)$. Рассмотрим случайный вектор 
$Z_n=(\xi_1,\xi_2,\ldots,\xi_n)$. Тогда $Z_n\in N(0,E_n)$, $E_n$ --- единичная матрица размера $n$. Показать, что $Z_n$ инвариантно относительно группы ортогональных преобразований. Заметим, что распределения 
$$
X_n \quad\text{ и } \quad \frac{Z_n}{\|Z_n \|_{{\mathbf R}^n}} \quad \text{ совпадают. }
$$
Поэтому имеет место равенство по распределению с.в. 
$$
Y_n=\frac{\xi_1}{\sqrt{\xi_1^2+\ldots+ \xi_n^2}} 
$$
$$
\Rightarrow \quad \sqrt{n}Y_n = \frac{\xi_1}{\sqrt{(\xi_1^2+\ldots+ \xi_n^2)/n}} . 
$$
Применить теорему Колмогорова для у.з.б.ч. для $\frac{\xi_1^2+\ldots+ \xi_n^2}{n}$. 

\end{remark}

\begin{problem}[Геометрическая интерпретация закона больших чисел]
Рассмотрим куб $C^n = [-1,1]$ в евклидовом пространстве $\mathbf{R}^n$. Пусть $\xi_i$, $i=1,\dots,n$ независимые центрированные одинаково распределенные случайные величины с равномерным распределением на $[-1,1]$. Найти геометрическую интерпретацию закона больших чисел.
\end{problem}

\begin{remark} 
Рассмотреть объем  следующего множества ---   пусть $\mathcal{H}$ часть гиперплоскости, содержащаяся в кубе и перпендикулярная главной диагонали куба, т.е.  $f(x) =\sum_{i=1}^n x_i = 0$. Необходимо подсчитать объем $\epsilon\sqrt{n}$-окрестности $\mathcal{H}$. 
\end{remark}
