\subsection{Моментные характеристики}

\begin{comment}
\begin{problem}
Сто паровозов выехали из города по однополосной линии, каждый с постоянной скоростью. Когда движение установилось, то из-за того, что быстрые догнали идущих впереди более медленных, образовались караваны (группы, движущиеся со скоростью лидера). Найдите м.о. и дисперсию числа караванов. Скорости различных паровозов независимы и одинаково распределены, а функция распределения скорости непрерывна.
\end{problem}

\begin{problem}
Согласно законам о трудоустройстве в городе \textit{N}, наниматели обязаны предоставить всем рабочим выходной, если хотя бы у одного из них день рождения, и принимать на службу рабочих независимо от их дня рождения. За исключением этих выходных рабочие трудятся весь год из 365 дней. Предприниматели хотят максимизировать среднее число человеко-дней в году. Сколько рабочих трудятся на фабрике в городе \textit{N}?

\end{problem}

\end{comment}

\begin{problem}
Существует ли случайная величина с конечным вторым моментом и бесконечным первым моментом.
\end{problem}

\begin{problem}
В начале карточной игры принято с помощью жребия определять первого сдающего. Жребий бросается так: колоду хорошо тасуют, и затем кто-нибудь сдает игрокам по карте до появления первого туза. Кому выпал туз -- тот и сдает в первой игре. На каком месте в среднем появляется первый туз, если в колоде 32 карты (то есть найти математическое ожидание случайной величины «Число карт, сданных до первого туза»)?

\begin{ordre} 
Задача на свойство линейности математического ожидания.
\end{ordre}

\end{problem}

\begin{problem}
Покажите, что неравенство Чебышева:
\[P\left\{\left|X-EX\right|>\varepsilon \right\}\le \frac{DX}{\varepsilon ^{2} } \] 
принципиально не улучшаемо.

\begin{ordre} 
Рассмотрите с.в. 
\[X=\left\{\begin{array}{cc} {a,} & {{\raise0.7ex\hbox{$ p $}\!\mathord{\left/ {\vphantom {p 2}} \right. \kern-\nulldelimiterspace}\!\lower0.7ex\hbox{$ 2 $}} } \\ {0,} & {1-p} \\ {-a,} & {{\raise0.7ex\hbox{$ p $}\!\mathord{\left/ {\vphantom {p 2}} \right. \kern-\nulldelimiterspace}\!\lower0.7ex\hbox{$ 2 $}} } \end{array}\right. \] 
Положите $\varepsilon =a-\delta $, где $\delta \to 0+$.
\end{ordre}

\end{problem}

\begin{problem}

 В лотерее на выигрыш уходит 40\% от стоимости проданных билетов. Каждый билет стоит 100 рублей. Доказать, что вероятность выиграть 5000 рублей (или больше) меньше 1\%.

\begin{ordre} 
Использовать неравенство Маркова.
\end{ordre}

Искомая вероятность зависит, конечно, от правил лотерее, но ни при каких условиях она не превосходит 0.8\%$<$1\%.
Приведите пример правил лотереи, где искомая вероятность минимальна и максимальна.

\end{problem}

\begin{comment}
\begin{problem}

Покажите, что все моменты распределения

 $p_{\lambda } \left(x\right)=\frac{1}{24} e^{-x^{{1\mathord{\left/ {\vphantom {1 4}} \right. \kern-\nulldelimiterspace} 4} } } \left(1-\lambda \sin x^{{1\mathord{\left/ {\vphantom {1 4}} \right. \kern-\nulldelimiterspace} 4} } \right)$, $x\ge 0$ при любом значении параметра $\lambda \in \left[0,1\right]$ совпадают.

\begin{remark}

Необходимое и достаточное условие того, чтобы моменты однозначно определяли распределение, вообще говоря, комплексной случайной величины $x$, имеет вид:

\noindent $\sum _{n=0}^{\infty }\left(M\left(\left|x\right|^{2n} \right)\right)^{{-1\mathord{\left/ {\vphantom {-1 \left(2n\right)}} \right. \kern-\nulldelimiterspace} \left(2n\right)} } =\infty  $ (условие Карлемана).
\end{remark}

\end{problem} 
\end{comment}

\begin{comment}
\begin{problem}
В каждую $i$-ую единицу времени живая клетка получает случайную дозу облучения $X_i$, причем $\{ X_i\}_{i=1}^{t}$ имеют 
одинаковую функцию распределения $F_X(x)$ и независимы в совокупности $\forall t$. Получив интегральную дозу облучения, 
равную $\nu$, клетка погибает. Оценить среднее время жизни клетки ${\mathbb E}T$. 
\end{problem}

\begin{ordre}

Показать тождество Вальда: 
$$
{\mathbb E}S_T={\mathbb E}X\cdot {\mathbb E}T, 
$$

введя вспомогательную случайную величину

$$
Y_j=\begin{cases}
1, &\text{ если }\quad X_1+\ldots +X_{j-1}=S_{j-1}<\nu, \\
0, &\text{ в остальных случаях }. 
\end{cases}
$$
 

\end{ordre}

\end{comment}


\begin{problem}
Восемь мальчиков и семь девочек купили билеты в кинотеатр на $15$ подряд идущих сидячих мест. Предположим, что все $15!$ 
возможных способов сесть равновероятны. Вычислите среднее число пар рядом сидящих мальчика и девочки. Например, 
м, м, м, м, м, м, ж, м, ж, ж, ж, ж, ж, ж содержит три такие пары. 
\end{problem}


\begin{problem}
На первом этаже семнадцатиэтажного общежития в лифт вошли десять человек. Предполагая, что каждый из вошедших может с равной вероятностью 
жить на любом из шестнадцати этажей (со $2$-го по $17$-ый), найдите среднее число остановок лифта. 
\end{problem}



\begin{problem}
Имеется $n$ пронумерованных писем и $n$ пронумерованных конвертов. Письма случайным образом раскладываются по конвертам (т.е. все $n!$ 
способов распределения $n$ писем по $n$ конвертам, так чтобы в каждом конверте было ровно по одному письму, считаются равновероятными). 
Найдите математическое ожидание случайной величины, равной числу совпадений (числу писем, лежащих в конвертах с теми же номерами). 
\end{problem}

\begin{comment}

\begin{problem}

Требуется определить начиная с какого этажа брошенный с балкона 100-этажного здания стеклянный шар разбивается. В наличии имеется два таких шара. Предложить метод нахождения граничного этажа, минимизирующий математическое ожидание числа бросков. Рассмотреть случай большего числа шаров.  

\end{problem}

\begin{problem}
На подоконнике лежат $N$ помидоров. У каждого из них предначертана своя судьба - сгнить вечером $i$-го дня ($1 \leqslant i \leqslant N$, причем ни у каких двух помидоров судьба не совпадает). Утром каждого дня приходит человек и случайным образом съедает один свежий помидор из оставшихся. Таким образом, каждый помидор либо сгнивает в свой предначертанный день, либо употребляется ранее в качестве пищи.

\begin{enumerate}
\item Получите рекуррентную формулу для математического ожидания съеденных помидоров от числа $N$.
\item Найдите асимптотическую оценку количества съеденных помидоров при $N \rightarrow \infty$.
\end{enumerate}

\end{problem}

\end{comment}