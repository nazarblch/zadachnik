

\subsection{Порядковые статистики}

\begin{problem}
$n$ человек разного роста случайным образом выстраиваются в шеренгу. Найти вероятность того, что: 
\begin{enumerate}
\item[а)] самый низкий окажется $i$-м слева; 
\item[б)] самый высокий окажется первым слева, а самый низкий --- последним слева; 
\item[в)] самый высокий и самый низкий окажутся рядом; 
\item[г)] между самым высоким и самым низким расположатся более $k$ человек. 
\end{enumerate}
\end{problem}


\begin{problem}
Урна содержит $N$ шаров с номерами от $1$ до $N$. Пусть $K$ --- наибольший номер, полученный при $n$ их поштучных извлечениях 
с возвращением. Найти 
\begin{enumerate}
\item[а)] распределение $K$, 
\item[б)] асимптотику математического ожидания ${\mathbb E}K$ при $N\to\infty$. 
\end{enumerate}
\end{problem}

\begin{ordre}
\[
{\mathbb E}K=\sum\limits_{j=0}^{N} {\mathbb P}\{ K>j\}
\]
\end{ordre}

\begin{problem}
Найти математическое ожидание ${\mathbb E}Z$, где $Z$ --- $k$-ая по величине из координат $n$ точек, взятых наудачу на отрезке 
$[0;1]$ $(k \leqslant n)$. 
\end{problem}

\begin{comment}
\begin{problem}[рекорды]
Пусть $X_1 ,X_2 ,\ldots $ - независимые 
случайные величины с одной и той же плотностью распределения вероятностей 
$p(x)$. Будем говорить, что наблюдается рекордное значение в момент времени 
n$>$1, если $X_n >\max \left[ {X_1 ,...,X_{n-1} } \right]$. Докажите 
следующие утверждения.

\begin{enumerate}
\item[\textbf{А)}] Вероятность того, что рекорд зафиксирован в момент времени $n$, 
равна $1/n$.

\item[\textbf{Б)}] Математическое ожидание числа рекордов до момента времени $n$ 
равно 
\[
\sum\limits_{1<i\le n} {\frac{1}{i}} \sim \ln n.
\]

\item[\textbf{В)}] Пусть $Y_n $ --- случайная величина, принимающая значение $1$, если 
в момент времени $n$ зафиксирован рекорд, и значение $0$ -- в противном случае. 
Тогда случайные величины $Y_1 ,Y_2 ,\ldots$ независимы в совокупности.

\item[\textbf{Г)}] Дисперсия числа рекордов до момента времени $n$ равна
\[
\sum\limits_{1<i\le n} {\frac{i-1}{i^2}} \sim \ln n.
\]

\item[\textbf{Д)}] Если $T$ -- момент появления первого рекорда после момента времени $1$, то $ET=\infty $.
\end{enumerate}
\end{problem}

\end{comment}

\begin{problem}
Покажите, что если независимые случайные величины $X_1,\ldots, X_n$ имеют показательное распределение, т.е. 
$$
f_{X_i}(x)=\begin{cases}
\lambda_i\exp(-\lambda_i x), \; x\geqslant 0 \\
0,\; x<0
\end{cases}
$$
(часто пишут $X_i\in\Exp(\lambda_i)$), то 
$$
\min\{ X_1,\ldots, X_n\}\in \Exp\Bigl( \sum\limits_{i=1}^{n}\lambda_i\Bigr) . 
$$
\end{problem}

\begin{problem}
В некотором Вузе проходит экзамен. Количество экзаменационных билетов $N$. Перед экзаменационной аудиторией выстроилась очередь из 
студентов, которые не знают, чему равно $N$. Согласно этой очереди студенты вызываются на экзамен (второй студент заходит в аудиторию 
после того как из нее выйдет первый и т.д.). Каждый студент с равной вероятностью может выбрать любой из $N$ билетов (в независимости 
от других студентов). Проэкзаменованные студенты, выходя из аудитории, сообщают оставшейся очереди номера своих билетов. 
Оцените (сверху), сколько студентов должно быть проэкзаменовано, чтобы оставшаяся к этому моменту очередь смогла оценить число 
экзаменационных билетов с точностью $10\%$ с вероятностью, не меньшей $0.95$. 
\end{problem}

