\subsection{Задачи повышенной сложности}

\begin{problem}
Из совокупности всех подмножеств множества $\{1,2,\ldots,N\}$ по схеме выбора с возвращением выбираются множества $A$ и $B$. 
Найти вероятность, что $A$ и $B$ не пересекаются. 
\end{problem}

\begin{problem}
В самолете $n$ мест. Есть $n$ пассажиров, выстроившихся друг за другом в очередь. Во главе очереди -- <<заяц>>. У всех, 
кроме <<зайца>>, есть билет, на котором указан номер посадочного места. Так как <<заяц>> входит первым, он случайным образом занимает 
некоторое место. Каждый следующий пассажир, входящий в салон самолета, действует по такому принципу: если его место свободно, то 
садится на него, если занято, то занимает с равной вероятностью любое свободное. Найдите вероятность того, что последний пассажир 
сядет на свое место. 
\end{problem}

\begin{problem}[``сто заключенных'']
В коридоре находятся 100 человек, у каждого свой номер (от 1 до 100). Их по одному заводят в комнату, в которой 
находится комод со 100 выдвижными ящиками. В ящики случайным образом 
разложены карточки с номерами (от 1 до 100). Каждому разрешается заглянуть в 
не более чем 50 ящиков. Цель каждого -- определить, в каком ящике находится 
его номер. Общаться и передавать друг другу информацию запрещается. 
Предложите стратегию, которая с вероятностью не меньшей $0.3$ (в 
предположении, что все $100!$ способов распределения карточек по ящикам 
равновероятны) приведет к выигрышу всей команды. Команда выигрывает, если 
все 100 участников верно определили ящик с карточкой своего номера.
\end{problem}

\textbf{Стратегия. }Каждый человек первым открывает ящик под его номером, 
вторым -- под номером, который указан на карточке, лежащей в ящике, открытом 
перед этим и т.д. Среднее число циклов длины r в случайной 
перестановке -- есть 1/r (покажите, используя, например, задачу ``про 
предельные меры''). Тогда среднее число циклов длины большей n/2 
есть $\sum\limits_{i=n \mathord{\left/ {\vphantom {n 2}} \right. 
\kern-\nulldelimiterspace} 2}^n {\frac{1}{i}} $. Это и есть вероятность 
существования цикла длины большей n/2. Поэтому вероятность успеха команды -- 
есть $1-\sum\limits_{i=51}^{100} {\frac{1}{i}} \approx 0,31$. Если 
же просто произвольно открывать ящики, то вероятность успеха будет 
$2^{-100}\approx 8\cdot 10^{-31}$. В случае, когда карточки 
разложены не случайным образом, то следует сделать случайной нумерацию 
ящиков, и далее следовать старой стратегии.

\begin{problem}
Симметричную монету независимо бросили $n$ раз. Результат бросания записали в виде последовательности нулей и единиц. Покажите, что с вероятностью стремящейся к единице при $n\to \infty $ длина максимальной подпоследовательности из подряд идущих единиц лежит в промежутке
\[\left(\log \sqrt{n} ,\; \log n^{2} \right).\] 
\end{problem}

\begin{problem}[пример о сходимости ряда]
Пусть $(\Omega,\Xi,{\mathbb P})$ --- вероятностное пространство, $\xi_1,\xi_2,\ldots$ --- некоторая последовательность с.в.. 
Обозначим $\Xi_n^{\infty}=\sigma(\xi_{n},\xi_{n+1},\ldots)$ --- $\sigma$-алгебру, порожденную с.в. $\xi_{n},\xi_{n+1},\ldots$ и пусть 
$$
{\mathcal X}=\bigcap\limits_{n=1}^{\infty} \Xi_{n}^{\infty} . 
$$
Поскольку пересечение $\sigma$-алгебр есть снова $\sigma$-алгебра, то ${\mathcal X}$ --- есть $\sigma$-алгебра. Эту $\sigma$-алгебру 
будем называть <<хвостовой>> или <<остаточной>>, в связи с тем, что всякое событие $A\in{\mathcal X}$ не зависит от значений с.в. 
$\xi_1,\xi_2,\ldots,\xi_n$ при любом конечном $n$, а определяется лишь <<поведением бесконечно далеких значений последовательности 
$\xi_1,\xi_2,\ldots$ >>. 

С помощью задачи $\ref{SigmaAlgebra}$ докажите справедливость следующего утверждения: 

Пусть $\xi_1,\xi_2,\ldots$ --- последовательность независимых в совокупности с.в. и $A\in{\mathcal X}$ 
(событие $A$ принадлежит <<хвостовой>> $\sigma$-алгебре). Тогда ${\mathbb P}(A)$ может принимать лишь два значения $0$ или $1$. 
\end{problem}

\begin{ordre}
Идея доказательства состоит в том, чтобы показать, что каждое <<хвостовое>> событие $A$ не зависит от самого себя и, значит, 
${\mathbb P}(A\cap A)={\mathbb P}(A)\cdot {\mathbb P}(A)$, т.е. ${\mathbb P}(A)={\mathbb P}^2(A)$, откуда 
${\mathbb P}(A)=0$ или $1$. 
\end{ordre}

\begin{problem}
Сто паровозов выехали из города по однополосной линии, каждый с постоянной скоростью. Когда движение установилось, то из-за того, что быстрые догнали идущих впереди более медленных, образовались караваны (группы, движущиеся со скоростью лидера). Найдите м.о. и дисперсию числа караванов. Скорости различных паровозов независимы и одинаково распределены, а функция распределения скорости непрерывна.
\end{problem}

\begin{problem}
Согласно законам о трудоустройстве в городе \textit{N}, наниматели обязаны предоставить всем рабочим выходной, если хотя бы у одного из них день рождения, и принимать на службу рабочих независимо от их дня рождения. За исключением этих выходных рабочие трудятся весь год из 365 дней. Предприниматели хотят максимизировать среднее число человеко-дней в году. Сколько рабочих трудятся на фабрике в городе \textit{N}?

\end{problem}

\begin{problem}
В каждую $i$-ую единицу времени живая клетка получает случайную дозу облучения $X_i$, причем $\{ X_i\}_{i=1}^{t}$ имеют 
одинаковую функцию распределения $F_X(x)$ и независимы в совокупности $\forall t$. Получив интегральную дозу облучения, 
равную $\nu$, клетка погибает. Оценить среднее время жизни клетки ${\mathbb E}T$. 
\end{problem}

\begin{ordre}

Показать тождество Вальда: 
$$
{\mathbb E}S_T={\mathbb E}X\cdot {\mathbb E}T, 
$$

введя вспомогательную случайную величину

$$
Y_j=\begin{cases}
1, &\text{ если }\quad X_1+\ldots +X_{j-1}=S_{j-1}<\nu, \\
0, &\text{ в остальных случаях }. 
\end{cases}
$$
 

\end{ordre}

\begin{problem}

Покажите, что все моменты распределения

 $p_{\lambda } \left(x\right)=\frac{1}{24} e^{-x^{{1\mathord{\left/ {\vphantom {1 4}} \right. \kern-\nulldelimiterspace} 4} } } \left(1-\lambda \sin x^{{1\mathord{\left/ {\vphantom {1 4}} \right. \kern-\nulldelimiterspace} 4} } \right)$, $x\ge 0$ при любом значении параметра $\lambda \in \left[0,1\right]$ совпадают.

\begin{remark}

Необходимое и достаточное условие того, чтобы моменты однозначно определяли распределение, вообще говоря, комплексной случайной величины $x$, имеет вид:

\noindent $\sum _{n=0}^{\infty }\left(M\left(\left|x\right|^{2n} \right)\right)^{{-1\mathord{\left/ {\vphantom {-1 \left(2n\right)}} \right. \kern-\nulldelimiterspace} \left(2n\right)} } =\infty  $ (условие Карлемана).
\end{remark}

\end{problem} 

\begin{problem}

Требуется определить начиная с какого этажа брошенный с балкона 100-этажного здания стеклянный шар разбивается. В наличии имеется два таких шара. Предложить метод нахождения граничного этажа, минимизирующий математическое ожидание числа бросков. Рассмотреть случай большего числа шаров.  

\end{problem}

\begin{problem}
На подоконнике лежат $N$ помидоров. У каждого из них предначертана своя судьба - сгнить вечером $i$-го дня ($1 \leqslant i \leqslant N$, причем ни у каких двух помидоров судьба не совпадает). Утром каждого дня приходит человек и случайным образом съедает один свежий помидор из оставшихся. Таким образом, каждый помидор либо сгнивает в свой предначертанный день, либо употребляется ранее в качестве пищи.

\begin{enumerate}
\item Получите рекуррентную формулу для математического ожидания съеденных помидоров от числа $N$.
\item Найдите асимптотическую оценку количества съеденных помидоров при $N \rightarrow \infty$.
\end{enumerate}

\end{problem}

\begin{problem}
\begin{enumerate}
\item[1)] Имеется монетка (несимметричная). Несимметричность монетки заключается в том, что либо орел выпадает в два раза чаще решки; 
либо наоборот (априорно (до проведения опытов) оба варианта считаются равновероятными). Монетку бросили $10$ раз. Орел выпал $7$ раз. 
Определите апостериорную вероятность того, что орел выпадает в два раза чаще решки (апостериорная вероятность считается с учетом 
проведенных опытов (иначе говоря, это просто условная вероятность)). 

\item[2)] Определите апостериорную вероятность того, что орел выпадает не менее чем в два раза чаще решки. Если несимметричность 
монетки заключается в том, что либо орел выпадает не менее чем в два раза чаще решки; либо наоборот (априорно оба варианта считаются 
равновероятными). 
\end{enumerate}
\end{problem}

\begin{problem}

На некоторой реке имеется 6 островов, соединенных между собой системой мостов. Во время летнего наводнения часть мостов была разрушена. При этом каждый мост разрушается с вероятностью ${1\mathord{\left/ {\vphantom {1 2}} \right. \kern-\nulldelimiterspace} 2} $, независимо от других мостов. Какова вероятность того, что после наводнения можно будет перейти с одного берега на другой, используя не разрушенные мосты?

\imgh{70mm}{graphs_bridges.pdf}{Схема мостов}

\end{problem}

\begin{problem}[Остовные деревья в полном графе]

Пусть имеется полный граф с $n$ вершинами $\{1,2,\ldots ,n\}$. Каждое из 
$\frac{n(n-1)}{2}$ ребер графа с вероятностью $\frac{1}{2}$ удаляется. 
Найдите вероятность того, что полученный после удаления ребер граф будет 
остовным деревом.

\end{problem}

\begin{problem}
Докажите, что при $n\to\infty$ 
$$
X_n\xrightarrow{L_2} X \,\Rightarrow\, X_n\xrightarrow{L_1}X \, \Rightarrow\, X_n\xrightarrow{P}X 
\, \Leftarrow\, X_n\xrightarrow{\text{ п.н. }}X , 
$$
$$
X_n\xrightarrow{P}X \, \Rightarrow\, X_n\xrightarrow{d}X . 
$$
С помощью контрпримеров покажите, что никакие другие стрелки импликации в эту схему в общем случае добавить нельзя. 
При каких дополнительных условиях можно утверждать, что 
$$
X_n\xrightarrow{\text{ п.н. }}X  \, \Rightarrow\, X_n\xrightarrow{L_1}X ?
$$
Кроме того, показать, что 
$$
X_n\xrightarrow{P} X \; (n\to\infty) \,\Leftrightarrow\, \rho_P(X_n,X)={\mathbb E}\Bigl( \frac{|X_n-X|}{1+|X_n-X|}\Bigr)
\xrightarrow{n\to\infty} 0 . 
$$
Также показать, что 
$$
X_n \xrightarrow{d}c\quad \Rightarrow \quad X_n \xrightarrow{P}c, \text{ где } c=\const \text{ (не с.в.) }
$$
\end{problem}

\begin{ordre}
$ $
\begin{enumerate}

\item Из сходимости по распределению не следует сходимость по вероятности, а также сходимость в $L_1$, $L_2$ и почти наверное. Проанализируйте следующий контр пример.  

\[
X,Y,X,Y,\ldots
\]

 где  случайные величины $X(\omega)=\omega$ и $Y(\omega)=1-\omega$ имеют одну и ту же функцию распределения 
$$
F_X(x)=F_Y(x)=x\cdot {\mathbb I}_{\{ x\in[0,1]\}} . 
$$

\item Из сходимости по вероятности не следует сходимость почти наверное. Проанализируйте в качестве контр примера серию бегущих импульсов.
 
$$
X_1={\mathbb I}_{[0,1/2]},\, X_2={\mathbb I}_{[1/2,1]},\, X_3={\mathbb I}_{[0,1/4]},\, X_4={\mathbb I}_{[1/4,1/2]}, \ldots, 
$$

\item Выполнена импликация 
$$
X_n\xrightarrow{\text{ п.н. }}X  \, \Rightarrow\, X_n\xrightarrow{L_1}X , 
$$
т.е. возможен предельный переход под знаком математического ожидания, если семейство с.в. $\{ X_n\}$ является равномерно интегрируемым: 
$$
\sup\limits_n {\mathbb E}\bigl[ |X_n|\cdot {\mathbb I}_{\{ |X_n|>c\}} \bigr]\xrightarrow{c\to +\infty}0 . 
$$

\item 


$\frac{|X_n-X|}{1+|X_n-X|}<1$, $\frac{|X_n-X|}{1+|X_n-X|}\leqslant |X_n-X|$. 


\begin{multline*}
{\mathbb E}\Bigl( \frac{|X_n-X|}{1+|X_n-X|}\Bigr)=\int\limits_{|X_n-X|\leqslant\varepsilon} 
\frac{|X_n(\omega)-X(\omega)|}{1+|X_n(\omega)-X(\omega)|}\, P(d\omega)+\\
+\int\limits_{|X_n-X|>\varepsilon}
\frac{|X_n(\omega)-X(\omega)|}{1+|X_n(\omega)-X(\omega)|}\, P(d\omega)
\leqslant \varepsilon+{\mathbb P}(|X_n-X|>\varepsilon) . 
\end{multline*}

\end{enumerate}

\end{ordre}

\begin{problem}[рекорды]
Пусть $X_1 ,X_2 ,\ldots $ - независимые 
случайные величины с одной и той же плотностью распределения вероятностей 
$p(x)$. Будем говорить, что наблюдается рекордное значение в момент времени 
n$>$1, если $X_n >\max \left[ {X_1 ,...,X_{n-1} } \right]$. Докажите 
следующие утверждения.

\begin{enumerate}
\item[\textbf{А)}] Вероятность того, что рекорд зафиксирован в момент времени $n$, 
равна $1/n$.

\item[\textbf{Б)}] Математическое ожидание числа рекордов до момента времени $n$ 
равно 
\[
\sum\limits_{1<i\le n} {\frac{1}{i}} \sim \ln n.
\]

\item[\textbf{В)}] Пусть $Y_n $ --- случайная величина, принимающая значение $1$, если 
в момент времени $n$ зафиксирован рекорд, и значение $0$ -- в противном случае. 
Тогда случайные величины $Y_1 ,Y_2 ,\ldots$ независимы в совокупности.

\item[\textbf{Г)}] Дисперсия числа рекордов до момента времени $n$ равна
\[
\sum\limits_{1<i\le n} {\frac{i-1}{i^2}} \sim \ln n.
\]

\item[\textbf{Д)}] Если $T$ -- момент появления первого рекорда после момента времени $1$, то $ET=\infty $.
\end{enumerate}
\end{problem}

\begin{problem}[распределение Коши]
Радиоактивный источник испускает 
частицы в случайном направлении (при этом все направления равновероятны). 
Источник находится на расстоянии $d$ от фотопластины, которая представляет 
собой бесконечную вертикальную плоскость.

\begin{enumerate}
\item[\textbf{А)}] При условии, что частица попадает в плоскость, покажите, что 
горизонтальная координата точки попадания (если начало координат выбирается 
в точке, ближайшей к источнику) имеет плотность распределения:
\[
p\left( x \right)=\frac{d}{\pi \left( {d^2+x^2} \right)}.
\]
Это распределение известно как \textit{распределение Коши}.

\item[\textbf{Б)}] Можно ли вычислить среднее (математическое ожидание) этого 
распределения?
\end{enumerate}
\end{problem}

\begin{problem}
Предположим, что с.в. $X\in L_2$, это означает ${\mathbb E}X^2<\infty$. Докажите, что 
\begin{equation}
\label{UMO}
\| X-{\mathbb E}(X|Y_1,\ldots,Y_n)\|_{L_2}=\min\limits_{\varphi\in H} \| X-\varphi(Y_1,\ldots,Y_n)\|_{L_2} , 
\end{equation}
где $H$ --- подпространство пространства $L_2$ всевозможных борелевских функций $\varphi(Y_1,\ldots,Y_n)\in L_2$; 
${\mathbb E}(X|Y_1,\ldots,Y_n)$ --- условное математическое ожидание с.в. $X$ относительно $\sigma$-алгебры, порожденной с.в. 
$Y_1,\ldots,Y_n$, часто говорят просто относительно с.в. $Y_1,\ldots,Y_n$; 
$$
\| X\|_{L_2}=\sqrt{\langle X,X\rangle_{L_2}}=\sqrt{{\mathbb E}(X\cdot X)}=\sqrt{{\mathbb E}(X^2)} . 
$$
\end{problem}

\begin{ordre}
Покажите, что $X-{\mathbb E}^{\mathcal A}X \bot \xi,\quad \forall\xi\in H$, т.е. ${\mathbb E}^{\mathcal A}$ 
является проектором на подпространство $H$ в $L_2$. 
\end{ordre}


\begin{problem}
Докажите, что если в условиях предыдущей задачи $(X,Y_1,\ldots,Y_n)^T$ --- является нормальным случайным вектором (без ограничения 
общности можно также считать, что $(Y_1,\ldots,Y_n)^T$  --- невырожденный нормальный случайный вектор), то в качестве $H$ можно взять 
подпространство всевозможных линейных комбинаций с.в. $Y_1,\ldots,Y_n$. Т.е. мы можем более конкретно сказать, на каком именно 
классе борелевских функций достигается минимум в $(\ref{UMO})$. 
\end{problem}

\begin{ordre}
Будем искать 
${\mathbb E}(X|Y_1,\ldots,Y_n)$ в виде 
\begin{equation}
\label{Gauss}
{\mathbb E}(X|Y_1,\ldots,Y_n)=c_1 Y_1+\ldots +c_n Y_n . 
\end{equation}

Докажите следующие утверждения:

\begin{enumerate}
\item $X-c_1 Y_1-\ldots-c_n Y_n, Y_1,\ldots, Y_n$ - независимы.
\item $X-c_1 Y_1-\ldots-c_n Y_n$ ортогонален подпространству $H$ пространства $L_2$ всевозможных борелевских функций $\varphi(Y_1,\ldots,Y_n)\in L_2$.
\end{enumerate}
 
\end{ordre}
