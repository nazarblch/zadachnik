%\subsection{Закон больших чисел}

\begin{problem}
Пусть случайная величина $X_n$ принимает значения 
$2^n$ и $-2^n$ с вероятностями $1/2$. Выполняется ли для последовательности независимых случайных величин 
$X_1$, $X_2$, $\ldots$ закон больших чисел? 
\end{problem}

\begin{ordre}
Покажите, что усредненная сумма последовательности не сходится по вероятности к своему математическому ожиданию, зафиксировав знаки последних двух слагаемых. 
\end{ordre}


\begin{problem}
Пусть случайная величина $X_n$ принимает значения 
$n$, $0$ и $-n$ с вероятностями $1/4$, $1/2$, $1/4$. Выполняется ли для последовательности независимых случайных величин 
$X_1$, $X_2$, $\ldots$ закон больших чисел? 
\end{problem}

\begin{ordre}
 
$$
S_n\xrightarrow{p}0 \,\Leftrightarrow\, S_n\xrightarrow{D}0 \,\Leftrightarrow\, \varphi_{S_n}(t)
\xrightarrow{n\to\infty}1 , 
$$

где $S_n=\frac{X_1+\ldots +X_n}{n}$

\end{ordre}


\begin{problem}
Пусть $\{ X_n\}_{n=1}^{\infty}$ -- последовательность независимых случайных величин, причем $X_n$ принимает значения 
$-\sqrt{n}$, $\sqrt{n}$ с вероятностями $1/2$ каждое. 
Выполняется для этой последовательности закон больших чисел? 
\end{problem}


\begin{problem}
При каких значениях $\alpha > 0$ к последовательности независимых случайных величин $\{ X_n\}_{n=1}^{\infty}$, 
таких что ${\mathbb P}\{ X_n=n^{\alpha}\}={\mathbb P}\{ X_n=-n^{\alpha}\}=1/2$, применим закон больших чисел? 
\end{problem}

\begin{comment}
\begin{ordre}
Докажите достаточное условие выполнения ЗБЧ:
 \[
Var S_n \xrightarrow {n\to\infty}0
\] 
\end{ordre}
\end{comment}

\begin{problem}
Пусть $\{ X_n\}_{n=1}^{\infty}$ --- последовательность случайных величин с дисперсиями $\sigma_i^2$. Доказать, что если все 
корреляционные моменты (ковариации) $R_{ij}$ случайных величин $X_i$ и $X_j$ неположительны и при  
$\frac{\sum\limits_{i=1}^{n} \sigma_i^2}{n^2}\to 0$, $n\to\infty$, то для последовательности $\{ X_n\}_{n=1}^{\infty}$ выполняется закон больших чисел. 
\end{problem}

\begin{problem}
Пусть $\{ X_n\}_{n=1}^{\infty}$ --- последовательность случайных величин с равномерно ограниченными дисперсиями, причем каждая 
случайная величина $X_n$ зависит только от $X_{n-1}$ и $X_{n+1}$, но не зависит от остальных $X_i$. Доказать выполнение для этой 
последовательности закона больших чисел.
\end{problem}

\begin{problem}
Книга объемом $500$ страниц содержит $50$ опечаток. Оценить вероятность того, что на случайно выбранной странице 
имеется не менее трех опечаток. (Использовать нормальное и пуассоновское приближения, сравнить результаты). 
\end{problem}

\begin{problem}
В тесто для выпечки булок с изюмом замешано $N$ изюмин. Всего из данного теста выпечено $K$ булок. Оценить вероятность того, 
что в случайно выбранной булке число изюмин находится в пределах от $a$ до $b$. 
\end{problem}

\begin{problem}
В поселке $N$ жителей, каждый из которых в среднем $n$ раз в месяц ездит в город, выбирая дни поездки независимо от остальных. 
Поезд из поселка в город идет один раз в сутки. Какова должна быть вместимость поезда для того, чтобы он переполнился с вероятностью, 
не превышающей заданного числа $\beta$? 
\end{problem}
