\subsection{Случайный вектор}



\begin{problem}
Пусть $X$ и $Y$ --- независимые случайные величины, равномерно распределенные на $(-b,b)$. 
Найдите вероятность $q_b$ того, что уравнение $t^2+tX+Y=0$ имеет действительные корни. Доказать, что 
существует $\lim\limits_{b\to\infty} q_b=q$. Найдите $q$. 
\end{problem}


\begin{problem}
Компоненты случайного вектора имеют нормальное распределение. Следует ли из этого, что вектор имеет нормальное распределение? 
\end{problem}

\begin{problem}
Двумерный случайный вектор $X=(X_1,X_2)$  имеет следующую функцию плотности распределения: 
$$
f(x_1,x_2)=\begin{cases}
\dfrac{c}{\sqrt{x_1^2+x_2^2}}, \text{ при } x_1^2+x_2^2\leqslant 1 , \\
\quad 0, \quad\text{ иначе }. 
\end{cases}
$$
\begin{enumerate}
\item[1)] Найдите $c$. 
\item[2)] Найдите частные и условные распределения его компонент. 
\item[3)] Являются ли они 
\begin{enumerate}
\item[а)] стохастически зависимыми; 
\item[б)] коррелированными? 
\end{enumerate}
\end{enumerate}
\end{problem}



\begin{problem}
Предположим, что с.в. $X\in L_2$, это означает ${\mathbb E}X^2<\infty$. Докажите, что 
\begin{equation}
\label{UMO}
\| X-{\mathbb E}(X|Y_1,\ldots,Y_n)\|_{L_2}=\min\limits_{\varphi\in H} \| X-\varphi(Y_1,\ldots,Y_n)\|_{L_2} , 
\end{equation}
где $H$ --- подпространство пространства $L_2$ всевозможных борелевских функций $\varphi(Y_1,\ldots,Y_n)\in L_2$; 
${\mathbb E}(X|Y_1,\ldots,Y_n)$ --- условное математическое ожидание с.в. $X$ относительно $\sigma$-алгебры, порожденной с.в. 
$Y_1,\ldots,Y_n$, часто говорят просто относительно с.в. $Y_1,\ldots,Y_n$; 
$$
\| X\|_{L_2}=\sqrt{\langle X,X\rangle_{L_2}}=\sqrt{{\mathbb E}(X\cdot X)}=\sqrt{{\mathbb E}(X^2)} . 
$$
\end{problem}

\begin{ordre}
Покажите, что $X-{\mathbb E}^{\mathcal A}X \bot \xi,\quad \forall\xi\in H$, т.е. ${\mathbb E}^{\mathcal A}$ 
является проектором на подпространство $H$ в $L_2$. 
\end{ordre}


\begin{problem}
Докажите, что если в условиях предыдущей задачи $(X,Y_1,\ldots,Y_n)^T$ --- является нормальным случайным вектором (без ограничения 
общности можно также считать, что $(Y_1,\ldots,Y_n)^T$  --- невырожденный нормальный случайный вектор), то в качестве $H$ можно взять 
подпространство всевозможных линейных комбинаций с.в. $Y_1,\ldots,Y_n$. Т.е. мы можем более конкретно сказать, на каком именно 
классе борелевских функций достигается минимум в $(\ref{UMO})$. 
\end{problem}

\begin{ordre}
Будем искать 
${\mathbb E}(X|Y_1,\ldots,Y_n)$ в виде 
\begin{equation}
\label{Gauss}
{\mathbb E}(X|Y_1,\ldots,Y_n)=c_1 Y_1+\ldots +c_n Y_n . 
\end{equation}

Докажите следующие утверждения:

\begin{enumerate}
\item $X-c_1 Y_1-\ldots-c_n Y_n, Y_1,\ldots, Y_n$ - независимы.
\item $X-c_1 Y_1-\ldots-c_n Y_n$ ортогонален подпространству $H$ пространства $L_2$ всевозможных борелевских функций $\varphi(Y_1,\ldots,Y_n)\in L_2$.
\end{enumerate}
 
\end{ordre}



\begin{problem}
$$
x=\begin{pmatrix}
x_1\\
x_2\\
x_3
\end{pmatrix}
\in N\left(
\begin{pmatrix}
2\\
3\\
1
\end{pmatrix}, 
\begin{Vmatrix}
5 & 2 & 7\\
2 & 5 & 7\\
7 & 7 & 14
\end{Vmatrix}
\right) . 
$$
\begin{enumerate}
\item[а)] Найдите распределение случайной величины $y_1=x_1+x_2-x_3$. 
\item[б)] Найдите распределение случайной величины $y_2=x_1+x_2+x_3$. 
\item[в)] Найдите ${\mathbb E}(y_2\, |\, x_1=5, x_2=3)$. 
\item[г)] Найдите ${\mathbb E}(y_2\, |\, x_1=5, x_2<3)$. 
\item[д)] Найдите ${\mathbb P}(y_2<10\, |\, x_1=5, x_2<3)$. 
\end{enumerate}
\end{problem}

\begin{ordre}
Проверьте следующее свойство нормального случайного вектора.
Пусть $y=c^Tx$. Тогда 
$$
{\mathbb E}y=c^T{\mathbb E}x, 
$$
$$
\Var y=c^T (\Var x) c 
$$
\end{ordre}



\begin{problem}
Случайный вектор $W=(W_1, W_2, \ldots, W_n)'$ имеет плотность распределения 
$$
f_W(w)=\begin{cases}
n! , & \text{ если } 0\leqslant w_1\leqslant w_2\leqslant \ldots \leqslant w_n\leqslant 1, \\
0, & \text{ в остальных случаях }. 
\end{cases}
$$
Найти распределение вектора $U=(U_1, U_2, \ldots, U_n)'$, если 
$$
U_1=W_1, \quad U_2=W_2-W_1, \ldots, U_n=W_n-W_{n-1} . 
$$
\end{problem}

\begin{ordre}
В общем случае: 
$$
U=\varphi(W) \,\Rightarrow\, f_U(u)=f_W\bigl(\varphi^{-1}(u)\bigr) \Bigl| J\Bigl( \frac{\partial W}{\partial U}\Bigr)\Bigr|, \quad 
J(\cdot)=\det\Bigl( \frac{\partial \varphi_i^{-1}(u)}{\partial u_j}\Bigr) . 
$$
\end{ordre}


\begin{problem}[переход к полярным координатам, якобиан преобразования]
Спортсмен стреляет по круговой мишени. Вертикальная и горизонтальная 
координаты точки попадания пули (при условии, что центр мишени -- начало 
координат) -- независимые случайные величины, каждая с распределением 
$N(0,1)$. Покажите, что расстояние от точки попадания до центра имеет 
плотность распределения вероятностей $r\exp \left( {-{r^2} \mathord{\left/ 
{\vphantom {{r^2} 2}} \right. \kern-\nulldelimiterspace} 2} \right)$ для 
$r\ge 0$. Найдите медиану этого распределения.
\end{problem}