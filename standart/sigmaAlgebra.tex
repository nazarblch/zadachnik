\subsection{Вероятностное пространство, $\sigma$–алгебра}

\begin{problem}
Показать, что борелевская $\sigma$-алгебра в ${\mathbb R}^1$, содержащая все числовые промежутки вида $[a,b)$, 
содержит все промежутки вида $(a,b)$, $(a,b]$, $[a,b]$ и отдельные точки прямой. 

\begin{ordre}
Учесть свойство замкнутости $\sigma$-алгебры относительно операций 
объединения, пересечения и вычитания. 
\end{ordre}

\end{problem}


\begin{problem}
Пусть $\Omega = [a, b]$, $F$ --- $\sigma$–алгебра, содержащая все отрезки 
$[a,b]$ $(a \leqslant \alpha < \beta \leqslant b)$ с вероятностной мерой 
${\mathbb P}\{ \omega\in[\alpha,\beta]\}=\dfrac{\mes[\alpha, \beta]}{\mes[a,b]}$. 
Показать, что 
\begin{enumerate}
\item[а)] ${\mathbb P}\{ \omega=c=\const\}=0$; 
\item[б)] ${\mathbb P}\{ \omega_1=\omega_2\}=0$. 
\end{enumerate}
Найти вероятность, что для трех исходов $\omega_1$, $\omega_2$, $\omega_3$ третий лежит между первыми двумя. 
\end{problem}


\begin{problem}
Число элементарных событий некоторого вероятностного пространства равно $n$. Указать минимальное и максимальное возможные значения 
для числа событий. 
\end{problem}

\begin{problem}
Может ли число всех событий какого-либо вероятностного пространства быть равным $129$; $130$; $128$? 
\end{problem}


\begin{problem}
В урне находится $3$ белых и $2$ черных шара (и только они). 
Эксперимент состоит в последовательном извлечении из урны всех шаров по одному наугад без возвращения. Построить вероятностное пространство. 
Описать $\sigma$-алгебру, порожденную случайной величиной $X$, если: 
\begin{enumerate}
\item[а)] $X$ --- число белых шаров, предшествующих первому черному шару; 
\item[б)] $X$ --- число черных шаров среди извлеченных; 
\item[в)] $X=X_1+X_2$, где $X_1$ --- число белых шаров, предшествующих первому черному шару, 
$X_2$ --- число черных шаров, предшествующих белому шару. 
\end{enumerate}
\end{problem}


\begin{problem}
\label{SigmaAlgebra}
Пусть $(\Omega,\Sigma,{\mathbb P})$ --- некоторое вероятностное пространство и $A$ --- алгебра подмножеств $\Omega$ такая, что 
$\sigma(A)=\Sigma$ ($\sigma(A)$ --- наименьшая $\sigma$-алгебра, содержащая алгебру $A$). Доказать, что 
$$
\forall\varepsilon>0,\, B\in\Sigma\quad \exists A_{\varepsilon}\in A:\quad {\mathbb P}(A_{\varepsilon}\bigtriangleup B)
\leqslant\varepsilon . 
$$
\end{problem}

\begin{ordre}
Рассмотрим совокупность множеств 
$$
{\mathcal B}=\bigl\{ B\in\Sigma\, | \, \forall\varepsilon>0 \; \exists A_B\in A:\; {\mathbb P}(A_B\bigtriangleup B)
\leqslant\varepsilon \bigr\} . 
$$

Покажите, что ${\mathcal B}$ является минимальной $\sigma$-алгеброй 

\end{ordre}

\begin{problem} (пример о сходимости ряда)
Пусть $(\Omega,\Xi,{\mathbb P})$ --- вероятностное пространство, $\xi_1,\xi_2,\ldots$ --- некоторая последовательность с.в.. 
Обозначим $\Xi_n^{\infty}=\sigma(\xi_{n},\xi_{n+1},\ldots)$ --- $\sigma$-алгебру, порожденную с.в. $\xi_{n},\xi_{n+1},\ldots$ и пусть 
$$
{\mathcal X}=\bigcap\limits_{n=1}^{\infty} \Xi_{n}^{\infty} . 
$$
Поскольку пересечение $\sigma$-алгебр есть снова $\sigma$-алгебра, то ${\mathcal X}$ --- есть $\sigma$-алгебра. Эту $\sigma$-алгебру 
будем называть <<хвостовой>> или <<остаточной>>, в связи с тем, что всякое событие $A\in{\mathcal X}$ не зависит от значений с.в. 
$\xi_1,\xi_2,\ldots,\xi_n$ при любом конечном $n$, а определяется лишь <<поведением бесконечно далеких значений последовательности 
$\xi_1,\xi_2,\ldots$ >>. 

С помощью задачи $\ref{SigmaAlgebra}$ докажите справедливость следующего утверждения: 

Пусть $\xi_1,\xi_2,\ldots$ --- последовательность независимых в совокупности с.в. и $A\in{\mathcal X}$ 
(событие $A$ принадлежит <<хвостовой>> $\sigma$-алгебре). Тогда ${\mathbb P}(A)$ может принимать лишь два значения $0$ или $1$. 
\end{problem}

\begin{ordre}
Идея доказательства состоит в том, чтобы показать, что каждое <<хвостовое>> событие $A$ не зависит от самого себя и, значит, 
${\mathbb P}(A\cap A)={\mathbb P}(A)\cdot {\mathbb P}(A)$, т.е. ${\mathbb P}(A)={\mathbb P}^2(A)$, откуда 
${\mathbb P}(A)=0$ или $1$. 
\end{ordre}