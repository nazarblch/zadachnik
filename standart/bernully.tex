
\subsection{Испытания Бернулли}

\begin{problem}
Опыт состоит в подбрасывании монеты до тех пор, пока два раза подряд она не выпадет одной и той же стороной. Каждому 
возможному исходу опыта припишем вероятность $1/2$ (монета <<правильная>>). 
Построить пространство элементарных событий и найти вероятности следующих событий: 
\begin{enumerate}
\item[а)] опыт окончится до шестого бросания; 
\item[б)] для завершения опыта потребуется четное число бросаний. 
\end{enumerate}
\end{problem}


\begin{problem}
При каждом подбрасывании монета падает вверх орлом с вероятностью $p>0$. Пусть $\pi _{n} $ - вероятность того, что число орлов после $n\in {\mathbb N}$ независимых подбрасываний будет чётно. Показав, что $\pi _{n+1} =\left(1-p\right)\cdot \pi _{n} +p\cdot \left(1-\pi _{n} \right)$, $n\in {\mathbb N}$, или иным способом найдите $\pi _{n} $. Число $0$ считаем чётным.
\end{problem}

\begin{problem}
Симметричную монету независимо бросили $n$ раз. Результат бросания записали в виде последовательности нулей и единиц. Покажите, что с вероятностью стремящейся к единице при $n\to \infty $ длина максимальной подпоследовательности из подряд идущих единиц лежит в промежутке
\[\left(\log \sqrt{n} ,\; \log n^{2} \right).\] 
\end{problem}
