\subsection{Сходимость случайных величин}

Introduction: quelque phrases de limit tipes.

\begin{problem}
Докажите, что при $n\to\infty$ 
$$
X_n\xrightarrow{L_2} X \,\Rightarrow\, X_n\xrightarrow{L_1}X \, \Rightarrow\, X_n\xrightarrow{P}X 
\, \Leftarrow\, X_n\xrightarrow{\text{ п.н. }}X , 
$$
$$
X_n\xrightarrow{P}X \, \Rightarrow\, X_n\xrightarrow{d}X . 
$$
С помощью контрпримеров покажите, что никакие другие стрелки импликации в эту схему в общем случае добавить нельзя. 
При каких дополнительных условиях можно утверждать, что 
$$
X_n\xrightarrow{\text{ п.н. }}X  \, \Rightarrow\, X_n\xrightarrow{L_1}X ?
$$
Кроме того, показать, что 
$$
X_n\xrightarrow{P} X \; (n\to\infty) \,\Leftrightarrow\, \rho_P(X_n,X)={\mathbb E}\Bigl( \frac{|X_n-X|}{1+|X_n-X|}\Bigr)
\xrightarrow{n\to\infty} 0 . 
$$
Также показать, что 
$$
X_n \xrightarrow{d}c\quad \Rightarrow \quad X_n \xrightarrow{P}c, \text{ где } c=\const \text{ (не с.в.) }
$$
\end{problem}

\begin{ordre}
$ $
\begin{enumerate}

\item Из сходимости по распределению не следует сходимость по вероятности, а также сходимость в $L_1$, $L_2$ и почти наверное. Проанализируйте следующий контр пример.  

\[
X,Y,X,Y,\ldots
\]

 где  случайные величины $X(\omega)=\omega$ и $Y(\omega)=1-\omega$ имеют одну и ту же функцию распределения 
$$
F_X(x)=F_Y(x)=x\cdot {\mathbb I}_{\{ x\in[0,1]\}} . 
$$

\item Из сходимости по вероятности не следует сходимость почти наверное. Проанализируйте в качестве контр примера серию бегущих импульсов.
 
$$
X_1={\mathbb I}_{[0,1/2]},\, X_2={\mathbb I}_{[1/2,1]},\, X_3={\mathbb I}_{[0,1/4]},\, X_4={\mathbb I}_{[1/4,1/2]}, \ldots, 
$$

\item Выполнена импликация 
$$
X_n\xrightarrow{\text{ п.н. }}X  \, \Rightarrow\, X_n\xrightarrow{L_1}X , 
$$
т.е. возможен предельный переход под знаком математического ожидания, если семейство с.в. $\{ X_n\}$ является равномерно интегрируемым: 
$$
\sup\limits_n {\mathbb E}\bigl[ |X_n|\cdot {\mathbb I}_{\{ |X_n|>c\}} \bigr]\xrightarrow{c\to +\infty}0 . 
$$

\item 


$\frac{|X_n-X|}{1+|X_n-X|}<1$, $\frac{|X_n-X|}{1+|X_n-X|}\leqslant |X_n-X|$. 


\begin{multline*}
{\mathbb E}\Bigl( \frac{|X_n-X|}{1+|X_n-X|}\Bigr)=\int\limits_{|X_n-X|\leqslant\varepsilon} 
\frac{|X_n(\omega)-X(\omega)|}{1+|X_n(\omega)-X(\omega)|}\, P(d\omega)+\\
+\int\limits_{|X_n-X|>\varepsilon}
\frac{|X_n(\omega)-X(\omega)|}{1+|X_n(\omega)-X(\omega)|}\, P(d\omega)
\leqslant \varepsilon+{\mathbb P}(|X_n-X|>\varepsilon) . 
\end{multline*}

\end{enumerate}

\end{ordre}