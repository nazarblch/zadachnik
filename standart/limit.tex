%\subsection{Сходимость случайных величин}

Introduction: quelque phrases de limit tipes.


\begin{problem}
Пусть $X_n \overset{a.s.}{\longrightarrow} X$, $Y_n \overset{a.s.}{\longrightarrow} Y$. Доказать справедливость соотношений:
\begin{enumerate}
\item $a X_n + b Y_n \overset{a.s.}{\longrightarrow} a X + b Y$, где  $a, b = const$
\item $|X_n| \overset{a.s.}{\longrightarrow} |X|$
\item $X_n Y_n \overset{a.s.}{\longrightarrow} XY$
\end{enumerate}

\end{problem}

\begin{problem}
Пусть $X_n \overset{d}{\longrightarrow} a$, где $a = const$. Доказать справедливость соотношения $X_n \overset{p}{\longrightarrow} a$.
\end{problem}

\begin{problem}
Привести пример, показывающий, что из сходимости по вероятности не следует сходимость в среднем порядка $p > 0$.
\end{problem}

\begin{problem}
Пусть $X_n \overset{L_2}{\longrightarrow} X$, причем $\Exp |X_n| < \infty$. Доказать справедливость соотношений $\Exp |X| < \infty$ и $\Exp (X_n)\longrightarrow \Exp(X)$.
\end{problem}

\begin{problem}
Пусть в вероятностном пространстве  $<$$\Omega$,F,P$>$  $\Omega$ = [0,1), \textit{F}  -- $\sigma$ - алгебра с полуинтервалами вида $\Omega _{in} =[(i-1)/n,\; i/n)$,  \textit{i=}1,\dots ,\textit{n}, \textit{n }= 1, 2, \dots  и P --мера  Лебега ($(\forall i,n:P\{ \omega \in \Omega _{in} \} =1/n)$.

Исследовать сходимость следующих последовательностей случайных величин 

$X_{1}^{\eqref{GrindEQ__1_}} \; (\omega ),\; \; X_{2}^{\eqref{GrindEQ__1_}} \; (\omega ),\; \; X_{2}^{\eqref{GrindEQ__2_}} \; (\omega ),\; \; X_{3}^{\eqref{GrindEQ__1_}} \; (\omega ),\; \; X_{3}^{\eqref{GrindEQ__2_}} \; (\omega ),\; \; X_{3}^{\eqref{GrindEQ__3_}} \; (\omega ),\ldots ,$где, 

\textit{а}) $X_{n}^{(i)} \; (\omega )=n\; \; {\rm ?@8}\; \; \omega \in \Omega _{in} ,\; X_{n}^{(i)} (\omega )=0\; \; {\rm ?@8}\; \; \omega \in \Omega \backslash \Omega _{in} ;$

\textit{б}) $X_{n}^{(i)} \; (\omega )=n^{-1} \; \; {\rm ?@8}\; \omega \in \Omega _{in} ,\; X_{n}^{(i)} (\omega )=0\; \; {\rm ?@8}\; \; \omega \in \Omega \backslash \Omega _{in} ;$

\textit{в}) $X_{n}^{(i)} (\omega )=n\; \; {\rm ?@8}\; \; \omega \in \Omega _{in} ,\; \, X_{n}^{(i)} \; (\omega )=n^{-1} \; {\rm ?@8}\; \; \omega \in \Omega \backslash \Omega _{in} ;$

\textit{г}) $X_{n}^{(i)} \; (\omega )=n^{-1} \; \; {\rm ?@8}\; \; \omega \in \Omega _{in} ,\; X_{n}^{(i)} \; (\omega )=1-n^{-1} \; {\rm ?@8}$ $\omega \in \Omega \backslash \Omega _{in} ;$

\textit{д}) $X_{n}^{(i)} \; (\omega )\in N(m_{in} ,\; \sigma _{in}^{2} ),\; \, {\rm 345}\; \mathop{\lim }\limits_{in\to \infty } \; m_{in} =3,\; \mathop{\lim }\limits_{in\to \infty } \sigma _{in}^{2} =1.$s
\end{problem}

\begin{problem}
Доказать, что не существует метрики, определенной на парах случайных величин, сходимость в которой равносильна сходимости почти наверное.
\end{problem}

\begin{problem} (добавить замечание)
Число $\alpha$ из отрезка $[0, 1]$ назовем нормально приближаемым рациональными числами, если найдутся $c,\varepsilon>0$ такие, что 
при любом натуральном $q$ 
\begin{equation}
\label{BorelKantel}
\min\limits_{p\in {\mathbb Z}} \Bigl|\alpha-\frac{p}{q} \Bigr|\geqslant \frac{c}{q^{2+\varepsilon}} . 
\end{equation}
Используя лемму Бореля-Кантелли, докажите, что множество нормально приближаемых чисел на отрезке $[0, 1]$ имеет Лебегову меру $1$. 

\end{problem}

\begin{comment}
\begin{problem}
Докажите, что при $n\to\infty$ 
$$
X_n\xrightarrow{L_2} X \,\Rightarrow\, X_n\xrightarrow{L_1}X \, \Rightarrow\, X_n\xrightarrow{P}X 
\, \Leftarrow\, X_n\xrightarrow{\text{ п.н. }}X , 
$$
$$
X_n\xrightarrow{P}X \, \Rightarrow\, X_n\xrightarrow{d}X . 
$$
С помощью контрпримеров покажите, что никакие другие стрелки импликации в эту схему в общем случае добавить нельзя. 
При каких дополнительных условиях можно утверждать, что 
$$
X_n\xrightarrow{\text{ п.н. }}X  \, \Rightarrow\, X_n\xrightarrow{L_1}X ?
$$
Кроме того, показать, что 
$$
X_n\xrightarrow{P} X \; (n\to\infty) \,\Leftrightarrow\, \rho_P(X_n,X)={\mathbb E}\Bigl( \frac{|X_n-X|}{1+|X_n-X|}\Bigr)
\xrightarrow{n\to\infty} 0 . 
$$
Также показать, что 
$$
X_n \xrightarrow{d}c\quad \Rightarrow \quad X_n \xrightarrow{P}c, \text{ где } c=\const \text{ (не с.в.) }
$$
\end{problem}

\begin{ordre}
$ $
\begin{enumerate}

\item Из сходимости по распределению не следует сходимость по вероятности, а также сходимость в $L_1$, $L_2$ и почти наверное. Проанализируйте следующий контр пример.  

\[
X,Y,X,Y,\ldots
\]

 где  случайные величины $X(\omega)=\omega$ и $Y(\omega)=1-\omega$ имеют одну и ту же функцию распределения 
$$
F_X(x)=F_Y(x)=x\cdot {\mathbb I}_{\{ x\in[0,1]\}} . 
$$

\item Из сходимости по вероятности не следует сходимость почти наверное. Проанализируйте в качестве контр примера серию бегущих импульсов.
 
$$
X_1={\mathbb I}_{[0,1/2]},\, X_2={\mathbb I}_{[1/2,1]},\, X_3={\mathbb I}_{[0,1/4]},\, X_4={\mathbb I}_{[1/4,1/2]}, \ldots, 
$$

\item Выполнена импликация 
$$
X_n\xrightarrow{\text{ п.н. }}X  \, \Rightarrow\, X_n\xrightarrow{L_1}X , 
$$
т.е. возможен предельный переход под знаком математического ожидания, если семейство с.в. $\{ X_n\}$ является равномерно интегрируемым: 
$$
\sup\limits_n {\mathbb E}\bigl[ |X_n|\cdot {\mathbb I}_{\{ |X_n|>c\}} \bigr]\xrightarrow{c\to +\infty}0 . 
$$

\item 


$\frac{|X_n-X|}{1+|X_n-X|}<1$, $\frac{|X_n-X|}{1+|X_n-X|}\leqslant |X_n-X|$. 


\begin{multline*}
{\mathbb E}\Bigl( \frac{|X_n-X|}{1+|X_n-X|}\Bigr)=\int\limits_{|X_n-X|\leqslant\varepsilon} 
\frac{|X_n(\omega)-X(\omega)|}{1+|X_n(\omega)-X(\omega)|}\, P(d\omega)+\\
+\int\limits_{|X_n-X|>\varepsilon}
\frac{|X_n(\omega)-X(\omega)|}{1+|X_n(\omega)-X(\omega)|}\, P(d\omega)
\leqslant \varepsilon+{\mathbb P}(|X_n-X|>\varepsilon) . 
\end{multline*}

\end{enumerate}

\end{ordre}
\end{comment}