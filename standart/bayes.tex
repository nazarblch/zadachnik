\subsection{Формула Байеса}


\begin{problem}
Имеются две урны. В одной из них находится один белый шар, в другой --- один черный шар (других шаров урны не содержат). Выбирается 
наугад одна урна. В нее добавляется один белый шар и после перемешивания один из шаров извлекается. Извлеченный шар оказался белым. 
Определить апостериорную вероятность того, что выбранной оказалась урна, которая первоначально содержала белый шар. 
\end{problem}


\begin{problem}
В первой урне содержится $a$ белых и $b$ черных шаров (и только они), во второй --- $c$ белых и $d$ черных шаров 
(и только они). Из выбранной наугад урны извлекается один шар, который обратно не возвращается. Извлеченный шар оказался белым. 
Найти вероятность того, что и второй шар, извлеченный из той же урны, окажется белым. 
\end{problem}


\begin{problem}
Известно, что $96\%$ выпускаемой продукции соответствует стандарту. Упрощенная схема контроля признает годным с вероятностью 
$0.98$ каждый стандартный экземпляр аппаратуры и с вероятностью $0.05$ --- каждый нестандартной экземпляр аппаратуры. Найти вероятность, 
что изделие, прошедшее контроль, соответствует стандарту. 
\end{problem}


\begin{problem}
Пусть отличник правильно решает задачу с вероятностью 0.9, а двоечник с вероятностью 0.1. Сколько задач нужно дать на зачете и сколько требовать решить, чтоб отличник не сдал зачет с вероятностью не большей 0.001, а двоечник сдал зачет с вероятностью не большей 0.1?
\end{problem}

\begin{problem}
В $m+1$ урне содержится по $m$ шаров, причем урна с номером $n$ содержит $n$ белых и $m-n$ черных шаров $(n = 0,1,\ldots,m)$. 
Случайным образом выбирается урна и из нее $k$ раз с возвращением извлекаются шары. Найти 
\begin{enumerate}
\item[а)] вероятность, что следующим также будет извлечен белый шар, при условии, что все $k$ шаров оказались белыми, 
\item[б)] ее предел при $m\to\infty$. 
\end{enumerate}
\end{problem}

\begin{ordre}
Применить формулу полной вероятности в следующем виде. 
$$
{\mathbb P}(B\, |\, A)=\sum\limits_{n=1} {\mathbb P}(B\, |\, H_n A){\mathbb P}(H_n\, |\, A)
$$
\end{ordre}

\begin{problem}
Известно, что $90\%$ выпускаемой продукции соответствует стандарту. Упрощенная схема контроля признает 
годным с вероятностью $0,88$ каждый стандартный экземпляр аппаратуры и с вероятностью $0,05$ --- 
каждый нестандартной экземпляр аппаратуры. Найдите вероятность, что изделие, прошедшее контроль, соответствует стандарту. 
\end{problem}


\begin{problem}
До проведения схемы испытаний Бернулли разыгрывается с.в. $p$, имеющая равномерное распределение на отрезке $[0.1, 0.9]$ 
(результаты розыгрыша нам неизвестны). После того как эта с.в. была разыграна, начинают проводиться опыты по схеме Бернулли 
(независимо $n=1000$ раз подкидывается монетка) с вероятность успеха (выпадения <<орла>>) в каждом опыте равной $p$ 
(после того как с.в. $p$ была разыграна, она уже приняла какое-то значения из отрезка $[0.1, 0.9]$ и рассматривается в серии опытов 
Бернулли уже как число, причем не меняющееся от опыта к опыту). В результате опыта было посчитано значение числа успехов $r=777$. 
Определите апостериорное распределение с.в. $p$, т.е. найдите условную плотность распределения $p(x|r=777)$. Оцените, как изменится 
ответ, если точное значение числа успехов нам неизвестно. Известно только, что $r\in[750, 790]$. Т.е. посчитайте условную 
плотность вероятности $p(x|r\in[750, 790])$. 
\end{problem}

\begin{ordre}

Условная функция распределения 
$$
F(x)={\mathbb P}(p<x|S_n=r)=\frac{{\mathbb P}(p<x, S_n=r)}{{\mathbb P}(S_n=r)}
$$

Условная плотность равна 
$$
p(x|r=777)=F'(x) 
$$

Исходя из ц.п.т.  $S_n=np+\xi\sqrt{np(1-p)}$, где $\xi\in {\mathcal N}(0,1)$

\end{ordre}


\begin{problem}
\begin{enumerate}
\item[1)] Имеется монетка (несимметричная). Несимметричность монетки заключается в том, что либо орел выпадает в два раза чаще решки; 
либо наоборот (априорно (до проведения опытов) оба варианта считаются равновероятными). Монетку бросили $10$ раз. Орел выпал $7$ раз. 
Определите апостериорную вероятность того, что орел выпадает в два раза чаще решки (апостериорная вероятность считается с учетом 
проведенных опытов (иначе говоря, это просто условная вероятность)). 

\item[2)] Определите апостериорную вероятность того, что орел выпадает не менее чем в два раза чаще решки. Если несимметричность 
монетки заключается в том, что либо орел выпадает не менее чем в два раза чаще решки; либо наоборот (априорно оба варианта считаются 
равновероятными). 
\end{enumerate}
\end{problem}