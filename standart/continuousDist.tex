
\subsection{Примеры непрерывных распределений}

\begin{problem}
Допустим, что вероятность столкновения молекулы с другими молекулами в промежутке времени $[t,t + \Delta t)$ 
равна $p = \lambda\Delta t+{\overline o}(\Delta t)$ и не зависит от времени, прошедшего после предыдущего столкновения $(\lambda = \const)$. 
Найти распределение времени свободного пробега молекулы и вероятность того, что это время превысит заданную величину $t^*$. 
\end{problem}

\begin{ordre}
Разобьем интервал $\Delta=[0,t)$ на $n$ отрезков равной длины.
Пусть $A_i$ --- событие, означающее, что на отрезке $\Delta_i$  молекула претерпит столкновение с другими молекулами. Можно представить вероятность отсутствия столкновения в виде произведения вероятностей событий $\overline{A_i}$.
\end{ordre}

\begin{problem}
Случайная величина $X$ имеет функцию распределения $F_X(u)$ и функцию плотности распределения $f_X(u)$. Найти функции распределения и плотности распределения (если последние существуют) для случайных величин: а) $Y = aX+b$; б) $Z =e^X$; в) $V = X^2$ (a и b – неслучайные величины). Конкретизировать решения при $X \in N(0,1)$.
\end{problem}

\begin{problem}
Пусть случайные величины $X$ и $Y$ связаны соотношением $Y = \phi(X)$ ($\phi^{-1}(X)$ – непрерывная функция). Найти распределение случайной величины $Y$, если случайная величина $X$ равномерно распределена на отрезке $[0,1]$.
\end{problem}

\begin{comment}
\begin{problem}[распределение Коши]
Радиоактивный источник испускает 
частицы в случайном направлении (при этом все направления равновероятны). 
Источник находится на расстоянии $d$ от фотопластины, которая представляет 
собой бесконечную вертикальную плоскость.

\begin{enumerate}
\item[\textbf{А)}] При условии, что частица попадает в плоскость, покажите, что 
горизонтальная координата точки попадания (если начало координат выбирается 
в точке, ближайшей к источнику) имеет плотность распределения:
\[
p\left( x \right)=\frac{d}{\pi \left( {d^2+x^2} \right)}.
\]
Это распределение известно как \textit{распределение Коши}.

\item[\textbf{Б)}] Можно ли вычислить среднее (математическое ожидание) этого 
распределения?
\end{enumerate}
\end{problem}
\end{comment}
