%\subsection{Независимость и условные вероятности}

\begin{problem}
Приведите пример вероятностного пространства и трёх событий на этом пространстве, которые попарно независимы, но зависимы в совокупности. Достаточно рассмотреть вероятностное пространство, порожденное бросанием шестигранного кубика. \textbf{б) }Предложите обобщение этой задачи, в котором любые \textit{n} из (\textit{n}+1) событий независимы в совокупности, а эти (\textit{n}+1) -- зависимы в совокупности.
\end{problem}



\begin{problem}
Показать, что из независимости событий $A$ и $B$ следует независимость событий $A$ и $\overline B$, $\overline A$ и $B$, 
$\overline A$ и $\overline B$. 
\end{problem}


\begin{problem}
Показать, что из равенства ${\mathbb P}(A\, |\, B)={\mathbb P}(A\, |\, \overline B)$ для ненулевых событий $A$ и $B$ следует 
равенство ${\mathbb P}(AB)={\mathbb P}(A){\mathbb P}(B)$, т.е. их независимость. 
\end{problem}


\begin{problem}
Подбрасываются три игральные кости. События $A$, $B$ и $C$ означают выпадение одинакового числа очков (соответственно) на первой и 
второй, на второй и третьей, на первой и третьей костях. Являются ли эти события независимыми 
\begin{enumerate}
\item[а)] попарно, 
\item[б)] в совокупности? 
\end{enumerate}
\end{problem}


\begin{problem}[парадокс транзитивности]
Будем говорить, что случайная величина $X$ больше по вероятности случайной величины $Y$, если ${\mathbb P}(X>Y)>{\mathbb P}(X\le Y)$. 
Пусть известно, что для случайных величин $X$, $Y$, $Z$, $W$ выполнена следующая цепочка равенств: 
$$
{\mathbb P}(X>Y)={\mathbb P}(Y>Z)={\mathbb P}(Z>W)=\alpha>\frac{1}{2} . 
$$
Верно ли, что $X$ больше по вероятности $W$ и почему? 
\end{problem}



\begin{problem}
Доказать справедливость равенства 
${\mathbb P}(A\bigtriangleup B)={\mathbb P}(A)+{\mathbb P}(B)-2{\mathbb P}(A\, B)$. 
\end{problem}


\begin{problem}
Пусть $A$, $B$, $C$ --- заданные события. Доказать справедливость неравенств 
\begin{enumerate}
\item[а)] ${\mathbb P}(AB)+{\mathbb P}(AC)+{\mathbb P}(BC)\geqslant {\mathbb P}(A)+{\mathbb P}(B)+{\mathbb P}(C)-1$; 
\item[б)] ${\mathbb P}(AB)+{\mathbb P}(AC)-{\mathbb P}(BC)\leqslant {\mathbb P}(A)$; 
\item[в)] ${\mathbb P}(A\bigtriangleup B)\leqslant {\mathbb P}(A\bigtriangleup C)+{\mathbb P}(C\bigtriangleup B)$ . 
\end{enumerate}
\end{problem}


\begin{problem}
Юноша собирается сыграть три теннисных матча со своими родителями, и он должен победить два раза подряд. 
Порядок матчей может быть следующим: отец--мать--отец, мать--отец--мать. Юноше нужно решить, какой порядок для него предпочтительней, 
учитывая, что отец играет лучше матери.

\end{problem}


\begin{problem}
Имеются две урны. В одной из них находится один белый шар, в другой --- один черный шар (других шаров урны не содержат). Выбирается 
наугад одна урна. В нее добавляется один белый шар и после перемешивания один из шаров извлекается. Извлеченный шар оказался белым. 
Определить апостериорную вероятность того, что выбранной оказалась урна, которая первоначально содержала белый шар. 
\end{problem}


\begin{problem}
В первой урне содержится $a$ белых и $b$ черных шаров (и только они), во второй --- $c$ белых и $d$ черных шаров 
(и только они). Из выбранной наугад урны извлекается один шар, который обратно не возвращается. Извлеченный шар оказался белым. 
Найти вероятность того, что и второй шар, извлеченный из той же урны, окажется белым. 
\end{problem}


\begin{problem}
Известно, что $96\%$ выпускаемой продукции соответствует стандарту. Упрощенная схема контроля признает годным с вероятностью 
$0.98$ каждый стандартный экземпляр аппаратуры и с вероятностью $0.05$ --- каждый нестандартной экземпляр аппаратуры. Найти вероятность, 
что изделие, прошедшее контроль, соответствует стандарту. 
\end{problem}


\begin{problem}
Пусть отличник правильно решает задачу с вероятностью 0.9, а двоечник с вероятностью 0.1. Сколько задач нужно дать на зачете и сколько требовать решить, чтоб отличник не сдал зачет с вероятностью не большей 0.001, а двоечник сдал зачет с вероятностью не большей 0.1?
\end{problem}

\begin{problem}
В $m+1$ урне содержится по $m$ шаров, причем урна с номером $n$ содержит $n$ белых и $m-n$ черных шаров $(n = 0,1,\ldots,m)$. 
Случайным образом выбирается урна и из нее $k$ раз с возвращением извлекаются шары. Найти 
\begin{enumerate}
\item[а)] вероятность, что следующим также будет извлечен белый шар, при условии, что все $k$ шаров оказались белыми, 
\item[б)] ее предел при $m\to\infty$. 
\end{enumerate}
\end{problem}

\begin{ordre}
Применить формулу полной вероятности в следующем виде. 
$$
{\mathbb P}(B\, |\, A)=\sum\limits_{n=1} {\mathbb P}(B\, |\, H_n A){\mathbb P}(H_n\, |\, A)
$$
\end{ordre}


\begin{problem}
До проведения схемы испытаний Бернулли разыгрывается с.в. $p$, имеющая равномерное распределение на отрезке $[0.1, 0.9]$ 
(результаты розыгрыша нам неизвестны). После того как эта с.в. была разыграна, начинают проводиться опыты по схеме Бернулли 
(независимо $n=1000$ раз подкидывается монетка) с вероятность успеха (выпадения <<орла>>) в каждом опыте равной $p$ 
(после того как с.в. $p$ была разыграна, она уже приняла какое-то значения из отрезка $[0.1, 0.9]$ и рассматривается в серии опытов 
Бернулли уже как число, причем не меняющееся от опыта к опыту). В результате опыта было посчитано значение числа успехов $r=777$. 
Определите апостериорное распределение с.в. $p$, т.е. найдите условную плотность распределения $p(x|r=777)$. Оцените, как изменится 
ответ, если точное значение числа успехов нам неизвестно. Известно только, что $r\in[750, 790]$. Т.е. посчитайте условную 
плотность вероятности $p(x|r\in[750, 790])$. 
\end{problem}

\begin{ordre}

Условная функция распределения 
$$
F(x)={\mathbb P}(p<x|S_n=r)=\frac{{\mathbb P}(p<x, S_n=r)}{{\mathbb P}(S_n=r)}
$$

Условная плотность равна 
$$
p(x|r=777)=F'(x) 
$$

Исходя из ц.п.т.  $S_n=np+\xi\sqrt{np(1-p)}$, где $\xi\in {\mathcal N}(0,1)$

\end{ordre}

\begin{comment}
\begin{problem}
\begin{enumerate}
\item[1)] Имеется монетка (несимметричная). Несимметричность монетки заключается в том, что либо орел выпадает в два раза чаще решки; 
либо наоборот (априорно (до проведения опытов) оба варианта считаются равновероятными). Монетку бросили $10$ раз. Орел выпал $7$ раз. 
Определите апостериорную вероятность того, что орел выпадает в два раза чаще решки (апостериорная вероятность считается с учетом 
проведенных опытов (иначе говоря, это просто условная вероятность)). 

\item[2)] Определите апостериорную вероятность того, что орел выпадает не менее чем в два раза чаще решки. Если несимметричность 
монетки заключается в том, что либо орел выпадает не менее чем в два раза чаще решки; либо наоборот (априорно оба варианта считаются 
равновероятными). 
\end{enumerate}
\end{problem}
\end{comment}

\begin{problem}
Пусть $Х$ и $Y$ – независимые случайные величины,  имеющие распределение Пуассона. Доказать, что случайная величина $Z = X + Y$ имеет распределение Пуассона. Выявить вид условного распределения случайной величины $Х$ при фиксированном значении случайной величины $Z$.
\end{problem}
