\subsection{Независимость и условные вероятности}

\begin{problem}
Приведите пример вероятностного пространства и трёх событий на этом пространстве, которые попарно независимы, но зависимы в совокупности. Достаточно рассмотреть вероятностное пространство, порожденное бросанием шестигранного кубика. \textbf{б) }Предложите обобщение этой задачи, в котором любые \textit{n} из (\textit{n}+1) событий независимы в совокупности, а эти (\textit{n}+1) -- зависимы в совокупности.
\end{problem}



\begin{problem}
Показать, что из независимости событий $A$ и $B$ следует независимость событий $A$ и $\overline B$, $\overline A$ и $B$, 
$\overline A$ и $\overline B$. 
\end{problem}


\begin{problem}
Показать, что из равенства ${\mathbb P}(A\, |\, B)={\mathbb P}(A\, |\, \overline B)$ для ненулевых событий $A$ и $B$ следует 
равенство ${\mathbb P}(AB)={\mathbb P}(A){\mathbb P}(B)$, т.е. их независимость. 
\end{problem}


\begin{problem}
Подбрасываются три игральные кости. События $A$, $B$ и $C$ означают выпадение одинакового числа очков (соответственно) на первой и 
второй, на второй и третьей, на первой и третьей костях. Являются ли эти события независимыми 
\begin{enumerate}
\item[а)] попарно, 
\item[б)] в совокупности? 
\end{enumerate}
\end{problem}


\begin{problem}[парадокс транзитивности]
Будем говорить, что случайная величина $X$ больше по вероятности случайной величины $Y$, если ${\mathbb P}(X>Y)>{\mathbb P}(X\le Y)$. 
Пусть известно, что для случайных величин $X$, $Y$, $Z$, $W$ выполнена следующая цепочка равенств: 
$$
{\mathbb P}(X>Y)={\mathbb P}(Y>Z)={\mathbb P}(Z>W)=\alpha>\frac{1}{2} . 
$$
Верно ли, что $X$ больше по вероятности $W$ и почему? 
\end{problem}



\begin{problem}
Доказать справедливость равенства 
${\mathbb P}(A\bigtriangleup B)={\mathbb P}(A)+{\mathbb P}(B)-2{\mathbb P}(A\, B)$. 
\end{problem}


\begin{problem}
Пусть $A$, $B$, $C$ --- заданные события. Доказать справедливость неравенств 
\begin{enumerate}
\item[а)] ${\mathbb P}(AB)+{\mathbb P}(AC)+{\mathbb P}(BC)\geqslant {\mathbb P}(A)+{\mathbb P}(B)+{\mathbb P}(C)-1$; 
\item[б)] ${\mathbb P}(AB)+{\mathbb P}(AC)-{\mathbb P}(BC)\leqslant {\mathbb P}(A)$; 
\item[в)] ${\mathbb P}(A\bigtriangleup B)\leqslant {\mathbb P}(A\bigtriangleup C)+{\mathbb P}(C\bigtriangleup B)$ . 
\end{enumerate}
\end{problem}


\begin{problem}
Юноша собирается сыграть три теннисных матча со своими родителями, и он должен победить два раза подряд. 
Порядок матчей может быть следующим: отец--мать--отец, мать--отец--мать. Юноше нужно решить, какой порядок для него предпочтительней, 
учитывая, что отец играет лучше матери.

\end{problem}
