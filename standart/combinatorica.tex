\subsection{Дискретное равномерное распределение}


\begin{problem}
Некто имеет $N$ ключей, из которых только один от его двери. Какова вероятность, что, используя ключи в случайном порядке, 
он откроет дверь 
\begin{enumerate}
\item[а)] первым ключом, 
\item[б)] последним ключом? 
\end{enumerate}
Найти вероятность, что потребуется не менее $k$ попыток, чтобы открыть дверь, если ключи, которые не подошли, 
\begin{enumerate}
\item[в)] откладываются, 
\item[г)] не откладываются. 
\end{enumerate}

\begin{ordre}
В пункте в) воспользоваться заменой вероятности бинарной величины на математическое ожидание.
\end{ordre}

\end{problem}


\begin{problem}
Ребенок играет с десятью буквами разрезной азбуки: А, А, А, Е, И, К, М, М, Т, Т. 
Какова вероятность того, что при случайном расположении букв в ряд он получит слово <<МАТЕМАТИКА>>? 
\end{problem}


\begin{problem}
Из урны, содержащей $a$ белых, $b$ черных и $c$ красных шаров (и только их), последовательно извлекаются три шара. Найти 
вероятность следующих событий: 
\begin{enumerate}
\item[а)] все три шара разного цвета; 
\item[б)] шары извлечены в последовательности белый, черный, красный; 
\item[в)] шары извлечены в обратной последовательности. 
\end{enumerate}
\end{problem}



\begin{problem}
Из урны, содержащей $a$ белых и $b$ черных шаров, извлекается наугад один шар и откладывается в сторону. Какова вероятность 
того, что извлеченный наугад второй шар окажется белым, если: 
\begin{enumerate}
\item[а)] первый извлеченный шар белый; 
\item[б)] цвет  первого извлеченного шара остается неизвестным? 
\end{enumerate}
\end{problem}




\begin{problem}
Партия продукции состоит из десяти изделий, среди которых два изделия дефектные. Какова вероятность того, что из пяти отобранных 
наугад и проверенных изделий: 
\begin{enumerate}
\item[а)] ровно одно изделие дефектное; 
\item[б)] ровно два изделия дефектные; 
\item[в)] хотя бы одно изделие дефектное? 
\end{enumerate} 

\begin{ordre}
В пункте в) удобнее искать вероятность противоположного события.
\end{ordre}

\end{problem}

\begin{problem}
Найти вероятность того, что из $50$ студентов, присутствующих на лекции, хотя бы двое имеют одну и ту же дату рождения. 

\begin{remark}
Для получения приближенного решения можно воспользоваться свойством линейности математического ожидания следующих событий $X_{ij}$ - у  i-го и  j-го человека дни рождения совпадают.  
\end{remark}

\end{problem}


\begin{problem}
В урне находится $m$ шаров, из которых $m_1$ белых и $m_2$ черных $(m_1 + m_2 = m)$. 
Производится $n$ извлечений одного шара с возвращением его (после определения его цвета) обратно в урну. Найти вероятность того, 
что ровно $r$ раз из $n$ будет извлечен белый шар. 
\end{problem}


\begin{problem}
Найти вероятность того, что при размещении $n$ различных шаров по $N$ ящикам заданный ящик будет содержать ровно 
$k$: $0\leqslant k\leqslant n$, шаров (все различимые размещения равновероятны). 
\end{problem}


\begin{problem}
В урне находится $m$ шаров, из которых $m_1$ --- первого цвета, $m_2$ --- второго цвета, $\ldots$, $m_s$ --- $s$-го цвета 
$(m_1+m_2+\ldots +m_s=m)$. 
Производится $n$ извлечений одного шара с возвращением его (после определения его цвета) обратно в урну. Найти вероятность того, 
что $r_1$ раз будет извлечен шар первого цвета, $r_2$ раз --- шар второго цвета, $\ldots$, $r_s$ раз --- шар $s$-го цвета 
$(r_1+r_2+\ldots +r_s=n)$. 
\end{problem}


\begin{problem}
В гардеробе все шляпы $N$ посетителей оказались случайным образом перепутанными. Шляпы не имеют внешних отличительных признаков. 
Какова вероятность того, что хотя бы один посетитель получит свою шляпу (рассмотреть случаи $N=4$, $N=10000$)? Найти математическое ожидание таких посетителей. 

\begin{ordre}
Воспользоваться формулой включений-исключений.
\end{ordre}

\end{problem}


\begin{problem}
Несколько раз бросается игральная кость. Какое событие более вероятно: 
\begin{enumerate}
\item[а)] сумма выпавших очков четна; 
\item[б)] сумма выпавших очков нечетна? 
\end{enumerate}
\end{problem}


\begin{problem}
Для уменьшения общего количества игр $2n$ команд спортсменов разбиваются на две подгруппы. Определить вероятности того, что 
две наиболее сильные команды окажутся: 
\begin{enumerate}
\item[а)] в одной подгруппе; 
\item[б)] в разных подгруппах. 
\end{enumerate}
\end{problem}


\begin{problem}
В урне находятся черные и белые шары, которые наугад по одному без возвращения извлекаются из урны до тех пор, пока урна не опустеет. 
Какое событие более вероятно: 
\begin{enumerate}
\item[а)] первый извлеченный шар белый; 
\item[б)] последний извлеченный шар белый? 
\end{enumerate}
\end{problem}



\begin{problem}
В урне находятся $a$ белых и $b$ черных шаров, Шары наугад по одному извлекаются из урны без возвращения. Найти вероятность того, 
что $k$-й вынутый шар оказался белым. 
\end{problem}


\begin{problem}
$30$ шаров размещаются по $8$ ящикам так, что для каждого шара одинаково возможно попадание в любой ящик. Найти вероятность 
размещения, при котором будет $3$ пустых ящика, $2$ ящика --- с тремя, $2$ ящика --- с шестью и $1$ ящик --- с двенадцатью шарами. 
\end{problem}


\begin{problem}
$N$ частиц случайно и независимо друг от друга размещаются в $k$ ячейках так, что каждая из них попадает 
в $i$-ую ячейку с вероятностью $p_i$ $(i=1,\ldots,k, \sum\limits_{i=1}^{k} p_i=1)$. Найти вероятность того, что число частиц в ячейках 
примет заданные значения $n_1$, $\ldots$, $n_i$, $\ldots$, $n_k$ (полиномиальное распределение). 
\end{problem}

\begin{problem}
Из $n$ лотерейных билетов $k$ --- выигрышные $(n\geqslant 2k)$. Какова вероятность, что среди $k$ купленных билетов по крайней мере 
один будет выигрышным? 
\end{problem}

\begin{problem}
Из совокупности всех подмножеств множества $\{1,2,\ldots,N\}$ по схеме выбора с возвращением выбираются множества $A$ и $B$. 
Найти вероятность, что $A$ и $B$ не пересекаются. 
\end{problem}


\begin{problem}
Сколькими способами можно так выбрать 
четыре различных 7-сочетания с повторениями из множества букв русского 
алфавита, чтобы в каждом из этих сочетаний присутствовала хотя бы 
одна буква из набора $ \{{\rm ж},{\rm а},{\rm б}\} $? 
\end{problem}

\begin{problem}
Пусть ${\cal M} = \{M_1,\,\ldots,\,M_s\}$~--- случайная совокупность, состоящая из различных $k$-сочетаний элементов множества~$\{1,\,\ldots,\,n\} $. Здесь $n,\,k,\,s$~фиксированы, а выбор совокупности осуществляется в соответствии с классическим определением вероятности (извлекаем её наудачу из множества всех совокупностей, которые состоят из $s$~различных $k$-элементных подмножеств множества~$\{1,\,\ldots,\,n\} $). Назовём $S\subset\{1,\,\ldots,\,n\}$ системой общих представителей (с.\,о.\,п.) для~${\cal M}$, если~$S\cap M_i\neq\emptyset$ для всех~$i=1,\,\dots,\,s$. Найдите математическое ожидание числа $l$-элементных с.\,о.\,п. для случайной совокупности~${\cal M}$.
\end{problem}

\begin{problem}
В каждую из $n$ пронумерованных ячеек в случайном порядке помещается один из $n$ так же пронумерованных шаров. 
Найти вероятность того, что ни в одной из ячеек номер шара не совпадет с номером ячейки. 

\begin{ordre}
Применить формулу включений-исключений.
\end{ordre}
\end{problem}

\begin{problem} 
Найдите вероятность $q_n$ того, что случайная $(0,1)$-матрица размера $n\times n$ является невырожденной над полем 
$GF_2=\{ 0,1\}$. Доказать, что существует $\lim\limits_{n\to\infty} q_n=q>0$. 
\end{problem}

\begin{ordre}
$k$-я строка невырожденной матрицы $A$ равно -- любая строка из $\{ 0,1\}^n$, 
кроме любой линейной комбинации первых $k-1$ строк. 
\end{ordre}

\begin{problem}
В самолете $n$ мест. Есть $n$ пассажиров, выстроившихся друг за другом в очередь. Во главе очереди -- <<заяц>>. У всех, 
кроме <<зайца>>, есть билет, на котором указан номер посадочного места. Так как <<заяц>> входит первым, он случайным образом занимает 
некоторое место. Каждый следующий пассажир, входящий в салон самолета, действует по такому принципу: если его место свободно, то 
садится на него, если занято, то занимает с равной вероятностью любое свободное. Найдите вероятность того, что последний пассажир 
сядет на свое место. 
\end{problem}

\begin{problem}
(Задача игрока де Мере) 
Что более вероятно: при одновременном бросании четырех игральных костей получить хотя бы одну единицу или при $24$ бросаниях 
по две игральные кости одновременно получить хотя бы один раз две единицы? Найти вероятности указанных событий. 
\end{problem}

\begin{problem}[``сто заключенных'']
В коридоре находятся 100 человек, у каждого свой номер (от 1 до 100). Их по одному заводят в комнату, в которой 
находится комод со 100 выдвижными ящиками. В ящики случайным образом 
разложены карточки с номерами (от 1 до 100). Каждому разрешается заглянуть в 
не более чем 50 ящиков. Цель каждого -- определить, в каком ящике находится 
его номер. Общаться и передавать друг другу информацию запрещается. 
Предложите стратегию, которая с вероятностью не меньшей $0.3$ (в 
предположении, что все $100!$ способов распределения карточек по ящикам 
равновероятны) приведет к выигрышу всей команды. Команда выигрывает, если 
все 100 участников верно определили ящик с карточкой своего номера.
\end{problem}

\textbf{Стратегия. }Каждый человек первым открывает ящик под его номером, 
вторым -- под номером, который указан на карточке, лежащей в ящике, открытом 
перед этим и т.д. Среднее число циклов длины r в случайной 
перестановке -- есть 1/r (покажите, используя, например, задачу ``про 
предельные меры''). Тогда среднее число циклов длины большей n/2 
есть $\sum\limits_{i=n \mathord{\left/ {\vphantom {n 2}} \right. 
\kern-\nulldelimiterspace} 2}^n {\frac{1}{i}} $. Это и есть вероятность 
существования цикла длины большей n/2. Поэтому вероятность успеха команды -- 
есть $1-\sum\limits_{i=51}^{100} {\frac{1}{i}} \approx 0,31$. Если 
же просто произвольно открывать ящики, то вероятность успеха будет 
$2^{-100}\approx 8\cdot 10^{-31}$. В случае, когда карточки 
разложены не случайным образом, то следует сделать случайной нумерацию 
ящиков, и далее следовать старой стратегии.

\begin{problem}
Найдите число различных ожерелий, которые можно 
составить из 14 изумрудов, алмазов и сапфиров (количество камней того или 
иного вида может быть любым). Одинаковыми считаются только те ожерелья, которые можно совместить  поворотом вокруг оси, перпендикулярной плоскости ожерелий.

\begin{ordre}
Воспользоваться формулой обращения Мебиуса.
\end{ordre}
   
\end{problem} 