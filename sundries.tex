\section{Разные задачи}

\begin{problem}
Напомним, что сингулярными мерами называются меры, функции распределения $F(x)$ которых непрерывны, но точки их роста ($x$ -- точка роста $F(x)$, если для любого $\varepsilon >0$ выполняется: $F(x+\varepsilon )-F(x+\varepsilon )>0$) образуют множество нулевой меры Лебега. Покажите, что мера, соответствующая функции Кантора, сингулярна по отношению к мере Лебега.
\end{problem}

