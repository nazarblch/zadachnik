\section{Разные задачи}

\begin{problem}
(Распределения канторовского типа).

Рассмотрим в $\sum _{k=1}^{\infty }2^{-k} X_{k}  $, где $X_{k} $ - взаимно независимые с.в., имеющие распределение Бернулли с параметром ${1\mathord{\left/ {\vphantom {1 2}} \right. \kern-\nulldelimiterspace} 2} $, сумму слагаемых с четными номерами, или, что с точностью до множителя 3 (в дальнейшем потребуется для удобства) есть $Y=3\sum _{s=1}^{\infty }4^{-s} X_{s}  $. Покажите, что функция распределения $F(x)=P\left\{Y\le x\right\}$ является сингулярной (когда не оговаривается относительно какой меры, подразумевается, что относительно меры Лебега).


\begin{ordre}
Можно рассматривать $Y$ как выигрыш игрока, который получает $3\cdot 4^{-k} $, когда $k$-е бросание симметричной монеты дает в результате решку. Ясно, что полный выигрыш лежит между 0 и $3\left(4^{-1} +4^{-2} +\ldots \right)=1$. Если первое подбрасывание монеты привело к решке, то полный выигрыш $\ge {3\mathord{\left/ {\vphantom {3 4}} \right. \kern-\nulldelimiterspace} 4} $, тогда как в противоположном случае $Y\le 3\left(4^{-2} +4^{-3} +\ldots \right)=4^{-1} $. То есть неравенство ${1\mathord{\left/ {\vphantom {1 4}} \right. \kern-\nulldelimiterspace} 4} <Y<{3\mathord{\left/ {\vphantom {3 4}} \right. \kern-\nulldelimiterspace} 4} $ не может быть осуществлено ни при каких обстоятельствах, значит $F(x)={1\mathord{\left/ {\vphantom {1 2}} \right. \kern-\nulldelimiterspace} 2} $ в интервале $x\in \left({1\mathord{\left/ {\vphantom {1 4}} \right. \kern-\nulldelimiterspace} 4} ,{3\mathord{\left/ {\vphantom {3 4}} \right. \kern-\nulldelimiterspace} 4} \right)$. Чтобы определить, как ведет себя функция распределения на интервале $x\in \left(0,{1\mathord{\left/ {\vphantom {1 4}} \right. \kern-\nulldelimiterspace} 4} \right)$, покажите, что на этом интервале график отличается только преобразованием подобия $F(x)={1\mathord{\left/ {\vphantom {1 2}} \right. \kern-\nulldelimiterspace} 2} F(4x)$.
\end{ordre}

\begin{remark}
Пример, когда свертка двух сингулярных распределений имеет непрерывную плотность: с.в. $X=\sum _{k=1}^{\infty }2^{-k} X_{k}  $ имеет равномерное распределение на $\left(0;1\right)$. Обозначим сумму членов ряда с четными и нечетными номерами через $U$ и $V$ соответственно. Ясно, что $U$ и $2V$ имеют одинаковое распределение и их распределение относится к канторовскому типу.
\end{remark}

\end{problem}


\begin{problem}
Напомним, что сингулярными мерами называются меры, функции распределения $F(x)$ которых непрерывны, но точки их роста ($x$ -- точка роста $F(x)$, если для любого $\varepsilon >0$ выполняется: $F(x+\varepsilon )-F(x+\varepsilon )>0$) образуют множество нулевой меры Лебега. Покажите, что мера, соответствующая функции Кантора, сингулярна по отношению к мере Лебега.

\end{problem}


\begin{problem} (повторяется)
Доказать неравенство Чернова:

\[P\left\{\sum _{i=1}^{n}X_{i} >(p+t)n \right\}\le \exp \left\{nH\left(\left\{p+t,q-t\right\},\left\{p,q\right\}\right)\right\},\quad 0\le t\le q,\] 
где $X_{i} $, $i=1,...,n$ - независимые случайные величины, имеющие распределение Бернулли:
\[X_{i} =\left\{\begin{array}{cc} {1,} & {p,} \\ {0,} & {q=1-p;} \end{array}\right. \] 
$H\left(P,Q\right)=\sum _{j=1}^{m}-P_{j} \log \frac{P_{j} }{Q_{j} }  $ - относительное энтропийное «расстояние» между двумя (дискретными) распределениями вероятностей $P=\left(P_{1} ,\ldots ,P_{m} \right)$ и $Q=\left(Q_{1} ,\ldots ,Q_{m} \right)$ на пространстве элементарных исходов размера $m$.

\begin{ordre}
 
Воспользуйтесь техникой получения неравенства Азумы, а именно 

\noindent \textbf{а)} перейдите к положительной случайной величине $e^{\lambda \sum _{i=1}^{n}X_{i}  } $($\lambda >0$ - некий параметр). В теории вероятностей функция $\varphi _{Y} (\lambda )=E\left[e^{\lambda Y} \right]$называется производящей функции моментов случайной величины $Y$, так как при разложении в ряд Тейлора $\varphi _{Y} (\lambda )=E\left[e^{\lambda Y} \right]=E\left[\sum _{i=0}^{\infty }\frac{\lambda ^{i} }{i!} Y^{i}  \right]=\sum _{i=0}^{\infty }\frac{\lambda ^{i} }{i!} E\left[Y^{i} \right] $, где $E\left[Y^{i} \right]$ - \textit{i}-ый момент случайной величины $Y$.

\noindent \textbf{б)} применив неравенство Маркова , получите $P\left\{\sum _{i=1}^{n}X_{i} > \; m\right\}=P\left\{e^{\lambda \sum _{i=1}^{n}X_{i}  } >e^{m} \right\}\le \left(\frac{pe^{\lambda } +q}{e^{\lambda (p+t)} } \right)^{n} $ с параметризацией $m=(p+t)n$.

\noindent \textbf{в)} подобрав оптимальное значение ($\frac{pe^{\lambda } +q}{e^{\lambda (p+t)} } \to \mathop{\min }\limits_{\lambda } $), получите 

\[
P\left\{\sum _{i=1}^{n}X_{i} >(p+t)n \right\}\le \left(\left(\frac{p}{p+t} \right)^{p+t} \left(\frac{q}{q-t} \right)^{q-t} \right)^{n} 
\]
\[
= \exp \left\{n\left[-(p+t)\ln \frac{p+t}{p} -(q-t)\ln \frac{q-t}{q} \right]\right\}.
\] 


\end{ordre}

\begin{remark}
C точки зрения математической статистики 

\[
H\left(\left\{p+t,q-t\right\},\left\{p,q\right\}\right)=-(p+t)\ln \frac{p+t}{p} -(q-t)\ln \frac{q-t}{q} 
\]

-энтропийное расстояние между апостериорным (полученным после эксперимента) распределением $\left\{p+t,q-t\right\}$и априорным $\left\{p,q\right\}$. Таким образом, «граница» Чернова уменьшается экспоненциально с показателем равным n-кратному энтропийному расстоянию между апостериорным и априорным распределением вероятностей.\textbf{}

\noindent Более удобная запись «границы» Чернова:
\[P\left\{\sum _{i=1}^{n}X_{i} >(p+t)n \right\}\le \exp \left\{-\frac{2t^{2} }{n} \right\},\] 
так как для функции $f(t)=(p+t)\ln \frac{p+t}{p} +(q-t)\ln \frac{q-t}{q} $ имеем $f(0)=f'(0)=0$, $f''(t)=\frac{1}{(p+t)(q-t)} \ge 4$ для любого $0\le t\le q$, значит разложение Тейлора 
\[f(t)=f(0)+f'(0)t+f''(\xi )\frac{t^{2} }{2!} \ge 2t^{2} ,\quad 0<\xi <t.\]

\end{remark}

\end{problem}

\begin{problem}

Доказать оценку для больших уклонений в Бернуллиевской модели:
\[X_{i} =\left\{\begin{array}{cc} {1,} & {p,} \\ {0,} & {q=1-p;} \end{array}\right. \] 
$P\left\{\sum _{i=1}^{n}X_{i} =\left[\alpha n\right] \right\}\sim \frac{1}{\sqrt{2\pi \alpha (1-\alpha )} } \exp \left\{nH(\{ p,1-p\} ,\{ \alpha ,1-\alpha \} )\right\}$, $0<\alpha <1$ и $\alpha \ne p,$

\noindent где $H\left(P,Q\right)=\sum _{j=1}^{m}-P_{j} \log \frac{P_{j} }{Q_{j} }  $ - относительное энтропийное «расстояние» между двумя (дискретными) распределениями вероятностей $P=\left(P_{1} ,\ldots ,P_{m} \right)$ и $Q=\left(Q_{1} ,\ldots ,Q_{m} \right)$ на пространстве элементарных исходов размера $m$.

\begin{ordre}

Напомним, что производящей функцией (п.ф.) некоторой целочисленной неотрицательной случайной величины (с.в.) $\xi $ называется $\varphi _{\xi } (z)=Ez^{\xi } =\sum _{j=0}^{\infty }P\left\{\xi =j\right\} z^{j} $. Зная распределение с.в. можно получить ее п.ф., и наоборот, задание п.ф. однозначно определяет ее распределение: (для этого можно воспользоваться теорией функций комплексного переменного -- теоремой Коши -- для определения коэффициентов в разложении аналитической функции в ряд Лорана) $P\left\{\xi =k\right\}=\left[z^{k} \right]\varphi _{\xi } (z)=\frac{1}{2\pi i} \oint _{\left|z\right|=1}\frac{\varphi _{\xi } (z)}{z^{k+1} } dz $. Также следует напомнить, что п.ф. для суммы независимых с.в. справедливо $\varphi _{\sum _{i=1}^{n}X_{i}  } (z)=Ez^{\sum _{i=1}^{n}X_{i}  } =E\left[\prod _{i=1}^{n}z^{X_{i} }  \right]=\prod _{i=1}^{n}Ez^{X_{i} }  =\prod _{i=1}^{n}\varphi _{X_{i} } (z) $. Таким образом, в данной задаче $P\left\{\sum _{i=1}^{n}X_{i} =\left[\alpha n\right] \right\}=\left[z^{\left[\alpha n\right]} \right]\varphi _{\sum _{i=1}^{n}X_{i}  } (z)=\frac{1}{2\pi i} \oint _{\left|z\right|=\rho }\frac{\left(pz+q\right)^{n} }{z^{\left[\alpha n\right]+1} } dz $,\textbf{ }последний интеграл можно оценить с помощью методе перевала: перейдя к полярным координатам $z=\rho e^{i\theta } $ и сделав замену, получим:\textbf{}
\[\frac{1}{2\pi i} \oint _{\left|z\right|=1}\frac{\left(pz+q\right)^{n} }{z^{\left[\alpha n\right]+1} } dz =\frac{1}{2\pi } \int _{-\pi }^{\pi }e^{nf\left(\rho e^{i\theta } \right)} d\theta  ,\] 
где $f(z)=\ln \left(pz+q\right)-\left(\alpha -\frac{\varepsilon _{n} }{n} \right)\ln z$ ($\left[\alpha n\right]=\alpha n-\varepsilon _{n} $). Далее, нужно найти седловую точку и ограничиться интегрированием по её малой окрестности, разлагая подынтегральное выражение в ряд Тейлора.

\end{ordre}

\begin{problem}[перколяция] 
В квадратном пруду (со стороной равной 1) 
выросли (случайным образом) $N\gg 1$ цветков лотоса, имеющих форму круга 
радиуса $r>0$. Назовем $r_N $ - \textit{радиусом перколяции}, если с вероятностью не меньшей 0.99 не 
любящий воду жук сможет переползти по цветкам лотоса с северного берега на 
южный, не замочившись.

Покажите, что $r_N \sim C \mathord{\left/ {\vphantom {C {\sqrt N }}} \right. 
\kern-\nulldelimiterspace} {\sqrt N }$. Оцените $C$.
\end{problem}

\end{problem}