\section{Разные задачи}

\begin{problem}
Напомним, что сингулярными мерами называются меры, функции распределения $F(x)$ которых непрерывны, но точки их роста ($x$ -- точка роста $F(x)$, если для любого $\varepsilon >0$ выполняется: $F(x+\varepsilon )-F(x+\varepsilon )>0$) образуют множество нулевой меры Лебега. Покажите, что мера, соответствующая функции Кантора, сингулярна по отношению к мере Лебега.
\end{problem}


\begin{problem}
(Распределения канторовского типа).

Рассмотрим в $\sum _{k=1}^{\infty }2^{-k} X_{k}  $, где $X_{k} $ - взаимно независимые с.в., имеющие распределение Бернулли с параметром ${1\mathord{\left/ {\vphantom {1 2}} \right. \kern-\nulldelimiterspace} 2} $, сумму слагаемых с четными номерами, или, что с точностью до множителя 3 (в дальнейшем потребуется для удобства) есть $Y=3\sum _{s=1}^{\infty }4^{-s} X_{s}  $. Покажите, что функция распределения $F(x)=P\left\{Y\le x\right\}$ является сингулярной (когда не оговаривается относительно какой меры, подразумевается, что относительно меры Лебега).


\begin{ordre}
Можно рассматривать $Y$ как выигрыш игрока, который получает $3\cdot 4^{-k} $, когда $k$-е бросание симметричной монеты дает в результате решку. Ясно, что полный выигрыш лежит между 0 и $3\left(4^{-1} +4^{-2} +\ldots \right)=1$. Если первое подбрасывание монеты привело к решке, то полный выигрыш $\ge {3\mathord{\left/ {\vphantom {3 4}} \right. \kern-\nulldelimiterspace} 4} $, тогда как в противоположном случае $Y\le 3\left(4^{-2} +4^{-3} +\ldots \right)=4^{-1} $. То есть неравенство ${1\mathord{\left/ {\vphantom {1 4}} \right. \kern-\nulldelimiterspace} 4} <Y<{3\mathord{\left/ {\vphantom {3 4}} \right. \kern-\nulldelimiterspace} 4} $ не может быть осуществлено ни при каких обстоятельствах, значит $F(x)={1\mathord{\left/ {\vphantom {1 2}} \right. \kern-\nulldelimiterspace} 2} $ в интервале $x\in \left({1\mathord{\left/ {\vphantom {1 4}} \right. \kern-\nulldelimiterspace} 4} ,{3\mathord{\left/ {\vphantom {3 4}} \right. \kern-\nulldelimiterspace} 4} \right)$. Чтобы определить, как ведет себя функция распределения на интервале $x\in \left(0,{1\mathord{\left/ {\vphantom {1 4}} \right. \kern-\nulldelimiterspace} 4} \right)$, покажите, что на этом интервале график отличается только преобразованием подобия $F(x)={1\mathord{\left/ {\vphantom {1 2}} \right. \kern-\nulldelimiterspace} 2} F(4x)$.
\end{ordre}

\begin{remark}
Пример, когда свертка двух сингулярных распределений имеет непрерывную плотность: с.в. $X=\sum _{k=1}^{\infty }2^{-k} X_{k}  $ имеет равномерное распределение на $\left(0;1\right)$. Обозначим сумму членов ряда с четными и нечетными номерами через $U$ и $V$ соответственно. Ясно, что $U$ и $2V$ имеют одинаковое распределение и их распределение относится к канторовскому типу.
\end{remark}

\end{problem}