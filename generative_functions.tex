\section{Производящие и характеристические функции}

\begin{problem}
Пусть $g(t)$ является характеристической функцией абсолютно непрерывной случайной величины. Являются ли $Re[g]$ и $Im[g]$ характеристическими функциями?
\end{problem}

\begin{problem}
Может ли функция $\varphi(t)=\begin{cases}1,\quad t\in[-T,T]\\
0,\quad t\notin[-T,T] \end{cases}$ --- 
быть характеристической функцией некоторой с.в.? Изменится ли ответ, если <<чуть-чуть>> размазать (сгладить) разрывы функции 
$\varphi(t)$ в точках $t=\pm T$? 
\end{problem}

\begin{ordre}
Характеристическая функция обладает следующими свойствами:
\begin{fixme}
Iterate the properties 
\end{fixme}
\end{ordre}



\begin{problem}
Пусть $\xi\in\Po(\lambda)$, $\lambda\gg 1$. Покажите, что 
$$
\sqrt{\xi}\approx \sqrt{\lambda}+N(0,\left. 1\right/4) . 
$$
\end{problem}


\begin{problem}
В течение года фирма осуществляет $K\in \Po(\lambda)$ сделок ($K$ -- с.в., имеющая распределение Пуассона с параметром  $\lambda=100000$ 
[сделок]). Каждая сделка приносит фирме прибыль $V_n\in R[a,b]$ ($V_n$ -- с.в., имеющая равномерное распределение на отрезке 
$[a,b]=[-50\$,100\$]$, $n$ -- номер сделки). Считая, что $K$, $V_1$, $V_2$, $\ldots$ --- независимые в совокупности с.в., оцените 
\begin{equation}
\label{ProbRatio}
\left. {\mathbb P}\Bigl(\sum\limits_{n=1}^{K} V_n\leqslant 0\Bigr)\right/{\mathbb P}\Bigl(\sum\limits_{n=1}^{K} V_n>0\Bigr) . 
\end{equation}
\end{problem}


\begin{problem}
Частица находится в начальный момент в вершине треугольника (Паскаля). Затем частица начинает двигаться (с вероятностью $p$ вправо и 
с вероятностью $q = 1-p$ --- влево). Определите вероятность того, что частица за $n$ шагов 
двигалась вправо ровно $k$ раз. 
\end{problem}
