\section{Производящие и характеристические функции}

\begin{problem}[Счастливые билеты]
Трамвайные билеты имеют шестизначные номера. Билет называют счастливым, если 
сумма его первых трех цифр равна сумме трех последних. Вычислите приближенно 
вероятность того, что Вам достанется счастливый билет (предполагается, что 
все билеты равновероятны).

\end{problem}

\begin{ordre}
Покажите, что число счастливых билетов совпадает с числом билетов, 
сумма цифр у которых равна 27. Запишите производящую функцию для 
последовательности $\left\{ {a_k } \right\}_{k=0}^{36} $, где $a_k $ - число 
билетов с суммой цифр равной $k$. Для нахождения коэффициента ПФ $a_{27} $ 
воспользуйтесь теоремой Коши (ТФКП). Для оценки полученного интеграла примените 
метод стационарной фазы.
\end{ordre}

\begin{problem}[Числа Каталана]
Пусть в очереди за покупкой товара ценой в 
50руб стоят $2n$ человек. Пусть у $n$ человек есть только купюра в 100руб., 
а у $n$ человек - купюра в 50руб., при этом изначально касса магазина пуста. 
Найдите вероятность события, что никто из людей в очереди не будет ждать 
свою сдачу.
\end{problem}

\begin{ordre}
Напишите ПФ последовательности чисел Каталана (числом Каталана 
$c_n $ называется число различных правильных скобочных структур из $n$ пар 
скобок).
\end{ordre}


\begin{problem}[Задача о беспорядках]
Группа из $n$ фанатов выигрывающей футбольной команды на радостях 
подбрасывают в воздух свои шляпы. Шляпы возвращаются в случайном порядке -- 
по одной к каждому болельщику. Какова вероятность того, что никому из 
фанатов не вернется своя шляпа?
\end{problem}

\begin{problem}[Задача с марками]
Пусть Вы хотите собрать коллекцию из $N$ 
марок, для этого Вы каждый день покупаете конверт со случайной маркой 
(равновероятное распределение марок в конверте).

А) Введем дискретную с.в. $X$, равную номеру впервые купленной Вами 
повторной марки. Найдите математическое ожидание с.в. $X$.

Б) Покажите, что распределение случайной величины $X$ имеет асимптотически 
распределение Релея при $n=t\sqrt N $ ($N\gg 1)$:
\[
P\left\{ {X>t\sqrt N } \right\}\sim e^{-\frac{t^2}{2}},
\quad
P\left\{ {X=t\sqrt N } \right\}\sim \frac{1}{\sqrt N }te^{-\frac{t^2}{2}}.
\]
В) Определить математическое ожидание номера купленной Вами марки, которая 
станет недостающей в собранной Вами коллекции марок. 

\end{problem}

\begin{ordre}

А) Покажите, что $P\{X>n\}=\frac{n!}{N^n}\left[ {z^n} \right]\left( {1+z} 
\right)^N=n!\left[ {z^n} \right]\left( {1+\frac{z}{N}} \right)^N$, а значит 
$MX=\sum\limits_{n=0}^\infty {n!\left[ {z^n} \right]} \left( {1+\frac{z}{N}} 
\right)^N=\int\limits_0^\infty {e^{-t}\left( {1+\frac{t}{N}} \right)^Ndt} $. 

Далее для приближенного вычисления интеграла $\int\limits_0^\infty 
{e^{-t}\left( {1+\frac{t}{N}} \right)^Ndt} =N\int\limits_0^\infty 
{e^{N\left( {\ln (1+u)-u} \right)}du} $ воспользуемся \textit{методом Лапласа}:
\[
\begin{array}{l}
 \int\limits_0^\infty {f(u)e^{NS(u)}du} \approx f(u_0 )e^{NS(u_0 
)}\int\limits_{u_0 -\delta }^{u_0 +\delta } {e^{\frac{NS''(u_0 )(u-u_0 
)^2}{2}}} du\quad \mathop {\mathop \to \limits^{x=\sqrt {-NS''(u_0 )} (u-u_0 
)} }\limits_{{\begin{array}{*{20}c}
 {N\to \infty } \hfill \\
 \Downarrow \hfill \\
 {\sqrt {-NS''(u_0 )} \delta \to \infty } \hfill \\
\end{array} }} \quad \frac{f(u_0 )e^{NS(u_0 )}}{\sqrt {-NS''(u_0 )} 
}\int\limits_{-\infty }^{+\infty } {e^{-\frac{x^2}{2}}} dx= \\ 
 =\frac{f(u_0 )e^{NS(u_0 )}}{\sqrt {-NS''(u_0 )} }\sqrt {2\pi } , \\ 
 \end{array}
\]
где $u_0 $ - единственная точка максимума вещественнозначной функции 
$S(u)$на полубесконечном интервале $(0;+\infty )$. Основная идея 
асимптотического представления интеграла Лапласа заключается в представлении 
функции $S(u)$ в окрестности точки максимума $u_0 $ в ряд Тейлора.

Б) Согласно теореме Коши (из курса ТФКП). 
\[
P(X>n)=n!\left[ {z^n} \right]\left( {1+\frac{z}{N}} 
\right)^N=\frac{1}{n!}\frac{1}{2i\pi }\oint\limits_{\vert z\vert =\rho } 
{\left( {1+\frac{z}{N}} \right)^N\frac{dz}{z^{n+1}}} 
\]
Перейдя к полярным координатам, получите, что:

$\frac{1}{2i\pi }\oint\limits_{\vert z\vert =\rho } {\left( {1+\frac{z}{N}} 
\right)^N\frac{dz}{z^{n+1}}} =\frac{1}{2\pi }\int\limits_{-\pi }^\pi {\left( 
{1+\frac{\rho e^{i\theta }}{N}} \right)^N\frac{d\theta }{\rho ^ne^{in\theta 
}}} =\frac{1}{2\pi }\int\limits_{-\pi }^\pi {e^{f(\rho e^{i\theta })}d\theta 
} ,$ где $f(z)=N\ln \left( {1+\frac{z}{N}} \right)-n\ln z,\quad z=\rho 
e^{i\theta };$

Воспользуйтесь \textit{методом ``перевала''}: разложите в ряд Тейлора функцию $f$ в окрестности седловой 
точки: $f(\rho e^{i\theta })=f(\rho )-\frac{1}{2}\beta (\rho )\theta 
^2+O(\theta ^3)$ для $\left| \theta \right|<\delta $ (разложение Тейлора), 
где $\beta (\rho )=\rho ^2\left. {\left[ {\left( {\frac{d}{dz}} 
\right)^2f(z)} \right]} \right|_{z=\rho } $.

Замените интеграл по всей окружности $\vert z\vert =\rho $ на интеграл по ее 
части: $\int\limits_{-\delta }^\delta {e^{f(\rho e^{i\theta })}d\theta } 
\approx e^{f(\rho )}\int\limits_{-\delta }^\delta {e^{-\frac{1}{2}\beta 
(\rho )\theta ^2}d\theta } =\frac{e^{f(\rho )}}{\sqrt {\beta (\delta )} 
}\int\limits_{-\delta \sqrt {\beta (\rho )} }^{\delta \sqrt {\beta (\rho )} 
} {e^{-\frac{1}{2}u^2}du} \mathop \to \limits_{\beta (\rho )\to \infty } 
\frac{e^{f(\rho )}}{\sqrt {\beta (\delta )} }\int\limits_{-\infty }^\infty 
{e^{-\frac{1}{2}u^2}du} =\sqrt {2\pi } \frac{e^{f(\rho )}}{\sqrt {\beta 
(\delta )} }.$В) Пусть $Y$ - дискретная с.в., равная номеру купленной Вами 
марки, которая станет недостающей в собранной Вами коллекции марок.

Тогда $P\{Y\le n\}=\frac{n!}{N^n}\left[ {z^n} \right](e^z-1)^N=n!\left[ 
{z^n} \right](e^{\frac{z}{N}}-1)^N$, а следовательно 
$MY=\sum\limits_{n=0}^\infty {n!\left[ {z^n} \right]\left( 
{e^z-(e^{\frac{z}{N}}-1)^N} \right)} =\int\limits_0^\infty {\left[ 
{1-(1-e^{-\frac{t}{N}})^N} \right]} dt$. Сделав замену переменных 
$y=1-e^{-\frac{t}{N}}$, получите, что $MY=N\left( {1+\frac{1}{2}+\cdots 
+\frac{1}{N}} \right)$.

\end{ordre}

\begin{problem}[Урновая схема]

Рассмотрим случайное размещение $n$ различных шаров по $m$ различным урнам. 
Пусть случайные величины MIN, MAX - размер наименее или наиболее заполненной 
урны в случайном размещении. Получите функции распределения этих случайных 
величин, а именно $P\{MAX\le l\}=\frac{n!\left[ {z^n} \right]e_l 
(z)^m}{m^n}=n!\left[ {z^n} \right]e_l \left( {\frac{z}{m}} \right)^m$,

$P\{MIN>l\}=n!\left[ {z^n} \right]\left( {e^{\frac{z}{m}}-e_l \left( 
{\frac{z}{m}} \right)} \right)^m,$ где $e_l (z)=1+z+\frac{z^2}{2!}+\cdots 
+\frac{z^l}{l!}$.

\end{problem}

\begin{remark}

 Пусть $m,n\to \infty $ и $\raise0.7ex\hbox{$n$} \!\mathord{\left/ 
{\vphantom {n 
m}}\right.\kern-\nulldelimiterspace}\!\lower0.7ex\hbox{$m$}=\alpha \in 
\left( {0;\infty } \right)$. Можно показать, что п.н. MIN=0, 

MAX$\sim \frac{\log n}{\log \log n}$. (см. Колчин, Севастьянов, Чистяков 
``Случайные размещения'')

\end{remark}

\begin{problem}[Задача о циклах в случайной перестановке]

\textbf{\textit{А)}} Найдите математическое ожидание числа циклов длины r в 
случайной перестановке длины n

\textbf{\textit{Б)}} Найдите математическое ожидание числа циклов в 
случайной перестановке длины n 

\textbf{\textit{В)}} Решите задачу о заключенных (раздел Стандартные задачи № ?)

\end{problem}

\begin{problem}
Найдите сумму $\sum\limits_{k=0}^n k^3 C_n^k \left(\frac{1}{17}\right)^k$.
\end{problem}

\begin{problem}
Пусть $f(n)$~такова, что $ f(n+2) - 2f(n+1) - 4f(n) = 0 $ и $ f(0) = 1 $, $ f(1) = -3 $. Найдите сумму $ \sum\limits_{k=0}^{\infty} f(k) (0.23)^k $.
\end{problem}

\begin{problem}
Пусть $\xi$ --- случайная величина, равномерно распределенная на множестве всех пар векторов 
$(x,y)\in \{ 0,1\}^n\otimes \{ 0,1\}^n$, равная $\xi(x,y)=(x,y)=\sum\limits_{k=1}^{n} x_k y_k$. Найдите: 
$$
P_k={\mathbb P}(\xi=k), \; {\mathbb E}\,\xi, \; \Var \xi . 
$$
\end{problem}

\begin{problem}
Сколько раз нужно подбросить монету, чтобы решка выпала два раза подряд?
\end{problem}

\begin{ordre}
Вероятностное пространство состоит из всех 
последовательностей букв Р и О, оканчивающихся на РР, но не содержащих двух 
Р подряд ранее. Любой элемент тогда имеет вид:
$$
(O+PO)*PP \text{ --- регулярное выражение }.
$$

Заменяя ${P}\to pz;\quad {O}\to qz$, получим производящую функцию.
\end{ordre}


\begin{problem}[игра У. Пенни, 1969]
Алиса и Билл играют в игру: они 
бросают монету до тех пор, пока не встретится РРО или РОО. Если первой 
появится последовательность РРО, выигрывает Алиса, если РОО -- Билл. Будет 
ли игра честной?
\end{problem}

\begin{ordre}

Справедливы следующие равенства:
\[
\mbox{1+N(Р+О)=N+А+В},
\]
\[
\mbox{NРРО=А},
\]
\[
\mbox{NРОО=В+АО},
\]

где $A$ - конфигурации, выигрышные для Алисы, $B$ -  конфигурации, выигрышные для Билла, $N$ - конфигурации последовательностей , для которых ни один из игроков не выиграл.

\end{ordre}


\begin{problem}
Теперь трое игроков: Алиса, Билл и Компьютер. Играют пока 
не выпадет одна из следующих последовательностей: А=РРОР, В=РОРР, С=ОРРР. 
Каковы шансы каждого выиграть?
\end{problem}


\begin{problem}
Рассматривается игра У. Пенни (см. предыдущие две задачи). Показать, что 
последовательность $a_1 a_2 \ldots a_l $ всегда уступает последовательности 
$\bar {a}_2 a_1 a_2 \ldots a_{l-1} $, $l\ge 3$.
\end{problem}


\begin{problem}
В вершине пятиугольника $ABCDE$ 
находится яблоко, а на расстоянии двух ребер, в вершине $C$, находится 
червяк. Каждый день червяк переползает в одну из двух соседних вершин с 
равной вероятностью. Так, через один день червяк окажется в вершине $B$ или 
$D$ с вероятностью $1 \mathord{\left/ {\vphantom {1 2}} \right. 
\kern-\nulldelimiterspace} 2$. По прошествии двух дней червяк может снова 
оказаться в $C$, поскольку он не запоминает своих предыдущих положений. 
Достигнув вершины $A$, червячок останавливается пообедать.

\begin{enumerate}
\item Чему равны математическое ожидание и дисперсия числа дней прошедших до обеда?
\item Какую оценку дает неравенство Чебышёва для вероятности $p$ того, что это число дней будет 100 или больше?
\item Что позволяют сказать о величине $p$ оценки из задачи ``об оценки хвостов''.
\end{enumerate}
\end{problem}


\begin{problem}
Пять человек стоят в вершинах пятиугольника $ABCDE$ и 
бросают друг другу диски Фрисби. У них имеется два диска, которые в 
начальный момент находятся в соседних вершинах. В очередной момент времени 
диски бросают либо налево, либо направо с одинаковой вероятностью. Процесс 
продолжается до тех пор, пока обе тарелки не окажутся в одной вершине.

\begin{enumerate}
\item Найдите математическое ожидание и дисперсию числа пар бросков.
\item Найдите ``замкнутое'' выражение через числа Фибоначчи для вероятности того, что игра продлиться более 100 шагов.
\end{enumerate}
\end{problem}


\begin{problem}
Обобщите предыдущую задачу на случай $m$-угольника и найдите 
математическое ожидание и дисперсию числа пар бросков до столкновения 
дисков. Докажите, что если $m$ нечетно, то ПФСВ для числа бросаний 
представимо в следующем виде:\footnote{ Воспользуйтесь подстановкой $z=1 
\mathord{\left/ {\vphantom {1 {\cos ^2\theta }}} \right. 
\kern-\nulldelimiterspace} {\cos ^2\theta }$.}
\[
G_m (z)=\prod\limits_{k=1}^{(m-1)/2} {\frac{p_k z}{1-q_k z}} ,
\]
где
\[
p_k =\sin ^2\frac{(2k-1)\pi }{2m},
\quad
q_k =\cos ^2\frac{(2k-1)\pi }{2m}.
\]
\end{problem}



\begin{problem}
Пусть $\xi$ --- с.в., равномерно распределенная на множестве бинарных матриц (т.е. матриц с элементами типа $0$ и $1$) порядка 
$m\times n$ и равная числу нулевых столбцов матрицы. Доказать, что 
\begin{enumerate}
\item[а)] $P_k(m,n)={\mathbb P}(\xi=k)=C_n^k\cdot \left.\bigl( 2^m -1\bigr)^{n-k}\right/ 2^{m\cdot n}$; 

\item[б)] ${\mathbb E}\xi=\left. n\right/2^m$; 
\item[в)] если $2^m-1=\alpha\cdot n$, где $\alpha$ не зависит от $n$, то 
$$
\lim\limits_{n\to\infty} P_k(m,n)=e^{-\lambda}\, \frac{\lambda^k}{k!} ,\; \text{ где } \lambda=\alpha^{-1} .
$$
\end{enumerate}
\end{problem}

\begin{ordre}

Рассмотрим следующие случайные величины: 
$$
\xi_i=\begin{cases}
1, & \text{ $i$-й столбец нулевой}, \\
0, & \text{ иначе }.
\end{cases} 
$$
Тогда $\xi_i\in\Be$, $\xi=\xi_1+\ldots +\xi_n$. 

Имеет место следующее мультипликативное свойство:
$$
\psi_{\xi}(z)=\bigl[\psi_{\xi_i}(z)\bigr]^n
$$

\end{ordre}


\begin{problem}
Может ли функция $\varphi(t)=\begin{cases}1,\quad t\in[-T,T]\\
0,\quad t\notin[-T,T] \end{cases}$ --- 
быть характеристической функцией некоторой с.в.? Изменится ли ответ, если <<чуть-чуть>> размазать (сгладить) разрывы функции 
$\varphi(t)$ в точках $t=\pm T$? 
\end{problem}

\begin{ordre}
Характеристическая функция обладает следующими свойствами:
\begin{fixme}
Iterate the properties 
\end{fixme}
\end{ordre}


\begin{problem}
Будет ли функция~$\cos(t^2)$ характеристической для какой-нибудь случайной величины?
\end{problem}

\begin{problem}
Пусть $\varphi_{\xi}$~--- характеристическая функция абсолютно непрерывной случайной величины~$\xi$ с плотностью~$p_{\xi}$. Рассмотрим $f_1 = \Re \varphi_{\xi}$ и~$f_2 = \Im \varphi_{\xi} $. Существуют ли случайные величины $\eta_1,\,\eta_2$, для которых $f_1,\,f_2$~являются их характеристическими функциями? 
\end{problem}

\begin{problem}
В течение года фирма осуществляет $K\in \Po(\lambda)$ сделок ($K$ -- с.в., имеющая распределение Пуассона с параметром  $\lambda=100000$ 
[сделок]). Каждая сделка приносит фирме прибыль $V_n\in R[a,b]$ ($V_n$ -- с.в., имеющая равномерное распределение на отрезке 
$[a,b]=[-50\$,100\$]$, $n$ -- номер сделки). Считая, что $K$, $V_1$, $V_2$, $\ldots$ --- независимые в совокупности с.в., оцените 
\begin{equation}
\label{ProbRatio}
\left. {\mathbb P}\Bigl(\sum\limits_{n=1}^{K} V_n\leqslant 0\Bigr)\right/{\mathbb P}\Bigl(\sum\limits_{n=1}^{K} V_n>0\Bigr) . 
\end{equation}
\end{problem}


\begin{problem}
Частица находится в начальный момент в вершине треугольника (Паскаля). Затем частица начинает двигаться (с вероятностью $p$ вправо и 
с вероятностью $q = 1-p$ --- влево). Определите вероятность того, что частица за $n$ шагов 
двигалась вправо ровно $k$ раз. 
\end{problem}

\begin{problem}
Реализуем $m$~раз схему из $n$~испытаний Бернулли с вероятностью успеха~$p$. Считаем, что все реализации схем взаимно независимы. На выходе получим $m$~случайных векторов ${\bf x}_1,\,\dots,\,{\bf x}_m$ с координатами $0$~и~$1$. Некоторые из этих векторов могут совпадать. Скажем, что векторы ${\bf x}_i,\,{\bf x}_j,\,{\bf x}_k$ образуют прямой угол с вершиной в~${\bf x}_k$, если~$({\bf x}_i-{\bf x}_k,\,{\bf x}_j - {\bf x}_k) = 0$ (помимо обычных прямых углов, под это определение попадают и <<вырожденные>>, т.\,е. образованные совпадающими векторами). Найдите математическое ожидание числа прямых углов во множестве~$\{{\bf x}_1,\,\dots,\,{\bf x}_m\}$.
\end{problem}

\begin{problem}[парадокс Стефана Банаха](усл из конкр мат)
В двух спичечных коробках имеется по $n$ спичек. На каждом шаге наугад выбирается коробок, и из него удаляется (используется) 
одна спичка. Найти вероятность того, что в момент, когда один из коробков опустеет, в другом останется $k$ спичек. 
\end{problem}

\begin{ordre}

Возможны следующие два варианта рассуждений, приводящих к различным ответам.

Событие, удовлетворяющее условию задачи --- выбран пустой коробок, а в другом коробке имеется $k$ спичек. 


Пусть $P(w,z)=\sum\limits_{m,n} p_{m,n} w^m z^n$, 
$P_k(w,z)=\sum\limits_{m,n} p_{k,m,n} w^m z^n$,
где $p_{m,n}$ есть вероятность, начав с $m$ спичек в одной коробке и $n$ --- в другой, получить обе пустые коробки, 
когда впервые выбирается пустая коробка, 
$p_{k,m,n}$ есть вероятность, начав с $m$ спичек в одной коробке и $n$ --- в другой, в момент выбрасывания  первой пустой коробки иметь вторую коробку с $k$ спичками. 

Получив рекуррентные соотношения для данных функций, покажите, что искомая вероятность равна 
$$
p_{k,n,n}=\frac{C_{2n-k}^n}{2^{2n-k}} . 
$$


Событие, удовлетворяющее условию задачи --- из выбранной коробки взяли последнюю спичку, а в другом коробке имеется $k$ спичек (второй вариант). 

Для нахождения вероятности этого события рассмотрите процесс изъятия спичек из коробок как последовательность нулей и единиц (например, нули соответствуют спичкам первой коробки, 
единицы -- второй коробки) длины $2n$, с числом нулей и единиц равным $n$. 
\end{ordre}

\begin{problem}
Пусть при любом $\lambda >0$ с.в. $\xi _{\lambda } $ имеет распределение Пуассона. Докажите, что $\frac{\xi _{\lambda } -\lambda }{\sqrt{\lambda } } $ слабо сходится (по распределению) к стандартному нормальному распределению при $\lambda \to \infty $.

\begin{ordre}
 Используйте аппарат характеристических функций и теорему о непрерывном соответствии (о том, что слабая сходимость эквивалентна равномерной сходимости соответствующих характеристических функций).
 \end{ordre}
\end{problem}

\begin{problem} 
Найдите вероятность того, что пара случайно выбранных из $E^n=\{ 0,1\}^n$ векторов является ортогональной 
\begin{enumerate}
\item[а)] над полем $F_2=\{ 0,1\}$; 

\item[б)] над полем действительных чисел. 
\end{enumerate}
\end{problem}

\begin{ordre}
a) Пусть 
$$
\xi_{x,y}=\begin{cases}
1, &\text{ если } (x,y)=0,\\
0, &\text{ если } (x,y)=1.
\end{cases}
$$
Для искомой вероятности тогда имеем 
$$
P=\frac{1}{2^n\cdot 2^n}\sum\limits_{x,y}\xi_{x,y}
$$
\end{ordre}



\begin{problem}[об оценке хвостов]
Пусть $\Psi (z)=Mz^X$ --- производящая 
функция случайной величины (ПФСВ) $X$. Докажите, что
\[
{\rm P}(X\le r)\le x^{-r}\Psi (x),\mbox{ для }0<x\le 1;
\]
\[
{\rm P}(X\ge r)\le x^{-r}\Psi (x),\mbox{ для }x\ge 1.
\]
\end{problem}


\begin{problem}[загадочный случайный суп]
Студент, решивший отобедать в 
столовой, может обнаружить в своей тарелке с супом случайное число $N$ 
инородных частиц $\Lambda $ со средним $\mu $ и конечной дисперсией. С 
вероятностью $p$ выбранная частица является мухой, иначе это таракан; типы 
разных частиц независимы. Пусть $F$ -- количество мух и $S$ -- количество тараканов.

\begin{enumerate}
\item[\textbf{А)}] Покажите, что производящая функция случайной величины $F$ равна
\[
\psi _F (s)=\psi _N (ps+1-p).
\]

\item[\textbf{Б)}] Предположим, что случайная величина N имеет пуассоновское 
(poisson) распределение с параметром $\mu $ (записывают $\mbox{N}\in 
\Po\left( \mu \right))$. Покажите, что $F$ имеет пуассоновское распределение с 
параметром $p\mu $, а случайные величины $F$ и $S$ независимы. Покажите, что
\[
\psi _N (s)=\psi _N^2 \Bigl( {\frac{1}{2}(1+s)} \Bigr).
\]
\end{enumerate}
Убедитесь, что случайная величина $N$ имеет пуассоновское распределение.
\end{problem}

\begin{problem} 
Обозначим через $E^n$ -- множество бинарных последовательностей длины n, или множество вершин 
единичного n-мерного куба, а через $E_k^n $ -- k-ый слой куба $E^n$, то есть 
подмножество точек $E^n$, имеющих ровно k единичных координат. Пусть 
$X=\left( {\vec {x},\vec {y}} \right)$ - случайная величина, где $\vec 
{x}\in E_p^n $, $\vec {y}\in E_q^n $ - независимые и равномерно 
распределенные на $E_p^n $ и $E_q^n $ соответственно векторы. Обозначим 
через $a_{p,q} (k)=P\left\{ {X=k} \right\}$. Доказать следующие утверждения:

\begin{enumerate}
\item[\textbf{1)}] $\sum\limits_{k=0}^n {a_{p,q} (k)z^k} =\frac{1}{2\pi i}C_n^p 
\oint\limits_{\left| u \right|=\rho } 
{\frac{(1+zu)^p(1+u)^{n-p}}{u^{q+1}}du} $.

\item[\textbf{2)}] $a_{p,q} (k)=\frac{C_p^k C_{n-p}^{q-k} }{C_n^q }$.

\item[\textbf{3)}] $EX=\frac{pq}{n}$.

\item[\textbf{4)}] $DX=\frac{pq}{n(n-1)}\left( {n+\frac{pq}{n}-(p+q)} \right)$.
\end{enumerate}
\end{problem}

\begin{fixme}
См. Леонтьева Избранные примеры.... 
\end{fixme}








