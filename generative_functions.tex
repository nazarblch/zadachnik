\section{Производящие и характеристические функции}

\begin{problem}
Пусть $g(t)$ является характеристической функцией абсолютно непрерывной случайной величины. Являются ли $Re[g]$ и $Im[g]$ характеристическими функциями?
\end{problem}

\begin{problem}
Может ли функция $\varphi(t)=\begin{cases}1,\quad t\in[-T,T]\\
0,\quad t\notin[-T,T] \end{cases}$ --- 
быть характеристической функцией некоторой с.в.? Изменится ли ответ, если <<чуть-чуть>> размазать (сгладить) разрывы функции 
$\varphi(t)$ в точках $t=\pm T$? 
\end{problem}

\begin{ordre}
Характеристическая функция обладает следующими свойствами:
\begin{fixme}
Iterate the properties 
\end{fixme}
\end{ordre}



\begin{problem}
В течение года фирма осуществляет $K\in \Po(\lambda)$ сделок ($K$ -- с.в., имеющая распределение Пуассона с параметром  $\lambda=100000$ 
[сделок]). Каждая сделка приносит фирме прибыль $V_n\in R[a,b]$ ($V_n$ -- с.в., имеющая равномерное распределение на отрезке 
$[a,b]=[-50\$,100\$]$, $n$ -- номер сделки). Считая, что $K$, $V_1$, $V_2$, $\ldots$ --- независимые в совокупности с.в., оцените 
\begin{equation}
\label{ProbRatio}
\left. {\mathbb P}\Bigl(\sum\limits_{n=1}^{K} V_n\leqslant 0\Bigr)\right/{\mathbb P}\Bigl(\sum\limits_{n=1}^{K} V_n>0\Bigr) . 
\end{equation}
\end{problem}


\begin{problem}
Частица находится в начальный момент в вершине треугольника (Паскаля). Затем частица начинает двигаться (с вероятностью $p$ вправо и 
с вероятностью $q = 1-p$ --- влево). Определите вероятность того, что частица за $n$ шагов 
двигалась вправо ровно $k$ раз. 
\end{problem}


\begin{problem}[парадокс Стефана Банаха](усл из конкр мат)
В двух спичечных коробках имеется по $n$ спичек. На каждом шаге наугад выбирается коробок, и из него удаляется (используется) 
одна спичка. Найти вероятность того, что в момент, когда один из коробков опустеет, в другом останется $k$ спичек. 
\end{problem}

\begin{ordre}

Возможны следующие два варианта рассуждений, приводящих к различным ответам.

Событие, удовлетворяющее условию задачи --- выбран пустой коробок, а в другом коробке имеется $k$ спичек. 


Пусть $P(w,z)=\sum\limits_{m,n} p_{m,n} w^m z^n$, 
$P_k(w,z)=\sum\limits_{m,n} p_{k,m,n} w^m z^n$,
где $p_{m,n}$ есть вероятность, начав с $m$ спичек в одной коробке и $n$ --- в другой, получить обе пустые коробки, 
когда впервые выбирается пустая коробка, 
$p_{k,m,n}$ есть вероятность, начав с $m$ спичек в одной коробке и $n$ --- в другой, в момент выбрасывания  первой пустой коробки иметь вторую коробку с $k$ спичками. 

Получив рекуррентные соотношения для данных функций, покажите, что искомая вероятность равна 
$$
p_{k,n,n}=\frac{C_{2n-k}^n}{2^{2n-k}} . 
$$


Событие, удовлетворяющее условию задачи --- из выбранной коробки взяли последнюю спичку, а в другом коробке имеется $k$ спичек (второй вариант). 

Для нахождения вероятности этого события рассмотрите процесс изъятия спичек из коробок как последовательность нулей и единиц (например, нули соответствуют спичкам первой коробки, 
единицы -- второй коробки) длины $2n$, с числом нулей и единиц равным $n$. 
\end{ordre}

\begin{problem}
Пусть при любом $\lambda >0$ с.в. $\xi _{\lambda } $ имеет распределение Пуассона. Докажите, что $\frac{\xi _{\lambda } -\lambda }{\sqrt{\lambda } } $ слабо сходится (по распределению) к стандартному нормальному распределению при $\lambda \to \infty $.

\begin{ordre}
 Используйте аппарат характеристических функций и теорему о непрерывном соответствии (о том, что слабая сходимость эквивалентна равномерной сходимости соответствующих характеристических функций).
 \end{ordre}
\end{problem}

\begin{problem} 
Найдите вероятность того, что пара случайно выбранных из $E^n=\{ 0,1\}^n$ векторов является ортогональной 
\begin{enumerate}
\item[а)] над полем $F_2=\{ 0,1\}$; 

\item[б)] над полем действительных чисел. 
\end{enumerate}
\end{problem}

\begin{ordre}
a) Пусть 
$$
\xi_{x,y}=\begin{cases}
1, &\text{ если } (x,y)=0,\\
0, &\text{ если } (x,y)=1.
\end{cases}
$$
Для искомой вероятности тогда имеем 
$$
P=\frac{1}{2^n\cdot 2^n}\sum\limits_{x,y}\xi_{x,y}
$$
\end{ordre}

