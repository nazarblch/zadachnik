\section{Элементы теории информации}

\begin{problem}
Пусть для некоторого пункта (скажем, для г. Долгопрудный) 
вероятность того, что $15$ июня будет дождь, равна $0.4$, а вероятность того, 
что дождя не будет, равна $0.6$. Пусть далее для этого же пункта вероятность 
дождя $15$ октября равна $0.8$, а вероятность отсутствия осадков равна $0.2$. 
Предположим, что определенный метод прогноза погоды $15$ июня оказывается 
верным в $3/5$ всех тех случаев, когда предсказывается дождь 
и в $4/5$ тех случаев, в которых прогнозируется отсутствие осадков. Применительно к 
погоде на $15$ октября этот метод оказывается правильным в $9/10$ тех случаев, 
когда предсказывается дождь, и в половине случаев, когда предсказывается его 
отсутствие. В какой из указанных двух дней прогноз дает нам больше 
информации о реальной погоде?
\end{problem}

\begin{problem}

Пусть казино делает $n$ бросаний, используя распределение вероятностей на бинарных словах длины $n$ $p\left(x\right)$, где $x=\left\{0,1\right\}^{n} $, известное игроку. При этом казино производит выплаты так, как если бы оно использовало распределении $q\left(x\right)$ (то есть выигранная ставка на 0, после выпадения $x$ исходов увеличивается в $\frac{q\left(x\right)}{q\left(x0\right)} $ раз, выигранная ставка на 1, после выпадения $x$ исходов увеличивается в $\frac{q\left(x\right)}{q\left(x1\right)} $ раз, см. семинар). Докажите, что у игрока есть стратегия, логарифм значения капитала которой равен расстоянию Кульбака-Лейблера $\sum _{x\in \left\{0,1\right\}^{n} }p\left(x\right)\log \frac{p\left(x\right)}{q\left(x\right)}  $ между распределениями $p$ и $q$.
\end{problem}


\subsection{Основные определения}
Алфавитом $A = \{a_1, ..., a_{|A|}\}$ будем называть конечное множество мощности A;\\
$n$-буквенную последовтаельность $X_1, ..., X_n$ из алфавита $A$ будем обозначать $x_1^n$ или $\textbf{x}$


\begin{definition} 
\textit{Энтропией} $H(\mathbb{P})$ вероятностного распределения $\mathbb{P} = \{\mathbb{P}(a), a \in A\}$ называется величина:
\begin{center}
$H(\mathbb{P}) = - \sum_{a \in A} \mathbb{P}(a)\log \mathbb{P}(a)$ (логарифм по основанию 2)
\end{center}
\end{definition}

\begin{definition}
\textit{Относительной энтропией} распределений $\mathbb{P}$ и $\mathbb{Q}$ на множестве $A$ (или расстоянием Кульбака-Лейблера между распределениями $\mathbb{P}$ и $\mathbb{Q}$) называется величина
\begin{center}
$D(\mathbb{P}||\mathbb{Q}) = \sum_{a \in A} \mathbb{P}(a) \log \frac{\mathbb{P}(a)}{\mathbb{Q}(a)}$.
\end{center}

\end{definition}


\begin{definition} 
\textit{Эмпирическим распределением} или типом $\hat{\mathbb{P}} = \hat{\mathbb{P}}_{x_1^n}$ слова $x_1^n\in A^n$ называется распределение вида:
\begin{center}
$\hat{\mathbb{P}}(a) = \frac{|\{i : x_i = a\}| }{n}$
\end{center}  
\end{definition}

\begin{definition} 
Будем обозначать $\mathcal{P}_n$ множество эмпирических распределений, построенных по всевозможным цепочкам длины $n$.
\end{definition}

\begin{example}
$A = \{0, 1\}$, тогда
\begin{center}
$\mathcal{P}_n = \{(\mathbb{P}(0), \mathbb{P}(1)) : (\frac{0}{n} , \frac{n}{n}), (\frac{1}{n}, \frac{n-1}{n}), ..., (\frac{n}{n}, \frac{0}{n})\}$
\end{center}
\end{example}

\begin{definition} 
Если $\mathbb{P} \in \mathcal{P}_n$, то множество всех слов длины $n$, таких что:
\begin{center}
$T(\mathbb{P}) = \{x_1^n \in A^n: \mathbb{P}_{x_1^n} = \mathbb{P}\}$
\end{center}
будем называть классом эквивалентности.
\end{definition}

\subsection{Задачи}


\begin{problem}
Доказать, что $|\mathcal{P}_n| \leq (n+1)^{|A|}$.
\textit{Замечание.} Этот несложный результат указывает на то, что $\forall n \in \mathbb{N}$ количество эмпирических распределений, полученных при помощи оценки частотности $\mathcal{P}_n$ полиномиально. 
\end{problem}

\begin{problem}
Пусть $\mathbb{Q}^n(x_1^n) = \prod_{i = 1}^{n} \mathbb{Q}(x_i)$. Показать, что если последовательность независимых одинаково распределенных cлучайных величин $\{X_i\}_{i = 1}^n$, $X_i \sim \mathbb{Q}$, то вероятность появления слова $x_1^n$ в такой модели зависит только от его типа и определяется как
\begin{center}
$\mathbb{Q}^n(x_1^n) = 2^{-nH(\mathbb{P}_{x_1^n}) + nD(\mathbb{P}_{x_1^n}||\mathbb{Q})}$.
\end{center}   
\end{problem}

\begin{problem}
Показать, что $\forall \mathbb{P} \in \mathcal{P}_n$ и для любого произвольного распрделения $\mathbb{Q}$ на множестве $A$, вероятность того, что модель $\mathbb{Q}^n$ сгенерирует слово из множества $T(\mathbb{P})$ удовлетворяет следующему неравенству:
\begin{center}
$\frac{1}{(n+1)^{|A|}} 2^{-nD(\mathbb{P}||\mathbb{Q})} \leq \mathbb{Q}^n(T(\mathbb{P})) \leq 2^{-nD(\mathbb{P}||\mathbb{Q})}$
\end{center}
\emph{Указание}:
\begin{enumerate}
\item $\mathbb{Q}^n(T(\mathbb{P})) = \sum_{x \in T(\mathbb{P})} \mathbb{Q}^n(x)$
\item воспользуйтесь результатом задачи 1.2 и неравенством \\
$\frac{1}{(n+1)^{|A|}} 2^{nH(\mathbb{P})} \leq |T(\mathbb{P})| \leq 2^{nH(\mathbb{P})} $
\end{enumerate}
\textit{Замечание}. Неравенство (*) говорит о том, что $\forall n \in \mathbb{N}$ число слов длины $n$, по котороым можно построить эмпирическое распределение $\mathbb{P}$ ($|T(\mathbb{P})|$) экспонециально.
\end{problem}

\begin{problem}
\textit{Закон больших чисел}
$\forall \mathbb{P} \in \mathcal{P}_n $, $\forall \mathbb{Q}$ на $A$, $\forall \epsilon > 0$ типичным $T_{\mathbb{Q}}^{\epsilon}$ множеством слов для распределения $\mathbb{Q}^n$ будем называть
\begin{center}
$T_{\mathbb{Q}}^{\epsilon} = \{x_1^n: D(\mathbb{P}_{x_1^n}|| \mathbb{Q}) \leq \epsilon \}$.
\end{center}
Пусть $X_1, ..., X_n$ независимые одинаково распределенные с.в. ($X_i \sim \mathbb{Q}$). Показать, что:
\begin{enumerate}
\item 
\begin{center}
$Pr\{D(\mathbb{P}_{x_1^n} || \mathbb{Q}) \geq \epsilon \} \leq 2^{-n(\epsilon - |A|\frac{\log(n+1)}{n})}$
\end{center}
\textit{Указание}:
Пользуясь результатами задач 1.1 и 1.3 оценить вероятность того, что последовательность $x_1^n$ - нетипична: $Pr\{x_1^n \not\in T_{\mathbb{Q}}^{\epsilon}\} = 1 - \mathbb{Q}^n(T_{\mathbb{Q}}^{\epsilon})$;\\
для этого представить $1 - \mathbb{Q}^n(T_{\mathbb{Q}}^{\epsilon}) = \sum_{\mathbb{P}: D(\mathbb{P}||\mathbb{Q}) > \epsilon}{ \mathbb{Q}^n(T(\mathbb{P}))}$ 
\item $D(\mathbb{P}_{x_1^n}||\mathbb{Q})\rightarrow^{a.s.}0$\\
\textit{Указание}: воспользуйтесь первой леммой Борелля-Кантелли.\\
Пусть $\{A_n, n \geq \}$ - последовательность событий в вероятностном пространстве. $A = \cup_{k = 1}^{\infty} \cap_{n = k}^{\infty} A_n$ - событие, говорящее о том, что из серии $\{A_n, n \geq \}$ реализовалось бесконечно много событий.
Если $\sum_{n=1}^{\infty}\mathbb{P}(A_n) < \infty$, то $\mathbb{P}(A) = 0$
\end{enumerate}
\end{problem}

\begin{problem} \textit{Теорема Санова или неравенство больших уклонений.} Сделать указанием.\\
Пусть $X_1, ..., X_n$ независимые одинаково распределенные с.в. ($X_i \sim \mathbb{Q}$). $E$ - множество вероятностных распределений. Доказать, что
\begin{equation}
\mathbb{Q}^n(E) = \mathbb{Q}^n(E\cap \mathcal{P}_n) \leq (n+1)^{|A|}2^{-nD(\mathbb{P}^*||\mathbb{Q})},
\end{equation}
где $\mathbb{Q}^n(E) = \mathbb{Q}^n(E\cap \mathcal{P}_n)$ $\mathbb{P}^* = arg\min_{\mathbb{P}\in E}D(\mathbb{P}||\mathbb{Q})$ - ближайшее к $\mathbb{Q}$ в смысле расстояния Кульбака-Лейблера  распрделение из семейства $E$.\\
Если $E$ - замкнуто, то
\begin{equation}
\frac{1}{n} \log \mathbb{Q}^n(E) \rightarrow D(\mathbb{P}^*|| \mathbb{Q})
\end{equation}
\end{problem}

\begin{problem} \textit{Неравенство больших уклонений}
Рассмотрим последовательность незовисимых одинаково распределенных случайных величин $X_1, ..., X_n$, $X_i~Be(q)$. Доказать, что $-\frac{1}{n}\log \mathbb{P}(\frac{1}{n}\sum X_i \geq p) \rightarrow p\log \frac{p}{q} + (1-p)\log \frac{1-p}{1-q} = D((p, 1-p)||(q, 1-q))$.
\begin{ordre}
\end{ordre}
\end{problem}

\begin{problem} 
\textit{Расстоянием по вариации} между двумя распределениями называется $||\mathbb{P}_1 - \mathbb{P}_2||_1 = \sum_{a \in A} |\mathbb{P}_1(a) - \mathbb{P}_2(a)|$. Доказать, что между ним и расстоянием Кульбака-Лейблера справедливо следующее соотношение:\\
$D(\mathbb{P}_1||\mathbb{P}_2) \geq \frac{1}{2\ln2} ||\mathbb{P}_1 - \mathbb{P}_2||_1^2$
\end{problem}