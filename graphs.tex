\section{Случайные модели графов}

\begin{problem}

На некоторой реке имеется 6 островов, соединенных между собой системой мостов. Во время летнего наводнения часть мостов была разрушена. При этом каждый мост разрушается с вероятностью ${1\mathord{\left/ {\vphantom {1 2}} \right. \kern-\nulldelimiterspace} 2} $, независимо от других мостов. Какова вероятность того, что после наводнения можно будет перейти с одного берега на другой, используя не разрушенные мосты?

\imgh{70mm}{graphs_bridges.pdf}{Схема мостов}

\end{problem}

\begin{problem}[Остовные деревья в полном графе]

Пусть имеется полный граф с $n$ вершинами $\{1,2,\ldots ,n\}$. Каждое из 
$\frac{n(n-1)}{2}$ ребер графа с вероятностью $\frac{1}{2}$ удаляется. 
Найдите вероятность того, что полученный после удаления ребер граф будет 
остовным деревом.

\end{problem}


\begin{ordre}

Обозначим $t_n $ - число остовных деревьев на пронумерованных вершинах 
$\{1,2,\ldots ,n\}$. Ясно, что искомая в задаче вероятность есть $\frac{t_n 
}{2^{\frac{n(n-1)}{2}}}$.

Выделим одну вершину и посмотрим на те связные компоненты или блоки, на 
которое разобьется остовное дерево, если проигнорировать все ребра, 
проходящие через выделенную вершину. Если невыделенные вершины образуют $m$ 
компонент размеров $k_1 ,k_2 ,\ldots ,k_m $, то их можно соединить с 
выделенной вершиной $k_1 k_2 \cdots k_m $ способами.

Такие рассуждения приводят к рекуррентному соотношению
\[
t_n =\sum\limits_{m>0} {\frac{1}{m!}} \;\sum\limits_{k_1 +k_2 +\ldots +k_m 
=n-1} {\left( {{\begin{array}{*{20}c}
 {n-1} \hfill \\
 {k_1 ,k_2 ,\ldots ,k_m } \hfill \\
\end{array} }} \right)} \,k_1 k_2 \cdots k_m \,t_{k_1 } t_{k_2 } \cdots 
t_{k_m } ,
\]
при любом $n>1$. 

Теперь обозначим $u_n =nt_n $, тогда рекуррентное соотношение примет 
следующий вид:
\[
\frac{u_n }{n!}=\sum\limits_{m>0} {\frac{1}{m!}} \;\sum\limits_{k_1 +k_2 
+\ldots +k_m =n-1} {\frac{u_{k_1 } }{k_1 !}\frac{u_{k_2 } }{k_2 !}\cdots 
\frac{u_{k_m } }{k_m !}} ,\quad n>1.
\]
Обозначим за $U(x)$ ЭПФ для последовательности $\left\{ {u_n } \right\}$ (то 
есть $U(x)=\sum\limits_{n=0}^\infty {\frac{u_n }{n!}x^n} )$. Таким образом,
\[
U(x)=xe^{U(x)}.
\]
Для нахождения явной формулы для этой последовательности, можно 
воспользоваться следующим уточнением теоремы Лагранжа.

$Теорема.$ Пусть функции $\varphi =\varphi (x)(\varphi (0)=0)$ и $\psi =\psi 
(z)$ связаны между собой уравнением Лагранжа
\[
\varphi (x)=x\psi \left( {\varphi (x)} \right).
\]
Тогда коэффициенты при $x^n$ в функции $\varphi $ равен коэффициенту при 
$z^{n-1}$ в разложении $\frac{1}{n}\psi ^n(z)$.

\end{ordre}


\begin{problem}

(Модель Эрдёша-Реньи).

 Пусть есть конечное множество (в дальнейшем множество вершин) $V$. $\xi _{vv'} $ - независимые с.в., занумерованные парами $\left\{v,v'\right\}\in V\times V$, $\vert V \vert = N$.
\[\xi _{vv'} =\left\{\begin{array}{cc} {1,} & {p} \\ {0,} & {1-p} \end{array}\right. .\] 
Таким образом, можно задать абстрактный случайный граф на фиксированном множестве вершин. Покажите, что 

\begin{enumerate}
\item  При $p=\frac{1}{N^{1+\varepsilon } } $, $\varepsilon >0$ среднее число не изолированных вершин в случайном графе $o\left(N\right)$.

\item  При $p=\frac{1}{N^{1-\varepsilon } } $, $\varepsilon >0$ с вероятностью близкой к единице ($N \gg 1$) существует связная компонента порядка $N$.
\end{enumerate}



\end{problem}

\begin{problem} 
Рассматривается конфигурация спинов $\omega =\left\{x_{mn} \right\}$ (где $x_{mn} $ - независимые бернуллиевские с.в. с параметром $p$) на двумерной решетке $\left\{(m,n)\right\}\in {\mathbb Z}^{2} $. Вершину $(m,n)$ назовем занятой, если $x_{mn} =1$. Соединим ребром все соседние (находящиеся на расстоянии 1) занятые вершины. Получится случайный граф $G=G\left(\omega \right)$. Назовем кластером графа $G$ максимальное подмножество $A$ вершин решетки такое, что для любых двух $v,v'\in A$ существует связывающий их путь по ребрам графа $G$. Докажите, что существует такое $0<\bar{p}<1$, что при $p<\bar{p}$ все кластеры конечны с вероятностью 1, а при $p>\bar{p}$ с положительной вероятностью есть хотя бы один бесконечный кластер.


\begin{ordre}
Покажите, что при достаточно малых значения $p$ вероятность события, что все кластеры конечны, равна 1. Покажите, что вероятность того, что кластер, содержащий начало координат и имеющий не менее $N$ вершин, не превосходит $\left(Cp\right)^{N} \mathop{\to }\limits_{N\to \infty } 0$, где $C$ - некоторая константа. А значит и событие: бесконечный кластер содержит начало координат - имеет нулевую вероятность.
\end{ordre}

\end{problem}


\begin{problem}
Найти математическое ожидание числа вершин, принадлежащих древесным компонентам.
\begin{ordre}
Число возможных деревьев на $k$ вершинах равно $k^{k-2}$. 
\end{ordre}
\end{problem}

\begin{problem}
Докажите, что количество лесов на $n$~вершинах с $r$~компонентами, каждая из которых содержит ровно по одной вершине из множества $ \{1, \ldots, r\} $, в точности $rn^{n-1-r}$~штук.
\end{problem}

\begin{problem}
Покажите, что число попарно неизоморфных деревьев на $n$ вершинах не меньше, чем $ c_n n^{-2.5} e^n $, где $ c_n \to \frac{1}{\sqrt{2\pi}}$.
\end{problem}

\begin{problem}[обобщенная схема размещений]
$ $

\begin{enumerate}
\item  Пусть для неотрицательных целочисленных с.в. $\eta _1 
$,{\ldots},$\eta _N $ существуют независимые одинаково распределенные с.в. 
$\xi _1 $,{\ldots},$\xi _N $ такие, что

$P\left( {\eta _1 =k_1 ,...,\eta _N =k_N } \right)=P\left( {\left. {\xi _1 
=k_1 ,...,\xi _N =k_N } \right|\xi _1 +...+\xi _N =n} \right).$ (**)

Введем независимые одинаково распределенные с.в. $\xi _1^{\left( r \right)} 
$,{\ldots},$\xi _N^{\left( r \right)} $, где $r$ целое неотрицательное число 
и
\[
P\left( {\xi _1^{\left( r \right)} =k} \right)=P\left( {\left. {\xi _1 =k} 
\right|\xi _1 \ne r} \right),
\quad
k=0,1,...
\]
Пусть $p_r =P\left( {\xi _1 =r} \right)$ и $S_N =\xi _1 +...+\xi _N $, 
$S_N^{\left( r \right)} =\xi _1^{\left( r \right)} +...+\xi _N^{\left( r 
\right)} $. Пусть $\mu _r \left( {n,N} \right)$ - число тех с.в. $\eta _1 
$,{\ldots},$\eta _N $, принявших значение $r$. Покажите, что с.в. типа $\mu 
_r \left( {n,N} \right)$ можно изучать с помощью \textit{обобщенной схемы размещений}: для любого $k=0,1,...,N$
\[
P\left( {\mu _r \left( {n,N} \right)=k} \right)=C_n^k p_r^k \left( {1-p_r } 
\right)^{N-k}\frac{P\left( {S_{N-k}^{\left( r \right)} =n-kr} 
\right)}{P\left( {S_N =n} \right)}.
\]
Напомним, что в классической схеме размещений $n$ различных частиц по $N$ 
различным ячейкам было доказано, что распределение заполнений ячеек $\eta _1 
$,{\ldots},$\eta _N $ имеет вид:
\[
P\left( {\eta _1 =k_1 ,...,\eta _n =k_n } \right)=\frac{n!}{k_1 !....k_N 
!N^n},
\]
где $k_1 $,{\ldots},$k_n $ - неотрицательные целые числа такие, что $k_1 
+....+k_n =n$. Если положить $\xi _1 $,{\ldots},$\xi _N \in Po\left( \lambda 
\right)$ - i.i.d. ($\lambda >0$ - произвольно), то получим (**).

\item Дан случайный граф (модель Эрдеша - Реньи) $G\left( {n,\;p} \right)$. Пусть 
$p=c\frac{\ln n}{n}$. Покажите, что при $c>1$ граф $G\left( {n,\;p} \right)$ 
почти наверное связен.


\end{enumerate}
\end{problem}

\begin{problem}
Graph social: Newmen's model with fixed (predefined) Exp of node degrees.
\end{problem}

\begin{problem}
Generalized graph model with restrictions (max H rule) 
\end{problem}