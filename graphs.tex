\section{Случайные модели графов}

\begin{problem}

На некоторой реке имеется 6 островов, соединенных между собой системой мостов. Во время летнего наводнения часть мостов была разрушена. При этом каждый мост разрушается с вероятностью ${1\mathord{\left/ {\vphantom {1 2}} \right. \kern-\nulldelimiterspace} 2} $, независимо от других мостов. Какова вероятность того, что после наводнения можно будет перейти с одного берега на другой, используя не разрушенные мосты?

\imgh{70mm}{graphs_bridges.pdf}{Схема мостов}

\end{problem}

\begin{problem}

(Модель Эрдёша-Реньи).

 Пусть есть конечное множество (в дальнейшем множество вершин) $V$. $\xi _{vv'} $ - независимые с.в., занумерованные парами $\left\{v,v'\right\}\in V\times V$, $\vert V \vert = N$.
\[\xi _{vv'} =\left\{\begin{array}{cc} {1,} & {p} \\ {0,} & {1-p} \end{array}\right. .\] 
Таким образом, можно задать абстрактный случайный граф на фиксированном множестве вершин. Покажите, что 

\begin{enumerate}
\item  При $p=\frac{1}{N^{1+\varepsilon } } $, $\varepsilon >0$ среднее число не изолированных вершин в случайном графе $o\left(N\right)$.

\item  При $p=\frac{1}{N^{1-\varepsilon } } $, $\varepsilon >0$ с вероятностью близкой к единице ($N \gg 1$) существует связная компонента порядка $N$.
\end{enumerate}



\end{problem}

\begin{problem} 
Рассматривается конфигурация спинов $\omega =\left\{x_{mn} \right\}$ (где $x_{mn} $ - независимые бернуллиевские с.в. с параметром $p$) на двумерной решетке $\left\{(m,n)\right\}\in {\mathbb Z}^{2} $. Вершину $(m,n)$ назовем занятой, если $x_{mn} =1$. Соединим ребром все соседние (находящиеся на расстоянии 1) занятые вершины. Получится случайный граф $G=G\left(\omega \right)$. Назовем кластером графа $G$ максимальное подмножество $A$ вершин решетки такое, что для любых двух $v,v'\in A$ существует связывающий их путь по ребрам графа $G$. Докажите, что существует такое $0<\bar{p}<1$, что при $p<\bar{p}$ все кластеры конечны с вероятностью 1, а при $p>\bar{p}$ с положительной вероятностью есть хотя бы один бесконечный кластер.


\begin{ordre}
Покажите, что при достаточно малых значения $p$ вероятность события, что все кластеры конечны, равна 1. Покажите, что вероятность того, что кластер, содержащий начало координат и имеющий не менее $N$ вершин, не превосходит $\left(Cp\right)^{N} \mathop{\to }\limits_{N\to \infty } 0$, где $C$ - некоторая константа. А значит и событие: бесконечный кластер содержит начало координат - имеет нулевую вероятность.
\end{ordre}

\end{problem}


\begin{problem}
Найти математическое ожидание числа вершин, принадлежащих древесным компонентам.
\begin{ordre}
Число возможных деревьев на $k$ вершинах равно $k^{k-2}$. 
\end{ordre}
\end{problem}

\begin{problem}
Докажите, что количество лесов на $n$~вершинах с $r$~компонентами, каждая из которых содержит ровно по одной вершине из множества $ \{1, \ldots, r\} $, в точности $rn^{n-1-r}$~штук.
\end{problem}

\begin{problem}
Покажите, что число попарно неизоморфных деревьев на $n$ вершинах не меньше, чем $ c_n n^{-2.5} e^n $, где $ c_n \to \frac{1}{\sqrt{2\pi}}$.
\end{problem}