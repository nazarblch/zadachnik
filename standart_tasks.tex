\section{Стандартные задачи}

\begin{problem}
При каждом подбрасывании монета падает вверх орлом с вероятностью $p>0$. Пусть $\pi _{n} $ - вероятность того, что число орлов после $n\in {\mathbb N}$ независимых подбрасываний будет чётно. Показав, что $\pi _{n+1} =\left(1-p\right)\cdot \pi _{n} +p\cdot \left(1-\pi _{n} \right)$, $n\in {\mathbb N}$, или иным способом найдите $\pi _{n} $. Число $0$ считаем чётным.
\end{problem}

\begin{problem}
Симметричную монету независимо бросили $n$ раз. Результат бросания записали в виде последовательности нулей и единиц. Покажите, что с вероятностью стремящейся к единице при $n\to \infty $ длина максимальной подпоследовательности из подряд идущих единиц лежит в промежутке
\[\left(\log \sqrt{n} ,\; \log n^{2} \right).\] 
\end{problem}

\begin{problem}
Ведущий приносит два одинаковых конверта и говорит, что в них лежат деньги, причем в одном вдвое больше, чем в другом. Двое участников берут конверты и тайком друг от друга смотрят, сколько в них денег. Затем один говорит другому: «Махнемся не глядя?» (предлагая поменяться конвертами). Стоит ли соглашаться?
\end{problem}

\begin{problem}
Приведите пример вероятностного пространства и трёх событий на этом пространстве, которые попарно независимы, но зависимы в совокупности. Достаточно рассмотреть вероятностное пространство, порожденное бросанием шестигранного кубика. \textbf{б) }Предложите обобщение этой задачи, в котором любые \textit{n} из (\textit{n}+1) событий независимы в совокупности, а эти (\textit{n}+1) -- зависимы в совокупности.
\end{problem}

\begin{problem}
Сто паровозов выехали из города по однополосной линии, каждый с постоянной скоростью. Когда движение установилось, то из-за того, что быстрые догнали идущих впереди более медленных, образовались караваны (группы, движущиеся со скоростью лидера). Найдите м.о. и дисперсию числа караванов. Скорости различных паровозов независимы и одинаково распределены, а функция распределения скорости непрерывна.
\end{problem}

\begin{problem}
Согласно законам о трудоустройстве в городе \textit{N}, наниматели обязаны предоставить всем рабочим выходной, если хотя бы у одного из них день рождения, и принимать на службу рабочих независимо от их дня рождения. За исключением этих выходных рабочие трудятся весь год из 365 дней. Предприниматели хотят максимизировать среднее число человеко-дней в году. Сколько рабочих трудятся на фабрике в городе \textit{N}?

\end{problem}

\begin{problem}
Существует ли случайная величина с конечным вторым моментом и бесконечным первым моментом.
\end{problem}

\begin{problem}
В начале карточной игры принято с помощью жребия определять первого сдающего. Жребий бросается так: колоду хорошо тасуют, и затем кто-нибудь сдает игрокам по карте до появления первого туза. Кому выпал туз -- тот и сдает в первой игре. На каком месте в среднем появляется первый туз, если в колоде 32 карты (то есть найти математическое ожидание случайной величины «Число карт, сданных до первого туза»)?

\begin{ordre} 
Задача на свойство линейности математического ожидания.
\end{ordre}

\end{problem}

\begin{problem}
(А.Я. Червоненкис) Покажите, что неравенство Чебышева:
\[P\left\{\left|X-EX\right|>\varepsilon \right\}\le \frac{DX}{\varepsilon ^{2} } \] 
принципиально не улучшаемо.

\begin{ordre} 
Рассмотрите с.в. 
\[X=\left\{\begin{array}{cc} {a,} & {{\raise0.7ex\hbox{$ p $}\!\mathord{\left/ {\vphantom {p 2}} \right. \kern-\nulldelimiterspace}\!\lower0.7ex\hbox{$ 2 $}} } \\ {0,} & {1-p} \\ {-a,} & {{\raise0.7ex\hbox{$ p $}\!\mathord{\left/ {\vphantom {p 2}} \right. \kern-\nulldelimiterspace}\!\lower0.7ex\hbox{$ 2 $}} } \end{array}\right. \] 
Положите $\varepsilon =a-\delta $, где $\delta \to 0+$.
\end{ordre}

\end{problem}

\begin{problem}

 (А. Шень) В лотерее на выигрыш уходит 40\% от стоимости проданных билетов. Каждый билет стоит 100 рублей. Доказать, что вероятность выиграть 5000 рублей (или больше) меньше 1\%.

\begin{ordre} 
Использовать неравенство Маркова.
\end{ordre}

Искомая вероятность зависит, конечно, от правил лотерее, но ни при каких условиях она не превосходит 0.8\%$<$1\%.
Приведите пример правил лотереи, где искомая вероятность минимальна и максимальна.

\end{problem}

\begin{problem}

Покажите, что все моменты распределения

 $p_{\lambda } \left(x\right)=\frac{1}{24} e^{-x^{{1\mathord{\left/ {\vphantom {1 4}} \right. \kern-\nulldelimiterspace} 4} } } \left(1-\lambda \sin x^{{1\mathord{\left/ {\vphantom {1 4}} \right. \kern-\nulldelimiterspace} 4} } \right)$, $x\ge 0$ при любом значении параметра $\lambda \in \left[0,1\right]$ совпадают.

\begin{remark}

Необходимое и достаточное условие того, чтобы моменты однозначно определяли распределение, вообще говоря, комплексной случайной величины $x$, имеет вид:

\noindent $\sum _{n=0}^{\infty }\left(M\left(\left|x\right|^{2n} \right)\right)^{{-1\mathord{\left/ {\vphantom {-1 \left(2n\right)}} \right. \kern-\nulldelimiterspace} \left(2n\right)} } =\infty  $ (условие Карлемана).
\end{remark}

\end{problem} 

\begin{problem}

На некоторой реке имеется 6 островов, соединенных между собой системой мостов. Во время летнего наводнения часть мостов была разрушена. При этом каждый мост разрушается с вероятностью ${1\mathord{\left/ {\vphantom {1 2}} \right. \kern-\nulldelimiterspace} 2} $, независимо от других мостов. Какова вероятность того, что после наводнения можно будет перейти с одного берега на другой, используя не разрушенные мосты?

\imgh{70mm}{graphs_bridges.pdf}{Схема мостов}

\end{problem}

\begin{problem}

(Модель Эрдёша-Реньи).

 Пусть есть конечное множество (в дальнейшем множество вершин) $V$. $\xi _{vv'} $ - независимые с.в., занумерованные парами $\left\{v,v'\right\}\in V\times V$, $\vert V \vert = N$.
\[\xi _{vv'} =\left\{\begin{array}{cc} {1,} & {p} \\ {0,} & {1-p} \end{array}\right. .\] 
Таким образом, можно задать абстрактный случайный граф на фиксированном множестве вершин. Покажите, что 

\begin{enumerate}
\item  При $p=\frac{1}{N^{1+\varepsilon } } $, $\varepsilon >0$ среднее число не изолированных вершин в случайном графе $o\left(N\right)$.

\item  При $p=\frac{1}{N^{1-\varepsilon } } $, $\varepsilon >0$ с вероятностью близкой к единице ($N \gg 1$) существует связная компонента порядка $N$.
\end{enumerate}



\end{problem}

\begin{problem} 
Рассматривается конфигурация спинов $\omega =\left\{x_{mn} \right\}$ (где $x_{mn} $ - независимые бернуллиевские с.в. с параметром $p$) на двумерной решетке $\left\{(m,n)\right\}\in {\mathbb Z}^{2} $. Вершину $(m,n)$ назовем занятой, если $x_{mn} =1$. Соединим ребром все соседние (находящиеся на расстоянии 1) занятые вершины. Получится случайный граф $G=G\left(\omega \right)$. Назовем кластером графа $G$ максимальное подмножество $A$ вершин решетки такое, что для любых двух $v,v'\in A$ существует связывающий их путь по ребрам графа $G$. Докажите, что существует такое $0<\bar{p}<1$, что при $p<\bar{p}$ все кластеры конечны с вероятностью 1, а при $p>\bar{p}$ с положительной вероятностью есть хотя бы один бесконечный кластер.


\begin{ordre}
Покажите, что при достаточно малых значения $p$ вероятность события, что все кластеры конечны, равна 1. Покажите, что вероятность того, что кластер, содержащий начало координат и имеющий не менее $N$ вершин, не превосходит $\left(Cp\right)^{N} \mathop{\to }\limits_{N\to \infty } 0$, где $C$ - некоторая константа. А значит и событие: бесконечный кластер содержит начало координат - имеет нулевую вероятность.
\end{ordre}

\end{problem}

\begin{problem}
Найти математическое ожидание числа вершин, принадлежащих древесным компонентам.
\begin{ordre}
Число возможных деревьев на $k$ вершинах равно $k^{k-2}$. 
\end{ordre}
\end{problem}


\subsection{Формула Байеса}

\begin{problem}
Пусть отличник правильно решает задачу с вероятностью 0.9, а двоечник с вероятностью 0.1. Сколько задач нужно дать на зачете и сколько требовать решить, чтоб отличник не сдал зачет с вероятностью не большей 0.001, а двоечник сдал зачет с вероятностью не большей 0.1?
\end{problem}



\subsection{Случайный вектор}

