\section{Стандартные задачи}

\begin{problem}
При каждом подбрасывании монета падает вверх орлом с вероятностью $p>0$. Пусть $\pi _{n} $ - вероятность того, что число орлов после $n\in {\mathbb N}$ независимых подбрасываний будет чётно. Показав, что $\pi _{n+1} =\left(1-p\right)\cdot \pi _{n} +p\cdot \left(1-\pi _{n} \right)$, $n\in {\mathbb N}$, или иным способом найдите $\pi _{n} $. Число $0$ считаем чётным.
\end{problem}

\begin{problem}
Симметричную монету независимо бросили $n$ раз. Результат бросания записали в виде последовательности нулей и единиц. Покажите, что с вероятностью стремящейся к единице при $n\to \infty $ длина максимальной подпоследовательности из подряд идущих единиц лежит в промежутке
\[\left(\log \sqrt{n} ,\; \log n^{2} \right).\] 
\end{problem}

\begin{problem}
Приведите пример вероятностного пространства и трёх событий на этом пространстве, которые попарно независимы, но зависимы в совокупности. Достаточно рассмотреть вероятностное пространство, порожденное бросанием шестигранного кубика. \textbf{б) }Предложите обобщение этой задачи, в котором любые \textit{n} из \textit{n}+1 событий независимы в совокупности, а эти (\textit{n}+1) -- зависимы в совокупности.
\end{problem}

\begin{problem}
Сто паровозов выехали из города по однополосной линии, каждый с постоянной скоростью. Когда движение установилось, то из-за того, что быстрые догнали идущих впереди более медленных, образовались караваны (группы, движущиеся со скоростью лидера). Найдите м.о. и дисперсию числа караванов. Скорости различных паровозов независимы и одинаково распределены, а функция распределения скорости непрерывна.
\end{problem}

\begin{problem}
Согласно законам о трудоустройстве в городе \textit{N}, наниматели обязаны предоставить всем рабочим выходной, если хотя бы у одного из них день рождения, и принимать на службу рабочих независимо от их дня рождения. За исключением этих выходных рабочие трудятся весь год из 365 дней. Предприниматели хотят максимизировать среднее число человеко-дней в году. Сколько рабочих трудятся на фабрике в городе \textit{N}?

\end{problem}

\begin{problem}
В начале карточной игры принято с помощью жребия определять первого сдающего. Жребий бросается так: колоду хорошо тасуют, и затем кто-нибудь сдает игрокам по карте до появления первого туза. Кому выпал туз -- тот и сдает в первой игре. На каком месте в среднем появляется первый туз, если в колоде 32 карты (то есть найти математическое ожидание случайной величины «Число карт, сданных до первого туза»)?

\begin{ordre} 
Задача на свойство линейности математического ожидания.
\end{ordre}

\end{problem}

\subsection{Sampling}

\begin{problem}
Покажите, что если с.в. $\eta $ равномерно распределена на отрезке $\left[0,1\right]$, то с.в. $\xi =F^{-1} \left(\eta \right)$ имеет функцию распределения $F\left(x\right)$. Предполагается, что $F\left(x\right)$ непрерывна и строго монотонна. Как выглядит формула для моделирования с.в. из показательного распределения с функцией распределения $F\left(x\right)=\left(1-e^{-\lambda x} \right)I\left\{x>0\right\}$? (Стандартный способ моделирования с.в. -- метод обратной функции)
\end{problem}

\begin{problem}

Пусть $\xi $ распределена на $\left[0,1\right]$ с плотностью $f_{\xi } (x)$, представимой в виде степенного ряда $\sum _{k=0}^{\infty }a_{k} x^{k}  $ с $a_{k} \ge 0$. Положим $p_{k} ={a_{k} \mathord{\left/ {\vphantom {a_{k}  (k+1)}} \right. \kern-\nulldelimiterspace} (k+1)} $. Тогда $f_{\xi } (x)=\sum _{k=0}^{\infty }p_{k} \cdot (k+1)x^{k}  $. Примените метод суперпозиции для моделирования с.в. $\xi $.

\begin{ordre}
Метод суперпозиции:

\noindent 1) Разыгрывается значение дискретной с.в., принимающей значения $k=0,1,2,...$ с вероятностями $p_{k} $.

\noindent 2) Моделируется с.в. с функцией распределения $F_{k} (x)$ (например, методом обратной функции).

\end{ordre}

\end{problem}

\begin{problem}
(Теорема Бернштейна) 

\textbf{а)} С помощью неравенства Чебышёва установите следующий результат из анализа: 

\[
\forall \; \; f\in C\left[0,1\right]\to \left\| f_{n} -f\right\| _{C\left[0,1\right]} \xrightarrow[{n\to \infty }]{} 0,
\] 

\[
f_{n} \left(x\right)=\sum_{k=0}^{n}f\left(\frac{k}{n} \right) C_{n}^{k} x^{k} \left(1-x\right)^{n-k} 
\]

\textbf{б)} Исходя из предыдущей задачи и п. а) предложите способ генерирования распределения с.в. $\xi $, имеющей плотность $f_{\xi } \left(x\right)$ с финитным носителем, для определенности, пусть носителем будет отрезок $\left[0,1\right]$.
\end{problem}

\begin{problem}
Как с помощью с.в. $\xi $, равномерно распределенной на отрезке $\left[0,1\right]$ ($\xi \in R\left[0;1\right]$), и симметричной монетки построить с.в. $X$, имеющую плотность распределения $f_{X} (x)=\frac{1}{4} \left(\frac{1}{\sqrt{x} } +\frac{1}{\sqrt{1-x} } \right)$, $x\in \left[0,1\right]$?
\end{problem}

\begin{problem}
(Метод фон Неймана) 

Пусть с.в. $\xi $ распределена на отрезке$\left[a,b\right]$, причем ее плотность распределения ограничена: $\mathop{\max }\limits_{x\in \left[a;b\right]} f_{\xi } (x)=C<\infty $. Пусть с.в. $\eta _{1} $, $\eta _{2} $, \dots  -- независимы и равномерно распределены на $\left[0,1\right]$, $X_{i} =a+\left(b-a\right)\eta _{2i-1} $, $Y_{i} =C\eta _{2i} $, $i=1,2,...$, т.е. пары $\left(X_{i} ,Y_{i} \right)$ независимы и равномерно распределены в прямоугольнике $\left[a,b\right]\times \left[0,C\right]$. Обозначим через $\nu $ номер первой точки с координатами $\left(X_{i} ,Y_{i} \right)$, попавшей под график плотности $f_{\xi } (x)$, т.е. $\nu =\min \left\{i:\quad Y_{i} \le f_{\xi } (X_{i} )\right\}$. Положим $X_{\nu } =\sum _{n=1}^{\infty }X_{n} I\left\{\nu =n\right\} $.

\textbf{а)} Покажите, что с.в. $X_{\nu } $ распределена также как $\xi $.

\textbf{б)} Сколько в среднем точек $\left(X_{i} ,Y_{i} \right)$ потребуется «вбросить» в прямоугольник $\left[a,b\right]\times \left[0,C\right]$ для получения одного значения $\xi $?
\end{problem}

\subsection{Формула Байеса}

\begin{problem}
Пусть отличник правильно решает задачу с вероятностью 0.9, а двоечник с вероятностью 0.1. Сколько задач нужно дать на зачете и сколько требовать решить, чтоб отличник не сдал зачет с вероятностью не большей 0.001, а двоечник сдал зачет с вероятностью не большей 0.1?
\end{problem}

\subsection{Графы}

\begin{problem}

На некоторой реке имеется 6 островов, соединенных между собой системой мостов. Во время летнего наводнения часть мостов была разрушена. При этом каждый мост разрушается с вероятностью ${1\mathord{\left/ {\vphantom {1 2}} \right. \kern-\nulldelimiterspace} 2} $, независимо от других мостов. Какова вероятность того, что после наводнения можно будет перейти с одного берега на другой, используя не разрушенные мосты?

\begin{fixme} 
Условие кажется некорректным: можно переформулировать "Какова вероятность того, что граф останется связным?" или "Какова вероятность того, что не будет вершин нулевой степени?"
\end{fixme}

\end{problem}

\subsection{Совместное распределение}

\begin{problem}
Пусть с.в. $\eta _{1} $, $\eta _{2} $ имеют равномерное распределение на отрезке $\left[0,1\right]$. Докажите, что с.в. $X$ и $Y$: $X=\sqrt{-2\ln \eta _{1} } \cos \left(2\pi \eta _{2} \right)$, $Y=\sqrt{-2\ln \eta _{1} } \sin \left(2\pi \eta _{2} \right)$ -- независимые и одинаково распределенные: стандартно нормально ${\rm {\mathcal N}}\left(0,1\right)$.

\begin{ordre}
Покажите, что
\[f_{XY} (x,y)=\frac{1}{\sqrt{2\pi } } e^{-\frac{x^{2} }{2} } \frac{1}{\sqrt{2\pi } } e^{-\frac{y^{2} }{2} } =\frac{1}{2\pi } e^{-\frac{x^{2} +y^{2} }{2} } .\] 
Перейдите к полярным координатам, не забыв о якобиане замены переменных.
\end{ordre}

\end{problem}

\begin{problem}

Если $X$ -- с.в., имеющая стандартное нормальное распределение, то $X^{-2} $ имеет устойчивую плотность:
\[\frac{1}{\sqrt{2\pi } } e^{-\frac{1}{2x} } x^{-\frac{3}{2} } , x>0.\] 
Используя это, покажите, что если $X$ и $Y$ -- независимые нормально распределенные с.в. с нулевым математическим ожиданием и дисперсиями $\sigma _{1}^{2} $ и $\sigma _{2}^{2} $, то величина $Z=\frac{XY}{\sqrt{X^{2} +Y^{2} } } $ нормально распределена с дисперсией $\sigma _{3}^{2} $, такой, что $\frac{1}{\sigma _{3}^{2} } =\frac{1}{\sigma _{1}^{2} } +\frac{1}{\sigma _{2}^{2} } $ (Л.Шепп).

\end{problem}