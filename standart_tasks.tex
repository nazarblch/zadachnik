\section{Стандартные задачи}

\begin{problem}
Некто имеет $N$ ключей, из которых только один от его двери. Какова вероятность, что, используя ключи в случайном порядке, 
он откроет дверь 
\begin{enumerate}
\item[а)] первым ключом, 
\item[б)] последним ключом? 
\end{enumerate}
Найти вероятность, что потребуется не менее $k$ попыток, чтобы открыть дверь, если ключи, которые не подошли, 
\begin{enumerate}
\item[в)] откладываются, 
\item[г)] не откладываются. 
\end{enumerate}

\begin{ordre}
В пункте в) воспользоваться заменой вероятности бинарной величины на математическое ожидание.
\end{ordre}

\end{problem}


\begin{problem}
Ребенок играет с десятью буквами разрезной азбуки: А, А, А, Е, И, К, М, М, Т, Т. 
Какова вероятность того, что при случайном расположении букв в ряд он получит слово <<МАТЕМАТИКА>>? 
\end{problem}


\begin{problem}
Из урны, содержащей $a$ белых, $b$ черных и $c$ красных шаров (и только их), последовательно извлекаются три шара. Найти 
вероятность следующих событий: 
\begin{enumerate}
\item[а)] все три шара разного цвета; 
\item[б)] шары извлечены в последовательности белый, черный, красный; 
\item[в)] шары извлечены в обратной последовательности. 
\end{enumerate}
\end{problem}



\begin{problem}
Из урны, содержащей $a$ белых и $b$ черных шаров, извлекается наугад один шар и откладывается в сторону. Какова вероятность 
того, что извлеченный наугад второй шар окажется белым, если: 
\begin{enumerate}
\item[а)] первый извлеченный шар белый; 
\item[б)] цвет  первого извлеченного шара остается неизвестным? 
\end{enumerate}
\end{problem}




\begin{problem}
Партия продукции состоит из десяти изделий, среди которых два изделия дефектные. Какова вероятность того, что из пяти отобранных 
наугад и проверенных изделий: 
\begin{enumerate}
\item[а)] ровно одно изделие дефектное; 
\item[б)] ровно два изделия дефектные; 
\item[в)] хотя бы одно изделие дефектное? 
\end{enumerate} 

\begin{ordre}
В пункте в) удобнее искать вероятность противоположного события.
\end{ordre}

\end{problem}

\begin{problem}
Найти вероятность того, что из $50$ студентов, присутствующих на лекции, хотя бы двое имеют одну и ту же дату рождения. 

\begin{remark}
Для получения приближенного решения можно воспользоваться свойством линейности математического ожидания следующих событий $X_{ij}$ - у  i-го и  j-го человека дни рождения совпадают.  
\end{remark}

\end{problem}


\begin{problem}
Опыт состоит в подбрасывании монеты до тех пор, пока два раза подряд она не выпадет одной и той же стороной. Каждому 
возможному исходу опыта припишем вероятность $1/2$ (монета <<правильная>>). 
Построить пространство элементарных событий и найти вероятности следующих событий: 
\begin{enumerate}
\item[а)] опыт окончится до шестого бросания; 
\item[б)] для завершения опыта потребуется четное число бросаний. 
\end{enumerate}
\end{problem}


\begin{problem}
В урне находится $m$ шаров, из которых $m_1$ белых и $m_2$ черных $(m_1 + m_2 = m)$. 
Производится $n$ извлечений одного шара с возвращением его (после определения его цвета) обратно в урну. Найти вероятность того, 
что ровно $r$ раз из $n$ будет извлечен белый шар. 
\end{problem}


\begin{problem}
Найти вероятность того, что при размещении $n$ различных шаров по $N$ ящикам заданный ящик будет содержать ровно 
$k$: $0\leqslant k\leqslant n$, шаров (все различимые размещения равновероятны). 
\end{problem}


\begin{problem}
В урне находится $m$ шаров, из которых $m_1$ --- первого цвета, $m_2$ --- второго цвета, $\ldots$, $m_s$ --- $s$-го цвета 
$(m_1+m_2+\ldots +m_s=m)$. 
Производится $n$ извлечений одного шара с возвращением его (после определения его цвета) обратно в урну. Найти вероятность того, 
что $r_1$ раз будет извлечен шар первого цвета, $r_2$ раз --- шар второго цвета, $\ldots$, $r_s$ раз --- шар $s$-го цвета 
$(r_1+r_2+\ldots +r_s=n)$. 
\end{problem}


\begin{problem}
В гардеробе все шляпы $N$ посетителей оказались случайным образом перепутанными. Шляпы не имеют внешних отличительных признаков. 
Какова вероятность того, что хотя бы один посетитель получит свою шляпу (рассмотреть случаи $N=4$, $N=10000$)? Найти математическое ожидание таких посетителей. 

\begin{ordre}
Воспользоваться формулой включений-исключений.
\end{ordre}

\end{problem}


\begin{problem}
Несколько раз бросается игральная кость. Какое событие более вероятно: 
\begin{enumerate}
\item[а)] сумма выпавших очков четна; 
\item[б)] сумма выпавших очков нечетна? 
\end{enumerate}
\end{problem}


\begin{problem}
Для уменьшения общего количества игр $2n$ команд спортсменов разбиваются на две подгруппы. Определить вероятности того, что 
две наиболее сильные команды окажутся: 
\begin{enumerate}
\item[а)] в одной подгруппе; 
\item[б)] в разных подгруппах. 
\end{enumerate}
\end{problem}


\begin{problem}
В урне находятся черные и белые шары, которые наугад по одному без возвращения извлекаются из урны до тех пор, пока урна не опустеет. 
Какое событие более вероятно: 
\begin{enumerate}
\item[а)] первый извлеченный шар белый; 
\item[б)] последний извлеченный шар белый? 
\end{enumerate}
\end{problem}



\begin{problem}
В урне находятся $a$ белых и $b$ черных шаров, Шары наугад по одному извлекаются из урны без возвращения. Найти вероятность того, 
что $k$-й вынутый шар оказался белым. 
\end{problem}


\begin{problem}
$30$ шаров размещаются по $8$ ящикам так, что для каждого шара одинаково возможно попадание в любой ящик. Найти вероятность 
размещения, при котором будет $3$ пустых ящика, $2$ ящика --- с тремя, $2$ ящика --- с шестью и $1$ ящик --- с двенадцатью шарами. 
\end{problem}


\begin{problem}
$N$ частиц случайно и независимо друг от друга размещаются в $k$ ячейках так, что каждая из них попадает 
в $i$-ую ячейку с вероятностью $p_i$ $(i=1,\ldots,k, \sum\limits_{i=1}^{k} p_i=1)$. Найти вероятность того, что число частиц в ячейках 
примет заданные значения $n_1$, $\ldots$, $n_i$, $\ldots$, $n_k$ (полиномиальное распределение). 
\end{problem}


\begin{problem}
При каждом подбрасывании монета падает вверх орлом с вероятностью $p>0$. Пусть $\pi _{n} $ - вероятность того, что число орлов после $n\in {\mathbb N}$ независимых подбрасываний будет чётно. Показав, что $\pi _{n+1} =\left(1-p\right)\cdot \pi _{n} +p\cdot \left(1-\pi _{n} \right)$, $n\in {\mathbb N}$, или иным способом найдите $\pi _{n} $. Число $0$ считаем чётным.
\end{problem}

\begin{problem}
Симметричную монету независимо бросили $n$ раз. Результат бросания записали в виде последовательности нулей и единиц. Покажите, что с вероятностью стремящейся к единице при $n\to \infty $ длина максимальной подпоследовательности из подряд идущих единиц лежит в промежутке
\[\left(\log \sqrt{n} ,\; \log n^{2} \right).\] 
\end{problem}

\begin{problem}
Некто обладает одной облигацией, которую намеревается продать в один из последующих четырех дней, в которых цена облигации 
принимает различные значения, априори неизвестные, но становящиеся известными в начале каждого дня продаж. Предполагается, что 
цены облигации независимы и их перестановки по торговым дням равновозможны. Какова стратегия продавца, состоящая в выборе дня 
продажи облигации и гарантирующая максимальную вероятность того, что он продаст облигацию в день ее наибольшей цены? 
\end{problem}

\begin{ordre}
Рассмотреть следующие возможные стратегии и сравнить вероятности продажи облигации в день наибольшей цены: 
\begin{enumerate}
\item[а)] на первом шаге (в первый день торгов) запомним имевшую место цену облигации, не продавая ее, а затем продадим 
облигацию в тот день, когда ее цена окажется большей цены, зафиксированной в первый день, или (когда такого дня не окажется) в 
последний (четвертый) день, независимо от цены этого дня (стратегия $S_1$); 

\item[б)] не продавая облигацию в первом и втором торговых днях, зафиксируем  максимальную цену из двух, имевших место для этих дней, 
и продадим облигацию в третьем торговом дне, если цена облигации в нем будет выше, чем указанная зафиксированная максимальная цена, 
или, в противном случае, в четвертом дне (стратегия $S_2$). 
\end{enumerate}
\end{ordre}

\begin{problem}
Из $n$ лотерейных билетов $k$ --- выигрышные $(n\geqslant 2k)$. Какова вероятность, что среди $k$ купленных билетов по крайней мере 
один будет выигрышным? 
\end{problem}

\begin{problem}
Из совокупности всех подмножеств множества $\{1,2,\ldots,N\}$ по схеме выбора с возвращением выбираются множества $A$ и $B$. 
Найти вероятность, что $A$ и $B$ не пересекаются. 
\end{problem}

\begin{problem}
Ведущий приносит два одинаковых конверта и говорит, что в них лежат деньги, причем в одном вдвое больше, чем в другом. Двое участников берут конверты и тайком друг от друга смотрят, сколько в них денег. Затем один говорит другому: «Махнемся не глядя?» (предлагая поменяться конвертами). Стоит ли соглашаться?
\end{problem}


\begin{problem}[парадокс Монти–Холла ]
Представьте, что вы стали участником игры, в которой находитесь перед тремя дверями. Ведущий поместил за одной из трех 
пронумерованных дверей автомобиль, а за двумя другими дверями --- по козе (козы тоже пронумерованы) случайным образом --– это значит, 
что все $3! = 6$ вариантов расположения автомобиля и коз за пронумерованными дверями равновероятны). У вас нет никакой информации 
о том, что за какой дверью находится. Ведущий говорит: <<Сначала вы должны выбрать одну из дверей. После этого я открою одну из 
оставшихся дверей (при этом если вы выберете дверь, за которой находится автомобиль, то я с вероятностью $1/2$ выберу дверь, 
за которой находится коза номер $1$, и с вероятностью $1-1/2=1/2$ дверь, за которой находится коза номер $2$). Затем я предложу 
вам изменить свой первоначальный выбор и выбрать оставшуюся закрытую дверь вместо той, которую вы выбрали сначала. Вы можете 
последовать моему совету и выбрать другую дверь, либо подтвердить свой первоначальный выбор. После этого я открою дверь, 
которую вы выбрали, и вы выиграете то, что находится за этой дверью.>> Вы выбираете дверь номер $3$. Ведущий открывает дверь номер $1$ 
и показывает, что за ней находится коза. Затем ведущий предлагает вам выбрать дверь номер $2$. Увеличатся ли ваши шансы 
выиграть автомобиль, если вы последуете его совету? 
\end{problem}


\begin{problem}[парадокс Стефана Банаха]
В двух спичечных коробках имеется по $n$ спичек. На каждом шаге наугад выбирается коробок, и из него удаляется (используется) 
одна спичка. Найти вероятность того, что в момент, когда один из коробков опустеет, в другом останется $k$ спичек. 
\end{problem}

\begin{solution}

{\textit{Первый вариант}}

Событие, удовлетворяющее условию задачи --- выбран пустой коробок, а в другом коробке имеется $k$ спичек. 

Рассмотрим производящую функцию $P(w,z)=\sum\limits_{m,n} p_{m,n} w^m z^n$, 
где $p_{m,n}$ есть вероятность, начав с $m$ спичек в одной коробке и $n$ --- в другой, получить обе пустые коробки, 
когда впервые выбирается пустая коробка. 

При $m=n=0$ имеем $p_{0,0}=1$. Получим 
$$
P(w,z)=1+\frac{1}{2}(w+z)P(w,z) \,\Rightarrow\, 
P(w,z)=\Bigl( 1-\frac{1}{2}(w+z) \Bigr)^{-1} . 
$$
Разлагая в ряд по степеням $w$ и $z$, находим 
$$
P(w,z)=1+\sum\limits_{k=1}^{\infty} \frac{(w+z)^k}{2^k} . 
$$
Отсюда $p_{m,n}=\frac{C_{m+n}^{n}}{2^{m+n}}$. 

Далее вводим производящую функцию $P_k(w,z)=\sum\limits_{m,n} p_{k,m,n} w^m z^n$, 
где $p_{k,m,n}$ есть вероятность, начав с $m$ спичек в одной коробке и $n$ --- в другой, в момент выбрасывания 
первой пустой коробки иметь вторую коробку с $k$ спичками. Тогда 
$$
P_k(w,z)=\frac{1}{2}(w^k + z^k) P(w,z) 
$$
$$
\Rightarrow\; p_{k,m,n}=\frac{C_{m+n-k}^m+C_{m+n-k}^n}{2^{1+m+n-k}} . 
$$
Для данной задачи искомая вероятность равна 
$$
p_{k,n,n}=\frac{C_{2n-k}^n}{2^{2n-k}} . 
$$


{\textit{Второй вариант}}

Событие, удовлетворяющее условию задачи --- из выбранной коробки взяли последнюю спичку, а в другом коробке имеется $k$ спичек. 

Рассмотрим процесс изъятия спичек из коробок как последовательность нулей и единиц (например, нули соответствуют спичкам первой коробки, 
единицы -- второй коробки) длины $2n$, с числом нулей и единиц равным $n$. Общее число возможных исходов равно $N=C_{2n}^n$. 

Исходы, удовлетворяющие условию задачи, устроены так: если сначала пустеет первый коробок, то на $2n-k$-м месте стоит нуль, 
а в первых $2n-k-1$ позициях имеется $n-1$ нулей в любом порядке. Аналогично когда сначала пустеет второй коробок. Число таких 
исходов равно $m=2\cdot C_{2n-k-1}^{n-1}$. Искомая вероятность равна 
$$
p=\frac{m}{N}=\frac{2\cdot C_{2n-k-1}^{n-1}}{C_{2n}^n}=\frac{2\cdot n!\, n!\, (2n-k-1)!}{(2n)!\, (n-1)!\, (n-k)!}=
\frac{n!\, (2n-k-1)!}{(2n-1)!\, (n-k)!}=\frac{C_{n}^k}{C_{2n-1}^k} . 
$$
\end{solution}

\begin{problem}
Приведите пример вероятностного пространства и трёх событий на этом пространстве, которые попарно независимы, но зависимы в совокупности. Достаточно рассмотреть вероятностное пространство, порожденное бросанием шестигранного кубика. \textbf{б) }Предложите обобщение этой задачи, в котором любые \textit{n} из (\textit{n}+1) событий независимы в совокупности, а эти (\textit{n}+1) -- зависимы в совокупности.
\end{problem}



\begin{problem}
Показать, что из независимости событий $A$ и $B$ следует независимость событий $A$ и $\overline B$, $\overline A$ и $B$, 
$\overline A$ и $\overline B$. 
\end{problem}


\begin{problem}
Показать, что из равенства ${\mathbb P}(A\, |\, B)={\mathbb P}(A\, |\, \overline B)$ для ненулевых событий $A$ и $B$ следует 
равенство ${\mathbb P}(AB)={\mathbb P}(A){\mathbb P}(B)$, т.е. их независимость. 
\end{problem}


\begin{problem}
Подбрасываются три игральные кости. События $A$, $B$ и $C$ означают выпадение одинакового числа очков (соответственно) на первой и 
второй, на второй и третьей, на первой и третьей костях. Являются ли эти события независимыми 
\begin{enumerate}
\item[а)] попарно, 
\item[б)] в совокупности? 
\end{enumerate}
\end{problem}


\begin{problem}[парадокс транзитивности]
Будем говорить, что случайная величина $X$ больше по вероятности случайной величины $Y$, если ${\mathbb P}(X>Y)>{\mathbb P}(X\le Y)$. 
Пусть известно, что для случайных величин $X$, $Y$, $Z$, $W$ выполнена следующая цепочка равенств: 
$$
{\mathbb P}(X>Y)={\mathbb P}(Y>Z)={\mathbb P}(Z>W)=\alpha>\frac{1}{2} . 
$$
Верно ли, что $X$ больше по вероятности $W$ и почему? 
\end{problem}



\begin{problem}
Юноша собирается сыграть три теннисных матча со своими родителями, и он должен победить два раза подряд. 
Порядок матчей может быть следующим: отец--мать--отец, мать--отец--мать. Юноше нужно решить, какой порядок для него предпочтительней, 
учитывая, что отец играет лучше матери.

\end{problem}

\begin{problem}
Показать, что борелевская $\sigma$-алгебра в ${\mathbb R}^1$, содержащая все числовые промежутки вида $[a,b)$, 
содержит все промежутки вида $(a,b)$, $(a,b]$, $[a,b]$ и отдельные точки прямой. 

\begin{ordre}
Учесть свойство замкнутости $\sigma$-алгебры относительно операций 
объединения, пересечения и вычитания. 
\end{ordre}

\end{problem}


\begin{problem}
Пусть $\Omega = [a, b]$, $F$ --- $\sigma$–алгебра, содержащая все отрезки 
$[a,b]$ $(a \leqslant \alpha < \beta \leqslant b)$ с вероятностной мерой 
${\mathbb P}\{ \omega\in[\alpha,\beta]\}=\dfrac{\mes[\alpha, \beta]}{\mes[a,b]}$. 
Показать, что 
\begin{enumerate}
\item[а)] ${\mathbb P}\{ \omega=c=\const\}=0$; 
\item[б)] ${\mathbb P}\{ \omega_1=\omega_2\}=0$. 
\end{enumerate}
Найти вероятность, что для трех исходов $\omega_1$, $\omega_2$, $\omega_3$ третий лежит между первыми двумя. 
\end{problem}


\begin{problem}
Число элементарных событий некоторого вероятностного пространства равно $n$. Указать минимальное и максимальное возможные значения 
для числа событий. 
\end{problem}

\begin{problem}
Может ли число всех событий какого-либо вероятностного пространства быть равным $129$; $130$; $128$? 
\end{problem}


\begin{problem}
В урне находится $3$ белых и $2$ черных шара (и только они). 
Эксперимент состоит в последовательном извлечении из урны всех шаров по одному наугад без возвращения. Построить вероятностное пространство. 
Описать $\sigma$-алгебру, порожденную случайной величиной $X$, если: 
\begin{enumerate}
\item[а)] $X$ --- число белых шаров, предшествующих первому черному шару; 
\item[б)] $X$ --- число черных шаров среди извлеченных; 
\item[в)] $X=X_1+X_2$, где $X_1$ --- число белых шаров, предшествующих первому черному шару, 
$X_2$ --- число черных шаров, предшествующих белому шару. 
\end{enumerate}
\end{problem}


\begin{problem}
\label{SigmaAlgebra}
Пусть $(\Omega,\Sigma,{\mathbb P})$ --- некоторое вероятностное пространство и $A$ --- алгебра подмножеств $\Omega$ такая, что 
$\sigma(A)=\Sigma$ ($\sigma(A)$ --- наименьшая $\sigma$-алгебра, содержащая алгебру $A$). Доказать, что 
$$
\forall\varepsilon>0,\, B\in\Sigma\quad \exists A_{\varepsilon}\in A:\quad {\mathbb P}(A_{\varepsilon}\bigtriangleup B)
\leqslant\varepsilon . 
$$
\end{problem}

\begin{ordre}
Рассмотрим совокупность множеств 
$$
{\mathcal B}=\bigl\{ B\in\Sigma\, | \, \forall\varepsilon>0 \; \exists A_B\in A:\; {\mathbb P}(A_B\bigtriangleup B)
\leqslant\varepsilon \bigr\} . 
$$

Покажите, что ${\mathcal B}$ является минимальной $\sigma$-алгеброй 

\end{ordre}

\begin{problem}
Пусть $(\Omega,\Xi,{\mathbb P})$ --- вероятностное пространство, $\xi_1,\xi_2,\ldots$ --- некоторая последовательность с.в.. 
Обозначим $\Xi_n^{\infty}=\sigma(\xi_{n},\xi_{n+1},\ldots)$ --- $\sigma$-алгебру, порожденную с.в. $\xi_{n},\xi_{n+1},\ldots$ и пусть 
$$
{\mathcal X}=\bigcap\limits_{n=1}^{\infty} \Xi_{n}^{\infty} . 
$$
Поскольку пересечение $\sigma$-алгебр есть снова $\sigma$-алгебра, то ${\mathcal X}$ --- есть $\sigma$-алгебра. Эту $\sigma$-алгебру 
будем называть <<хвостовой>> или <<остаточной>>, в связи с тем, что всякое событие $A\in{\mathcal X}$ не зависит от значений с.в. 
$\xi_1,\xi_2,\ldots,\xi_n$ при любом конечном $n$, а определяется лишь <<поведением бесконечно далеких значений последовательности 
$\xi_1,\xi_2,\ldots$ >>. 

С помощью задачи $\ref{SigmaAlgebra}$ докажите справедливость следующего утверждения: 

Пусть $\xi_1,\xi_2,\ldots$ --- последовательность независимых в совокупности с.в. и $A\in{\mathcal X}$ 
(событие $A$ принадлежит <<хвостовой>> $\sigma$-алгебре). Тогда ${\mathbb P}(A)$ может принимать лишь два значения $0$ или $1$. 
\end{problem}

\begin{ordre}
Идея доказательства состоит в том, чтобы показать, что каждое <<хвостовое>> событие $A$ не зависит от самого себя и, значит, 
${\mathbb P}(A\cap A)={\mathbb P}(A)\cdot {\mathbb P}(A)$, т.е. ${\mathbb P}(A)={\mathbb P}^2(A)$, откуда 
${\mathbb P}(A)=0$ или $1$. 
\end{ordre}


\begin{problem}
Доказать справедливость равенства 
${\mathbb P}(A\bigtriangleup B)={\mathbb P}(A)+{\mathbb P}(B)-2{\mathbb P}(A\, B)$. 
\end{problem}


\begin{problem}
Пусть $A$, $B$, $C$ --- заданные события. Доказать справедливость неравенств 
\begin{enumerate}
\item[а)] ${\mathbb P}(AB)+{\mathbb P}(AC)+{\mathbb P}(BC)\geqslant {\mathbb P}(A)+{\mathbb P}(B)+{\mathbb P}(C)-1$; 
\item[б)] ${\mathbb P}(AB)+{\mathbb P}(AC)-{\mathbb P}(BC)\leqslant {\mathbb P}(A)$; 
\item[в)] ${\mathbb P}(A\bigtriangleup B)\leqslant {\mathbb P}(A\bigtriangleup C)+{\mathbb P}(C\bigtriangleup B)$ . 
\end{enumerate}
\end{problem}


\begin{problem}
В каждую из $n$ пронумерованных ячеек в случайном порядке помещается один из $n$ так же пронумерованных шаров. 
Найти вероятность того, что ни в одной из ячеек номер шара не совпадет с номером ячейки. 

\begin{ordre}
Применить формулу включений-исключений.
\end{ordre}
\end{problem}

\begin{problem}
Сто паровозов выехали из города по однополосной линии, каждый с постоянной скоростью. Когда движение установилось, то из-за того, что быстрые догнали идущих впереди более медленных, образовались караваны (группы, движущиеся со скоростью лидера). Найдите м.о. и дисперсию числа караванов. Скорости различных паровозов независимы и одинаково распределены, а функция распределения скорости непрерывна.
\end{problem}

\begin{problem}
Согласно законам о трудоустройстве в городе \textit{N}, наниматели обязаны предоставить всем рабочим выходной, если хотя бы у одного из них день рождения, и принимать на службу рабочих независимо от их дня рождения. За исключением этих выходных рабочие трудятся весь год из 365 дней. Предприниматели хотят максимизировать среднее число человеко-дней в году. Сколько рабочих трудятся на фабрике в городе \textit{N}?

\end{problem}

\begin{problem}
Существует ли случайная величина с конечным вторым моментом и бесконечным первым моментом.
\end{problem}

\begin{problem}
В начале карточной игры принято с помощью жребия определять первого сдающего. Жребий бросается так: колоду хорошо тасуют, и затем кто-нибудь сдает игрокам по карте до появления первого туза. Кому выпал туз -- тот и сдает в первой игре. На каком месте в среднем появляется первый туз, если в колоде 32 карты (то есть найти математическое ожидание случайной величины «Число карт, сданных до первого туза»)?

\begin{ordre} 
Задача на свойство линейности математического ожидания.
\end{ordre}

\end{problem}

\begin{problem}
(А.Я. Червоненкис) Покажите, что неравенство Чебышева:
\[P\left\{\left|X-EX\right|>\varepsilon \right\}\le \frac{DX}{\varepsilon ^{2} } \] 
принципиально не улучшаемо.

\begin{ordre} 
Рассмотрите с.в. 
\[X=\left\{\begin{array}{cc} {a,} & {{\raise0.7ex\hbox{$ p $}\!\mathord{\left/ {\vphantom {p 2}} \right. \kern-\nulldelimiterspace}\!\lower0.7ex\hbox{$ 2 $}} } \\ {0,} & {1-p} \\ {-a,} & {{\raise0.7ex\hbox{$ p $}\!\mathord{\left/ {\vphantom {p 2}} \right. \kern-\nulldelimiterspace}\!\lower0.7ex\hbox{$ 2 $}} } \end{array}\right. \] 
Положите $\varepsilon =a-\delta $, где $\delta \to 0+$.
\end{ordre}

\end{problem}

\begin{problem}

 (А. Шень) В лотерее на выигрыш уходит 40\% от стоимости проданных билетов. Каждый билет стоит 100 рублей. Доказать, что вероятность выиграть 5000 рублей (или больше) меньше 1\%.

\begin{ordre} 
Использовать неравенство Маркова.
\end{ordre}

Искомая вероятность зависит, конечно, от правил лотерее, но ни при каких условиях она не превосходит 0.8\%$<$1\%.
Приведите пример правил лотереи, где искомая вероятность минимальна и максимальна.

\end{problem}

\begin{problem}

Покажите, что все моменты распределения

 $p_{\lambda } \left(x\right)=\frac{1}{24} e^{-x^{{1\mathord{\left/ {\vphantom {1 4}} \right. \kern-\nulldelimiterspace} 4} } } \left(1-\lambda \sin x^{{1\mathord{\left/ {\vphantom {1 4}} \right. \kern-\nulldelimiterspace} 4} } \right)$, $x\ge 0$ при любом значении параметра $\lambda \in \left[0,1\right]$ совпадают.

\begin{remark}

Необходимое и достаточное условие того, чтобы моменты однозначно определяли распределение, вообще говоря, комплексной случайной величины $x$, имеет вид:

\noindent $\sum _{n=0}^{\infty }\left(M\left(\left|x\right|^{2n} \right)\right)^{{-1\mathord{\left/ {\vphantom {-1 \left(2n\right)}} \right. \kern-\nulldelimiterspace} \left(2n\right)} } =\infty  $ (условие Карлемана).
\end{remark}

\end{problem} 

\begin{problem}

На некоторой реке имеется 6 островов, соединенных между собой системой мостов. Во время летнего наводнения часть мостов была разрушена. При этом каждый мост разрушается с вероятностью ${1\mathord{\left/ {\vphantom {1 2}} \right. \kern-\nulldelimiterspace} 2} $, независимо от других мостов. Какова вероятность того, что после наводнения можно будет перейти с одного берега на другой, используя не разрушенные мосты?

\imgh{70mm}{graphs_bridges.pdf}{Схема мостов}

\end{problem}

\begin{problem}

(Модель Эрдёша-Реньи).

 Пусть есть конечное множество (в дальнейшем множество вершин) $V$. $\xi _{vv'} $ - независимые с.в., занумерованные парами $\left\{v,v'\right\}\in V\times V$, $\vert V \vert = N$.
\[\xi _{vv'} =\left\{\begin{array}{cc} {1,} & {p} \\ {0,} & {1-p} \end{array}\right. .\] 
Таким образом, можно задать абстрактный случайный граф на фиксированном множестве вершин. Покажите, что 

\begin{enumerate}
\item  При $p=\frac{1}{N^{1+\varepsilon } } $, $\varepsilon >0$ среднее число не изолированных вершин в случайном графе $o\left(N\right)$.

\item  При $p=\frac{1}{N^{1-\varepsilon } } $, $\varepsilon >0$ с вероятностью близкой к единице ($N \gg 1$) существует связная компонента порядка $N$.
\end{enumerate}



\end{problem}

\begin{problem} 
Рассматривается конфигурация спинов $\omega =\left\{x_{mn} \right\}$ (где $x_{mn} $ - независимые бернуллиевские с.в. с параметром $p$) на двумерной решетке $\left\{(m,n)\right\}\in {\mathbb Z}^{2} $. Вершину $(m,n)$ назовем занятой, если $x_{mn} =1$. Соединим ребром все соседние (находящиеся на расстоянии 1) занятые вершины. Получится случайный граф $G=G\left(\omega \right)$. Назовем кластером графа $G$ максимальное подмножество $A$ вершин решетки такое, что для любых двух $v,v'\in A$ существует связывающий их путь по ребрам графа $G$. Докажите, что существует такое $0<\bar{p}<1$, что при $p<\bar{p}$ все кластеры конечны с вероятностью 1, а при $p>\bar{p}$ с положительной вероятностью есть хотя бы один бесконечный кластер.


\begin{ordre}
Покажите, что при достаточно малых значения $p$ вероятность события, что все кластеры конечны, равна 1. Покажите, что вероятность того, что кластер, содержащий начало координат и имеющий не менее $N$ вершин, не превосходит $\left(Cp\right)^{N} \mathop{\to }\limits_{N\to \infty } 0$, где $C$ - некоторая константа. А значит и событие: бесконечный кластер содержит начало координат - имеет нулевую вероятность.
\end{ordre}

\end{problem}

\begin{problem}
Найти математическое ожидание числа вершин, принадлежащих древесным компонентам.
\begin{ordre}
Число возможных деревьев на $k$ вершинах равно $k^{k-2}$. 
\end{ordre}
\end{problem}

\begin{problem}
Допустим, что вероятность столкновения молекулы с другими молекулами в промежутке времени $[t,t + \Delta t)$ 
равна $p = \lambda\Delta t+{\overline o}(\Delta t)$ и не зависит от времени, прошедшего после предыдущего столкновения $(\lambda = \const)$. 
Найти распределение времени свободного пробега молекулы и вероятность того, что это время превысит заданную величину $t^*$. 
\end{problem}

\begin{ordre}
Разобьем интервал $\Delta=[0,t)$ на $n$ отрезков равной длины.
Пусть $A_i$ --- событие, означающее, что на отрезке $\Delta_i$  молекула претерпит столкновение с другими молекулами. Можно представить вероятность отсутствия столкновения в виде произведения вероятностей событий $\overline{A_i}$.
\end{ordre}


\begin{problem}
В каждую $i$-ую единицу времени живая клетка получает случайную дозу облучения $X_i$, причем $\{ X_i\}_{i=1}^{t}$ имеют 
одинаковую функцию распределения $F_X(x)$ и независимы в совокупности $\forall t$. Получив интегральную дозу облучения, 
равную $\nu$, клетка погибает. Оценить среднее время жизни клетки ${\mathbb E}T$. 
\end{problem}

\begin{ordre}

Показать тождество Вальда: 
$$
{\mathbb E}S_T={\mathbb E}X\cdot {\mathbb E}T, 
$$

введя вспомогательную случайную величину

$$
Y_j=\begin{cases}
1, &\text{ если }\quad X_1+\ldots +X_{j-1}=S_{j-1}<\nu, \\
0, &\text{ в остальных случаях }. 
\end{cases}
$$
 

\end{ordre}


\begin{problem}
Урна содержит $N$ шаров с номерами от $1$ до $N$. Пусть $K$ --- наибольший номер, полученный при $n$ их поштучных извлечениях 
с возвращением. Найти 
\begin{enumerate}
\item[а)] распределение $K$, 
\item[б)] асимптотику математического ожидания ${\mathbb E}K$ при $N\to\infty$. 
\end{enumerate}
\end{problem}

\begin{ordre}
\[
{\mathbb E}K=\sum\limits_{j=0}^{N} {\mathbb P}\{ K>j\}
\]
\end{ordre}


\begin{problem}
Найдите вероятность $q_n$ того, что случайная $(0,1)$-матрица размера $n\times n$ является невырожденной над полем 
$GF_2=\{ 0,1\}$. Доказать, что существует $\lim\limits_{n\to\infty} q_n=q>0$. 
\end{problem}

\begin{ordre}
$k$-я строка невырожденной матрицы $A$ равно -- любая строка из $\{ 0,1\}^n$, 
кроме любой линейной комбинации первых $k-1$ строк. 
\end{ordre}

\begin{problem}
В некотором Вузе проходит экзамен. Количество экзаменационных билетов $N$. Перед экзаменационной аудиторией выстроилась очередь из 
студентов, которые не знают, чему равно $N$. Согласно этой очереди студенты вызываются на экзамен (второй студент заходит в аудиторию 
после того как из нее выйдет первый и т.д.). Каждый студент с равной вероятностью может выбрать любой из $N$ билетов (в независимости 
от других студентов). Проэкзаменованные студенты, выходя из аудитории, сообщают оставшейся очереди номера своих билетов. 
Оцените (сверху), сколько студентов должно быть проэкзаменовано, чтобы оставшаяся к этому моменту очередь смогла оценить число 
экзаменационных билетов с точностью $10\%$ с вероятностью, не меньшей $0.95$. 
\end{problem}


\begin{problem}
Восемь мальчиков и семь девочек купили билеты в кинотеатр на $15$ подряд идущих сидячих мест. Предположим, что все $15!$ 
возможных способов сесть равновероятны. Вычислите среднее число пар рядом сидящих мальчика и девочки. Например, 
м, м, м, м, м, м, ж, м, ж, ж, ж, ж, ж, ж содержит три такие пары. 
\end{problem}


\begin{problem}
На первом этаже семнадцатиэтажного общежития в лифт вошли десять человек. Предполагая, что каждый из вошедших может с равной вероятностью 
жить на любом из шестнадцати этажей (со $2$-го по $17$-ый), найдите среднее число остановок лифта. 
\end{problem}

\begin{problem}
В самолете $n$ мест. Есть $n$ пассажиров, выстроившихся друг за другом в очередь. Во главе очереди -- <<заяц>>. У всех, 
кроме <<зайца>>, есть билет, на котором указан номер посадочного места. Так как <<заяц>> входит первым, он случайным образом занимает 
некоторое место. Каждый следующий пассажир, входящий в салон самолета, действует по такому принципу: если его место свободно, то 
садится на него, если занято, то занимает с равной вероятностью любое свободное. Найдите вероятность того, что последний пассажир 
сядет на свое место. 
\end{problem}

\begin{problem}
Имеется $n$ пронумерованных писем и $n$ пронумерованных конвертов. Письма случайным образом раскладываются по конвертам (т.е. все $n!$ 
способов распределения $n$ писем по $n$ конвертам, так чтобы в каждом конверте было ровно по одному письму, считаются равновероятными). 
Найдите математическое ожидание случайной величины, равной числу совпадений (числу писем, лежащих в конвертах с теми же номерами). 
\end{problem}

\begin{problem}
(Задача игрока де Мере) 
Что более вероятно: при одновременном бросании четырех игральных костей получить хотя бы одну единицу или при $24$ бросаниях 
по две игральные кости одновременно получить хотя бы один раз две единицы? Найти вероятности указанных событий. 
\end{problem}

\begin{problem}
$n$ человек разного роста случайным образом выстраиваются в шеренгу. Найти вероятность того, что: 
\begin{enumerate}
\item[а)] самый низкий окажется $i$-м слева; 
\item[б)] самый высокий окажется первым слева, а самый низкий --- последним слева; 
\item[в)] самый высокий и самый низкий окажутся рядом; 
\item[г)] между самым высоким и самым низким расположатся более $k$ человек. 
\end{enumerate}
\end{problem}


\begin{problem}
Найти математическое ожидание ${\mathbb E}Z$, где $Z$ --- $k$-ая по величине из координат $n$ точек, взятых наудачу на отрезке 
$[0;1]$ $(k \leqslant n)$. 
\end{problem}

\begin{problem}[рекорды]
Пусть $X_1 ,X_2 ,\ldots $ - независимые 
случайные величины с одной и той же плотностью распределения вероятностей 
$p(x)$. Будем говорить, что наблюдается рекордное значение в момент времени 
n$>$1, если $X_n >\max \left[ {X_1 ,...,X_{n-1} } \right]$. Докажите 
следующие утверждения.

\begin{enumerate}
\item[\textbf{А)}] Вероятность того, что рекорд зафиксирован в момент времени $n$, 
равна $1/n$.

\item[\textbf{Б)}] Математическое ожидание числа рекордов до момента времени $n$ 
равно 
\[
\sum\limits_{1<i\le n} {\frac{1}{i}} \sim \ln n.
\]

\item[\textbf{В)}] Пусть $Y_n $ --- случайная величина, принимающая значение $1$, если 
в момент времени $n$ зафиксирован рекорд, и значение $0$ -- в противном случае. 
Тогда случайные величины $Y_1 ,Y_2 ,\ldots$ независимы в совокупности.

\item[\textbf{Г)}] Дисперсия числа рекордов до момента времени $n$ равна
\[
\sum\limits_{1<i\le n} {\frac{i-1}{i^2}} \sim \ln n.
\]

\item[\textbf{Д)}] Если $T$ -- момент появления первого рекорда после момента времени $1$, то $ET=\infty $.
\end{enumerate}
\end{problem}


\begin{problem}[распределение Коши]
Радиоактивный источник испускает 
частицы в случайном направлении (при этом все направления равновероятны). 
Источник находится на расстоянии $d$ от фотопластины, которая представляет 
собой бесконечную вертикальную плоскость.

\begin{enumerate}
\item[\textbf{А)}] При условии, что частица попадает в плоскость, покажите, что 
горизонтальная координата точки попадания (если начало координат выбирается 
в точке, ближайшей к источнику) имеет плотность распределения:
\[
p\left( x \right)=\frac{d}{\pi \left( {d^2+x^2} \right)}.
\]
Это распределение известно как \textit{распределение Коши}.

\item[\textbf{Б)}] Можно ли вычислить среднее (математическое ожидание) этого 
распределения?
\end{enumerate}
\end{problem}


\subsection{Формула Байеса}


\begin{problem}
Имеются две урны. В одной из них находится один белый шар, в другой --- один черный шар (других шаров урны не содержат). Выбирается 
наугад одна урна. В нее добавляется один белый шар и после перемешивания один из шаров извлекается. Извлеченный шар оказался белым. 
Определить апостериорную вероятность того, что выбранной оказалась урна, которая первоначально содержала белый шар. 
\end{problem}


\begin{problem}
В первой урне содержится $a$ белых и $b$ черных шаров (и только они), во второй --- $c$ белых и $d$ черных шаров 
(и только они). Из выбранной наугад урны извлекается один шар, который обратно не возвращается. Извлеченный шар оказался белым. 
Найти вероятность того, что и второй шар, извлеченный из той же урны, окажется белым. 
\end{problem}


\begin{problem}
Известно, что $96\%$ выпускаемой продукции соответствует стандарту. Упрощенная схема контроля признает годным с вероятностью 
$0.98$ каждый стандартный экземпляр аппаратуры и с вероятностью $0.05$ --- каждый нестандартной экземпляр аппаратуры. Найти вероятность, 
что изделие, прошедшее контроль, соответствует стандарту. 
\end{problem}


\begin{problem}
Пусть отличник правильно решает задачу с вероятностью 0.9, а двоечник с вероятностью 0.1. Сколько задач нужно дать на зачете и сколько требовать решить, чтоб отличник не сдал зачет с вероятностью не большей 0.001, а двоечник сдал зачет с вероятностью не большей 0.1?
\end{problem}

\begin{problem}
В $m+1$ урне содержится по $m$ шаров, причем урна с номером $n$ содержит $n$ белых и $m-n$ черных шаров $(n = 0,1,\ldots,m)$. 
Случайным образом выбирается урна и из нее $k$ раз с возвращением извлекаются шары. Найти 
\begin{enumerate}
\item[а)] вероятность, что следующим также будет извлечен белый шар, при условии, что все $k$ шаров оказались белыми, 
\item[б)] ее предел при $m\to\infty$. 
\end{enumerate}
\end{problem}

\begin{ordre}
Применить формулу полной вероятности в следующем виде. 
$$
{\mathbb P}(B\, |\, A)=\sum\limits_{n=1} {\mathbb P}(B\, |\, H_n A){\mathbb P}(H_n\, |\, A)
$$
\end{ordre}

\begin{problem}
Известно, что $90\%$ выпускаемой продукции соответствует стандарту. Упрощенная схема контроля признает 
годным с вероятностью $0,88$ каждый стандартный экземпляр аппаратуры и с вероятностью $0,05$ --- 
каждый нестандартной экземпляр аппаратуры. Найдите вероятность, что изделие, прошедшее контроль, соответствует стандарту. 
\end{problem}


\begin{problem}
До проведения схемы испытаний Бернулли разыгрывается с.в. $p$, имеющая равномерное распределение на отрезке $[0.1, 0.9]$ 
(результаты розыгрыша нам неизвестны). После того как эта с.в. была разыграна, начинают проводиться опыты по схеме Бернулли 
(независимо $n=1000$ раз подкидывается монетка) с вероятность успеха (выпадения <<орла>>) в каждом опыте равной $p$ 
(после того как с.в. $p$ была разыграна, она уже приняла какое-то значения из отрезка $[0.1, 0.9]$ и рассматривается в серии опытов 
Бернулли уже как число, причем не меняющееся от опыта к опыту). В результате опыта было посчитано значение числа успехов $r=777$. 
Определите апостериорное распределение с.в. $p$, т.е. найдите условную плотность распределения $p(x|r=777)$. Оцените, как изменится 
ответ, если точное значение числа успехов нам неизвестно. Известно только, что $r\in[750, 790]$. Т.е. посчитайте условную 
плотность вероятности $p(x|r\in[750, 790])$. 
\end{problem}

\begin{ordre}

Условная функция распределения 
$$
F(x)={\mathbb P}(p<x|S_n=r)=\frac{{\mathbb P}(p<x, S_n=r)}{{\mathbb P}(S_n=r)}
$$

Условная плотность равна 
$$
p(x|r=777)=F'(x) 
$$

Исходя из ц.п.т.  $S_n=np+\xi\sqrt{np(1-p)}$, где $\xi\in {\mathcal N}(0,1)$

\end{ordre}


\begin{problem}
\begin{enumerate}
\item[1)] Имеется монетка (несимметричная). Несимметричность монетки заключается в том, что либо орел выпадает в два раза чаще решки; 
либо наоборот (априорно (до проведения опытов) оба варианта считаются равновероятными). Монетку бросили $10$ раз. Орел выпал $7$ раз. 
Определите апостериорную вероятность того, что орел выпадает в два раза чаще решки (апостериорная вероятность считается с учетом 
проведенных опытов (иначе говоря, это просто условная вероятность)). 

\item[2)] Определите апостериорную вероятность того, что орел выпадает не менее чем в два раза чаще решки. Если несимметричность 
монетки заключается в том, что либо орел выпадает не менее чем в два раза чаще решки; либо наоборот (априорно оба варианта считаются 
равновероятными). 
\end{enumerate}
\end{problem}


\subsection{Случайный вектор}



\begin{problem}
Пусть $X$ и $Y$ --- независимые случайные величины, равномерно распределенные на $(-b,b)$. 
Найдите вероятность $q_b$ того, что уравнение $t^2+tX+Y=0$ имеет действительные корни. Доказать, что 
существует $\lim\limits_{b\to\infty} q_b=q$. Найдите $q$. 
\end{problem}


\begin{problem}
Компоненты случайного вектора имеют нормальное распределение. Следует ли из этого, что вектор имеет нормальное распределение? 
\end{problem}

\begin{problem}
Двумерный случайный вектор $X=(X_1,X_2)$  имеет следующую функцию плотности распределения: 
$$
f(x_1,x_2)=\begin{cases}
\dfrac{c}{\sqrt{x_1^2+x_2^2}}, \text{ при } x_1^2+x_2^2\leqslant 1 , \\
\quad 0, \quad\text{ иначе }. 
\end{cases}
$$
\begin{enumerate}
\item[1)] Найдите $c$. 
\item[2)] Найдите частные и условные распределения его компонент. 
\item[3)] Являются ли они 
\begin{enumerate}
\item[а)] стохастически зависимыми; 
\item[б)] коррелированными? 
\end{enumerate}
\end{enumerate}
\end{problem}



\begin{problem}
Предположим, что с.в. $X\in L_2$, это означает ${\mathbb E}X^2<\infty$. Докажите, что 
\begin{equation}
\label{UMO}
\| X-{\mathbb E}(X|Y_1,\ldots,Y_n)\|_{L_2}=\min\limits_{\varphi\in H} \| X-\varphi(Y_1,\ldots,Y_n)\|_{L_2} , 
\end{equation}
где $H$ --- подпространство пространства $L_2$ всевозможных борелевских функций $\varphi(Y_1,\ldots,Y_n)\in L_2$; 
${\mathbb E}(X|Y_1,\ldots,Y_n)$ --- условное математическое ожидание с.в. $X$ относительно $\sigma$-алгебры, порожденной с.в. 
$Y_1,\ldots,Y_n$, часто говорят просто относительно с.в. $Y_1,\ldots,Y_n$; 
$$
\| X\|_{L_2}=\sqrt{\langle X,X\rangle_{L_2}}=\sqrt{{\mathbb E}(X\cdot X)}=\sqrt{{\mathbb E}(X^2)} . 
$$
\end{problem}

\begin{ordre}
Покажите, что $X-{\mathbb E}^{\mathcal A}X \bot \xi,\quad \forall\xi\in H$, т.е. ${\mathbb E}^{\mathcal A}$ 
является проектором на подпространство $H$ в $L_2$. 
\end{ordre}


\begin{problem}
Докажите, что если в условиях предыдущей задачи $(X,Y_1,\ldots,Y_n)^T$ --- является нормальным случайным вектором (без ограничения 
общности можно также считать, что $(Y_1,\ldots,Y_n)^T$  --- невырожденный нормальный случайный вектор), то в качестве $H$ можно взять 
подпространство всевозможных линейных комбинаций с.в. $Y_1,\ldots,Y_n$. Т.е. мы можем более конкретно сказать, на каком именно 
классе борелевских функций достигается минимум в $(\ref{UMO})$. 
\end{problem}

\begin{ordre}
Будем искать 
${\mathbb E}(X|Y_1,\ldots,Y_n)$ в виде 
\begin{equation}
\label{Gauss}
{\mathbb E}(X|Y_1,\ldots,Y_n)=c_1 Y_1+\ldots +c_n Y_n . 
\end{equation}

Докажите следующие утверждения:

\begin{enumerate}
\item $X-c_1 Y_1-\ldots-c_n Y_n, Y_1,\ldots, Y_n$ - независимы.
\item $X-c_1 Y_1-\ldots-c_n Y_n$ ортогонален подпространству $H$ пространства $L_2$ всевозможных борелевских функций $\varphi(Y_1,\ldots,Y_n)\in L_2$.
\end{enumerate}
 
\end{ordre}



\begin{problem}
$$
x=\begin{pmatrix}
x_1\\
x_2\\
x_3
\end{pmatrix}
\in N\left(
\begin{pmatrix}
2\\
3\\
1
\end{pmatrix}, 
\begin{Vmatrix}
5 & 2 & 7\\
2 & 5 & 7\\
7 & 7 & 14
\end{Vmatrix}
\right) . 
$$
\begin{enumerate}
\item[а)] Найдите распределение случайной величины $y_1=x_1+x_2-x_3$. 
\item[б)] Найдите распределение случайной величины $y_2=x_1+x_2+x_3$. 
\item[в)] Найдите ${\mathbb E}(y_2\, |\, x_1=5, x_2=3)$. 
\item[г)] Найдите ${\mathbb E}(y_2\, |\, x_1=5, x_2<3)$. 
\item[д)] Найдите ${\mathbb P}(y_2<10\, |\, x_1=5, x_2<3)$. 
\end{enumerate}
\end{problem}

\begin{ordre}
Проверьте следующее свойство нормального случайного вектора.
Пусть $y=c^Tx$. Тогда 
$$
{\mathbb E}y=c^T{\mathbb E}x, 
$$
$$
\Var y=c^T (\Var x) c 
$$
\end{ordre}

\begin{problem}
Покажите, что если независимые случайные величины $X_1,\ldots, X_n$ имеют показательное распределение, т.е. 
$$
f_{X_i}(x)=\begin{cases}
\lambda_i\exp(-\lambda_i x), \; x\geqslant 0 \\
0,\; x<0
\end{cases}
$$
(часто пишут $X_i\in\Exp(\lambda_i)$), то 
$$
\min\{ X_1,\ldots, X_n\}\in \Exp\Bigl( \sum\limits_{i=1}^{n}\lambda_i\Bigr) . 
$$
\end{problem}


\begin{problem}
Случайный вектор $W=(W_1, W_2, \ldots, W_n)'$ имеет плотность распределения 
$$
f_W(w)=\begin{cases}
n! , & \text{ если } 0\leqslant w_1\leqslant w_2\leqslant \ldots \leqslant w_n\leqslant 1, \\
0, & \text{ в остальных случаях }. 
\end{cases}
$$
Найти распределение вектора $U=(U_1, U_2, \ldots, U_n)'$, если 
$$
U_1=W_1, \quad U_2=W_2-W_1, \ldots, U_n=W_n-W_{n-1} . 
$$
\end{problem}

\begin{solution}
В общем случае: 
$$
U=\varphi(W) \,\Rightarrow\, f_U(u)=f_W\bigl(\varphi^{-1}(u)\bigr) \Bigl| J\Bigl( \frac{\partial W}{\partial U}\Bigr)\Bigr|, \quad 
J(\cdot)=\det\Bigl( \frac{\partial \varphi_i^{-1}(u)}{\partial u_j}\Bigr) . 
$$
В нашем случае $w_1=u_1$, $w_2=u_1+u_2$, $\ldots$, $w_n=u_1+\ldots +u_n$ и 
$$
J(\cdot)=\det\Bigl( \frac{\partial \varphi_i^{-1}(u)}{\partial u_j}\Bigr)=1 . \quad \text{ Отсюда }
$$
$$
f_U(u)=\begin{cases}
n! , & \text{ если } \forall u_i\geqslant 0, \; \sum\limits_{i=1}^{n} u_i\leqslant 1; \\
0, & \text{ в прочих случаях }. 
\end{cases}
$$
\end{solution}


\begin{problem}[переход к полярным координатам, якобиан преобразования]
Спортсмен стреляет по круговой мишени. Вертикальная и горизонтальная 
координаты точки попадания пули (при условии, что центр мишени -- начало 
координат) -- независимые случайные величины, каждая с распределением 
$N(0,1)$. Покажите, что расстояние от точки попадания до центра имеет 
плотность распределения вероятностей $r\exp \left( {-{r^2} \mathord{\left/ 
{\vphantom {{r^2} 2}} \right. \kern-\nulldelimiterspace} 2} \right)$ для 
$r\ge 0$. Найдите медиану этого распределения.
\end{problem}
