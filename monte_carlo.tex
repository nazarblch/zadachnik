\section{Метод Монте-Карло}

\begin{problem}
Вычисление значения интеграла:

 \textbf{а)} Требуется вычислить с заданной точностью $\varepsilon $ и с заданной доверительной вероятностью $\gamma $ абсолютно сходящийся интеграл $J=\int _{\left[0,\; 1\right]^{m} }f\left(\vec{x}\right)d\vec{x} $. Считайте, что $\forall \; \; \vec{x}\in \left[0,\; 1\right]^{m} \to \left|f\left(\vec{x}\right)\right|\le 1$.

\noindent \textbf{Пояснение. }Введем случайный m-вектор $\vec{X}\in R\left(\left[0,\; 1\right]^{m} \right)$ и с.в. $\xi =f\left(\vec{X}\right)$. Тогда $M\xi =\int _{\left[0,\; 1\right]^{m} }f\left(\vec{x}\right)d\vec{x} =J$. Поэтому получаем оценку интеграла $\bar{J}_{n} =\frac{1}{n} \sum _{k=1}^{n}f\left(\vec{x}^{k} \right) $, где $\vec{x}^{k} $, $k=1,...,n$ -- повторная выборка значений случайного вектора $\vec{X}$ (т.е. все $\vec{x}^{k} $, $k=1,...,n$ -- независимы и одинаково распределены: также как и вектор $\vec{X}$). В задаче требуется оценить сверху число $n$ ($n\gg m$), начиная с которого $P\left(\left|J-\bar{J}_{n} \right|\le \varepsilon \right)\ge \gamma $.

\textbf{б) }Решите задачу из п. а) при дополнительном предположении липшецевости функции $f\left(\vec{x}\right)$, разбив единичный куб на $n=N^{m} $ одинаковых кубиков со стороной ${1\mathord{\left/ {\vphantom {1 N}} \right. \kern-\nulldelimiterspace} N} $, и используя оценку $\bar{J}_{n} =\frac{1}{n} \sum _{k=1}^{n}f\left(\vec{x}^{k} \right) $, где $\vec{x}^{k} $ -- имеет равномерное распределение в \textit{k}-м кубике.

\textbf{в) (метод выделения главной части, метод замены меры (метод существенной выборки), метод включения особенности в плотность) }Решите задачу п. а) не предполагая, что $f\left(\vec{x}\right)$ -- ограниченная функция на единичном кубе. Предложите способы уменьшения дисперсии полученной оценки интеграла. Как можно использовать информацию об особенностях функции $f\left(\vec{x}\right)$?

\end{problem}