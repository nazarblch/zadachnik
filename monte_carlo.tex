\section{Метод Монте Карло}

\begin{problem}
На плоскости дано ограниченное измеримое по Лебегу множество $S$. Требуется найти площадь (меру Лебега) этого множества с заданной точностью 
$\varepsilon$. 

Поскольку по условию множество ограничено, то вокруг него можно описать квадрат со стороной $a$. Выберем декартову систему координат 
в одной из вершин квадрата с осями, параллельными сторонам квадрата. Рассмотрим $n$  независимых с.в. $\{ X_k\}_{k=1}^{n}$,  имеющих 
одинаковое равномерное распределение в этом квадрате, т.е. $X_k\in R([0,a]^2)$. Введем с.в. 
$$
Y_k=I(X_k\in S)=\begin{cases}
1,\quad X_k\in S\\
0, \quad X_k\notin S
\end{cases} . 
$$
Тогда $\{ Y_k\}_{k=1}^{n}$ --- независимые одинаково распределенные с.в.. Ясно, что $Y_k\in\Be(p(S))$. Следовательно, по у.з.б.ч. 
$$
\frac{Y_1+\ldots+Y_n}{n} \xrightarrow{\text{ п.н. }} {\mathbb E}(Y_1)=p(S)=\frac{\mu(S)}{a^2} \quad \text{ при  } n\to\infty . 
$$
Оцените сверху следующую вероятность
$$
{\mathbb P}\Bigl( \Bigl| \frac{Y_1+\ldots+Y_n}{n}-\frac{\mu(S)}{a^2}\Bigr|>\delta \Bigr) . 
$$
\end{problem}

\begin{ordre}

Согласно неравенству Берри-Эссена для всякого значения $x$ выполнено
$$
\Bigl| {\mathbb P}\Bigl( \frac{S_n-{\mathbb E}S_n}{\sqrt{\Var S_n}}< x \Bigr)-\Phi(x)\Bigr| \leqslant
\frac{C_0 \mu^3}{\sigma^3\sqrt{n}} . 
$$
Здесь $C_0<0.7056$, $\,\sigma^2=\Var X_k$, 
$
\mu^3={\mathbb E}|X_k-p|^3. 
$
\end{ordre}


\begin{problem}
Вычисление значения интеграла:

 \textbf{а)} Требуется вычислить с заданной точностью $\varepsilon $ и с заданной доверительной вероятностью $\gamma $ абсолютно сходящийся интеграл $J=\int _{\left[0,\; 1\right]^{m} }f\left(\vec{x}\right)d\vec{x} $. Считайте, что $\forall \; \; \vec{x}\in \left[0,\; 1\right]^{m} \to \left|f\left(\vec{x}\right)\right|\le 1$.

\noindent \textbf{Пояснение. }Введем случайный m-вектор $\vec{X}\in R\left(\left[0,\; 1\right]^{m} \right)$ и с.в. $\xi =f\left(\vec{X}\right)$. Тогда $M\xi =\int _{\left[0,\; 1\right]^{m} }f\left(\vec{x}\right)d\vec{x} =J$. Поэтому получаем оценку интеграла $\bar{J}_{n} =\frac{1}{n} \sum _{k=1}^{n}f\left(\vec{x}^{k} \right) $, где $\vec{x}^{k} $, $k=1,...,n$ -- повторная выборка значений случайного вектора $\vec{X}$ (т.е. все $\vec{x}^{k} $, $k=1,...,n$ -- независимы и одинаково распределены: также как и вектор $\vec{X}$). В задаче требуется оценить сверху число $n$ ($n\gg m$), начиная с которого $P\left(\left|J-\bar{J}_{n} \right|\le \varepsilon \right)\ge \gamma $.

\textbf{б) }Решите задачу из п. а) при дополнительном предположении липшецевости функции $f\left(\vec{x}\right)$, разбив единичный куб на $n=N^{m} $ одинаковых кубиков со стороной ${1\mathord{\left/ {\vphantom {1 N}} \right. \kern-\nulldelimiterspace} N} $, и используя оценку $\bar{J}_{n} =\frac{1}{n} \sum _{k=1}^{n}f\left(\vec{x}^{k} \right) $, где $\vec{x}^{k} $ -- имеет равномерное распределение в \textit{k}-м кубике.

\textbf{в) (метод выделения главной части, метод замены меры (метод существенной выборки), метод включения особенности в плотность) }Решите задачу п. а) не предполагая, что $f\left(\vec{x}\right)$ -- ограниченная функция на единичном кубе. Предложите способы уменьшения дисперсии полученной оценки интеграла. Как можно использовать информацию об особенностях функции $f\left(\vec{x}\right)$?

\end{problem}

\begin{problem}
Пусть с.в. $\eta _{1} $, $\eta _{2} $ имеют равномерное распределение на отрезке $\left[0,1\right]$. Докажите, что с.в. $X$ и $Y$: $X=\sqrt{-2\ln \eta _{1} } \cos \left(2\pi \eta _{2} \right)$, $Y=\sqrt{-2\ln \eta _{1} } \sin \left(2\pi \eta _{2} \right)$ -- независимые и одинаково распределенные: стандартно нормально ${\rm {\mathcal N}}\left(0,1\right)$.

\begin{ordre}
Покажите, что
\[f_{XY} (x,y)=\frac{1}{\sqrt{2\pi } } e^{-\frac{x^{2} }{2} } \frac{1}{\sqrt{2\pi } } e^{-\frac{y^{2} }{2} } =\frac{1}{2\pi } e^{-\frac{x^{2} +y^{2} }{2} } .\] 
Перейдите к полярным координатам, не забыв о якобиане замены переменных.
\end{ordre}

\end{problem}

\begin{comment}
\begin{problem}

Если $X$ -- с.в., имеющая стандартное нормальное распределение, то $X^{-2} $ имеет устойчивую плотность:
\[\frac{1}{\sqrt{2\pi } } e^{-\frac{1}{2x} } x^{-\frac{3}{2} } , x>0.\] 
Используя это, покажите, что если $X$ и $Y$ -- независимые нормально распределенные с.в. с нулевым математическим ожиданием и дисперсиями $\sigma _{1}^{2} $ и $\sigma _{2}^{2} $, то величина $Z=\frac{XY}{\sqrt{X^{2} +Y^{2} } } $ нормально распределена с дисперсией $\sigma _{3}^{2} $, такой, что $\frac{1}{\sigma _{3}^{2} } =\frac{1}{\sigma _{1}^{2} } +\frac{1}{\sigma _{2}^{2} } $ (Л.Шепп).

\end{problem}
\end{comment}



\begin{problem}

Пусть $\xi $ распределена на $\left[0,1\right]$ с плотностью $f_{\xi } (x)$, представимой в виде степенного ряда $\sum _{k=0}^{\infty }a_{k} x^{k}  $ с $a_{k} \ge 0$. Положим $p_{k} ={a_{k} \mathord{\left/ {\vphantom {a_{k}  (k+1)}} \right. \kern-\nulldelimiterspace} (k+1)} $. Тогда $f_{\xi } (x)=\sum _{k=0}^{\infty }p_{k} \cdot (k+1)x^{k}  $. Примените метод суперпозиции для моделирования с.в. $\xi $.

\begin{ordre}
Метод суперпозиции:

\noindent 1) Разыгрывается значение дискретной с.в., принимающей значения $k=0,1,2,...$ с вероятностями $p_{k} $.

\noindent 2) Моделируется с.в. с функцией распределения $F_{k} (x)$ (например, методом обратной функции).

\end{ordre}

\end{problem}

\begin{comment}
\begin{problem}[Теорема Бернштейна] 

\textbf{а)} С помощью неравенства Чебышёва установите следующий результат из анализа: 

\[
\forall \; \; f\in C\left[0,1\right]\to \left\| f_{n} -f\right\| _{C\left[0,1\right]} \xrightarrow[{n\to \infty }]{} 0,
\] 

\[
f_{n} \left(x\right)=\sum_{k=0}^{n}f\left(\frac{k}{n} \right) C_{n}^{k} x^{k} \left(1-x\right)^{n-k} 
\]

\textbf{б)} Исходя из предыдущей задачи и п. а) предложите способ генерирования распределения с.в. $\xi $, имеющей плотность $f_{\xi } \left(x\right)$ с финитным носителем, для определенности, пусть носителем будет отрезок $\left[0,1\right]$.
\end{problem}
\end{comment}

\begin{comment}
\begin{problem}
Как с помощью с.в. $\xi $, равномерно распределенной на отрезке $\left[0,1\right]$ ($\xi \in R\left[0;1\right]$), и симметричной монетки построить с.в. $X$, имеющую плотность распределения $f_{X} (x)=\frac{1}{4} \left(\frac{1}{\sqrt{x} } +\frac{1}{\sqrt{1-x} } \right)$, $x\in \left[0,1\right]$?
\end{problem}

\begin{problem}[Метод фон Неймана] 

Пусть с.в. $\xi $ распределена на отрезке$\left[a,b\right]$, причем ее плотность распределения ограничена: $\mathop{\max }\limits_{x\in \left[a;b\right]} f_{\xi } (x)=C<\infty $. Пусть с.в. $\eta _{1} $, $\eta _{2} $, \dots  -- независимы и равномерно распределены на $\left[0,1\right]$, $X_{i} =a+\left(b-a\right)\eta _{2i-1} $, $Y_{i} =C\eta _{2i} $, $i=1,2,...$, т.е. пары $\left(X_{i} ,Y_{i} \right)$ независимы и равномерно распределены в прямоугольнике $\left[a,b\right]\times \left[0,C\right]$. Обозначим через $\nu $ номер первой точки с координатами $\left(X_{i} ,Y_{i} \right)$, попавшей под график плотности $f_{\xi } (x)$, т.е. $\nu =\min \left\{i:\quad Y_{i} \le f_{\xi } (X_{i} )\right\}$. Положим $X_{\nu } =\sum _{n=1}^{\infty }X_{n} I\left\{\nu =n\right\} $.

\begin{enumerate}
\item Покажите, что с.в. $X_{\nu } $ распределена также как $\xi $.

\item Сколько в среднем точек $\left(X_{i} ,Y_{i} \right)$ потребуется «вбросить» в прямоугольник $\left[a,b\right]\times \left[0,C\right]$ для получения одного значения $\xi $?

\item Предложите модификацию рассмотренного метода для генерации дискретной случайной величины, принимающей значения $\lbrace 1, 2, ... , k \rbrace$ с одинаковой вероятностью, имея в распоряжении монету (генератор бинарной случайной величины).   
\end{enumerate}
\end{problem}
\end{comment}

\begin{problem}[Алгоритм Кнута–Яо]
С помощью бросаний симметричной монетки требуется сгенерировать распределение заданной дискретной с.в., принимающей конечное число значений. Обобщите описанную ниже схему на общий случай. Предположим, что нам нужно сгенерировать распределение с.в., принимающей три значения 1, 2, 3 с равными вероятностями 1/3. Действуем таким образом. Два раза кидаем монетку: если выпало 00, то считаем, что выпало значение 1, если 01, то 2, если 11, то 3. Если 10, то еще два раза кидаем монетку и повторяем рассуждения. Покажите, что можно сгенерировать распределение дискретной с.в., принимающей, вообще говоря, с разными вероятностями n значений в среднем с помощью не более чем $\log_2 (n - 1) + 2$ подбрасываний симметричной монетки.
\end{problem}

\begin{comment}

\subsection{ Markov chain Monte Carlo }

\begin{problem}
Чтобы построить однородный дискретный марковский процесс с конечным числом состояний, имеющий заданную инвариантную (стационарную) меру $\pi$ , переходные вероятности ищутся в следующим виде: $p_{ij} = p_{ij}^0 a_{ij}$ , 
$i \neq j$; $p_{ii} = 1 - \sum p_{ij}$ , где $p^0$ – некоторая затравочная матрица, которую будем далее предполагать симметричной. Покажите, что матрица $p$ имеет инвариантную (стационарную) меру  $\pi$, если
\[
\frac{a_{ij}}{a_{ji}} = \frac{\pi_{j} p^0_{ji}}{\pi_{i} p^0_{ij}} = \frac{\pi_{j}}{\pi_{i}} 
\]

Чтобы найти  $a_{ij}$ достаточно найти функцию F: $\mathbb{R_+} \rightarrow [0,1]$ такую, что

\[
\frac{F(z)}{F(1/z)} = z; a_{ij}  \leftarrow F( \frac{\pi_{j}}{\pi_{i}} )
\]

Пример функции $F(z) = \min(z,1)$ определяет алгоритм Метрополиса. 

\end{problem}

\begin{problem}

В руки опытных криптографов попалось закодированное письмо (10 000 символов). Чтобы это письмо прочитать нужно его декодировать. Для этого берется стохастическая матрица переходных вероятностей $P$
(линейный размер которой определяется числом возможных символов (букв, знаков препинания и т.п.) в языке на котором до шифрования было написано письмо – этот язык известен и далее будет называться базовым). $P_{ij}$ отвечает за вероятность появления символа с номером j сразу после символа под номером i . Такая матрица может быть идентифицирована с помощью статистического анализ  большого текста, скажем,  Войны и мира Л.Н. Толстого.

Пускай способ (де)шифрования определяется некоторой неизвестной функцией $\overline{f}$ – преобразование (перестановка) множества кодовых букв во множество символов базового языка.
В качестве, “начального приближения” выбирается какая-то функция f , например,
полученная исходя из легко осуществимого частотного анализа. Далее рассчитывается вероятность выпадения полученного закодированного текста, сгенерированного при заданной функции f  (правдоподобие выборки).

Случайно  выбираются два аргумента у функции f и значения функции при этих аргументах меняются местами. Если в результате правдоподобие возрастает, то замена аргументов фиксируется, иначе бросается монетка с вероятность выпадения орла равной отношению правдоподобий. 

Объясните, почему предложенный алгоритм “сходится” именно к $\overline{f}$ ? Почему сходимость оказывается такой быстрой (0.01 сек. на современном PC)?


\end{problem}



\subsection{Gibbs Sampler}

\subsection{Variational Inference}

\end{comment}