\newpage

% Требование РИО "Литература" вместо "Список литературы"
\renewcommand\refname{Литература}
% В самом списке 1. вместо [1]
\makeatletter
\renewcommand{\@biblabel}[1]{#1.}
\makeatother

\begin{thebibliography} {20}

\bibitem{1} 
Боровков А.А. Теория вероятностей. – М.: Наука, 1986. –: 352 с. (и более поздние издания)

\bibitem{2} 
Гнеденко Б.В. Курс теории вероятностей. – М: Наука, 1988. – 446 с. (и более поздние издания)

\bibitem{3}
Колмогоров А.Н. Основные понятия теории вероятностей. – М.: Наука. 1974. – 120 с.

\bibitem{4} 
Ширяев А.Н. Вероятность. – М.: Наука, 1989. – 640 с. (и более поздние издания)

\bibitem{5} 
Натан А.А., Горбачев О.Г., Гуз С.А. Теория вероятностей: Учеб. пособие. – М.: МЗ Пресс – МФТИ, 2007. – 253 с.

\bibitem{6} 
Розанов Ю.А. Лекции по теории вероятностей. - Долгопрудный: Издательский дом “Интеллект”, 2008. – 136 с.

\bibitem{7} 
Ширяев А.Н. Задачи по теории вероятностей. – М.: МЦНМО, 2006. – 416 с.

\bibitem{8} 
Гмурман В.Е. Руководство к решению задач по теории вероятностей и математической статистике. – М.: Высшая школа, 1979. – 400 с. (и более поздние издания)

\bibitem{9} 
Зубков А.М., Севастьянов Б.А., Чистяков В.П. Сборник задач по теории вероятностей. – М.: Наука, 1989. – 320 с.

\bibitem{10} 
Прохоров А.В., Ушаков В.Г., Ушаков Н.Г. Задачи по теории вероятностей. Основные понятия. Предельные теоремы. Случайные процессы. – М.: Наука, 1986. – 328 с.

\bibitem{11} 
Стоянов Й. Контрпримеры в теории вероятностей. – М.: Факториал, 1999. – 288 с.

\bibitem{12} 
Секей Г. Парадоксы в теории вероятностей и математической статистике. – М.: РХД, 2003. – 272 с.

\bibitem{13} 
Flajolet P., Sedgewick R. Analytic combinatorics. Cambridge University Press, 2008. 

\bibitem{14} 
Ledoux M. Concentration of measure phenomenon, Providence, RI, Amer. Math. Soc., 2001 (Math. Surveys Monogr. V. 89)

\bibitem{15} 
Алон Н., Спенсер Дж. Вероятностный метод. Бином, 2006.

\bibitem{16} 
Кендалл М., Моран П. Геометрические вероятности. М.: Наука, 1972.

\bibitem{17} 
Motwani R., Raghavan P. Randomized algorithms. Cambridge Univ. Press, 1995.

\bibitem{18} 
Cover T.M., Thomas J.A. Elements of Information theory. Wiley-Interscience, 2006. 

\bibitem{19} 
Durrett R. Probability: Theory and Examples. Cambridge Univ. Press, 2010.

\bibitem{20} 
Кац М. Вероятность и смежные вопросы в физике. М.: Мир, 1965.


\end{thebibliography}


