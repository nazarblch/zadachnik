\begin{center}

\textbf{ПРОГРАММА УЧЕБНОГО КУРСА  \\ 
«Основы теории вероятностей  \\
 и стохастических процессов»  \\}
 

\end{center}
 
 
Интуитивные предпосылки теории вероятностей. Множество элементарных исходов опыта, событие. Классическое и статистическое определение вероятности. Математическое определение вероятности. Алгебра и сигма-алгебра событий. Аксиомы теории вероятностей и следствия из них. Вероятностное пространство.

Теорема непрерывности вероятности. Теорема сложения вероятностей. Зависимые и независимые события. Условная вероятность события. Формула полной вероятности. Формула Байеса.

Случайная величина как измеримая функция. Функция распределения случайной величины. Дискретные и непрерывные случайные величины. Плотность распределения вероятностей.

Конкретные распределения случайных величин. Схема Бернулли, геометрическое и биномиальное распределение. Простейший поток событий и распределение Пуассона. Показательное, равномерное, нормальное, log-нормальное и отрицательно-биномиальное распределения. Бета-распределение и гамма-распределение.

Случайный вектор. Функция распределения случайного вектора. Зависимые и независимые случайные величины, условные законы распределения. Функции случайных величин. Невырожденное функциональное преобразование случайного вектора.
Интеграл Стилтьеса. Математическое ожидание и дисперсия случайной величины. Моменты случайной величины. Неравенство Ляпунова. Условное математическое ожидание. Корреляционная матрица случайного вектора. Коэффициент корреляции двух случайных величин.

Характеристическая функция и ее свойства. Связь моментов случайной величины с ее характеристической функцией. Разложение характеристической функции в ряд.
Сходимость последовательностей случайных величин с вероятностью единица (почти наверное), в среднем квадратичном, по вероятности, по распределению. Соотношение между различными типами сходимости.

Неравенство Чебышева. Закон больших чисел. Критерий Колмогорова. Теоремы Хинчина и Чебышева. Леммы Бореля-Кантелли. Усиленный закон больших чисел. Теорема Колмогорова и Бореля. Оценивание скорости сходимости частоты к вероятности в схеме Бернулли.

Интегральная и локальная теоремы Myавра-Лапласа. Дискретная поправка. Теорема Линдберга. Центральная предельная теорема для одинаково распределенных случайных величин. Условие Ляпунова. Теорема Гливенко.