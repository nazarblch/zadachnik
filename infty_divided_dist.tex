\subsection{ Безгранично делимые распределения }

\begin{problem}
Пусть с.в. $X$ имеет экспоненциальное распределение с параметром $\lambda$ ($X\in\Exp(\lambda)$), т.е. 
${\mathbb P}(X>x)=e^{-\lambda x},\; x\geqslant 0$. Покажите, что имеет место <<отсутствие последействия>>: 
$$
{\mathbb P}(X>x+y\,|\, X>x)={\mathbb P}(X>y) . 
$$
\end{problem}


\begin{problem}
Представьте, что вы владелец киоска по продаже мороженного (одного вида, скажем, пломбира) и хотите оценить, сколько мороженного $K(T)$ 
вам удастся продать за рабочий день $T$. Имеет место формула 
$$
K(T)=\max\Bigl\{ n:\; \sum\limits_{k=1}^{n} X_k<T \Bigr\} , 
$$
где $X_1, X_2, X_3,\ldots$ --- независимые одинаково распределенные по закону $\Exp(\lambda)$ с.в. ($X_k$ интерпретируется как время между 
$k-1$ и $k$ сделкой (продажей)). Покажите, что 
\begin{enumerate}
\item вероятность ${\mathbb P}(K(T+t)-K(T)=k)$, где $t\geqslant 0$ и $k=0,1,2,\ldots$, не зависит от $T\geqslant 0$; 

\item $\forall n\geqslant 1$, $0\leqslant t_1\leqslant t_2\leqslant \ldots\leqslant t_n$ 
$\quad\to\quad \bigl\{ K(t_k)-K(t_{k-1})\bigr\}_{k=1}^{n}$ --- независимые с.в.; 

\item ${\mathbb P}(K(t)>1)=o(t)$, $t>0$; 

\item $\forall n\geqslant 1$, $0\leqslant t_1\leqslant t_2\leqslant \ldots\leqslant t_n$, 
$0\leqslant k_1\leqslant k_2\leqslant \ldots\leqslant k_n$, $k_1,\ldots, k_n\in {\mathbb N}\cup \{ 0\}$ 
\begin{multline*}
{\mathbb P}(K(t_1) = k_1,\ldots, K(t_n)=k_n)= \\
e^{-\lambda t_1} \frac{(\lambda t_1)^{k_1}}{k_1!}\cdot 
e^{-\lambda(t_2-t_1)} \frac{(\lambda(t_2-t_1))^{k_2-k_1}}{(k_2-k_1)!}\cdot \ldots \\
\cdot e^{-\lambda(t_n-t_{n-1})}\frac{(\lambda(t_n-t_{n-1}))^{k_n-k_{n-1}}}{(k_n-k_{n-1})!} , 
\end{multline*}
в частности, $K(T)\in \Po(\lambda T)$. 
\end{enumerate}
\end{problem}



\begin{problem}
В течение рабочего дня фирма осуществляет $K(T)\in \Po(\lambda T)$ сделок ($K(T)$ --- с.в., имеющая распределение Пуассона с параметром 
$\lambda T = 100 [\text{сделок/час}] * 10 [\text{часов}]$). Каждая сделка приносит доход $V_n\in R[a,b]$ ($V_n$ --- с.в., имеющая 
равномерное распределение на отрезке $[a,b]=[\$ 10, \$ 100]$, $n$ --- номер сделки). Считая, что $K$, $V_1$, $V_2$, $\ldots$ --- 
независимые в совокупности с.в., найдите математическое ожидание и дисперсию выручки за день $Q(T)=\sum\limits_{k=1}^{K(T)} V_k$. Докажите соотношение для характеристической функции $Q(T)$
$$
\varphi_{Q(T)}(t)=\exp\{ \lambda T(\varphi_{V_k}(t)-1)\} 
$$
   
\end{problem}

\begin{ordre}

Примените формулу для условного математического ожидания  
$$
{\mathbb E}X= \Exp{( {\mathbb E}(X|Y) )} 
$$
Установите справедливость следующего соотношения.
$$
\Var X=\Var({\mathbb E}(X|Y))+{\mathbb E}(\Var(X|Y)) . 
$$

\end{ordre}


\begin{problem}
В течение трех лет фирма из предыдущей задачи работала $N=1000$ дней (длина рабочего дня и параметры спроса не менялись). 
Оцените распределение с.в. 
$$Q^N=\sum\limits_{k=1}^{N} Q_k(T), 
$$
где $Q_k(T)$ --- выручка за $k$–ый день. Верно ли, что с.в. $Q^N$ и $Q_k(NT)$ одинаково распределены? 
\end{problem}


\begin{problem}
В течение года фирма осуществляет $K\in \Po(\lambda)$ сделок ($K$ -- с.в., имеющая распределение Пуассона с параметром  $\lambda=100000$ 
[сделок]). Каждая сделка приносит фирме прибыль $V_n\in R[a,b]$ ($V_n$ -- с.в., имеющая равномерное распределение на отрезке 
$[a,b]=[-50\$,100\$]$, $n$ -- номер сделки). Считая, что $K$, $V_1$, $V_2$, $\ldots$ --- независимые в совокупности с.в., оцените 
\begin{equation}
\label{ProbRatio}
\left. {\mathbb P}\Bigl(\sum\limits_{n=1}^{K} V_n\leqslant 0\Bigr)\right/{\mathbb P}\Bigl(\sum\limits_{n=1}^{K} V_n>0\Bigr) . 
\end{equation}
\end{problem}

\begin{problem}[пуассоновский поток событий]
Рассмотрим интервал 
$\left[ {-N,N} \right]$ и бросим на него независимо и случайно (точнее 
равномерно) $M=\left[ {\rho N} \right]$ точек, где $\rho >0$ - некоторая 
константа, называемая плотностью. Легко вычислить биномиальную вероятность 
$P_{N,M} \left( {k,I} \right)$ того, что в конечный интервал $I\subset 
\left[ {-N,N} \right]$ попадет ровно $k$ точек. Покажите, что $P_{N,M} 
\left( {k,I} \right)$ стремится при $N\to \infty $ к пуассоновскому 
выражению:
\[
P\left( {k,I} \right)\mathop =\limits^{def} \mathop {\lim }\limits_{N\to 
\infty } P_{N,M\left( N \right)} \left( {k,I} \right)=\frac{\left( {\rho 
\left| I \right|} \right)^k}{k!}e^{-\rho \left| I \right|},
\quad
k=0,1,...
\]
Покажите также, что если $I_1 ,I_2 \subset \left[ {-N,N} \right]$ и $I_1 
\cap I_2 =\emptyset $, то
\[
P\left( {k_1 ,I_1 ;k_2 ,I_2 } \right)\mathop =\limits^{def} \mathop {\lim 
}\limits_{N\to \infty } P_{N,M\left( N \right)} \left( {k_1 ,I_1 ;k_2 ,I_2 } 
\right)=P\left( {k,I_1 } \right)P\left( {k,I_2 } \right).
\]
\end{problem}




\begin{problem}[процесс Леви]
Покажите, что если $X(t)$ --- процесс Леви, то: 
$$
\exists ! (b,c,\nu(dx)): \; 
c\geqslant 0, \; \int_{-\infty}^{\infty} \max(1,x^2)\, \nu(dx)<\infty, \; \nu(dx)\geqslant 0 : 
$$
\[
\forall t\geqslant 0 \;  \varphi_{X(t)}(\mu)={\mathbb E}e^{i\mu X(t)}=  
\]
\[
\exp \Bigl\{ t\Bigl[ i\mu b-c\mu^2\left.\right/2+
\int_{-\infty}^{\infty} \bigl( e^{i\mu x}-1-i\mu x I(|x|\leqslant 1) \bigr)\, \nu(dx) 
\Bigr] \Bigr\} . 
\]

\end{problem}

\begin{remark}
При помощи данного свойства можно обосновать представление безгранично делимой случайной величины в виде смеси распределений: 
\[
\alpha + \beta N(m(t), \sigma^2(t)) + \gamma Q(t),
\]
\noindent где Q(t) - обобщенный пуассоновский процесс.
\end{remark}


\begin{problem}

Рассмотрим простую и классическую схему блуждания точки по прямой, соответствующую правилам игры в орлянку:
\[\eta (0)=0,\] 
\[\eta (t+1)=\left\{\begin{array}{cc} {\eta (t)+1,} & {p={1\mathord{\left/ {\vphantom {1 2}} \right. \kern-\nulldelimiterspace} 2} } \\ {\eta (t)-1,} & {p={1\mathord{\left/ {\vphantom {1 2}} \right. \kern-\nulldelimiterspace} 2} } \end{array}\right. .\] 
Занумеруем в порядке возрастания все моменты времени, когда $\eta (t)=0$. Получим с вероятностью 1 бесконечную последовательность $0=\tau _{0} <\tau _{1} <\tau _{2} <...$ Рассмотрим разности $\xi _{i} =\tau _{i} -\tau _{i-1} $, $i=1,2,...$ -- последовательность независимых одинаково распределенных с.в.

\textbf{а)} Найдите распределение $\xi _{i} =\tau _{i} -\tau _{i-1} $, т.е. $P\left\{\xi _{i} =2m\right\}$

\textbf{б)} Покажите, что математическое ожидание с.в. $\xi _{i} =\tau _{i} -\tau _{i-1} $ равно бесконечности. Этот результат можно проинтерпретировать так: среднее время до первого возвращения блуждания в 0 бесконечно.

Тем не менее суммы $\tau _{n} =\sum _{i=1}^{n}\xi _{i}  $ при надлежащей нормировке подчинены предельному распределению: 
\[\mathop{\lim }\limits_{n\to \infty } P\left\{\frac{2\tau _{n} }{\pi n^{2} } <z\right\}=\left\{\begin{array}{cc} {\frac{1}{\sqrt{2\pi } } \int _{0}^{z}e^{-\frac{1}{2x} } x^{-\frac{3}{2} }  dx,} & {z>0} \\ {0,} & {z<0} \end{array}\right. .\] 
Из теоремы о каноническом представлении устойчивых законов (теорема Леви-Хинчина): Для того чтобы функция распределения была устойчивой, необходимо и достаточно, чтобы логарифм ее характеристической функции представлялся формулой:
\[\ln \varphi (t)=i\gamma t-c|t|^{\alpha } \left(1+i\beta \frac{t}{|t|} \omega (t,\alpha )\right),\] 
где $\gamma $ -- любое действительное число, $-1\le \beta \le 1$, $0<\alpha \le 2$, $c\ge 0$ и
\[\omega (t,\alpha )=\left\{\begin{array}{cc} {tg\left(\frac{\pi }{2} \alpha \right),} & {\alpha \ne 1,} \\ {\frac{2}{\pi } \ln |t|,} & {\alpha =1} \end{array}\right. ,\] 
следует, что такое распределение соответствует каноническому представлению с $\alpha ={1\mathord{\left/ {\vphantom {1 2}} \right. \kern-\nulldelimiterspace} 2} $, $\beta =1$, $\gamma =0$, $c=1$, и принадлежит семейству кривых Пирсона (показано Н. В. Смирновым).

\end{problem}


