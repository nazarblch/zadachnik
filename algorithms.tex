\section{Вероятностные методы в Computer Science}

\subsection{Рандомизированные алгоритмы}

\begin{problem}

Требуется определить начиная с какого этажа брошенный с балкона 100-этажного здания стеклянный шар разбивается. В наличии имеется два таких шара. Предложить метод нахождения граничного этажа, минимизирующий математическое ожидание числа бросков. Рассмотреть случай большего числа шаров.  

\end{problem}


\begin{problem}
Имеется неизвестное число от 1 до $n$ ($n\ge 2$). Разрешается задавать любые вопросы с ответами ДА/НЕТ. При этом при ответе ДА мы платим 1 рубль, при ответе НЕТ -- 2 рубля. Сколько необходимо и достаточно заплатить для отгадывания числа?
\end{problem}

\begin{problem}
При нахождении скрытых величин генератора методом максимизации правдоподобия решается следующая оптимизационная задача.

\[
\underset{C}{\max} \; P(C \vert X) \varpropto P(X \vert C),
\]

\noindent где $X$ - множество известных сгенерированных случайных величин, $C$ - множество скрытых  сгенерированных величин.

Генератор представляет собой набор отношений (связей) между величинами из множеств $X$ и $C$. 
Каждая связь является распределением, участвующем в процессе генерации. 

Для решения указанной оптимизационной задачи предлагается подобрать распределение $q(C \vert \lambda)$ схожее с $P(C \vert X)$ по типам зависимостей. Далее, минимизируя расстояние Кульбака-Лейблера $\sum q\left(x\right)\log \frac{q\left(x\right)}{P\left(x\right)}  $ между распределениями $P$ и $q$, найти значения вектора параметров $\lambda$. Вычислив $\lambda$, можно найти сами скрытые переменные $C$ ввиду простой структуры распределения $q(C \vert \lambda)$, которое обычно задается в виде:

\[
q(C \vert \lambda) = \prod q_i(C_i \vert \lambda_i) 
\]        

Рассмотрим пример конкретный пример генератора.

\[
\theta_d \sim Dirichlet(\alpha) \; d = \overline{1,m}
\]
\[
\phi_t \sim Dirichlet(\beta) \; t = \overline{1,K}
\]
\[
z_{di} \sim Categorical(\theta_d) \; i = \overline{1,n_d}
\]
\[
w_{di} \sim Categorical(\phi_{z_{di}}) \; i = \overline{1,n_d},
\]
 
где $W = X$, $C = (\theta, \phi, Z)$, 

$\alpha, \beta, m, n_d, K$ - известные параметры генератора.  

Пусть 
\[
q(\theta \phi, Z \vert \lambda) = \underset{d}{\prod}\underset{t}{\prod}\underset{i}{\prod} Dir(\theta_d | \lambda_{d}) Dir(\phi_t | \eta_{t}) Cat(z_{di} | \lambda_{di})
\]

Требуется найти конечный вид системы уравнений 

\[
\frac{\partial \sum q\left(x\right)\log \frac{q\left(x\right)}{P\left(x\right)} } {\partial \lambda} = 0
\]  

для данного генератора.

\end{problem}

\begin{problem}
Пять философов сидят за круглым столом. В центре стола находится чаша со
спагетти. Между каждой парой соседних философов лежит вилка. Философы чередуют размышления с приемами пищи. Каждый философ может либо есть, либо размышлять. Приём пищи не ограничен
количеством оставшихся спагетти — подразумевается бесконечный запас. Однако
для того, чтобы вытащить спагетти из чаши и донести их до рта философу требуются
две вилки. Таким образом, философ может есть только взяв вилки слева и справа от
себя. 

Каждый философ может взять вилку рядом с ним (если она доступна), или положить
- если он уже держит её. Взятие каждой вилки и возвращение её на стол являются
раздельными действиями, которые должны выполняться одно за другим. Если
требуемая вилка занята соседом, голодный философ вынужден ждать - он не
может вернуться к размышлениям, не поев. После окончания еды философ кладет
обе вилки на стол для того, чтобы ими могли воспользоваться другие философы.

Время одного приема пищи одним философом равномерно распределено на отрезке [0, a]. 
Время одного размышления равномерно распределено на отрезке [0, b].

Данный процесс подвержен взаимной блокировке (Dead Lock): например, если каждый возьмет по левой вилке, то начнется вечное голодание. Для избежания блокировки каждый философ ложит первую вилку, если за время t после ее взятия вторая не освободилась.

Требуется определить распределение времени t, минимизирующее среднее время ожидания после размышления и перед приемом пищи.

\end{problem}



\subsection{Кодирование}
\subsection{Теория игр}
\begin{problem}
Один игрок прячет (зажимает в кулаке) одну или две монеты достоинством 10 рублей. Другой игрок должен отгадать сколько денег у первого спрятано. Если отгадывает, то получает деньги, если нет -- платит 15 рублей. Каковы ``должны'' быть стратегии игроков при многократном повторении игры?

\end{problem}