\section{Алгоритмы}

\subsection{Вероятностный анализ алгоритмов}

\begin{problem}

Требуется определить начиная с какого этажа брошенный с балкона 100-этажного здания стеклянный шар разбивается. В наличии имеется два таких шара. Предложить метод нахождения граничного этажа, минимизирующий математическое ожидание числа бросков. Рассмотреть случай большего числа шаров.  

\end{problem}

\begin{problem}
Один игрок прячет (зажимает в кулаке) одну или две монеты достоинством 10 рублей. Другой игрок должен отгадать сколько денег у первого спрятано. Если отгадывает, то получает деньги, если нет -- платит 15 рублей. Каковы ``должны'' быть стратегии игроков при многократном повторении игры?

\end{problem}

\begin{problem}
Имеется неизвестное число от 1 до $n$ ($n\ge 2$). Разрешается задавать любые вопросы с ответами ДА/НЕТ. При этом при ответе ДА мы платим 1 рубль, при ответе НЕТ -- 2 рубля. Сколько необходимо и достаточно заплатить для отгадывания числа?
\end{problem}

\begin{problem}
При нахождении скрытых величин генератора методом максимизации правдоподобия решается следующая оптимизационная задача.

\[
\underset{C}{\max} \; P(C \vert X) \varpropto P(X \vert C),
\]

где $X$ - множество известных сгенерированных случайных величин, $C$ - множество скрытых  сгенерированных величин.

Генератор представляет собой набор отношений (связей) между величинами из множеств $X$ и $C$. 
Каждая связь является распределением, участвующем в процессе генерации. 

Для решения указанной оптимизационной задачи предлагается подобрать распределение $q(C \vert \lambda)$ схожее с $P(C \vert X)$ по типам зависимостей. Далее, минимизируя расстояние Кульбака-Лейблера $\sum q\left(x\right)\log \frac{q\left(x\right)}{P\left(x\right)}  $ между распределениями $P$ и $q$, найти значения вектора параметров $\lambda$. Вычислив $\lambda$, можно найти сами скрытые переменные $C$ ввиду простой структуры распределения $q(C \vert \lambda)$, которое обычно задается в виде:

\[
q(C \vert \lambda) = \prod q_i(C_i \vert \lambda_i) 
\]        

Рассмотрим пример конкретный пример генератора.

\[
\theta_d \sim Dirichlet(\alpha) \; d = \overline{1,m}
\]
\[
\phi_t \sim Dirichlet(\beta) \; t = \overline{1,K}
\]
\[
z_{di} \sim Categorical(\theta_d) \; i = \overline{1,n_d}
\]
\[
w_{di} \sim Categorical(\phi_{z_{di}}) \; i = \overline{1,n_d},
\]
 
где $W = X$, $C = (\theta, \phi, Z)$, 

$\alpha, \beta, m, n_d, K$ - известные параметры генератора.  

Пусть 
\[
q(\theta \phi, Z \vert \lambda) = \underset{d}{\prod}\underset{t}{\prod}\underset{i}{\prod} Dir(\theta_d | \lambda_{d}) Dir(\phi_t | \eta_{t}) Cat(z_{di} | \lambda_{di})
\]

Требуется найти конечный вид системы уравнений 

\[
\frac{\partial \sum q\left(x\right)\log \frac{q\left(x\right)}{P\left(x\right)} } {\partial \lambda} = 0
\]  

для данного генератора.

\end{problem}

\begin{problem}
Пять философов сидят за круглым столом. В центре стола находится чаша со
спагетти. Между каждой парой соседних философов лежит вилка. Философы чередуют размышления с приемами пищи. Каждый философ может либо есть, либо размышлять. Приём пищи не ограничен
количеством оставшихся спагетти — подразумевается бесконечный запас. Однако
для того, чтобы вытащить спагетти из чаши и донести их до рта философу требуются
две вилки. Таким образом, философ может есть только взяв вилки слева и справа от
себя. 

Каждый философ может взять вилку рядом с ним (если она доступна), или положить
- если он уже держит её. Взятие каждой вилки и возвращение её на стол являются
раздельными действиями, которые должны выполняться одно за другим. Если
требуемая вилка занята соседом, голодный философ вынужден ждать - он не
может вернуться к размышлениям, не поев. После окончания еды философ кладет
обе вилки на стол для того, чтобы ими могли воспользоваться другие философы.

Время одного приема пищи одним философом равномерно распределено на отрезке [0, a]. 
Время одного размышления равномерно распределено на отрезке [0, b].

Данный процесс подвержен взаимной блокировке (Dead Lock): например, если каждый возьмет по левой вилке, то начнется вечное голодание. Для избежания блокировки каждый философ ложит первую вилку, если за время t после ее взятия вторая не освободилась.

Требуется определить распределение времени t, минимизирующее среднее время ожидания после размышления и перед приемом пищи.

\end{problem}


\subsection{Генерация случайных величин}

\begin{problem}
Покажите, что если с.в. $\eta $ равномерно распределена на отрезке $\left[0,1\right]$, то с.в. $\xi =F^{-1} \left(\eta \right)$ имеет функцию распределения $F\left(x\right)$. Предполагается, что $F\left(x\right)$ непрерывна и строго монотонна. Как выглядит формула для моделирования с.в. из показательного распределения с функцией распределения $F\left(x\right)=\left(1-e^{-\lambda x} \right)I\left\{x>0\right\}$? (Стандартный способ моделирования с.в. -- метод обратной функции)
\end{problem}

\begin{problem}

Пусть $\xi $ распределена на $\left[0,1\right]$ с плотностью $f_{\xi } (x)$, представимой в виде степенного ряда $\sum _{k=0}^{\infty }a_{k} x^{k}  $ с $a_{k} \ge 0$. Положим $p_{k} ={a_{k} \mathord{\left/ {\vphantom {a_{k}  (k+1)}} \right. \kern-\nulldelimiterspace} (k+1)} $. Тогда $f_{\xi } (x)=\sum _{k=0}^{\infty }p_{k} \cdot (k+1)x^{k}  $. Примените метод суперпозиции для моделирования с.в. $\xi $.

\begin{ordre}
Метод суперпозиции:

\noindent 1) Разыгрывается значение дискретной с.в., принимающей значения $k=0,1,2,...$ с вероятностями $p_{k} $.

\noindent 2) Моделируется с.в. с функцией распределения $F_{k} (x)$ (например, методом обратной функции).

\end{ordre}

\end{problem}

\begin{problem}
(Теорема Бернштейна) 

\textbf{а)} С помощью неравенства Чебышёва установите следующий результат из анализа: 

\[
\forall \; \; f\in C\left[0,1\right]\to \left\| f_{n} -f\right\| _{C\left[0,1\right]} \xrightarrow[{n\to \infty }]{} 0,
\] 

\[
f_{n} \left(x\right)=\sum_{k=0}^{n}f\left(\frac{k}{n} \right) C_{n}^{k} x^{k} \left(1-x\right)^{n-k} 
\]

\textbf{б)} Исходя из предыдущей задачи и п. а) предложите способ генерирования распределения с.в. $\xi $, имеющей плотность $f_{\xi } \left(x\right)$ с финитным носителем, для определенности, пусть носителем будет отрезок $\left[0,1\right]$.
\end{problem}

\begin{problem}
Как с помощью с.в. $\xi $, равномерно распределенной на отрезке $\left[0,1\right]$ ($\xi \in R\left[0;1\right]$), и симметричной монетки построить с.в. $X$, имеющую плотность распределения $f_{X} (x)=\frac{1}{4} \left(\frac{1}{\sqrt{x} } +\frac{1}{\sqrt{1-x} } \right)$, $x\in \left[0,1\right]$?
\end{problem}

\begin{problem}
(Метод фон Неймана) 

Пусть с.в. $\xi $ распределена на отрезке$\left[a,b\right]$, причем ее плотность распределения ограничена: $\mathop{\max }\limits_{x\in \left[a;b\right]} f_{\xi } (x)=C<\infty $. Пусть с.в. $\eta _{1} $, $\eta _{2} $, \dots  -- независимы и равномерно распределены на $\left[0,1\right]$, $X_{i} =a+\left(b-a\right)\eta _{2i-1} $, $Y_{i} =C\eta _{2i} $, $i=1,2,...$, т.е. пары $\left(X_{i} ,Y_{i} \right)$ независимы и равномерно распределены в прямоугольнике $\left[a,b\right]\times \left[0,C\right]$. Обозначим через $\nu $ номер первой точки с координатами $\left(X_{i} ,Y_{i} \right)$, попавшей под график плотности $f_{\xi } (x)$, т.е. $\nu =\min \left\{i:\quad Y_{i} \le f_{\xi } (X_{i} )\right\}$. Положим $X_{\nu } =\sum _{n=1}^{\infty }X_{n} I\left\{\nu =n\right\} $.

\textbf{а)} Покажите, что с.в. $X_{\nu } $ распределена также как $\xi $.

\textbf{б)} Сколько в среднем точек $\left(X_{i} ,Y_{i} \right)$ потребуется «вбросить» в прямоугольник $\left[a,b\right]\times \left[0,C\right]$ для получения одного значения $\xi $?
\end{problem}

\subsection{Кодирование}