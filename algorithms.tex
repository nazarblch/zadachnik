\section{Вероятностные методы в Computer Science}

\subsection{Рандомизированные алгоритмы}

\begin{comment}
\begin{problem}

Пять философов сидят за круглым столом. В центре стола находится чаша со
спагетти. Между каждой парой соседних философов лежит вилка. Философы чередуют размышления с приемами пищи, не отвлекаясь на второстепенные занятия. Однако
для того, чтобы вытащить спагетти из чаши и донести их до рта философу требуются
две вилки. Каждый философ может взять вилку рядом с ним (если она доступна), или положить - если он уже держит её. Если требуемая вилка занята соседом, голодный философ вынужден ждать - он не может вернуться к размышлениям, не поев. После окончания еды философ кладет обе вилки на стол.

Время одного приема пищи одним философом равномерно распределено на отрезке [0, a]. 
Время одного размышления равномерно распределено на отрезке [0, b].

Данный процесс подвержен взаимной блокировке (Dead Lock): например, если каждый возьмет по левой вилке, то начнется вечное голодание. Для избежания блокировки каждый философ кладет первую вилку, если за время t после ее взятия вторая не освободилась.

Требуется определить распределение времени t, минимизирующее среднее время ожидания после размышления и перед приемом пищи.

\end{problem}
\end{comment}

\begin{problem}
\noindent Алгоритм быстрой сортировки основан на парадигме «разделяй и властвуй». Выбирается из элементов массива опорный элемент, относительно которого переупорядочиваются все остальные элементы. Желательно выбрать опорный элемент близким к значению медианы, чтобы он разбивал список на две примерно равные части. Переупорядочивание элементов относительно опорного, происходит так, что все переставленные элементы, лежащие левее опорного, меньше его, а те, что правее -- больше или равны опорному. Далее процедура быстрой сортировки рекурсивно применяется к левому и правому списку для их упорядочивания по отдельности.

Наихудшие входные данные для описанного алгоритма быстрой сортировки (предполагается, что в качестве опорного элемента выбирается последний элемент обрабатываемого массива) -- элементы уже упорядоченные по возрастанию. 
Откуда следует, что асимптотика времени работы быстрой сортировки в худшем случае $\Theta (n^{2} )$.

Оценить время работы алгоритма быстрой сортировки в среднем. 

\begin{ordre}
Получить рекуррентное соотношение для математического ожидания времени работы, введя индикаторную функцию позиции опорного элемента. 
Воспользоваться соотношением:
\[\begin{array}{l} {\sum _{k=1}^{n-1}k\log k \le \log \frac{n}{2} \sum _{k=1}^{\left\lceil \frac{n}{2} \right\rceil -1}k +\log n\sum _{k=\left\lceil \frac{n}{2} \right\rceil }^{n-1}k =} \\ {=\frac{n(n-1)}{2} \log n-\frac{\left\lceil \frac{n}{2} \right\rceil \left(\left\lceil \frac{n}{2} \right\rceil -1\right)}{2} \le \frac{1}{2} n^{2} \log n-\frac{n^{2} }{8} } \end{array}\] 
 
\end{ordre}

Показать неулучшаемость оценки для произвольного алгоритма сортировки. Привести способ сортировки с асимптотикой $O(n \log n)$ в худшем случае.

\end{problem}


\begin{problem}[Задача поиска k-ой порядковой статистики].

Рекурсивное применение процедуры, основанной на методе быстрой сортировки, позволяет быстро (в среднем) находить k-ую порядковую статистику. Задача вычисления порядковых статистик состоит в следующем: дан список (массив) из $n$ чисел, необходимо найти значение, которое стоит в k-ой позиции в отсортированном в возрастающем порядке списке. 

\noindent Модифицируем алгоритм быстрой сортировки:


 Выбираем опорный элемент. Делим список на две группы. В первой -- элементы меньше опорного, во второй -- больше либо равны.
 Если размер (число элементов) первой группы больше либо равен k, то к ней снова применяется эта процедура. Иначе -- вызывается процедура для второй группы.
 
\noindent Покажите, используя ту же технику, что и при анализе в среднем алгоритма быстрой сортировки, что среднее время работы такого алгоритма линейно.

\begin{ordre}

\noindent Покажите, что выполняется оценка среднего времени работы алгоритма:

\[
E[T(n)]\le E\left\{\sum _{k=1}^{n}X_{k} \left[T\left(\max (k-1,n-k)\right)+O(n)\right] \right\} 
\]
\[ \le \frac{2}{n} \sum _{k=\left\lfloor \frac{n}{2} \right\rfloor }^{n-1}E[T(k)] +O(n)
\]



\end{ordre}
\end{problem}

\begin{remark}

Пусть $a\ge 1$ и $b>1$- константы, $f(n)$ - произвольная функция, $T(n)$ - функция, определенная на множестве неотрицательных целых чисел с помощью рекуррентного соотношения: $T(n)=aT\left(\frac{n}{b} \right)+f(n)$

\noindent где выражение $\frac{n}{b} $интерпретируется либо как $\left\lfloor \frac{n}{b} \right\rfloor $, либо как$\left\lceil \frac{n}{b} \right\rceil $. Тогда асимптотическое поведение функции $T(n)$ можно выразить следующим образом. 

\begin{enumerate}
\item  Если $f(n)=O(n^{\log _{b} a-\varepsilon } )$ для некоторой константы $\varepsilon >0$, то $T(n)=\Theta \left(n^{\log _{b} a} \right)$.

\item  Если $f(n)=\Theta \left(n^{\log _{b} a} \right)$, то $T(n)=\Theta \left(n^{\log _{a} b} \lg n\right)$.

\item  Если $f(n)=\Omega (n^{\log _{b} a+\varepsilon } )$ для некоторой константы $\varepsilon >0$, и для некоторой константы $c<1$ и достаточно больших n выполнено: $af\left(\frac{n}{b} \right)\le cf(n)$, то $T(n)=\Theta \left(f(n)\right)$.
\end{enumerate}

\end{remark}




\begin{problem}

Даны три матрицы $A,B,C$размера $n\times n$. Требуется проверить равенство $AB=C$.

Простой детерминированный алгоритм перемножает матрицы $A$, $B$ и сравнивает результат с $C$. Время работы такого алгоритма при использовании обычного перемножения матриц составляет $O(n^{3} )$, при использовании быстрого - $O(n^{2,376} )$. Вероятностный алгоритм Фрейвалда с односторонней ошибкой проверяет равенство за время $O(n^{2} )$.

Описание вероятностного алгоритма:

\begin{enumerate}
\item \textbf{ }взять случайный вектор $x\in \left\{0,1\right\}^{n} $

\item  вычислить $y=Bx$

\item  вычислить $z=Ay$

\item  вычислить $t=Cx$

\item  если $z=t$ вернуть «да», иначе «нет».
\end{enumerate}

Покажите, что для предъявленного алгоритма выполняется 
\[\begin{array}{l} {P\left\{z=t \vert AB=C\right\}=1,} \\ {P\left\{z \neq t \vert AB\ne C\right\}\ge {\raise0.7ex\hbox{$ 1 $}\!\mathord{\left/ {\vphantom {1 2}} \right. \kern-\nulldelimiterspace}\!\lower0.7ex\hbox{$ 2 $}} .} \end{array}\] 

\begin{ordre} (добавить что надо несколько раз)
Оценить вероятность ошибочного ответа на одной ненулевой строке матрицы $D = AB - C$. 
\end{ordre}

\end{problem}

\begin{comment}

\begin{problem}

Требуется сравнить две битовые строки $a,b$, потребовав как можно меньше информации от обеих строк. Основная идея -- сравнивать не сами строки, а функции от них. Так сравниваются $a\; mod\; p$ и $b\; mod\; p$, для некоторого простого числа $p$. Для этого требуется передать $2\log _{2} p$ бит информации.

Описание алгоритма сравнения строк:

\begin{enumerate}
\item  Пусть $\left|a\right|=\left|b\right|=n$, $N=n^{2} \log _{2} n^{2} $

\item  Выбираем случайное простое число $p$из интервала $\left[2..N\right]$ 

\item  Выдать «да», если $a\; mod\; p=b\; mod\; p\Leftrightarrow (a-b)\equiv 0\; mod\; p$, иначе выдать «нет».
\end{enumerate}

\noindent  Обоснуйте выбор именно простого числа на шаге 2 и предложите способ его генерации.  

\noindent Покажите что,

\[\begin{array}{l} {P\left\{(a-b)\equiv 0,mod(p)|a=b\right\}=1,} \\ {P\left\{(a-b)\equiv 0,mod(p)|a\ne b\right\}=O\left({\raise0.7ex\hbox{$ 1 $}\!\mathord{\left/ {\vphantom {1 n}} \right. \kern-\nulldelimiterspace}\!\lower0.7ex\hbox{$ n $}} \right),} \end{array}\] 
 
При этом необходимое количество переданных бит равно $O\left(\log _{2} n\right)$.

\begin{ordre}

Воспользоваться асимптотическим законом распределения простых чисел:
\[\mathop{\lim }\limits_{n\to \infty } \frac{\pi \left(n\right)}{{\raise0.7ex\hbox{$ n $}\!\mathord{\left/ {\vphantom {n \ln n}} \right. \kern-\nulldelimiterspace}\!\lower0.7ex\hbox{$ \ln n $}} } =1,\] 
где $\pi \left(n\right)$ - функция распределения простых чисел, равная количеству простых чисел, не превосходящих $n$.

\end{ordre}

\end{problem}

\begin{remark} (мелким шрифтом)
В предыдущей задаче потребуется проверка простоты числа. 
Согласно малой теореме Ферма, если $N$ - простое число и целое $a$ не делится на $N$, то  
\[a^{N-1} \equiv 1\; mod\; N                         \; \; \;            \left(*\right)\] 

Отсюда следует, что если при каком-то $a$ сравнение $\left(*\right)$ нарушается, то можно утверждать, что $N$ - составное. 
К сожалению,  простой вариант подбора $a$ не всегда позволяет эффективно выявить составное число. Имеются составные числа $N$, обладающие свойством $\left(*\right)$ для любого целого $a$ с условием $\left(a,N\right)=1$ ($a$ и $N$ - взаимно простые). Такие числа называются числами Кармайкла.

В 1976г. Миллер предложил заменить проверку $\left(*\right)$ проверкой несколько иного условия. Если $N$ - простое число, то $N-1=2^{s} t$, где $t$ нечетно, то согласно малой теоремы Ферма для каждого a с условием $\left(a,N\right)=1$ хотя бы одна из скобок в произведении 
\[\left(a^{t} -1\right)\left(a^{t} +1\right)\left(a^{2t} +1\right)\times \ldots \times \left(a^{2^{s-1} t} +1\right)=a^{N-1} -1\] 
делится на $N$. 

Пусть $N$ - нечетное составное число, $N-1=2^{s} t$, где \textbf{$t$ }нечетно. Назовем целое число $a$, $1<a<N$ «выявляющим» для $N$, если нарушается одно из двух условий:

I) $N$ не делится на $a$

II) $a^{t} \equiv 1\; mod\; N$ или существует целое $k$, $0\le k<s$ такое, что 

$a^{2^{k} t} \equiv -1\; mod\; N$.

Если $N$ составное число, то согласно теореме Рабина  существует не менее $\frac{3}{4} \left(N-1\right)$  выявляющих чисел.

\end{remark}

\begin{problem}

Пусть $f(x_{1} ,...,x_{n} )=C_{1} \vee \cdots \vee C_{m} $ - булева формула в дизъюнктивной нормальной форме (ДНФ), где каждая скобка $C_{i} $ - есть конъюнкция $L_{1} \wedge \cdots \wedge L_{k_{i} } $ $k_{i} $ литералов (литерал есть либо переменная, либо ее отрицание). Набор значений переменных $a=(a_{1} ,...,a_{n} )$ называется выполняющим для $f$, если $f(a_{1} ,...,a_{n} )=1$. Требуется найти число выполняющих наборов для данной ДНФ.

\noindent $V$ - множество всех двоичных наборов длины $n$.

\noindent $G$ - множество выполняющих наборов.


\noindent  Проведем $N$ независимых испытаний:

\noindent Выбираем случайно $v_{i} \in V$ ( в соответствии с равномерным распределением).
\noindent $y_{i} =f(v_{i} )$. Заметим, что $P\left\{y_{i} =1\right\}=\frac{\left|G\right|}{\left|V\right|} =p$.
Рассмотрим сумму независимых случайных величин $Y=\sum _{i=1}^{N}y_{i}  $. В качестве аппроксимации $\left|G\right|$ возьмем величину $\frac{Y}{N} \left|V\right|$.

\noindent Оцените необходимое число испытаний $N$ как функцию от $|V|$, $|G|$ и точности аппроксимации $\varepsilon$. 

\begin{ordre}
Докажите следующее утверждение. Пусть $X_{1} ,...,X_{n} $ - независимые случайные величины, принимающие значения 0 или 1, при этом $P\left\{X_{i} =1\right\}=p,\quad P\left\{X_{i} =0\right\}=1-p$. Тогда для $X=\sum _{i=1}^{N}X_{i}  $ и для любого $0<\delta <1$, выполнены неравенства
\[\begin{array}{l} {P\left\{X>(1+\delta )EX\right\}\le e^{-\frac{\delta ^{2} }{3} EX} } \\ {P\left\{X<(1-\delta )EX\right\}\le e^{-\frac{\delta ^{2} }{2} EX} } \end{array}\] 
\end{ordre}

\end{problem}


\begin{problem}

(Задача о покрытии). Дано конечное множество из m элементов и система его подмножеств $S_{1} ,...,S_{n} $. Требуется найти минимальную по числу подмножеств подсистему $S_{1} ,...,S_{n} $, покрывающую все множество объектов. 

\noindent Сформулируем ее в терминах булевых матриц и целочисленного линейного программирования:
\[\left\{\begin{array}{l} {cx\to \min ,} \\ {Ax\ge b,} \\ {\forall j\; x_{j} \in \{ 0,1\} .} \end{array}\right. \] 
Здесь переменные $x_{1} ,...,x_{n} $ соответствуют включению подмножеств $S_{1} ,...,S_{n} $ в решение-покрытие, матрица $A$ - матрица инцидентности, $c=(1...1)^{T} \in {\mathbb R}^{n} ,\quad b=(1...1)^{T} \in {\mathbb R}^{m} $ - векторы стоимости и ограничений.

\noindent Пусть элементы матрицы инцидентности -- независимые случайные величины с бернулевским распределением:$P\{ a_{ij} =1\} =p,$ $P\{ a_{ij} =0\} =1-p$. 

\noindent Для решения задачи применяется жадный алгоритм: на каждом шаге выбирается подмножество, максимально покрывающее еще не покрытые объекты. 

Доказать следующее утверждение. Пусть для случайной матрицы $A$, определенной выше, выполнены соотношения:
\[\begin{array}{l} {\forall \gamma >0:} \\ {\frac{\ln n}{m^{\gamma } } \mathop{\to }\limits_{n\to \infty } 0,} \\ {\frac{\ln m}{n} \mathop{\to }\limits_{n\to \infty } 0.} \end{array}\] 
Тогда для $\forall \varepsilon >0:$ $P\left\{\frac{Z}{M} \le 1+\varepsilon \right\}\mathop{\to }\limits_{n\to \infty } 1$, где Z -- решение жадного алгоритма, M -- величина минимального покрытия. 

\begin{fixme}
Добавить указание.
\end{fixme}

\end{problem}

\end{comment}

\begin{problem}
А) Пусть имеется генератор случайных чисел, в 
результате обращения к которому появляется 0 или 1 с одинаковой вероятностью 
равной 1/2 (аналог подбрасывания симметричной монеты). Пусть задано 
вещественное число $0\le p\le 1$. С помощью имеющегося генератора определить 
генератор randp, в результате обращения к которому появляется 0 или 1 с 
вероятностями $p$ и $1-p$ соответственно (незначительные отклонения 
допустимы). Оцените сложность в среднем алгоритма получения одного 
случайного числа с помощью randp (затраты определяются числом обращений к 
изначально имеющемуся генератору). 

Б) Пусть имеется генератор 
случайных чисел randp (описанный выше). Известно, что $p\ne 0,\quad p\ne 1$. 
Как с помощью него сконструировать генератор, в результате обращения 
которому появляется 0 или 1 с одинаковой вероятностью 1/2. 

В) Чему равно математическое ожидание числа обращений к изначально имеющемуся 
генератору случайных чисел при построении последовательности пар до 
появления 0,1 или 1,0? Найти сложность в среднем алгоритма получения k 
``равновероятных'' нулей и единиц с помощью сконструированного генератора 
(затраты определяются количеством обращений к изначально имеющемуся 
генератору). Можно ли указать значения $p$, для которых эта сложность имеет 
минимальное и, соответственно, максимальное значение?
\end{problem}


\begin{problem}[интерактивные доказательства]. 

А) (изоморфизм графов). Алисе известен изоморфизм $\phi$ графов $G_0$ и $G_1$. Но она посылает Бобу граф $H =\psi(G_0)$, либо $H =\psi(G_1)$, где $\psi$ – некоторый другой изоморфизм, не равный $\phi$. Боб бросает симметричную монетку и просит изоморфизм либо $H : G_0$, либо $H : G_1$. В первом случае Алиса посылает  $\psi$, во втором – $\psi \phi^{-1}$. Таких партий разыгрывается $N$ штук. Заметим, что в каждой новой партии Алиса придумывает новую перестановку $\psi$   вершин графа $G_0$. Если $\phi$  – действительно изоморфизм $G_0 : G_1$, то все проверки Боба будут положительны.
Покажите, что если $\phi$ – блеф, то с вероятностью  $1 - 2^{-N}$  хотя бы одна проверка обнаружит это (та проверка, в которой Боб попросил $H : G_1$).

\begin{remark}
Этот пример поучителен с точки зрения «криптографического фокуса» – Алиса убедила Боба в $G_0 : G_1$ так и не огласив самого изоморфизма $\phi$. Если $\phi$ – пароль, то диалог можно вести даже в открытую, что служит примером криптосистемы с нулевым разглашением.
\end{remark}

Б) (неизоморфизм графов). Теперь наделим Алису сверхъестественными вычислительными способностями. Для удобства переименуем игроков: «Prover» и «Verifier». Verifier выбирает случайно (равновероятно) $ i \in \lbrace 0, 1 \rbrace$ и некоторую перестановку $\pi$ вершин графа $G_i$, затем посылает граф $H =\psi(G_i)$ и требует, чтобы Prover определил $i$. Таких партий разыгрывается $N$ штук. Аналогично предыдущему примеру, $\pi$ в каждой новой партии свое. Если графы неизоморфны, то Prover всегда верно определит индекс $i$. Все тесты будут пройдены. Покажите, что иначе с вероятностью $2^{-N}$  Prover ошибется хотя бы один раз (хотя бы в одной партии).
 
\end{problem}

\begin{problem}[доказательство знаний при их отсутствии].
\end{problem}

\subsection{Кодирование}

\begin{problem}
Пусть буква $X$ --- дискретная с.в., принимающая значения из алфавита $(x_1,\ldots,x_m)$ с вероятностями $(p_1,\ldots,p_m)$. 
Имеется случайный текст из $n$ букв $X$ (предполагается, что буквы в тексте независимы друг от друга). Общее количество таких 
текстов $2^{n\log m}$. Поэтому можно закодировать все эти слова, используя $n\log m$ бит. Однако, используя то обстоятельство, что 
$(p_1,\ldots,p_m)$ --- в общем случае неравномерное распределение, предложите лучший способ кодирования, основанный 
на усиленном законе больших чисел.
\end{problem}

\begin{ordre}
Пусть $\Omega=\{ \omega:\; \omega=(X_1,X_2,\ldots, X_n),\, X_i\in 1,2,\ldots,m\}$ --- пространство элементарных исходов. 
Вероятность появления слова $\omega=(X_1,X_2,\ldots, X_n)$ равна $p(\omega)=p_{X_1}\cdot\ldots\cdot p_{X_n}$. По теореме Колмогорова об 
у.з.б.ч. 
$$
-\frac{1}{n}\log p(\omega)=-\frac{1}{n}\sum\limits_{i=1}^{n}\log p_{X_i} \xrightarrow{\text{ п.н. }} 
-{\mathbb E}p(\omega)=-\sum\limits_{i=1}^{m}p_i\log p_i=H(p) 
$$
В частности, $\frac{S_n}{n}\xrightarrow{P}H(p)$, где $S_n=-\log p(\omega)=-\sum\limits_{i=1}^{n}\log p_{X_i}$. Это можно записать в виде 

$$
{\mathbb P}\Bigl( \Bigl| \frac{S_n-nH(p)}{n}\Bigr|>\delta \Bigr)  \xrightarrow{n\to\infty} 0 
$$
\end{ordre}

\begin{comment}

\begin{problem}[о шляпах Тода Эберта (1998) ]

Трех игроков отводят в комнату, где на них надевают (случайно и независимо) белые и черные шляпы. Каждый видит 
цвет других шляп и должен написать на бумажке одно из трех слов: <<белый>>, <<черный>>, <<пас>> 
(не советуясь с другими и не показывая им свою бумажку). Команда выигрывает, если хотя бы один из игроков назвал правильный 
цвет своей шляпы и ни один не назвал неправильного. Как им сговориться, чтобы увеличить шансы? 
Решите эту же задачу, если игроков $n=2^m -1$ $( m\in {\mathbb N} )$. 
\end{problem}

\begin{ordre}
Воспользуйтесь понятием кода Хемминга.
\end{ordre}

\begin{remark}
Докажем для случая трех игроков, что стратегий лучше (вероятность выигрыша больше  $ \frac{3}{4}$) не бывает . 

Единственная информация, которой владеет $i$-й игрок --- это цвета шляп двух других. Поэтому стратегия для $i$-го игрока должна зависеть 
только от этих двух цветов. В каждом случае имется три варианта ответа для игрока: $0$, $1$ или <<пас>>, т.е. всего $3^{12}$ различных 
стратегий. Поскольку есть $8$ вариантов расположения шляп на игроках, более выгодная стратегия должна обеспечивать выигрыш в $7$ вариантах. 
Тогда один из игроков должен угадать свой цвет в $3$ ситуациях. Значит, имеются для него ответы $\alpha_{i_1 j_1}$, 
$\alpha_{i_2 j_2}$, не являющиеся пасами. Но тогда в ситуациях $\overline{\alpha_{i_1 j_1}} i_1 j_1$ и 
$\overline{\alpha_{i_2 j_2}} i_2 j_2$ он ошибется, что противоречит предположению о $7$ выигрышных ситуациях. 

Таким образом, максимальная вероятность выигрыша равна $\frac{3}{4}$. 

\end{remark}

\end{comment}

 \begin{problem} 
Имеется неизвестное число от 1 до $n$ ($n\ge 2$). Разрешается задавать любые вопросы с ответами ДА/НЕТ. При ответе ДА мы платим 1 рубль, при ответе НЕТ -- 2 рубля. Сколько необходимо и достаточно заплатить для отгадывания числа?
\end{problem}

\subsection{Теория игр}
\begin{problem}
Один игрок прячет (зажимает в кулаке) одну или две монеты достоинством 10 рублей. Другой игрок должен отгадать сколько денег у первого спрятано. Если отгадывает, то получает деньги, если нет -- платит 15 рублей. Каковы  должны быть стратегии игроков при многократном повторении игры?

\end{problem}


\begin{comment}
\begin{problem}
Некто обладает одной облигацией, которую намеревается продать в один из последующих четырех дней, в которых цена облигации 
принимает различные значения, априори неизвестные, но становящиеся известными в начале каждого дня продаж. Предполагается, что 
цены облигации независимы и их перестановки по торговым дням равновозможны. Какова стратегия продавца, состоящая в выборе дня 
продажи облигации и гарантирующая максимальную вероятность того, что он продаст облигацию в день ее наибольшей цены? 
\end{problem}

\begin{ordre}
Рассмотреть следующие возможные стратегии и сравнить вероятности продажи облигации в день наибольшей цены: 
\begin{enumerate}
\item[а)] на первом шаге (в первый день торгов) запомним имевшую место цену облигации, не продавая ее, а затем продадим 
облигацию в тот день, когда ее цена окажется большей цены, зафиксированной в первый день, или (когда такого дня не окажется) в 
последний (четвертый) день, независимо от цены этого дня (стратегия $S_1$); 

\item[б)] не продавая облигацию в первом и втором торговых днях, зафиксируем  максимальную цену из двух, имевших место для этих дней, 
и продадим облигацию в третьем торговом дне, если цена облигации в нем будет выше, чем указанная зафиксированная максимальная цена, 
или, в противном случае, в четвертом дне (стратегия $S_2$). 
\end{enumerate}
\end{ordre}
\end{comment}

\begin{problem}
Ведущий приносит два одинаковых конверта и говорит, что в них лежат деньги, причем в одном вдвое больше, чем в другом. Двое участников берут конверты и тайком друг от друга смотрят, сколько в них денег. Затем один говорит другому: «Махнемся не глядя?» (предлагая поменяться конвертами). Стоит ли соглашаться?
\end{problem}

\begin{comment}
\begin{problem}[парадокс Монти–Холла ](сократить)
Представьте, что вы стали участником игры, в которой находитесь перед тремя дверями. Ведущий поместил за одной из трех 
пронумерованных дверей автомобиль, а за двумя другими дверями --- по козе (козы тоже пронумерованы) случайным образом --– это значит, 
что все $3! = 6$ вариантов расположения автомобиля и коз за пронумерованными дверями равновероятны). У вас нет никакой информации 
о том, что за какой дверью находится. Ведущий говорит: <<Сначала вы должны выбрать одну из дверей. После этого я открою одну из 
оставшихся дверей (при этом если вы выберете дверь, за которой находится автомобиль, то я с вероятностью $1/2$ выберу дверь, 
за которой находится коза номер $1$, и с вероятностью $1-1/2=1/2$ дверь, за которой находится коза номер $2$). Затем я предложу 
вам изменить свой первоначальный выбор и выбрать оставшуюся закрытую дверь вместо той, которую вы выбрали сначала. Вы можете 
последовать моему совету и выбрать другую дверь, либо подтвердить свой первоначальный выбор. После этого я открою дверь, 
которую вы выбрали, и вы выиграете то, что находится за этой дверью.>> Вы выбираете дверь номер $3$. Ведущий открывает дверь номер $1$ 
и показывает, что за ней находится коза. Затем ведущий предлагает вам выбрать дверь номер $2$. Увеличатся ли ваши шансы 
выиграть автомобиль, если вы последуете его совету? 
\end{problem}


 \begin{problem}
 В аудитории находится невеста, которая хочет выбрать себе жениха. За дверью выстроилась очередь из $N$ женихов. Относительно любых 
 двух женихов невеста может сделать вывод, какой из них для неё предпочтительнее. Таким образом, невеста задает на множестве женихов 
 отношение порядка (естественно считать, что если $A$ предпочтительнее $B$, а $B$ предпочтительнее $C$, то $A$ предпочтительнее $C$). 
 Предположим, что все $N!$ вариантов очередей равновероятны и невеста об этом знает (равно, как и число $N$). Женихи запускаются 
 в аудиторию по очереди. Невеста видит каждого из них в первый раз! Если на каком-то женихе невеста остановится (сделает свой выбор), 
 то оставшаяся очередь расходится. Невеста хочет выбрать наилучшего жениха (исследуя $k$–го по очереди жениха, невеста лишь может 
 сравнить его со всеми предыдущими, которых она уже просмотрела и пропустила). Оцените (при $N\to\infty$) вероятность того, что невесте 
 удастся выбрать наилучшего жениха, если она придерживается следующей стратегии: просмотреть (пропустить) первых по очереди $[N/e]$ 
 кандидатов и затем выбрать первого кандидата, который лучше всех предыдущих (впрочем, такого кандидата может и не оказаться, тогда, 
 очевидно, невеста не смогла выбрать наилучшего жениха). 
 \end{problem}
 
 \end{comment}
 

\begin{problem}
В аудитории находится 100 человек (игроков). Каждого просят написать  число от 1 до 100. Победителем окажется тот участник, который написал число наиболее близкое к $\frac{2}{3}$ от среднего арифметического всех чисел. Найти оптимальную стратегию, при условии, что каждый участник думает на $x \sim Poisson(2)$  хода  вперед.  
\end{problem}

 \begin{comment}
\begin{problem}[сублинейный приближенный вероятностный алгоритм для 
матричных игр; Григориадис -- Хачиян, 1995]
Рассматривается симметричная 
антагонистическая игра двух лиц X и Y. Смешанные стратегии X и Y будем 
обозначать соответственно $\vec {x}$ и $\vec {y}$. При этом $x_k $ - 
вероятность того, что игрок X выберет стратегию с номером k, аналогично 
определяется $y_k $. Таким образом, $\vec {x},\vec {y}\in S=\left\{ {\vec 
{x}\in {\rm R}^n:\;\;\vec {e}^T\vec {x}=1,\;\vec {x}\ge \vec {0}} \right\}$, 
где $\vec {e}=\left( {1,...,1} \right)^T$. Выигрыш игрока X: $V_X \left( 
{\vec {x},\vec {y}} \right)=\vec {y}^TA\vec {x}$, а выигрыш игрока Y: $V_Y 
\left( {\vec {x},\vec {y}} \right)=-\vec {y}^TA\vec {x}$ (игра 
антагонистическая). Каждый игрок стремится максимизировать свой выигрыш, при 
заданном ходе оппонента. Равновесием Нэша (в смешанных стратегиях) 
называется такая пара стратегий $\left( {\vec {x}^\ast ,\;\vec {y}^\ast } 
\right)$, что
\[
\vec {x}^\ast \in \mbox{Arg}\mathop {\max }\limits_{\vec {x}\in S} \vec 
{y}^{\ast T} A\vec {x},
\quad
\vec {y}^\ast \in \mbox{Arg}\mathop {\min }\limits_{\vec {y}\in S} \vec 
{y}^TA\vec {x}^\ast .
\]
Ценой игры называют $\mathop {\max }\limits_{\vec {x}\in S} \mathop {\min 
}\limits_{\vec {y}\in S} \vec {y}^TA\vec {x}=\mathop {\min }\limits_{\vec 
{y}\in S} \mathop {\max }\limits_{\vec {x}\in S} \vec {y}^TA\vec {x}=\vec 
{y}^{\ast T}A\vec {x}^\ast $. Поскольку, по условию, игра также симметричная, 
то $A=-A^T$ - матрица$n\times n$. С помощью стандартной редукции можно 
свести к этому случаю общий случай произвольной матричной игры. В 
рассматриваемом же случае цена игры (выигрыш игроков в положении равновесия 
Нэша) есть 0, а множества оптимальных стратегий игроков совпадают. Требуется 
найти с точностью $\varepsilon >0$ положение равновесия Нэша (оптимальную 
стратегию), т.е. требуется найти такой вектор $\vec {x}$, что $A\vec {x}\le 
\varepsilon \vec {e}$, $\vec {x}\in S$. Покажите, считая элементы матрицы 
$A$ равномерно ограниченными, скажем, единицей, что приводимый ниже алгоритм 
находит с вероятностью не меньшей $1 \mathord{\left/ {\vphantom {1 2}} 
\right. \kern-\nulldelimiterspace} 2$ (вместо $1 \mathord{\left/ {\vphantom 
{1 2}} \right. \kern-\nulldelimiterspace} 2$ можно взять любое положительное 
число меньшее единицы) такой $\vec {x}$ за время ${\rm O}\left( {\varepsilon 
^{-2}n\log ^2n} \right)$, т.е. в определенном смысле даже не вся матрица (из 
$n^2$ элементов) просматривается. Отметим также, что в классе 
детерминированных алгоритмов, время работы растет с ростом $n$ не медленнее 
чем $\sim n^2$ (эта нижняя оценка получается из информационных соображений). 
Другими словами, никакой детерминированный алгоритм не может также 
асимптотически быстро находить приближенно равновесие Нэша. Точнее говоря, 
описанный ниже вероятностный алгоритм дает почти квадратичное ускорение по 
сравнению с детерминированными.

\underline {\textbf{Алгоритм}}

\begin{enumerate}
\item \textbf{Инициализация:} $\vec {x}=\vec {U}=\vec {0}$, $\vec {p}={\vec {e}} \mathord{\left/ {\vphantom {{\vec {e}} n}} \right. \kern-\nulldelimiterspace} n$, $t=0$.
\item \textbf{Повторить:}
\item \textbf{Счетчик итераций: }$t:=t+1$.
\item \textbf{Датчик случайных чисел:} выбираем $k\in \left\{ {1,...,n} \right\}$ с вероятностью $p_k $.
\item \textbf{Модификация }$\vec {X}$\textbf{: }$X_k :=X_k +1$.
\item \textbf{Модификация }$\vec {U}$\textbf{:} $U_i :=U_i +a_{ik} $, $i=1,...,n$.
\item \textbf{Модификация }$\vec {p}$\textbf{:} $p_i :={p_i \exp \left( {\varepsilon {a_{ik} } \mathord{\left/ {\vphantom {{a_{ik} } 2}} \right. \kern-\nulldelimiterspace} 2} \right)} \mathord{\left/ {\vphantom {{p_i \exp \left( {\varepsilon {a_{ik} } \mathord{\left/ {\vphantom {{a_{ik} } 2}} \right. \kern-\nulldelimiterspace} 2} \right)} {\left( {\sum\limits_{j=1}^n {p_j \exp \left( {\varepsilon {a_{jk} } \mathord{\left/ {\vphantom {{a_{jk} } 2}} \right. \kern-\nulldelimiterspace} 2} \right)} } \right)}}} \right. \kern-\nulldelimiterspace} {\left( {\sum\limits_{j=1}^n {p_j \exp \left( {\varepsilon {a_{jk} } \mathord{\left/ {\vphantom {{a_{jk} } 2}} \right. \kern-\nulldelimiterspace} 2} \right)} } \right)}$, $i=1,...,n$.
\item \textbf{Критерий останова:} если ${\vec {U}} \mathord{\left/ {\vphantom {{\vec {U}} t}} \right. \kern-\nulldelimiterspace} t\le 
\varepsilon \vec {e}$, то останавливаемся и печатаем ${\vec {x}=\vec {X}} \mathord{\left/ {\vphantom {{\vec {x}=\vec {X}} t}} \right. \kern-\nulldelimiterspace} t$.
\end{enumerate}
\textbf{Указание.} Покажите, что с вероятностью не меньшей, чем $1 
\mathord{\left/ {\vphantom {1 2}} \right. \kern-\nulldelimiterspace} 2$ 
алгоритм остановится через $t^\ast =4\varepsilon ^{-2}\ln n$ итераций. Для 
этого введите $P_i \left( t \right)=\exp \left( {{\varepsilon U_i \left( t 
\right)} \mathord{\left/ {\vphantom {{\varepsilon U_i \left( t \right)} 2}} 
\right. \kern-\nulldelimiterspace} 2} \right)$ и $\Phi \left( t 
\right)=\sum\limits_{i=1}^n {P_i \left( t \right)} $. Покажите, что

$M\left[ {\left. {\Phi \left( {t+1} \right)} \right|\vec {P}\left( t \right)} 
\right]=\Phi \left( t \right)\sum\limits_{i,k=1}^n {p_i \left( t \right)} 
p_k \left( t \right)\exp \left( {{\varepsilon a_{ik} } \mathord{\left/ 
{\vphantom {{\varepsilon a_{ik} } 2}} \right. \kern-\nulldelimiterspace} 2} 
\right)$ и $\exp \left( {{\varepsilon a_{ik} } \mathord{\left/ {\vphantom 
{{\varepsilon a_{ik} } 2}} \right. \kern-\nulldelimiterspace} 2} \right)\le 
1+{\varepsilon a_{ik} } \mathord{\left/ {\vphantom {{\varepsilon a_{ik} } 
2}} \right. \kern-\nulldelimiterspace} 2+{\varepsilon ^2} \mathord{\left/ 
{\vphantom {{\varepsilon ^2} 6}} \right. \kern-\nulldelimiterspace} 6$.

Используя это и кососимметричность матрицы $A$, покажите
\[
M\left[ {\Phi \left( {t+1} \right)} \right]\le M\left[ {\Phi \left( t 
\right)} \right]\left( {1+{\varepsilon ^2} \mathord{\left/ {\vphantom 
{{\varepsilon ^2} 6}} \right. \kern-\nulldelimiterspace} 6} \right).
\]
Следовательно, $M\left[ {\Phi \left( t \right)} \right]\le n\exp \left( 
{{t\varepsilon ^2} \mathord{\left/ {\vphantom {{t\varepsilon ^2} 6}} \right. 
\kern-\nulldelimiterspace} 6} \right)$ и $M\left[ {\Phi \left( {t^\ast } 
\right)} \right]\le n^{5 \mathord{\left/ {\vphantom {5 3}} \right. 
\kern-\nulldelimiterspace} 3}$. Отсюда по неравенству Маркова имеем, что 
($n\ge 8)$
\[
P\left( {\Phi \left( {t^\ast } \right)\le n^2} \right)\ge P\left( {\Phi 
\left( {t^\ast } \right)\le 2n^{5 \mathord{\left/ {\vphantom {5 3}} \right. 
\kern-\nulldelimiterspace} 3}} \right)\ge 1 \mathord{\left/ {\vphantom {1 
2}} \right. \kern-\nulldelimiterspace} 2.
\]
Тогда $P\left( {{\varepsilon U_i \left( {t^\ast } \right)} \mathord{\left/ 
{\vphantom {{\varepsilon U_i \left( {t^\ast } \right)} 2}} \right. 
\kern-\nulldelimiterspace} 2\le 2\ln n,\;i=1,...,n} \right)\ge 1 
\mathord{\left/ {\vphantom {1 2}} \right. \kern-\nulldelimiterspace} 2$. 
Откуда уже следует, что $P\left( {\vec {x}\left( {t^\ast } \right)\le 
\varepsilon \vec {e}} \right)\ge 1 \mathord{\left/ {\vphantom {1 2}} \right. 
\kern-\nulldelimiterspace} 2$.
\end{problem}
 \end{comment}

\begin{problem}
Два участника аукциона конкурируют за покупку некоторого объекта. Ценности объекта    $v_1$ и $v_2$ для участников являются независимыми случайными величинами, равномерно распределенными на отрезке [0, 1]. Участник  имеет точною информацию своем значении $v_i$, но не знает $v_j$. Участники делают ставки из диапазона [0, 1] одновременно и независимо друг от друга. В данном аукционе побеждает тот, кто поставил большую ставку. При равенстве ставок бросается жребий. Каждый обязан заплатить по средней ставке даже, если ему объект не достается! (Отказаться от участия в этом аукционе нельзя.)
\begin{enumerate}
\item Выпишите функции выигрыша в данной игре.
\item Найдите оптимальные стратегии в данной игре в классе квадратичных стратегий: $b_i(v_i) = cv_i^2$, где $c > 0$.
\item Покажите, что в этой игре нет других оптимальных решений с гладкими возрастающими стратегиями.
\end{enumerate}
\end{problem}