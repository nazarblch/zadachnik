\section{Предельные теоремы}



\begin{problem}

Рассмотрим простую и классическую схему блуждания точки по прямой, соответствующую правилам игры в орлянку:
\[\eta (0)=0,\] 
\[\eta (t+1)=\left\{\begin{array}{cc} {\eta (t)+1,} & {p={1\mathord{\left/ {\vphantom {1 2}} \right. \kern-\nulldelimiterspace} 2} } \\ {\eta (t)-1,} & {p={1\mathord{\left/ {\vphantom {1 2}} \right. \kern-\nulldelimiterspace} 2} } \end{array}\right. .\] 
Занумеруем в порядке возрастания все моменты времени, когда $\eta (t)=0$. Получим с вероятностью 1 бесконечную последовательность $0=\tau _{0} <\tau _{1} <\tau _{2} <...$ Рассмотрим разности $\xi _{i} =\tau _{i} -\tau _{i-1} $, $i=1,2,...$ -- последовательность независимых одинаково распределенных с.в.

\textbf{а)} Найдите распределение $\xi _{i} =\tau _{i} -\tau _{i-1} $, т.е. $P\left\{\xi _{i} =2m\right\}$

\textbf{б)} Покажите, что математическое ожидание с.в. $\xi _{i} =\tau _{i} -\tau _{i-1} $ равно бесконечности. Этот результат можно проинтерпретировать так: среднее время до первого возвращения блуждания в 0 бесконечно.

Тем не менее суммы $\tau _{n} =\sum _{i=1}^{n}\xi _{i}  $ при надлежащей нормировке подчинены предельному распределению: 
\[\mathop{\lim }\limits_{n\to \infty } P\left\{\frac{2\tau _{n} }{\pi n^{2} } <z\right\}=\left\{\begin{array}{cc} {\frac{1}{\sqrt{2\pi } } \int _{0}^{z}e^{-\frac{1}{2x} } x^{-\frac{3}{2} }  dx,} & {z>0} \\ {0,} & {z<0} \end{array}\right. .\] 
Из теоремы о каноническом представлении устойчивых законов (теорема Леви-Хинчина): Для того чтобы функция распределения была устойчивой, необходимо и достаточно, чтобы логарифм ее характеристической функции представлялся формулой:
\[\ln \varphi (t)=i\gamma t-c|t|^{\alpha } \left(1+i\beta \frac{t}{|t|} \omega (t,\alpha )\right),\] 
где $\gamma $ -- любое действительное число, $-1\le \beta \le 1$, $0<\alpha \le 2$, $c\ge 0$ и
\[\omega (t,\alpha )=\left\{\begin{array}{cc} {tg\left(\frac{\pi }{2} \alpha \right),} & {\alpha \ne 1,} \\ {\frac{2}{\pi } \ln |t|,} & {\alpha =1} \end{array}\right. ,\] 
следует, что такое распределение соответствует каноническому представлению с $\alpha ={1\mathord{\left/ {\vphantom {1 2}} \right. \kern-\nulldelimiterspace} 2} $, $\beta =1$, $\gamma =0$, $c=1$, и принадлежит семейству кривых Пирсона (показано Н. В. Смирновым).

\end{problem}

\begin{problem}
Bывод распределения Хольцмарка: Рассмотрим шар радиуса $r$ с центром в начале координат и $n$ звезд (точек), расположенных в нем наудачу и независимо друг от друга. Пусть каждая звезда имеет единичную массу. Обозначим $X_{1} ,...,X_{n} $ -компоненты гравитационных сил, соответствующие отдельным звездам, и положим $S_{n} =X_{1} +...+X_{n} $. Устремим $r$ и $n$ к бесконечности так, чтобы $\frac{4}{3} \pi r^{3} n^{-1} \to \lambda $. Показать, что распределение величины $S_{n} $ стремится к симметричному устойчивому распределению с $\alpha ={3\mathord{\left/ {\vphantom {3 2}} \right. \kern-\nulldelimiterspace} 2} $. Можно показать, что задача по существу не изменится, если массу каждой звезды считать с.в. с единичным математическим ожиданием и массы различных звезд предполагать взаимно независимыми с.в. и не зависящими также от их расположения.
\end{problem}

\begin{problem}
Напомним определение сходимости по распределению (частный случай слабой сходимости): $\xi _{n} \mathop{\to }\limits_{}^{d} \xi $ - для любого $x$, такого что функция $F_{\xi } (x)$ непрерывна в точке $x$, имеет место сходимость функций распределения: $F_{\xi _{n} } (x)\to F_{\xi } (x)$ при $n\to \infty $, где $F_{\xi } (x)$ - функция распределения с.в. $\xi $.

\noindent \textbf{а)} Объясните, почему нет слабой сходимости такой последовательности с.в.:
\[\xi _{n} =\left\{\begin{array}{cc} {-n,} & {{\raise0.7ex\hbox{$ 1 $}\!\mathord{\left/ {\vphantom {1 2}} \right. \kern-\nulldelimiterspace}\!\lower0.7ex\hbox{$ 2 $}} ,} \\ {n,} & {{\raise0.7ex\hbox{$ 1 $}\!\mathord{\left/ {\vphantom {1 2}} \right. \kern-\nulldelimiterspace}\!\lower0.7ex\hbox{$ 2 $}} ;} \end{array}\right. \] 

\begin{ordre} 

\noindent $F_{\xi _{n} } (x)=\left\{\begin{array}{cc} {0,} & {x\le -n} \\ {{\raise0.7ex\hbox{$ 1 $}\!\mathord{\left/ {\vphantom {1 2}} \right. \kern-\nulldelimiterspace}\!\lower0.7ex\hbox{$ 2 $}} ,} & {-n<x\le n} \\ {1,} & {x>n} \end{array}\right. $ сходятся к функции $G(x)\equiv \frac{1}{2} $.

\end{ordre} 

\noindent \textbf{б)} Пусть $\xi _{1} ,\xi _{2} ,\xi _{3} ,...$ - последовательность независимых одинаково распределенных с.в. с конечной ненулевой дисперсией. Обозначим $S_{n} =\sum _{i=1}^{n}\xi _{i}  $. При каких значениях $c$ имеет или не имеет место сходимость $P\left(\frac{S_{n} }{n} <c\right)\to P\left(E\xi _{1} <c\right)$?


\end{problem}

\begin{problem}

Доказать локальную предельную теорему:

\noindent Пусть $0<p<1$ и $X_{i} $, $i=1,...,n$ - независимые случайные величины, имеющие распределение:
\[X_{i} =\left\{\begin{array}{cc} {1,} & {p,} \\ {-1,} & {q=1-p;} \end{array}\right. \] 
Тогда равномерно по всем $x=O\left(\sqrt{n} \right)$ таким, что $(p-q)n+x$ целое неотрицательное число
\[P\left\{\sum _{i=1}^{n}X_{i} =(p-q)n+x \right\}\sim \frac{1}{\sqrt{2\pi npq} } \exp \left\{-\frac{x^{2} }{2npq} \right\}\] 
при $n\to \infty $. 

\begin{ordre}
Воспользоваться формулой Стирлинга
\[
n! \sim \sqrt{2 \pi n} \frac{n^n}{e^n} 
\]
\end{ordre}

\begin{remark}
Пусть $n=2k$ и $p=\frac{1}{2} $, тогда вероятность того, что число единиц в точности рано числу минус единиц мало (но не экспоненциально мало):
\[P\left\{\sum _{i=1}^{2k}X_{i} =k \right\}\sim \frac{1}{\sqrt{\pi k} } \] 
\end{remark}

\end{problem}


\begin{problem}
Книга объемом $500$ страниц содержит $50$ опечаток. Оценить вероятность того, что на случайно выбранной странице 
имеется не менее трех опечаток. (Использовать нормальное и пуассоновское приближения, сравнить результаты). 
\end{problem}

\begin{problem}
В тесто для выпечки булок с изюмом замешано $N$ изюмин. Всего из данного теста выпечено $K$ булок. Оценить вероятность того, 
что в случайно выбранной булке число изюмин находится в пределах от $a$ до $b$. 
\end{problem}

\begin{problem}
В поселке $N$ жителей, каждый из которых в среднем $n$ раз в месяц ездит в город, выбирая дни поездки независимо от остальных. 
Поезд из поселка в город идет один раз в сутки. Какова должна быть вместимость поезда для того, чтобы он переполнился с вероятностью, 
не превышающей заданного числа $\beta$? 
\end{problem}

\begin{problem}
В игре в рулетку колесо разделено на 38 равных секторов: 18 красных, 18 белых и два сектора (0 и 00) зеленого цвета. Пусть ставка игрока на каждом шаге равна 1\$. Обозначим $X_{i} $ - выигрыш в i-ой игре. Тогда $X_{1} ,X_{2} ,X_{3} ,...$ - независимые с.в., имеющие распределение: 
\[X_{i} =\left\{\begin{array}{cc} {+1,} & {p={\raise0.7ex\hbox{$ 18 $}\!\mathord{\left/ {\vphantom {18 38}} \right. \kern-\nulldelimiterspace}\!\lower0.7ex\hbox{$ 38 $}} ,} \\ {-1,} & {p={\raise0.7ex\hbox{$ 20 $}\!\mathord{\left/ {\vphantom {20 38}} \right. \kern-\nulldelimiterspace}\!\lower0.7ex\hbox{$ 38 $}} ;} \end{array}\right. \] 
Пусть сыграно $n=361=19^{2} $ партий. С помощью ЦПТ $P\left(\sum _{i=1}^{n}X_{i}  \ge 0\right)$ и  неравенства Берри-Эссена оцените погрешность приближения. (перенести определение неравенства)
\end{problem}

\begin{problem}
(Петербургский парадокс) $X_{1} ,X_{2} ,X_{3} ,...$ - независимые с.в., имеющие распределение $P\left(X_{i} =2^{k} \right)=2^{-k} $, $k=1,2,3,...$ То есть, если в игре в орлянку $k$ раз выпал «орел», то выигрыш будет $2^{k} $. Справедливой ценой за игру называют математическое ожидание выигрыша. Но здесь $EX_{i} =\infty $, однако, для этого нужно играть бесконечное число раз и иметь бесконечно много денег. Покажите, что $\frac{S_{n} }{n\log _{2} n} \mathop{\to }\limits^{p} 1$ при $n\to \infty $, где $S_{n} =\sum _{k=1}^{n}X_{k}  $. Проинтерпретируйте результат как цена за $n$ игр?

\begin{ordre} 

Пусть для каждого $n$ с.в. $X_{nk} $, $1\le k\le n$ независимы. Пусть также $b_{n} >0$ с $b_{n} \to \infty $ и $\bar{X}_{nk} =X_{nk} {\rm I} \left\{X_{nk} \le b_{n} \right\}$.(где ${\rm I} \left\{...\right\}$ - индикаторная функция). Предположим, что выполняются условия:
\[1) \sum _{k=1}^{n}P\left\{\left|X_{nk} \right|>b_{n} \right\} \mathop{\to }\limits_{n\to \infty } 0\] 
\[2) \frac{\sum _{k=1}^{n}D\bar{X}_{nk}  }{b_{n} ^{2} } \mathop{\to }\limits_{n\to \infty } 0.\] 
Тогда $\frac{\sum _{k=1}^{n}X_{nk}  -\sum _{k=1}^{n}E\bar{X}_{nk}  }{b_{n} } \mathop{\to }\limits^{p} 0$ при $n\to \infty $.

\noindent Для этой теоремы положите $X_{nk} =X_{k} $. В качестве $b_{n} >0$ возьмите $b_{n} =2^{m(n)} $, где $m(n)$ - целое число, которое можно представить в виде $m(n)=\log _{2} n+K(n)$, $K(n)\to \infty $ при $n\to \infty $. Например, если $K(n)\le \log \log n$, то результатом применения теоремы будет $\frac{S_{n} }{n\log _{2} n} \mathop{\to }\limits^{p} 1$ при $n\to \infty $.

\end{ordre} 

\end{problem}

\begin{problem}

Случайная величина (размер выигрыша) принимает значение $2^{k} -1$ с вероятностью $p_{k} =\frac{1}{2^{k} k(k+1)} $ для $k=1,2,3,...$ и значение\textit{ $-1$} с вероятностью $p_{0} =1-\sum _{k=1}^{\infty }p_{k}  $. Проверьте, что математическое ожидание выигрыша равно нулю. Применив теорему из предыдущей задачи, покажите, что для суммарного размера выигрыша за $n$ партий ($S_{n} $) справедливо$\frac{S_{n} }{{\raise0.7ex\hbox{$ n $}\!\mathord{\left/ {\vphantom {n \log _{2} n}} \right. \kern-\nulldelimiterspace}\!\lower0.7ex\hbox{$ \log _{2} n $}} } \mathop{\to }\limits^{p} 1$.

\begin{ordre}  
Положите $b_{n} =2^{m(n)} $,

 где $m(n)=\min \left\{m:\; 2^{-m} \frac{1}{\sqrt{m^{3}}} \le n^{-1} \right\}$
\end{ordre} 

\end{problem}

\begin{problem}

Определение: Распределение $G\left(x\right)$ называется max-устойчивым, если для любых $n=1,2,...$ существуют $a_{n} >0$ и $b_{n} \in {\mathbb R}$, такие что $G^{n} \left(a_{n} x+b_{n} \right)=G\left(x\right)$.


(MAX-устойчивые распределения: Гумбеля, Фреше, Вейбулла). Пусть есть независимые одинаково распределенные с.в. $X_{1} ,...,X_{n} $ с распределением $F\left(x\right)$. Обозначим $X_{\left(n\right)} =\max \left\{X_{1} ,...,X_{n} \right\}$. Распределение такой с.в. $F_{X_{\left(n\right)} } \left(x\right)=\left[F\left(x\right)\right]^{n} $.

\begin{enumerate}
\item  Пусть $\mathop{\lim }\limits_{x\to \infty } e^{\alpha x} \left(1-F\left(x\right)\right)=\beta $, где $\alpha ,\beta >0$ и $x\in {\mathbb R}$. Покажите, что $X_{\left(n\right)} -\frac{1}{\alpha } \ln \left(\beta n\right)\mathop{\to }\limits^{d} \chi $, где $\chi $ имеет распределение Гумбеля: $P\left(\chi \le x\right)=e^{-e^{-\alpha x} } $, $x\in {\mathbb R}$.

\item  Пусть $\mathop{\lim }\limits_{x\to \infty } x^{\alpha } \left(1-F\left(x\right)\right)=\beta $, где $\alpha ,\beta >0$ и $x\in {\mathbb R}_{+} $. Покажите, что $X_{\left(n\right)} \left(\beta n\right)^{-\frac{1}{\alpha } } \mathop{\to }\limits^{d} \eta $, где $\eta $ имеет распределение Фреше: $P\left(\eta \le x\right)=e^{-x^{-\alpha } } $, $x>0$.

\item  Пусть $\mathop{\lim }\limits_{x\to \infty } \left(c-x\right)^{\alpha } \left(1-F\left(x\right)\right)=\beta $, $F\left(c\right)=1$,где $\alpha ,\beta >0$, $c\in {\mathbb R}$ и $x\in {\mathbb R}$. Покажите, что $\left(X_{\left(n\right)} -c\right)\left(\beta n\right)^{\frac{1}{\alpha } } \mathop{\to }\limits^{d} \gamma $, где $\gamma $ имеет распределение Вейбулла: $P\left(\gamma \le x\right)=e^{-(-x)^{-\alpha } } $, $x<0$.

\end{enumerate}

\noindent Покажите, что распределения Гумбеля, Фреше, Вейбулла являются max-устойчивыми.

\begin{remark}
В классе max-устойчивых распределений распределения Гумбеля, Фреше, Вейбулла исчерпывают все возможные типы предельных распределений.
\end{remark}

\end{problem}


\begin{problem}

Пусть есть независимо одинаково распределенные с.в. $X_{1} ,...,X_{n} $ с распределением Коши $\alpha =1$, т.е. $F\left(x\right)=\frac{2}{\pi } \int _{-\infty }^{x}\frac{dy}{1+y^{2} }  $. Воспользовавшись предыдущей задачей, найдите предельное распределение для должным образом нормированных с.в. $X_{\left(n\right)} $.

\end{problem}

\begin{problem}
Пусть $\vec{X}_{n} \in {\mathbb R}^{m} $, $n=1,2,...$ - независимые одинаково распределенные случайные векторы. $M\vec{X}_{n} =\vec{0}$, $M\vec{X}_{n} \vec{X}_{n}^{T} =R$ ($R$ - неотрицательно определенная матрица - по определению, однако, мы дополнительно будем считать, что $R$ - положительно определенная). С помощью аппарата характеристических функций докажите, что тогда для любого борелевского множества $B\subseteq {\mathbb R}^{m} $:
\[\mathop{\lim }\limits_{N\to \infty } P\left(\frac{1}{\sqrt{N} } \sum _{n=1}^{N}\vec{X}_{n}  \in B\right)=\left(\left(2\pi \right)^{m} \det R\right)^{-{1\mathord{\left/ {\vphantom {1 2}} \right. \kern-\nulldelimiterspace} 2} } \int _{B}e^{-\left(\vec{x},R\vec{x}\right)} d\vec{x} .\] 

\end{problem}





\begin{problem}

Пусть $x_1,\ldots,x_n$ --- независимые одинаково распределенные с.в.. Пусть также характеристическая функция с.в. $x_k$ представляется 
в окрестности $t=0$ в виде 
$$
\varphi_{x_k}(t)={\mathbb E}(e^{it x_k})=1+imt+o(t). 
$$
Используя то, что 
$$
S_n \xrightarrow{d}c\quad \Rightarrow \quad S_n \xrightarrow{P}c, \text{ где } c=\const \text{ (не с.в.) }
$$
и 
$$
S_n \xrightarrow{d}S\quad \Leftrightarrow \quad \varphi_{S_n}(t) \to \varphi_S(t), \text{ равномерно по $t$ в окрестности $t=0$ } , 
$$
найдите 
$$
S_n=\frac{1}{n}\sum\limits_{i=1}^{n} x_i \xrightarrow{P} ?
$$
\end{problem}


\begin{problem}
Пусть $x_1, x_2, x_3, \ldots$ --- последовательность независимых одинаково распределенных с.в.. Положим 
$S_n=\sum\limits_{k=1}^{n} x_k$. Покажите, что 

\begin{enumerate}
\item[1)](з.б.ч.) если ${\mathbb E}(|x_k|)<\infty$, то $S_n/n\xrightarrow{P} m$ при $n\to\infty$, где $m={\mathbb E}(x_k)$; 

\item[2)](ц.п.т.) если ${\mathbb E}(x_k^2)<\infty$, то $(S_n-m\cdot n)/\sqrt{n\cdot D}\xrightarrow{d} N(0,1)$ при $n\to\infty$, 
где $D=\Var x_k$. 

\item[3)](задача математической статистики) Предположим, что независимо $n$ раз кидается монетка с вероятностью выпадения орла в каждом 
опыте равной $p$ (точного значения $p$ мы не знаем, а знаем лишь то, что $0.1\leqslant p\leqslant 0.9$), т.е. $x_k\in\Be(p)$. 
Сколько раз нужно кинуть монетку (оцените $p$), чтобы оценка ${\bar p}(x)=\frac{\sum\limits_{k=1}^{n}x_k}{n}$ с вероятностью 
$\gamma\geqslant 0.95$ отличалась от истинного значения $p$ не более, чем на величину $\delta=0.01$? Применить неравенство Чебышева 
и предельную теорему (точность, которую дает ц.п.т., оцените с помощью неравенства Берри – Эссена). Сравнить результаты. 
\end{enumerate}
\end{problem}

\begin{ordre}

Неравенство Берри – Эссена: 
$$
\sup\limits_{x} \Bigl|{\mathbb P}\Bigl(\sum\limits_{k=1}^{n}\frac{x_k-np}{\sigma\sqrt{n}}<x\Bigr)-\Phi(x)\Bigr|\leqslant 
\frac{C_0\mu^3}{\sigma^3\sqrt{n}} . 
$$
Здесь $C_0<0.7056$, $\,\sigma^2=\Var x_k=p(1-p)$, $\Phi(x)=\int\limits_{-\infty}^{x}\tfrac{e^{-\frac{t^2}{2}}}{\sqrt{2\pi}}\, dt$, 
$$
\mu^3={\mathbb E}|x_k-p|^3. 
$$

\end{ordre}


\begin{problem}
Пусть при каждом $n\geqslant 1$ независимые с.в. $x_{1n}, x_{2n},\ldots, x_{nn}$ таковы, что $x_{kn}\in \Be(p_{kn})$, где 
$\max\limits_{1\leqslant k\leqslant n} p_{kn}\xrightarrow{n\to\infty}0$, $\sum\limits_{k=1}^{n}p_{kn}\xrightarrow{n\to\infty}\lambda$. 
Тогда 
\begin{equation}
\label{TPois}
{\mathbb P}(S_n=m)\xrightarrow{n\to\infty} e^{-\lambda}\frac{\lambda^m}{m!}, \quad m=0,1,2,\ldots, \quad 
\text{ где } S_n=\sum\limits_{k=1}^{n} x_{kn} . 
\end{equation}
\end{problem}

\begin{problem}
В течение дня игрок в казино участвует в $N=100$ независимых розыгрышах. В каждом розыгрыше он выигрывает с вероятностью 
$p=0.01$. Оцените вероятность того, что игрок ни разу не выиграет, выиграет ровно 
один раз и ровно три раза.

Предположим, что игра происходит в течении $n=100$ дней. 
Оцените вероятность того, что за эти $100$ дней в общей сложности реализуется не менее $100$ выигрышей, не менее $300$ выигрышей. 


\end{problem}



\begin{problem}

Показать, что при бросании симметричной монеты $n$ раз отношение числа выпадений герба к числу выпадений решки почти наверное стремится 
к $1$ при $n\to\infty$, а вероятность того, что число выпадений герба в точности равняется числу выпадений решки, при четном числе 
бросаний, стремится к $0$ при $n\to\infty$. 
\end{problem}



\begin{problem} (переместить к 7)
Пусть с.в. $x_n\in \Gamma(\lambda,n)$. Покажите, что из ц.п.т. следует 
$$
\frac{x_n-m(\lambda)\cdot n}{\sigma(\lambda)\cdot\sqrt{n}} \xrightarrow{d} N(0,1) \text{ при } n\to\infty . 
$$
Найдите $m(\lambda)$, $\sigma(\lambda)$. 
\end{problem}



\begin{problem}
Пусть $X_n$ --- последовательность независимых с.в., сходящаяся по вероятности к с.в. $X:\; X_n\xrightarrow{P}X$. Докажите, 
что с.в. $X$ вырождена, т.е. $X\equiv x$, где $x$ --- некоторое число. 
\end{problem}

\begin{ordre}

Справедливы следующие утверждения:

\begin{enumerate}
\item
Из любой сходящейся по мере (в частности, по вероятностной) последовательности 
измеримых функций (в частности, с.в.) можно выделить подпоследовательность, сходящуюся почти всюду (п.н.). 

\item
Из закона нуля и единицы Колмогорова для всякого разбиения прямой ${\mathbb R}$ на борелевские множества $\{ B_m\}_{m\geqslant 1}$ 
ровно для одного $m=m_0:$ $\quad {\mathbb P}(A_{B_{m_0}})=1$, для остальных $m:\quad {\mathbb P}(A_{B_m})=0$, где 
$$
A_{B_m}=\{ \omega: \, X=\lim\limits_{k\to\infty} X_{n_k}\in B_m \} . 
$$
\end{enumerate}
\end{ordre}


\begin{problem} (добавить замечание)
Число $\alpha$ из отрезка $[0, 1]$ назовем нормально приближаемым рациональными числами, если найдутся $c,\varepsilon>0$ такие, что 
при любом натуральном $q$ 
\begin{equation}
\label{BorelKantel}
\min\limits_{p\in {\mathbb Z}} \Bigl|\alpha-\frac{p}{q} \Bigr|\geqslant \frac{c}{q^{2+\varepsilon}} . 
\end{equation}
Используя лемму Бореля-Кантелли, докажите, что множество нормально приближаемых чисел на отрезке $[0, 1]$ имеет Лебегову меру $1$. 

\end{problem}

\begin{problem}
В некотором городе прошел второй тур выборов. Выбор был между двумя кандидатами $A$ и $B$ (графы <<против всех>> на этих выборах не было). 
Сколько человек надо опросить на выходе с избирательных участков, чтобы исходя из ответов можно было определить долю проголосовавших 
за кандидата $A$ с точностью $5\%$ и с вероятностью не меньшей $0.99$. Считайте, что исходя из голосования в первом туре, известно, 
что каждый из кандидатов наберет не меньше $30\%$ голосов избирателей. 
\end{problem}


\begin{problem}

На множестве $n!$ перестановок $n$ различных элементов задано равномерное распределение. Обозначим через $\xi_k$ случайную величину, 
равную числу инверсий, образованных элементом с номером $k$, т.е. равную числу элементов с номерами меньшими чем $k$, 
которые стоят в перестановке правее элемента с номером $k$. Покажите, что 
$$
\frac{\sum\limits_{k=1}^{n}\xi_k -\left.n^2\right/4}{\left.n^{3/2}\right/6}\xrightarrow{d} N(0,1) \quad \text{ при } n\to\infty . 
$$
\end{problem}


\begin{ordre}

$ $

\begin{enumerate}

\item
 Введем с.в. 
$$
\xi_{k,i}={\mathcal I}(\text{<<$k$ находится левее числа $i$>>}) . 
$$

Тогда 
$$
\xi_k=\xi_{k,1}+\xi_{k,2}+\ldots +\xi_{k,k-1}, 
$$

Покажите, что
$${\mathbb E}\xi_k=\left.(k-1)\right/2$$,

$$
{\mathbb E}\xi_k^2={\mathbb E}\xi_{k,1}^2+\ldots+{\mathbb E}\xi_{k,k-1}^2+2\sum\limits_{i<j<k}{\mathbb E}(\xi_{k,i}\xi_{k,j})=
$$
$$
=\frac{k-1}{2}+2\cdot\frac{(k-1)(k-2)}{2}\cdot\frac{1}{3}=\frac{2k^2-3k+1}{6} , 
$$


с.в. $\xi_k$ и $\xi_m$ некоррелированы, $k\ne m$. 


\item 
Для всякой сл.в. $X_k=\xi_k-{\mathbb E}\xi_k$ характеристическая функция имеет вид 
\[
\varphi_{X_k}(t)=1-\frac{t^2 \Var\xi_k}{2}+{\overline o}(t^2)
\]

\end{enumerate}
\end{ordre}

