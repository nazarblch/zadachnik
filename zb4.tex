\section{Сходимость и закон больших чисел}

\begin{problem}
Пусть случайный вектор $X^{n} $ имеет равномерное распределение на единичной сфере в ${\mathbb R}^{n} $. Пусть $Y^{n} $ -- проекция $X^{n} $ на первую координатную ось. Докажите, что последовательность $\sqrt{n} Y^{n} $ сходится по распределению к стандартной нормальной случайной величине.

\begin{ordre} 
 Воспользоваться леммой Пуанкаре.
\end{ordre}
\end{problem}

\begin{problem}

Рассмотрим простую и классическую схему блуждания точки по прямой, соответствующую правилам игры в орлянку:
\[\eta (0)=0,\] 
\[\eta (t+1)=\left\{\begin{array}{cc} {\eta (t)+1,} & {p={1\mathord{\left/ {\vphantom {1 2}} \right. \kern-\nulldelimiterspace} 2} } \\ {\eta (t)-1,} & {p={1\mathord{\left/ {\vphantom {1 2}} \right. \kern-\nulldelimiterspace} 2} } \end{array}\right. .\] 
Занумеруем в порядке возрастания все моменты времени, когда $\eta (t)=0$. Получим с вероятностью 1 бесконечную последовательность $0=\tau _{0} <\tau _{1} <\tau _{2} <...$ Рассмотрим разности $\xi _{i} =\tau _{i} -\tau _{i-1} $, $i=1,2,...$ -- последовательность независимых одинаково распределенных с.в.

\textbf{а)} Найдите распределение $\xi _{i} =\tau _{i} -\tau _{i-1} $, т.е. $P\left\{\xi _{i} =2m\right\}$

\textbf{б)} Покажите, что математическое ожидание с.в. $\xi _{i} =\tau _{i} -\tau _{i-1} $ равно бесконечности. Этот результат можно проинтерпретировать так: среднее время до первого возвращения блуждания в 0 бесконечно.

Тем не менее суммы $\tau _{n} =\sum _{i=1}^{n}\xi _{i}  $ при надлежащей нормировке подчинены предельному распределению: 
\[\mathop{\lim }\limits_{n\to \infty } P\left\{\frac{2\tau _{n} }{\pi n^{2} } <z\right\}=\left\{\begin{array}{cc} {\frac{1}{\sqrt{2\pi } } \int _{0}^{z}e^{-\frac{1}{2x} } x^{-\frac{3}{2} }  dx,} & {z>0} \\ {0,} & {z<0} \end{array}\right. .\] 
Из теоремы о каноническом представлении устойчивых законов (см. лекции,\textbf{ теорема Леви-Хинчина}): Для того чтобы функция распределения была устойчивой, необходимо и достаточно, чтобы логарифм ее характеристической функции представлялся формулой:
\[\ln \varphi (t)=i\gamma t-c|t|^{\alpha } \left(1+i\beta \frac{t}{|t|} \omega (t,\alpha )\right),\] 
где $\gamma $ -- любое действительное число, $-1\le \beta \le 1$, $0<\alpha \le 2$, $c\ge 0$ и
\[\omega (t,\alpha )=\left\{\begin{array}{cc} {tg\left(\frac{\pi }{2} \alpha \right),} & {\alpha \ne 1,} \\ {\frac{2}{\pi } \ln |t|,} & {\alpha =1} \end{array}\right. ,\] 
следует, что такое распределение соответствует каноническому представлению с $\alpha ={1\mathord{\left/ {\vphantom {1 2}} \right. \kern-\nulldelimiterspace} 2} $, $\beta =1$, $\gamma =0$, $c=1$, и принадлежит семейству кривых Пирсона (показано Н. В. Смирновым).

\end{problem}

\begin{problem}
Bывод распределения Хольцмарка: Рассмотрим шар радиуса $r$ с центром в начале координат и $n$ звезд (точек), расположенных в нем наудачу и независимо друг от друга. Пусть каждая звезда имеет единичную массу. Обозначим $X_{1} ,...,X_{n} $ x-компоненты гравитационных сил, соответствующие отдельным звездам, и положим $S_{n} =X_{1} +...+X_{n} $. Устремим $r$ и $n$ к бесконечности так, чтобы $\frac{4}{3} \pi r^{3} n^{-1} \to \lambda $. Показать, что распределение величины $S_{n} $ стремится к симметричному устойчивому распределению с $\alpha ={3\mathord{\left/ {\vphantom {3 2}} \right. \kern-\nulldelimiterspace} 2} $. Можно показать, что задача по существу не изменится, если массу каждой звезды считать с.в. с единичным математическим ожиданием и массы различных звезд предполагать взаимно независимыми с.в. и не зависящими также от их расположения.
\end{problem}